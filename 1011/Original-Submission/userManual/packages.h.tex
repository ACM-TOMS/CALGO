% !TeX encoding = UTF-8 Unicode
% !TeX root = manual.tex
% !TeX spellcheck = en_GB

%Babel zuerst
%===============
%\usepackage[utf8]{inputenc}
\usepackage[ngerman,english]{babel}		%ngerman für Neue Deutsche Rechtschreibung, english für Englisch
\def\ensuretext{\textrm}



\usepackage{graphicx}
%\usepackage{hyperref}



\setcounter{secnumdepth}{1}
\setcounter{tocdepth}{2}

%=============================================
\usepackage{amsmath, amsthm, amssymb, amsfonts, mathtools}
\usepackage[a4paper, top=3cm, bottom=3cm, left=2cm, right=2cm]{geometry}
\usepackage{xcolor}
%\usepackage[verbose]{placeins}	%Verhinder Floating über subsections
\usepackage{graphicx}
\usepackage[autostyle=true]{csquotes}
\usepackage{mathdots} %Rede­fines \ddots and \vdots, and de­fines \id­dots.
\usepackage{upquote} %needed for correct quotes in the listings
\usepackage{listings}

\usepackage[inline]{enumitem} %For enumeration
\newlist{param}{description}{10}
\setlist[param]{style=unboxed, font=\texttt, labelindent=\parindent,leftmargin=*, itemsep=0ex, itemindent=!}
%\usepackage{upquote}
%\usepackage{multicol}

%\usepackage{soul}		%um gesperrt zu schreiben mit Befehl \so{gesperrter Text} 
%\usepackage[inline]{enumitem}
%\usepackage{cnenum}

\usepackage[bookmarks=true, breaklinks=true, colorlinks=false, pdfborder={0 0 0}, pdfa, draft=false]{hyperref} 
\providecommand\phantomsection{} %allows to comment out hyperref package
% Optionen weg wenn was nicht geht, Muss als letztes Package da stehn




%Design Änderungen, Compile Änderungen
%=======================================
%Sorgt dafuer, dass Bilder erst auf einzelner Seite stehn wenn weniger als 20% Text darauf waer
%\renewcommand{\floatpagefraction}{0.8}	
\vfuzz2pt                 % Don't report over-full v-boxes if over-edge is small
\hfuzz2pt                 % Don't report over-full h-boxes if over-edge is small

                          
%Weitere Wichtige Befehle
%==========================
\definecolor{darkyellow}{rgb}{0.65, 0.5, 0.05}
\definecolor{darkgreen}{rgb}{0.05, 0.5, 0.05}
\newcommand{\XX}[1]{%
    {\color{red}\footnotesize%
    \phantomsection\addcontentsline{lof}{section}{\protect\framebox{\textbf{XX}}\protect\color{red} #1}%
    \framebox{\textbf{XX}}\emph{\textbf{#1}}}}%


%%%%%%%%%%%%%%%%%%%
% Specific stuff
%%%%%%%%%%%%%%%%%%%%%%

\newcommand{\fwsym}[1]{\parbox[c]{.9em}{\protect\ensuremath{#1}}}
\newcommand{\ppmm}{\fwsym{\pm}}
\newcommand{\Circ}{{\raisebox{-0.2ex}{\scalebox{1.4}{$\circ$}}}}
\newcommand{\pp}{\fwsym{+}}
\newcommand{\mm}{\fwsym{-}}
\newcommand{\nn}{\fwsym{\Circ}}
\newcommand{\seqj}[0]{{\,\boldsymbol{j}}}
\newcommand{\nablal}[0]{{\nabla_{\!l}\,}}
\newcommand{\nablak}[0]{{\nabla^{k}\,}}
\newcommand{\nablakp}[0]{{\nabla^{k+1}\,}}
\newcommand{\nablamu}[0]{{\tilde{\nabla}^{\mu}\,}}
\newcommand{\assS}[0]{\ensuretext{\textbf{S}}}
%\newcommand{\coast}{\operatorname{co}_\ast}
%\DeclareMathOperator{\hausdorff}{h}

\newcommand{\FWFtext}[1]{{\color{gray}#1}}

\let\subsetneq\undefined %undefine subsetneq, since we only want to use \subsetneqq
\let\subset\undefined %undefine subset, since we only want to use \subseteq

\DeclareMathOperator{\JSR}{JSR}
\DeclareMathOperator{\LSR}{LSR}
\DeclareMathOperator{\RSR}{RSR}
\DeclareMathOperator{\capacity}{cap}

%\swapnumbers %To write the numbers before Theorem/Definition/...

%%%%%% Redefine :=
%\mathchardef\ordinarycolon\mathcode`\:
%\mathcode`\:=\string"8000
%\begingroup \catcode`\:=\active
%  \gdef:{\mathrel{\mathop\ordinarycolon}}
%\endgroup

\numberwithin{equation}{section}

\hyphenation{bal-anc-ing}
\hyphenation{Dau-bechies}
\hyphenation{wave-lets}
\hyphenation{wave-let}

%%%%%%%%%%%%%%%
% Mathematical stuff
%%%%%%%%%%%%%%%
%
%Theorems
%=======================

\newtheorem{theorem}{Theorem}[section]
\newtheorem{lemma}[theorem]{Lemma}
\newtheorem{assumption}[theorem]{Assumption}
\newtheorem{proposition}[theorem]{Proposition}
\newtheorem{corollary}[theorem]{Corollary}
%\newtheorem{claim}[theorem]{Claim}
\newtheorem{conjecture}[theorem]{Conjecture}
\newtheorem{goal}{Goal}[]

\theoremstyle{definition}
%\newtheorem*{claim*}{Claim}
\newtheorem{definition}[theorem]{Definition}
\newtheorem{remark}[theorem]{Remark}
\newtheorem{example}[theorem]{Example}
\newtheorem{algorithm}[theorem]{Algorithm}
\newtheorem{algdef}[theorem]{Algorithm and Definition}

%\newtheorem{exampleX}[theorem]{Example}
%\newcommand{\examplesymbol}{\ensuremath{\triangle}}
%\newenvironment{example}%To add a symbol at the end of the example environment
%{%
%    \pushQED{\qed}%
%    \renewcommand{\qedsymbol}{\examplesymbol}
%    \exampleX%
%}
%{
%    \popQED%
%    \endexampleX%
%}%

%% Various commands
%=======================
\newcommand\numberthis{\addtocounter{equation}{1}\tag{\theequation}}
%\newcommand{\pbrackets}[1]{\left( #1 \right)}
%\newcommand{\Pbrackets}[1]{\left( #1 \vphantom{\left(#1\right)^A} \right)}
\newcommand*\justify{% for texttt % use as \texttt{\justify lorem ip sum)},
%  \fontdimen2\font=0.4em% interword space
%  \fontdimen3\font=0.2em% interword stretch
%  \fontdimen4\font=0.1em% interword shrink
%  \fontdimen7\font=0.1em% extra space
  \hyphenchar\font=`\-% allowing hyphenation
}



%% Symbols of functions
%====================
%\DeclareMathOperator{\FT}{\mathcal{F}}
\DeclareMathOperator{\spn}{span}
\DeclareMathOperator{\supp}{supp}
\newcommand{\lebesgue}[1]{\lambda\left( #1 \right)} \let\lebesgue\lebesgue
\newcommand{\Fourierhat}[1]{#1\!\widehat{\phantom{x}}} \let\Fourierhat\Fourierhat  %Fourier-Transform Sign with hat
\newcommand{\floor}[1]{\left\lfloor #1 \right\rfloor} 
\DeclareMathOperator*{\esssup}{ess\,sup}
\DeclareMathOperator{\closure}{cl}
\newcommand{\interior}{\circ}
%\newcommand{\boundary}{\partial} %semms to be defined somewhere
\DeclareMathOperator*\infp{\vphantom{p}inf} %add depth to "inf", so that the subscripts under "sup inf" are at the same height.
\DeclareMathOperator*\limp{\vphantom{p}lim} %add depth to "inf", so that the subscripts under "sup inf" are at the same height.
%\DeclareMathOperator*{\trunc}{trunc}
%\newcommand{\ceil}{[1]{\rotatebox{180}{\reflectbox{\lfloor \rotatebox{180}{\reflectbox{\lfloor #1\rfloor}}\rfloor}}}

%% Symbols of any other mathemical stuff
%==============================
\newcommand{\norm}[1]{\left|\left|#1\right|\right|}
\newcommand{\abs}[1]{\left|#1\right|}
\newcommand{\set}[1]{\left\lbrace #1 \right\rbrace}

%% Greek Letters
%================================
\newcommand{\Zeta}{\textrm{Z}} \let\Zeta\Zeta %Zeta
\newcommand{\Nu}{\mathcal{V}} \let\Nu\Nu %Nu
%\newcommand{\eps}{\varepsilon} \let\eps\eps
\newcommand{\Varepsilon}{\mathcal{E}}  \let\Varepsilon\Varepsilon
\newcommand{\Eps}{\Varepsilon}  \let\Eps\Eps
\newcommand{\omikron}{\mathcal{o}}  \let\omikron\omikron
\newcommand{\Omikron}{\mathcal{O}}  \let\Omikron\Omikron

\newcommand{\vardot}{\mathord{\,\cdot\,}}  \let\vardot\vardot

\newlength{\CapLen}
\AtBeginDocument{\settoheight{\CapLen}{A}} % after \normalsize
\newcommand{\Varmu}{\resizebox{!}{\CapLen}{$\mu$}} \let\Varmu\Varmu %Varmu
%================================
%% with double line
\newcommand{\CC}{\mathbb{C}}  \let\CC\CC
%\newcommand{\CCx}{\mathbb{C}_\times}  \let\CCx\CCx
\newcommand{\RR}{\mathbb{R}}  \let\RR\RR
\newcommand{\QQ}{\mathbb{Q}}  \let\QQ\QQ
\newcommand{\NN}{\mathbb{N}}  \let\NN\NN
\newcommand{\ZZ}{\mathbb{Z}}  \let\ZZ\ZZ



\newcommand{\absco}{\operatorname{co}}
\newcommand{\coast}{\operatorname{co}_{\ast}}
\newcommand{\comin}{\operatorname{co}_-}

%\newcommand{\numberthis}{\addtocounter{equation}{1}\tag{\theequation}}
%\newcommand{\Circ}{{\raisebox{-0.2ex}{\scalebox{1.4}{$\circ$}}}}

\newcommand{\defval}[1]{default: #1}


\setcounter{MaxMatrixCols}{20}

\makeatletter
\newcommand\footnoteref[1]{\protected@xdef\@thefnmark{\ref{#1}}\@footnotemark}
\makeatother



%% Matrices
%===================
\setcounter{MaxMatrixCols}{30} %Increase maximal number of columns in 'matrix'
%\newcommand{\tpmatrix}[2][r]{\begin{pmatrix*}[#1]#2\end{pmatrix*}}
%\newcommand{\tpsmatrix}[2][r]{\left(\begin{smallmatrix*}[#1]#2\end{smallmatrix*}\right)}
\renewenvironment{pmatrix}{}{\typeout{Use bmatrix instead of pmatrix.}}
\newcommand{\tbmatrix}[2][r]{\begin{bmatrix*}[#1]#2\end{bmatrix*}}
\newcommand{\tbsmatrix}[2][r]{\left[\begin{smallmatrix*}[#1]#2\end{smallmatrix*}\right]}
\newcommand{\tmatrix}[2][r]{\begin{matrix*}[#1]#2\end{matrix*}}
\newcommand{\tsmatrix}[2][r]{\begin{smallmatrix*}[#1]#2\end{smallmatrix*}}

%%%%%%%%%%%%%%%%%%%%%%%%%%%%%%
%% EOF %%%%%%%%%%%%%%%%%%%%%%%
%%%%%%%%%%%%%%%%%%%%%%%%%%%%%%