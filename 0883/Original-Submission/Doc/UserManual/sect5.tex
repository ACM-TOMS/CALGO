% Skim 5/15/07
\section{Description of main and principal subfunctions}
%The main MATLAB functions and important subfunctions}
\label{mainFunctions}

The main function and principal subfunctions are described in terms of input and output
arguments in this section. 

\subsection{The MATLAB functions sparsePOP.m, SDPrelaxation.m, and SDPrelaxationMex.m} 


The main function sparsePOP.m, its principal subfunctions SDPrelaxation.m 
% Skim 5/15/07
and SDPrelaxationMex.m  shown in Figure~\ref{structure} 
have the following function declarations: 
\begin{verbatim}
function [param,SDPobjValue,POP,cpuTime,SeDuMiInfo,SDPinfo] = ...
    sparsePOP(objPoly,ineqPolySys,lbd,ubd,param); 

function [param,SDPobjValue,POP,cpuTime,SeDuMiInfo,SDPinfo] = ...
    SDPrelaxation(param,objPoly,ineqPolySys,lbd,ubd);

function [param,SDPobjValue,POP,cpuTime,SeDuMiInfo,SDPinfo] = ...
    SDPrelaxationMex(param,objPoly,ineqPolySys,lbd,ubd);
\end{verbatim}
respectively. These three functions have the same input and output 
arguments. Among the input arguments, {\sf param} contains a set of parameters
% Skim 5/15/07 
% Kojima 5/18/07 
% for choosing certain algorithmic procedures. % their behavior.
% More 
whose detailed description is included in Section \ref{PARAM}. 
The other input arguments, if all of them are specified,   describe a POP 
in the SparsePOP format, as presented in Section~4.2. 

Although sparsePOP.m is defined with 5 input arguments, using 1 or 2 input arguments
is also possible as mentioned in Section~\ref{sample}. 
%the following usages with 1 or 2 input arguments are also possible: 
\begin{itemize}
\item ${\bf >>}$ {\sf sparsePOP('example1.gms')} for solving a POP described in the GAMS scalar format 
with the default {\sf param}. 
\item ${\bf >>}$ {\sf sparsePOP('example1.gms',param)} for solving a POP described in the GAMS scalar 
format with the user-specified {\sf param}. 
\item ${\bf >>}$ {\sf sparsePOP('example1')} for solving a POP described in the SparsePOP  format with 
the default {\sf param}. 
\item ${\bf >>}$ {\sf sparsePOP('example1',param)} for solving a POP described in the SparsePOP format 
with the user-specified {\sf param}. 
\end{itemize}
%These have been illustrated in Section~\ref{sample}. 

If the SDPrelaxation.m or the SDPrelaxationMex.m is to be utilized directly,
%On the other hand,
 either a set of the 5 input arguments 
\[
\mbox{ {\sf param}, {\sf objPoly}, {\sf ineqPolySys}, {\sf lbd} and {\sf ubd} } 
\]
or a set of the 3 input arguments 
\[
\mbox{ {\sf param}, {\sf objPoly} and {\sf ineqPolySys}} 
\]
needs to be specified. %if the user utilize the sparsePOPmain.m or the sparsePOPmainMex.m directly. 
%In the latter  3 input arguments case, 
In the case that 3 input arguments are prescribed,
the functions SDPrelaxation.m and SDPrelaxationMex.m assign the default values 
% Skim 5/24/07
{\sf lbd}$(i) = -1.0$e+$10$ and  {\sf ubd}$(i) = +1.0$e+$10$ $(i=1,2,\ldots,n)$, which implies that 
$-\infty < x_i < \infty$ $(i=1,2,\ldots,n)$.  

For the output arguments, user-specified or default values for the parameters are stored  in {\sf param}.
{\sf SDPobjValue} contains a lower bound for the optimal objective value of the POP (\ref{POP0}). 
% Skim 5/15/07
\mbox{For every feasible solution $\x$ of the POP (\ref{POP0})},
\begin{eqnarray}
\mbox{{\sf SDPobjValue}}  \leq f_0(\x) \label{lowerBound} 
\end{eqnarray}
holds. %\mbox{for every feasible solution $\x$ of (\ref{POP0})}. 
The output argument {\sf POP} has four components: 
\begin{itemize}
\item {\sf POP.xVect}: a candidate $\x^{\omega}$ of an optimal solution of  the POP (\ref{POP0}). 
\item {\sf POP.objValue}: 
the objective function value $f_0(\x^{\omega})$ at $\x^{\omega} = $ {\sf POP.xVect}.
\item {\sf POP.absError}: an  absolute feasibility error at  $\x^{\omega}$. 
\item {\sf POP.scaledError}: a scaled feasibility error $\x^{\omega}$. 
\end{itemize}
(Recall that the relaxation order $\omega = $ {\sf param.relaxOrder} determines the 
quality of the SDP relaxation of the POP). % to be solved). 
Here the absolute feasibility error 
at $\x^{\omega}$ is given by 
\begin{eqnarray*}
\min \left\{ \min\{ f_i(\x^{\omega}), \ 0 \} \ (i=1,2,\ldots,\ell), \ - |f_j(\x^{\omega})| \ (j=\ell+1,\ldots,m) \right\}, 
\end{eqnarray*}
and the scaled feasibility error  
is given by 
\begin{eqnarray*}
\min \left\{ \min\{ f_i(\x^{\omega})/\sigma_i(\x^{\omega}), \ 0 \} \ (i=1,2,\ldots,\ell), \ - |f_j(\x^{\omega})|/\sigma_j(\x^{\omega}) \ (j=\ell+1,\ldots,m) \right\}, 
\end{eqnarray*}
where $\sigma_i(\x^{\omega})$ denotes the maximum of the absolute values 
of all monomials 
of $f_i(\x)$ evaluated at $\x^{\omega}$ if the maximum is greater than $1$, 
or $\sigma_i(\x^{\omega}) = 1$ otherwise $(i=1,2,\ldots,m)$. 
Note that both errors are always nonpositive. 
The relative error in the objective value at $\x^{\omega}$ in the output of sparsePOP.m, 
which has been illustrated in Section 4,  is computed as 
\[
\mbox{{\sf   relative obj error}} = \frac{{\sf POP.objValue} - {\sf SDPobjValue}}{\max \{ 1, |{\sf POP.objValue}| \}}. 
\]
If {\sf POP.scaledError} $\leq 0$ is close to $0$, say $-1.0$e-$6 \leq $  {\sf POP.scaledError} $\leq 0$, 
we may regard that $\x^{\omega}$ is feasible approximately. If, in addition, 
{\sf   relative obj error} $ \geq 0$ is close to $0$, say $0 \leq $ {\sf   relative obj error} $\leq 1.0$e-$6$, 
% Skim 5/15/07 provides
$\x^{\omega}$ is an approximate optimal solution of the POP (\ref{POP0}). 

The output argument {\sf cpuTime} shows various cpu times consumed by 
% Skim 5/24/07
%the execution of
 sparsePOP.m:
\begin{itemize}
\item  {\sf cpuTime.conversion}: 
the cpu time consumed to convert the POP into its SDP relaxation. 
\item {\sf cpuTime.SeDuMi}: the cpu time consumed by SeDuMi to solve the SDP.
\item {\sf cpuTime.Total}: the cpu time for the entire process.
\end{itemize}

The output argument {\sf SDPinfo} has information of the SDP relaxation problem solved 
by SeDuMi.
\begin{itemize}
\item {\sf SDPinfo.rowSizeA}:
the number of rows  of the coefficient matrix $\A$ of the SDP.  
\item {\sf SDPinfo.colSizeA}:
the number of  columns of the coefficient matrix $\A$. % of the primal SDP 
\item {\sf SDPinfo.nonzeroInA}: the number of nonzeros of the 
coefficient matrix $A$.
\item {\sf SDPinfo.noOfLPvariables}:
the number of LP variables of the SDP.
\item {\sf SDPinfo.noOfFRvariables}: the number of free variables of the SDP. 
\item {\sf   SDPinfo.SDPblock}:
the row vector of the sizes of SDP blocks.
\end{itemize}

Finally, the output argument 
{\sf SeDuMiInfo} contains {\sf SeDuMiInfo.numerr}, {\sf SeDuMiInfo.pinf},
and {\sf SeDuMi.dinf}, which
are equivalent to info.numerr, info.pinf, and info.dinf in SeDuMi output.
See \cite{STRUM99} for the details.

\subsection{The MATLAB function readGMS.m} 

The MATLAB function readGMS.m has the following function declaration:
\begin{verbatim}
function [objPoly,ineqPolySys,lbd,ubd] = readGMS(fileName,symbolicMath);
\end{verbatim}
% Skim 5/15/07
The first argument {\sf fileName}, a string in MATLAB,
is the name of the file where a problem is described
in the \GMS format. It must have the extension .gms such as 'example1.gms'. 
The second input argument {\sf symbolicMath} is set to be $1$ by default, assuming that 
the Symbolic Math Tool is available. It should be set to $0$ if it is not available. 
The output of {\sf objPoly}, {\sf ineqPolySys}, {\sf lbd}, and {\sf ubd} 
is a POP data in  the SparsePOP format, and can be passed to SDPrelaxation.m
or SDPrelaxationMex.m.


\subsection{The MATLAB function printSolution.m} 

The function printSolution.m for printing the results has
the following function declaration.
\begin{verbatim}
function printSolution(fileId,printLevel,dataFileName,param,SDPobjValue,...
        POP,cpuTime,SeDuMiInfo,SDPinfo);
\end{verbatim}
The meaning of each input argument is as follows.
\begin{itemize}
\item {\sf fileId}: fileId where output is printed. If fileId is $1$,
then the result is displayed on the screen (i.e., the standard output).
If the result in a file is desired,
% Skim 5/15/07 
the file should be open in writable mode before specifying it in fileId.
\item {\sf printLevel}: a larger value of printLevel gives more detailed description
of the result. Default is $2$.
\item {\sf dataFileName}: the name of the problem solved.
\end{itemize}
The rest of the input arguments, i.e.,
{\sf param},
{\sf SDPobjValue},
{\sf POP},
{\sf cpuTime},
{\sf SeDuMiInfo}, and 
{\sf SDPinfo} should be the output of 
SDPrelaxation.m and SDPrelaxationMex.m.



