\section{SDP relaxation of POPs using csp graphs} \label{sec:csp}

% Skim 5/15/07
We briefly explain
the sparse SDP relaxation of the POP (\ref{POP0})
used in SparsePOP.  For simplicity of discussions, 
the POP (\ref{POP0}) considered here  does not have
equality constraint, any lower bound and any upper bound on the variables $x_i$ $(i=1,2,\ldots,n)$; 
$\ell = m$, $\mbox{lbd}_i = -\infty$ and $\mbox{ubd}_i = \infty$ $(i=1,2,\ldots,n)$. 
See \cite{WAKI04} for details of more general cases.

Let $N = \{1,2,\ldots,n\}$. 
For every $k=1,2,\ldots,m$, let 
\begin{eqnarray*}
      F_k = \left\{ i \in N : \alpha_i \geq 1 \ \mbox{for some }  \balpha \in \FC_k \right\}.
\end{eqnarray*}
Then, a graph $G(N,E)$ 
representing the sparsity structure of
the POP (\ref{POP0}) is constructed. More specifically, 
a pair $\{i, \ j\}$ with $i \neq \ j$ selected from
the node set $N$ 
is an edge or $\{i, \ j\} \in E$ if and only if either there is an $\balpha \in \FC_0$ such that
$\alpha_i > 0$ and $\alpha_j > 0$ or $i, \ j \in F_k$ for some $k=1,2,\ldots,m$.
The graph $G(N, E)$ is called 
\textit{a correlative sparsity pattern (csp) graph}.
By construction, each $F_k$ is a clique of $G(N,E)$ $(k=1,2,\ldots,m)$.
% Skim 5/15/07
We then generate a chordal extension $G(N,E^{\prime})$ of $G(N,E)$.
(See, for example, \cite{BLAIR93} for the definition and basic properties of chordal graphs). 
Let $C_1,C_2,\ldots,C_p$ be the maximum cliques of $G(N, E^{\prime})$. 
Note that  $C_1,C_2,\ldots,C_p$ can  be easily computed because 
$G(N, E^{\prime})$ is a chordal graph.

For every 
$C \subset N$ and $\psi \in \Integer_+$, define 
\begin{eqnarray*}
\AC_{\psi}^C & = & 
\left\{ \balpha \in \Integer^n_+ : 
\alpha_j = 0 \ \mbox{if } j \not\in C, \
\sum_{i \in C} \alpha_i \leq \psi
 \right\}, 
\end{eqnarray*}
and let $\u(\x,\AC_{\psi}^C)$ denote a column vector consisting of 
the monomials $\x^{\balpha}$ $(\balpha \in \AC_{\psi}^C)$. We assume 
that the elements $\x^{\balpha}$ $(\balpha \in \AC_{\psi}^C)$ in the vector
% Skim 5/15/07 
$\u(\x,\AC_{\psi}^C)$ are arranged according to  lexicographically  
increasing order of $\balpha$'s. Notice that $\0 \in \AC_{\psi}^C$;
hence the first element of the column vector 
$\u(\x,\AC_{\psi}^C)$ is always $\x^{\0} = 1$. 

For every $k=0,1,2,\ldots,m$, let
$\omega_k = \lceil \deg(f_k(\x)) / 2 \rceil$, 
\begin{equation}
\omega_{\max} = \max \{ \omega_k : k =0,1,\ldots,m \}, 
\label{omegaMax}
\end{equation}
and let $\widetilde{C}_k$ be the union of some of 
the maximal cliques $C_1, C_2,\ldots,C_p$ of $G(N,E^{\prime})$ 
such that $F_k \subset \widetilde{C}_k$. 

% Skim 5/15/07
To derive an SDP relaxation, 
the POP  (\ref{POP0}) is first transformed  into  an equivalent polynomial SDP (PSDP)
\begin{equation}
\left.
\begin{array}{lll}
\mbox{minimize } & f_0(\x) \\
\mbox{subject to } &  \u(\x,\AC^{\widetilde{C}_k}_{\omega-\omega_k})
    \u(\x,\AC^{\widetilde{C}_k}_{\omega-\omega_k})^T f_k(\x)
\succeq \O 
\ (k=1,2,\ldots,m), \vspace{0.2cm} \\
& \u(\x,\AC^{C_{\ell}}_{\omega}) \u(\x,\AC^{C_{\ell}}_{\omega})^T
\succeq \O 
\ (\ell=1,2,\ldots,p)
\end{array}
\right\} \label{PolySDP0}
\end{equation}
with  some {\it relaxation order}  $\omega \geq \omega_{\max}$. 
Here $\B \succeq \O$ denotes that a real symmetric matrix $\B$ 
is positive semidefinite. 
The matrices $\u(\x,\AC^{\widetilde{C}_k}_{\omega-\omega_k})
    \u(\x,\AC^{\widetilde{C}_k}_{\omega-\omega_k})^T$ 
$(k=1,2,\ldots,m)$
and $\u(\x,\AC^{C_{\ell}}_{\omega}) \u(\x,\AC^{C_{\ell}}_{\omega})^T$
$(\ell=1,2,\ldots,p)$ are  positive semidefinite symmetric matrices of
rank one
for any $\x \in \Real^n$, and have  the element $1$ in their upper left corner.
These ensure the equivalence between the POP (\ref{POP0}) and
the PSDP (\ref{PolySDP0}). 

Since the objective function of the PSDP (\ref{PolySDP0}) is a real-valued polynomial and 
the left hand side 
of the matrix inequality constraints of the PSDP (\ref{PolySDP0}) are 
real symmetric matrix valued polynomials, we can rewrite the PSDP (\ref{PolySDP0}) as
\[
\begin{array}{lllllll}
\mbox{minimize } & \displaystyle \sum_{\balpha \in \widetilde{\FC} }
\tilde{c}_0(\balpha) \x^{\salpha} &
\mbox{subject to } &  \displaystyle
\M(\0,\omega) + \sum_{\balpha \in \widetilde{\FC} } \M(\balpha,\omega) \x^{\salpha}
\succeq \O.
\end{array}
\]
for some $\widetilde{\FC} \subset \Integer^n_+$, some 
$\tilde{c}_0(\balpha) \in \Real$ $(\balpha \in \widetilde{\FC})$ and some 
real symmetric matrices 
$\M(\balpha,\omega)$ $(\balpha \in \widetilde{\FC} \bigcup \{ \0\})$. 
Note that the size of the matrices $\M(\balpha,\omega)$ $(\balpha \in \widetilde{\FC} \bigcup \{ \0\})$ 
and the number of variables $y_{\salpha}$ $(\balpha \in \widetilde{\FC})$ are determined by 
the relaxation order $\omega$. 
Each monomial $\x^{\salpha}$ is replaced by
a single real variable $y_{\salpha}$, and we have an SDP relaxation problem of the POP 
(\ref{POP0}):
\begin{equation}
\begin{array}{lllllll}
\mbox{minimize } & \displaystyle \sum_{\balpha \in \widetilde{\FC} }
\tilde{c}_0(\balpha) y_{\salpha} &
\mbox{subject to } &  \displaystyle
\M(\0,\omega) + \sum_{\balpha \in \widetilde{\FC} } \M(\balpha,\omega) y_{\salpha}
\succeq \O,
\end{array}
\label{PrimalSDP}
\end{equation}
% The problem (\ref{PrimalSDP}) is an SDP relaxation, 
which can be solved by standard SDP solvers \cite{STRUM99,SDPT3,YAMASHITA03}.
SparsePOP calls SeDuMi \cite{STRUM99} 
to solve the resulting SDP (\ref{PrimalSDP}) and/or outputs 
the data of (\ref{PrimalSDP}) in the SDPA sparse format. 

% Skim 5/15/07
From the solution obtained by SeDuMi, we can find an approximate solution.
Let 
$\zeta_{\omega}$ denote the optimal objective value of the SDP (\ref{PrimalSDP}), and 
$y_{\salpha}^{\omega}$ $(\balpha \in \widetilde{\FC})$ an optimal solution of the SDP (\ref{PrimalSDP}). 
We extract 
$ 
y_{\salpha}^{\omega} \ (\balpha \in \{ \e^{i} : i \in N\} ), 
$ 
which are induced from 
the variable $x_i$ $(i \in N)$ in the original POP (\ref{POP0}), to form 
an approximate optimal solution $\x^{\omega}$ of the POP (\ref{POP0}): 
\begin{eqnarray*}
\x^{\omega} & = & (x_1^{\omega},x_2^{\omega},\ldots,x_n^{\omega}), \ 
x_{i}^{\omega}  = y_{\e^i} \ (i \in N). 
\end{eqnarray*}
The relation $\zeta_{\omega} \leq \zeta_{\omega'} \leq \zeta^* \leq f_0(\x)$ holds if 
$\omega_{\max} \leq \omega \leq \omega'$ and if $\x \in \Real^n$ is a feasible solution 
of the POP (\ref{POP0}). (Recall that $\zeta^*$ denotes the optimal value of the POP 
(\ref{POP0})). Therefore, if $f_0(\x^{\omega})-\zeta_{\omega}$ is zero (or sufficiently small) 
and if $\x^{\omega}$ satisfies (or approximately satisfies) the constraints 
$f_k(\x) \geq 0$ $(k=1,2,\ldots,m),$ then $\x^{\omega}$ is an optimal solution 
(or an approximate optimal solution, respectively) of the POP (\ref{POP0}).  
Theoretically, $\zeta_{\omega}$ converges $\zeta^*$ under a moderate assumption \cite{LAS06}.  

The relaxation order $\omega$,
which 
corresponds to the parameter {\sf param.relaxOrder} used in SparsePOP, 
determines both the quality and the size 
of the SDP relaxation (\ref{PrimalSDP}) of the POP (\ref{POP0}). 
We expect to obtain an approximate optimal solution of the POP (\ref{POP0})
% Skim 5/15/07 quality -> accuracy 
with  higher accuracy (more precisely,  not lower accuracy) by solving  the SDP 
relaxation (\ref{PrimalSDP}) as  a larger  value is chosen for $\omega$. 
In practice, however, the size of  the SDP relaxation (\ref{PrimalSDP}) becomes larger and 
the cost of solving the SDP relaxation (\ref{PrimalSDP}) increases
very rapidly. 
Therefore, we usually start 
the SDP relaxation (\ref{PrimalSDP})  with the relaxation order 
$\omega=\omega_{\max}$, and successively increase $\omega$ by $1$ if 
the approximate 
solution $\x^{\omega}$ obtained is not accurate. 

% Skim 5/15/07
We mention that SparsePOP can solve POPs with polynomial equality constraints, 
lower, and upper bounds for the variables $x_i$ $(i \in N)$ as in the 
POP (\ref{POP0}), and also an unconstrained POP that only objective 
polynomial $f_0 (\x)$ is specified. 



