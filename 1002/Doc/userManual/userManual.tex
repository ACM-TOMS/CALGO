\documentclass{article}

\usepackage{hyperref}
\usepackage{verbatim}

\setlength{\parindent}{0em}
\setlength{\parskip}{1em}

\begin{document}

\title{CPPR Manual}
\author{}
\date{}

\maketitle

\section{Description}

cppr is an implementation of the colored parallel push-relabel algorithm.
There is also a sequential fifo push-relabel algorithm implemented mostly
using same routines. Two different graph representations are implemented,
adjacency list and star graph representations.

This code is structured as a library but it also contains a standalone tool
to use as a solver for DIMACS maximum flow problem files. Solution is then
given as a DIMACS maximum flow solution file with optional flow value
assignments. See the provided samples as an example for these file formats.

This documentation assumes a UNIX based operating system with essential
build tools installed and a POSIX compatible shell for commands. You may
need to adjust these to your setup for other combinations.

This same content can be read from both \texttt{README} and \texttt{doc/manual.pdf} files.

\section{Building}

Run \texttt{make} to compile the tool.

Run \texttt{make run} to run with the sample input.

Run \texttt{make test} to compile and run unit tests.

See \texttt{Makefile} for compile time options and their default values.

\section{Installation}

If you want to install the standalone tool, you can simply copy the binary
to a directory in \texttt{PATH} variable:

\begin{verbatim}
cp bin/cppr /usr/local/bin
\end{verbatim}

If you want to install the library, you can copy header files in \texttt{include}
directory to a system include directory:

\begin{verbatim}
mkdir -p /usr/local/include/cppr
cp include/* /usr/local/include/cppr
\end{verbatim}

You can then include these files with the corresponding prefix in your
source code:

\begin{verbatim}
...
#include <cppr/dimacs.hpp>
#include <cppr/push_relabel.hpp>
...
\end{verbatim}

You may need to elevate permission to use these commands (e.g. \texttt{sudo}).

\section{Compiler}

You need a compiler with at least C++11 and OpenMP 3.1 support for
compilation. This roughly corresponds to version 4.8 onwards for GCC and
version 3.9 onwards for Clang. Detailed feature support information for GCC
and Clang can be found in following pages if you need to use older versions:

\begin{itemize}
    \item \url{https://gcc.gnu.org/projects/cxx-status.html}
    \item \url{https://gcc.gnu.org/wiki/openmp}
    \item \url{https://clang.llvm.org/cxx_status.html}
    \item \url{https://openmp.llvm.org/}
\end{itemize}

\textbf{Note:} Although Clang 3.8 supports the necessary specifications, there is a
bug crashing the frontend when references are used in OpenMP reduction
clauses which seems to be fixed in 3.9 onwards.

\section{Usage}

cppr tool can read the input either from a file given as the first argument
or from stdin when an argument is not provided. Output is given to stdout by
default which can instead be written to a file either using \texttt{-o} option with
a file name or using shell redirections.

An example DIMACS input file and the corresponding DIMACS output file are
provided in \texttt{test/sample} directory. DIMACS file format for the maximum flow
problem is described here:

\begin{itemize}
    \item \url{http://lpsolve.sourceforge.net/5.5/DIMACS_maxf.htm}
\end{itemize}

The example input file is as follows:

\begin{verbatim}
$ cat test/sample/inp.max
c This is a simple example file to demonstrate the DIMACS
c input file format for maximum flow problems. The solution
c vector is [5,10,5,0,5,5,10,5] with cost at 15.
c Problem line (nodes, links)
p max 6 8
c source
n 1 s
c sink
n 6 t
c Arc descriptor lines (from, to, capacity)
a 1 2 5
a 1 3 15
a 2 4 5
a 2 5 5
a 3 4 5
a 3 5 5
a 4 6 15
a 5 6 5
c
c End of file
\end{verbatim}

This can be run with cppr tool as follows:

\begin{verbatim}
$ bin/cppr test/sample/inp.max
c
c cppr v1.0.0
c
c # nodes : 6
c # arcs  : 8
c
c graph     : star
c problem   : maximum flow
c algorithm : colored parallel push-relabel (8 threads)
c
c # pushes          : 7 + 1 = 8
c # relabels        : 1 + 0 = 1
c # discharges      : 1 + 0 = 1
c # global relabels : 1 + 1 = 2
c # colors          : 2
c # color ticks     : 2 + 1 = 3
c
c solution is not checked
c
s 15
c
c total measured time           : 0.000802795
c - init time                   : 4.0107e-05
c   - read time                 : 3.7393e-05
c   - convert residual time     : 2.714e-06
c - solve time                  : 0.000747983 + 1.4705e-05 = 0.000762688
c   - color graph time          : 4.432e-06
c   - discharge time (par)      : 3.8084e-05 + 4.134e-06 = 4.2218e-05
c   - global relabel time (par) : 0.000705467 + 1.0571e-05 = 0.000716038
c - report time                 : 0
c   - check solution time       : 0
c   - write time                : 0
\end{verbatim}

There are also some small examples generated with the tools used in DIMACS
challenge in \texttt{test/dimacs} directory. Output from cppr tool for these
examples can be found in the same directory.

Run \texttt{bin/cppr -h} to see all runtime options along with the usage.

\section{OpenMP Options}

You can use \texttt{OMP\_NUM\_THREADS} variable to set the number of threads used for
the parallel algorithm:

\begin{verbatim}
OMP_NUM_THREADS=2 bin/cppr test/sample/inp.max
\end{verbatim}

You can use \texttt{OMP\_PROC\_BIND} variable to set processor affinities to prevent
thread migrations (recommended):

\begin{verbatim}
OMP_PROC_BIND='TRUE' bin/cppr test/sample/inp.max
\end{verbatim}

When these variables are not set, default values of these options are
implementation dependent.

\section{Intel TBB Allocator}

You can replace the standard memory allocator with scalable memory allocator
from Intel TBB library to avoid false sharing (recommended). You can either
do this at compile time by setting \texttt{LDLIBS} while compiling:

\begin{verbatim}
LDLIBS=-ltbbmalloc_proxy make && bin/cppr test/sample/inp.max
\end{verbatim}

Or at load time using \texttt{LD\_PRELOAD} trick:

\begin{verbatim}
make && LD_PRELOAD=libtbbmalloc_proxy.so.2 bin/cppr test/sample/inp.max
\end{verbatim}

If TBB is installed in a non-standard location, you may need to add
\texttt{-ltbbmalloc} to \texttt{LDLIBS} and the path to TBB library to \texttt{LIBRARY\_PATH} and
\texttt{LD\_LIBRARY\_PATH} beforehand:

\begin{verbatim}
LIBRARY_PATH="/path/to/tbb:$LIBRARY_PATH" \
LDLIBS='-ltbbmalloc -ltbbmalloc_proxy' \
make && \
LD_LIBRARY_PATH="/path/to/tbb:$LD_LIBRARY_PATH" \
bin/cppr test/sample/inp.max
\end{verbatim}

Or if you want to use the second method, then you only need to add the path
to \texttt{LD\_LIBRARY\_PATH} before running:

\begin{verbatim}
make && \
LD_LIBRARY_PATH="/path/to/tbb:$LD_LIBRARY_PATH" \
LD_PRELOAD=libtbbmalloc_proxy.so.2 \
bin/cppr test/sample/inp.max
\end{verbatim}

See the related section in TBB documentation for more information:

\begin{itemize}
    \item \url{https://software.intel.com/en-us/node/506096}
\end{itemize}

\section{Files}

The code is structured as a header only C++ library and also includes a
standalone tool. If you want to use it as a library, you only need to
include appropriate header files in your source. You can use the standalone
tool as an example for this purpose. Also, if you create a new source file
under tool directory, it will automatically be compiled as a separate tool.

Brief descriptions of subdirectories are as follows:

\begin{verbatim}
.
|-- bin/                         -- tool binaries (after building)
|   `-- cppr*
|-- build/                       -- build files (after building)
|   |-- dep/                     -- dependency files (after building)
|   |   |-- color_queue_test.d
|   |   |-- cppr.d
|   |   |-- multi_queue_test.d
|   |   `-- push_relabel_test.d
|   `-- test/                    -- test binaries (after building)
|       |-- color_queue_test*
|       |-- multi_queue_test*
|       `-- push_relabel_test*
|-- doc                          -- documentation
|   |-- manual.pdf
|   `-- manual.tex
|-- include/                     -- header files
|   |-- color_queue.hpp
|   |-- common.hpp
|   |-- dimacs.hpp
|   |-- greedy_coloring.hpp
|   |-- list_graph.hpp
|   |-- multi_queue.hpp
|   |-- push_relabel.hpp
|   `-- star_graph.hpp
|-- Makefile
|-- README
|-- test/                        -- source files for tests
|   |-- color_queue_test.cpp
|   |-- multi_queue_test.cpp
|   |-- push_relabel_test.cpp
|   `-- sample/                  -- sample DIMACS input/solution
|       |-- inp.max
|       `-- sol.max
`-- tool/                        -- source files for tools
    `-- cppr.cpp

9 directories, 26 files
\end{verbatim}

\end{document}
