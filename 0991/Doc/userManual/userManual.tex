\documentclass[11pt]{article} 

\usepackage[titlenumbered,ruled]{algorithm2e} % Algorithms
\usepackage[margin=1.2in]{geometry}% Changing the Margins 
\usepackage{natbib} % Bibliography
\usepackage{graphicx} % Figures
\usepackage{booktabs}% For Cross-Tab Titles
\usepackage{url} % Links 

% Define mathematical functions 
\usepackage{amsmath,amsthm,amssymb} % Mathematics 
\usepackage{bbold} % Indicator Functions 
\newcommand\floor[1]{\lfloor#1\rfloor}
\theoremstyle{definition}
\newtheorem{definition}{Definition}[section]
\newtheorem{theorem}{Theorem}[section]
\DeclareMathOperator*{\argmin}{arg\,min}
\DeclareMathOperator*{\argmax}{arg\,max}
\usepackage{listings}

\begin{document}

\title{\large{Algorithm xxx: The 2D Tree Sliding Window Discrete Fourier Transform}}
\author{Lee F. Richardson \and William F. Eddy}
\date{\today}
\maketitle

\section{Introduction}
This package implements of the 2D Tree Radix-2 Sliding Window Discrete Fourier Transform (SWDFT). This short manual describes installation, compiling the executable programs, callable routines, and shows how to incorporate the package into a larger program. 

\section{Installation}
Download the zip file, unzip it, then navigate to the directory where the files are extracted. There are three executable programs to compile: one ({\tt driver2d}) is located in the {\it src/} directory, and the other two ({\tt timing} and {\tt stability}) are located in the {\it tests/} directory. Compile and run the {\tt driver2d} executable with:

\begin{lstlisting}
	cd src
	make 
	./driver2d
\end{lstlisting}

The expected output is contained in the file {\it src/driver2d-expected-output.txt}. Similarly, compile and run the {\tt timing} and {\tt stability} executables with:

\begin{lstlisting}
	cd tests
	make 
	./timing
	./stability
\end{lstlisting}

The expected outputs are in {\it tests/timing-expected-output.txt} and {\it tests/stability-expected-output.txt}, respectively. 

\section{Routines and Executables}
This package contains three executable programs, which call our C functions implementing 2D SWDFT algorithms. This section gives brief descriptions of both the C functions and the executable programs that call them. 

\subsection{C Functions}
Our package includes one primary C function: {\tt tswdft2d}. This function implements the 2D Radix-2 Tree Sliding Window Discrete Fourier Transform, described in the corresponding manuscript. Detailed documentation for {\tt tswdft2d} is provided in the {\it src/tswdft2d.c} file. All the macros and constants used in this function are defined in {\it src/tswdft2d.h}. The inputs for the {\tt tswdft2d} function are:

\begin{itemize}
	\item {\bf x}. double complex *. Row-major 2D array with dimensions $N_0 \times N_1$
	\item {\bf n0}. int. Window size in row direction (must be a power of two)
	\item {\bf n1}. int. Window size in column direction (must be a power of two)
	\item {\bf N0}. int. Number of rows in {\bf x}
	\item {\bf N1}. int. Number of columns in {\bf x}.
\end{itemize}

The output is:

\begin{itemize}
	\item {\bf a}. double complex *. Row-major 4D array with dimensions $(N_0 - n_0 + 1) \times (N_1 - n_1 + 1) \times n_0 \times n_1$. The first two dimensions correspond to window position, and the final two dimensions correspond to frequency. 
\end{itemize}

We also include two other functions, {\tt swdft2d} and {\tt swfft2d}, for testing purposes. These functions the same input and output as {\tt tswdft2d}. 

\subsection{Executables}
We provide the following three executables:

\begin{itemize}
	\item {\tt driver2d}. Randomly generate a 2D complex-valued array, run the {\tt tswdft2d} function on the 2D array, print the output. 
	\item {\tt timing}. Generate a $100 \times 100$ array, run three different algorithms on the array with varying window sizes, print out how long each algorithm takes. 
	\item {\tt stability}. Verifies that the {\tt tswdft2d} and {\tt swfft2d} functions give identical results. 
\end{itemize}

\subsection{Larger Programs}
The {\tt tswdft2d} function can be used in larger programs. First, include the following header files:

\begin{lstlisting}
	#include "tswdft2d.h" 
	#include "complex.h"
\end{lstlisting}

\noindent Then allocate a double complex pointer, run the program, and free the memory:

\begin{lstlisting}
 	double complex *a;
 	a = tswdft2d(*x, m0, m1, N0, N1);	
 	free(a);
\end{lstlisting}

where {\bf x} is a 2D complex array, and the rest of the inputs are integers. See the driver program {\it src/driver2d.c} for an example. This example also shows how to access/print the elements of the array using some pre-defined macros. 

\end{document}
