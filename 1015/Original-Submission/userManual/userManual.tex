% workaround (https://tex.stackexchange.com/questions/239444/sideways-table-not-centered-in-the-page)
\RequirePackage[counterclockwise]{rotating}
\documentclass[format=acmsmall,review=false, screen=true]{acmart}

\usepackage{booktabs} % For formal tables

\usepackage{pgfplots}
\pgfplotsset{compat=1.14}
\usepgfplotslibrary{statistics}
\usepgfplotslibrary{colorbrewer}
\usepackage{xcolor}

\usetikzlibrary{external}
\tikzexternalize[prefix=figures/]

\usepackage{multirow}
\usepackage{units}
\usepackage[normalem]{ulem}

% TOG prefers author-name bib system with square brackets
\citestyle{acmauthoryear}
\setcitestyle{square}


\usepackage[ruled,linesnumbered]{algorithm2e} % For algorithms
\renewcommand{\algorithmcfname}{ALGORITHM}
\SetAlFnt{\small}
\SetAlCapFnt{\small}	
\SetAlCapNameFnt{\small}
\SetAlCapHSkip{0pt}
\IncMargin{-\parindent}

\usepackage{subcaption}


% Metadata Information
\acmJournal{TOMS}
%\acmVolume{9}
%\acmNumber{4}
%\acmArticle{39}
%\acmYear{2010}
%\acmMonth{3}
%\copyrightyear{2009}

%\acmArticleSeq{9}

% Copyright
%\setcopyright{acmcopyright}
\setcopyright{acmlicensed}
%\setcopyright{rightsretained}
%\setcopyright{usgov}
%\setcopyright{usgovmixed}
%\setcopyright{cagov}
%\setcopyright{cagovmixed}

% DOI
\acmPrice{15.00}
\acmDOI{10.1145/3442348}

% Paper history
\received{June 2018}
\received[revised]{July 2019}
\received[revised]{June 2020}
\received[revised]{November 2020}
\received[accepted]{December 2020}

%%\usepackage{epsfig}

\begin{comment}
\newtheorem{theorem}{Theorem}[section]
\newtheorem{proposition}[theorem]{Proposition}
\newtheorem{lemma}[theorem]{Lemma}
\newtheorem{corollary}[theorem]{Corollary}
\newtheorem{definition}[theorem]{Definition}
\end{comment}

\usepackage{enumerate}

%\usepackage{psfrag}

\usepackage{subfigure}

%\usepackage{graphicx}
%\usepackage{caption}
%\usepackage{subcaption}

\usepackage{epsfig}

\usepackage{color}% Include colors for document elements (required for psfrag)
\usepackage{psfrag}

\newcommand{\psdir}{./sections}
\newcommand{\ldir}{./sections}
\newcommand{\fig}{./sections/fig}
\newcommand{\comm}[1]{}


\def\IL{{\it left}}
\def\IM{{\it middle}}
\def\IR{{\it right}}
\def\IB{{\it bottom}}
\def\IT{{\it top}}


\newcommand{\algref}[1]{Algorithm~\ref{#1}}
\newcommand{\defref}[1]{Definition~\ref{#1}}
\newcommand{\secref}[1]{Section~\ref{#1}}
\newcommand{\figref}[1]{Figure~\ref{#1}}
\newcommand{\tabref}[1]{Table~\ref{#1}}
\newcommand{\thmref}[1]{Theorem~\ref{#1}}
\newcommand{\propref}[1]{Proposition~\ref{#1}}
\newcommand{\efct}[1]{$\langle$ #1 $\rangle$}

\newcommand{\chapref}[1]{Chapter~\ref{#1}}



\newcommand{\atlasMgrid}{Atlas MultiGrid}
\newcommand{\cg}{$\cal{G}$}
\newcommand{\cgraph}{active constraint graph}
\newcommand{\sgraph}{stratification graph}
\newcommand{\cs}{$\cal{S}$}
\newcommand{\cd}{colldet}
\newcommand{\pconfig}{parametrized configuration}
\newcommand{\orient}{pose}
\newcommand{\point}{point}
\newcommand{\helix}{molecular composite}
\newcommand{\atom}{atom marker}
\newcommand{\dumbell}{dumbell}
\newcommand{\EASAL}{\texttt{EASAL}}
\newcommand{\Cr}{Cartesian realization}

\newcommand{\tns}{T} %total number of samples
\newcommand{\C}{C} %the collection of interesting atom pairs
\newcommand{\m}{m} %size of S
\newcommand{\stp}{s}


\newcommand{\tol}{$\tau$}
\newcommand{\AC}{A}
\newcommand{\mU}{M}
\newcommand{\aU}{atom marker}
\newcommand{\acg}{G}

\newcommand{\cover}{boundaries}


\newcommand{\aview}{atlas view}
\newcommand{\pview}{parametrized chart view}
\newcommand{\rview}{realization view} %configuration space view
\newcommand{\param}{parameter}
\newcommand{\idialog}{intervention dialog}
\newcommand{\ndialog}{node selection dialog}
\newcommand{\atlas}{atlas}
\newcommand{\threeRealizable}{$3$-realizable}
\newcommand{\chart}{chart}

% for lists
\renewcommand{\labelitemi}{$\bullet$}



\def\jorg#1{\textcolor{blue}{#1}}
% \def\aysegul#1{\textcolor{red}{#1}}
\def\mred#1{\textcolor{red}{#1}}


% for review
\definecolor{seagreen}{RGB}{48,178,139}
\definecolor{yelloworange}{RGB}{249,146,0}
\definecolor{plum}{RGB}{127,0,126}

\newcommand{\aysegul}[1]{{\color{blue}#1}}
\newcommand{\meera}[1]{{\color{magenta}#1}}
\newcommand{\troy}[1]{{\color{seagreen}#1}}
\newcommand{\rahul}[1]{{\color{cyan}#1}}
\newcommand{\joel}[1]{{\color{yelloworange}#1}}
\newcommand{\ruijin}[1]{{\color{plum}#1}}


% Document starts
\begin{document}
% Title portion. Note the short title for running heads
\title[Algorithm XXX: A Fast Scalable Solver for the Dense Linear (Sum) Assignment Problem]{Algorithm XXX: A Fast Scalable Solver for the Dense Linear (Sum) Assignment Problem}

\author{Stefan Guthe}
\email{stefan.guthe@tu-darmstadt.de}
\orcid{0000-0001-5539-9096}
\affiliation{\institution{TU Darmstadt}\department{Graphical Interactive Systems Group}\streetaddress{Fraunhofer Str. 5}\postcode{64283}\city{Darmstadt}\country{Germany}}
\affiliation{\institution{Fraunhofer IGD}\country{Germany}}

\author{Daniel Thuerck}
\email{daniel.thuerck@neclab.eu}
\affiliation{\institution{NEC Laboratories}\country{Germany}}
%\affiliation{\institution{TU Darmstadt}\country{Germany}}
\authornote{Work done while at TU Darmstadt.}

\acmSubmissionID{TOMS-2018-0046}

\renewcommand\shortauthors{Guthe, S., Thuerck, D.}

\begin{abstract}
This document contains the instructions for building and testing the software bundle as well as instructions on how to use the software in your own project.
\end{abstract}

%
% The code below should be generated by the tool at
% http://dl.acm.org/ccs.cfm
% Please copy and paste the code instead of the example below.
%
 \begin{CCSXML}
<ccs2012>
<concept>
<concept_id>10002950.10003624.10003625.10003628</concept_id>
<concept_desc>Mathematics of computing~Combinatorial algorithms</concept_desc>
<concept_significance>500</concept_significance>
</concept>
<concept>
<concept_id>10002950.10003624.10003625.10003630</concept_id>
<concept_desc>Mathematics of computing~Combinatorial optimization</concept_desc>
<concept_significance>500</concept_significance>
</concept>
</ccs2012>
\end{CCSXML}

\ccsdesc[500]{Mathematics of computing~Combinatorial algorithms}
\ccsdesc[500]{Mathematics of computing~Combinatorial optimization}
%
% End generated code
%

\keywords{Successive Shortest Path Algorithm, Parallel Processing, Epsilon Scaling}

\maketitle

\section{Linux}
\label{sec:linux}
This section contains the instructions for building and running under Linux. For Windows, refer to Section \ref{sec:windows}.

\subsection{Requirements}
The following software is required to build the source code that comes with this publication:
\begin{itemize}
\item CMake 3.5
\item GCC
\item CUDA 10 (or later, optional for the GPU build)
\end{itemize}

\subsection{Build Instructions}
To build the test package that was used to generate all the performance measurements in the main publication, run the following commands inside the \texttt{lap\_solver} directory:
\begin{itemize}
\item \small{\texttt{mkdir build}}
\item \small{\texttt{cd build}}
\item \small{\texttt{cmake ../gcc}}
\item \small{\texttt{make}}
\end{itemize}
Since the makefile is set up to compile the same code with multiple sets of defines, it is not possible to use the parallel build, e.g. \texttt{make -j4}, as this will cause the build to fail.

\section{Windows}
\label{sec:windows}

\subsection{Requirements}
The following software is required to build the source code that comes with this publication:
\begin{itemize}
\item Visual Studio (at least Community Edition) 2014, 2017 or 2019
\item CUDA 10.2 (other versions require build files to be patched manually)
\end{itemize}

\subsection{Build Instructions}
To build the test package that was used to generate all the performance measurements in the main publication, the following steps are required:
\begin{itemize}
\item load solution from \texttt{vc14}, \texttt{vc17} or \texttt{vc19}
\item press \texttt{Ctrl+Shift+B}
\end{itemize}

\section{Running Tests}
The following commands will re-produce the data found in the figures and tables of the paper (Figure 1, 2 \& 3 require special debug builds) using Linux. For Windows, the relative path is different (\texttt{../../../../../images/}, when executing in the build directory).
\begin{itemize}
\raggedright
\item Table 1 \& Figure 4:\\
\small{\texttt{./test\_cpu\_evaluated -table\_min 1000 -table\_max 32000 -sanity -random -geometric -geometric\_disjoint -random\_low\_rank -rank\_min 1 -rank\_max 8 -double -single -runs 5 -omp}\\
\texttt{./test\_cpu\_evaluated -table\_min 1000 -table\_max 128000 -sanity -random -geometric -geometric\_disjoint -random\_low\_rank -rank\_min 1 -rank\_max 8 -double -epsilon -runs 5 -omp}}
\item Figure 5:\\
\small{\texttt{./test\_cpu -table\_min 1000 -table\_max 32000 -sanity -geometric\_disjoint -random\_low\_rank -rank\_min 1 -rank\_max 8 -double -single -runs 5}\\
\texttt{./test\_cpu -table\_min 1000 -table\_max 128000 -random -geometric -double -single -runs 5}\\
\texttt{./test\_cpu -table\_min 1000 -table\_max 128000 -sanity -random -geometric\\ -geometric\_disjoint  -random\_low\_rank -rank\_min 1 -rank\_max 8\\ -double -epsilon -runs 5}}
\item Figure 6:\\
\small{\texttt{./test\_cpu -memory 257698037760 -cached\_min 1000 -cached\_max 1024000 -geometric\_cached -geometric\_disjoint\_cached -sanity\_cached -random\_low\_rank\_cached -rank\_min 1 -rank\_max 8 -double -epsilon -runs 5 -omp}}
\item Figure 7:\\
\small{\texttt{./test\_gpu -memory 3221225472 -table\_min 1000 -table\_max 128000 -random -double -epsilon -runs 5}\\
\texttt{./test\_gpu -memory 3221225472 -cached\_min 1000 -cached\_max 1024000 -geometric\_cached -geometric\_disjoint\_cached -sanity\_cached -random\_low\_rank\_cached -rank\_min 1 -rank\_max 8 -double -epsilon -runs 5}}
\item Figure 8:\\
\small{\texttt{./test\_gpu -memory 15032385536 -table\_min 1000 -table\_max 128000 -random -double -epsilon -runs 5}\\
\texttt{./test\_gpu -memory 15032385536 -cached\_min 1000 -cached\_max 1024000 -geometric\_cached -geometric\_disjoint\_cached -sanity\_cached -random\_low\_rank\_cached -rank\_min 1 -rank\_max 8 -double -epsilon -runs 5}}
\item Figure 9:\\
\small{\texttt{./test\_cpu -memory 128849018880 -img ../../images/img1s.ppm -img ../../images/img2s.ppm -img ../../images/img3s.ppm -img ../../images/img4s.ppm -img ../../images/img5s.ppm -img ../../images/img6s.ppm -img ../../images/img7s.ppm -img ../../images/img8s.ppm -img ../../images/img9s.ppm -img ../../images/img10s.ppm -float -single}\\
\texttt{./test\_cpu -memory 128849018880 -img ../../images/img1m.ppm -img ../../images/img2m.ppm -img ../../images/img3m.ppm -img ../../images/img4m.ppm -img ../../images/img5m.ppm -img ../../images/img6m.ppm -img ../../images/img7m.ppm -img ../../images/img8m.ppm -img ../../images/img9m.ppm -img ../../images/img10m.ppm -float -single}\\
\texttt{./test\_cpu -memory 128849018880 -img ../../images/img1s.ppm -img ../../images/img2s.ppm -img ../../images/img3s.ppm -img ../../images/img4s.ppm -img ../../images/img5s.ppm -img ../../images/img6s.ppm -img ../../images/img7s.ppm -img ../../images/img8s.ppm -img ../../images/img9s.ppm -img ../../images/img10s.ppm -float -epsilon -omp}\\
\texttt{./test\_cpu -memory 128849018880 -img ../../images/img1m.ppm -img ../../images/img2m.ppm -img ../../images/img3m.ppm -img ../../images/img4m.ppm -img ../../images/img5m.ppm -img ../../images/img6m.ppm -img ../../images/img7m.ppm -img ../../images/img8m.ppm -img ../../images/img9m.ppm -img ../../images/img10m.ppm -float -epsilon -omp}\\
\texttt{./test\_cpu -memory 257698037760 -img ../../images/img1l.ppm -img ../../images/img2l.ppm -img ../../images/img3l.ppm -img ../../images/img4l.ppm -img ../../images/img5l.ppm -img ../../images/img6l.ppm -img ../../images/img7l.ppm -img ../../images/img8l.ppm -img ../../images/img9l.ppm -img ../../images/img10l.ppm -float -epsilon -omp}\\
\texttt{/test\_gpu -memory 3221225472 -img ../../images/img1s.ppm -img ../../images/img2s.ppm -img ../../images/img3s.ppm -img ../../images/img4s.ppm -img ../../images/img5s.ppm -img ../../images/img6s.ppm -img ../../images/img7s.ppm -img ../../images/img8s.ppm -img ../../images/img9s.ppm -img ../../images/img10s.ppm -float -epsilon}\\
\texttt{/test\_gpu -memory 3221225472 -img ../../images/img1m.ppm -img ../../images/img2m.ppm -img ../../images/img3m.ppm -img ../../images/img4m.ppm -img ../../images/img5m.ppm -img ../../images/img6m.ppm -img ../../images/img7m.ppm -img ../../images/img8m.ppm -img ../../images/img9m.ppm -img ../../images/img10m.ppm -float -epsilon}\\
\texttt{/test\_gpu -memory 3221225472 -img ../../images/img1l.ppm -img ../../images/img2l.ppm -img ../../images/img3l.ppm -img ../../images/img4l.ppm -img ../../images/img5l.ppm -img ../../images/img6l.ppm -img ../../images/img7l.ppm -img ../../images/img8l.ppm -img ../../images/img9l.ppm -img ../../images/img10l.ppm -float -epsilon}\\
\texttt{/test\_gpu -memory 15032385536 -img ../../images/img1s.ppm -img ../../images/img2s.ppm -img ../../images/img3s.ppm -img ../../images/img4s.ppm -img ../../images/img5s.ppm -img ../../images/img6s.ppm -img ../../images/img7s.ppm -img ../../images/img8s.ppm -img ../../images/img9s.ppm -img ../../images/img10s.ppm -float -epsilon}\\
\texttt{/test\_gpu -memory 15032385536 -img ../../images/img1m.ppm -img ../../images/img2m.ppm -img ../../images/img3m.ppm -img ../../images/img4m.ppm -img ../../images/img5m.ppm -img ../../images/img6m.ppm -img ../../images/img7m.ppm -img ../../images/img8m.ppm -img ../../images/img9m.ppm -img ../../images/img10m.ppm -float -epsilon}\\
\texttt{/test\_gpu -memory 15032385536 -img ../../images/img1l.ppm -img ../../images/img2l.ppm -img ../../images/img3l.ppm -img ../../images/img4l.ppm -img ../../images/img5l.ppm -img ../../images/img6l.ppm -img ../../images/img7l.ppm -img ../../images/img8l.ppm -img ../../images/img9l.ppm -img ../../images/img10l.ppm -float -epsilon}}
\item Table 2: Requires Auction solver which is not included in this package.
\item Table 3: Data found in other tables except for limiting the number of threads to 8
\end{itemize}

\section{Own Project}
In order to use the software package in your own project, you need to include the \texttt{lap.h} file after setting the desired defines in your project. To get the same behaviour as the \texttt{test\_cpu}  program, use the following:
\begin{verbatim}
// enable OpenMP support
#ifdef _OPENMP
#  define LAP_OPENMP
#endif
// quiet mode
#define LAP_QUIET
// increase numerical stability for non-integer costs
#define LAP_MINIMIZE_V
\end{verbatim}
In case you would like to use GPU support, use the following as a starting point:
\begin{verbatim}
// enable CUDA support
#define LAP_CUDA
// OpenMP required for multiple devices
#define LAP_CUDA_OPENMP
// quiet mode
#define LAP_QUIET
// increase numerical stability for non-integer costs
#define LAP_MINIMIZE_V
\end{verbatim}

\subsection{High-Level Interface}
The high-level interface can be found in the \texttt{test\_cpu.cpp} and \texttt{test\_gpu.cu} files. The CPU functions are:
\begin{verbatim}
template <class SC, class TC, class CF, class TP>
void solveAdaptiveOMP(TP &start_time, int N1, int N2, CF &get_cost, int *rowsol, 
                      int entries, bool epsilon)
\end{verbatim}
and
\begin{verbatim}
template <class SC, class TC, class CF, class TP>
void solveAdaptive(TP &start_time, int N1, int N2, CF &get_cost, int *rowsol, 
                   int entries, bool epsilon)
\end{verbatim}
Where:
\begin{itemize}\raggedright
\item \texttt{SC} is the type used within the solver.
\item \texttt{TC} is the type for storing the cost values.
\item \texttt{TP \&start\_time} is starting time of the test returned by  \texttt{std::chrono::high\_resolution\_clock::now();}
\item \texttt{int N1} and \texttt{int N2} specify the size of the problem
\item \texttt{CF \&get\_cost} is the cost function lambda that takes two parameters \texttt{int x, int y} ($0 \leq x < N1$ and $0 \leq y < N2$) and returns the cost of type \texttt{TC}
\item \texttt{int *rowsol} points to the row solution being returned
\item \texttt{bool epsilon} enables out $\epsilon$-Pricing and should always be enabled.
\end{itemize}

The corresponding GPU functions are:
\begin{verbatim}
template <class SC, class TC, class CF, class STATE, class TP>
void solveCUDA(TP& start_time, int N1, int N2, CF& get_cost_gpu, STATE* state, 
               lap::cuda::Worksharing& ws, long long max_memory, int* rowsol, 
               bool epsilon, bool silent)
\end{verbatim}
and
\begin{verbatim}
template <class SC, class TC, class CF, class TP>
void solveTableCUDA(TP& start_time, int N1, int N2, CF& get_cost_cpu, 
                    lap::cuda::Worksharing& ws, long long max_memory, 
                    int* rowsol, bool epsilon, bool sequential, bool pinned, 
                    bool silent)
\end{verbatim}
The difference between these functions is that the first one uses a device lambda \texttt{CF \&get\_cost\_gpu} with parameters \texttt{(int x, int y, STATE \&state)} while the second function uses a regular cpu lambda as in the cpu code above.
Additional parameters are:
\begin{itemize}
\item A work sharing struct \texttt{ws}, constructed using \texttt{lap::cuda::Worksharing ws(int N1, int multiple, std::vector<int> \&devices, int max\_devices, bool silent);} with \texttt{multiple} usually set to 256 for better memory and thread alignment
\item \texttt{STATE *state} is a used defined per GPU state passed to the \texttt{get\_cost\_gpu} function, including pointer to memory locations used inside the device lambda
\item \texttt{long long max\_memory} defines how much memory should be allocated on a single GPU for holding the cost values (either cache or table)
\item \texttt{bool sequential} specifies if the \texttt{get\_cost\_cpu} lambda can only be called from a single thread
\item \texttt{bool pinned} if true, the entire CPU cost table will be stored in pinned memory
\end{itemize} 

\subsection{Low-Level Interface}
The low-level interface can be found in the \texttt{lap.h} include files. The single threaded CPU code consists of the following functions for solving the linear assignment and calculating the final costs:
\begin{verbatim}
namespace lap
{
  template <class SC, class CF, class I> void solve(
    int dim, CF &costfunc, I &iterator, int *rowsol, bool use_epsilon);
  template <class SC, class CF, class I> void solve(
    int dim, int dim2, CF &costfunc, I &iterator, int *rowsol, 
    bool use_epsilon);
  template <class SC, class CF> SC cost(
    int dim, CF &costfunc, int *rowsol);
  template <class SC, class CF> SC cost(
    int dim, int dim2, CF &costfunc, int *rowsol);
}
\end{verbatim}
The multi threaded CPU code uses the following interface:
\begin{verbatim}
namespace lap
{
  namespace omp
  {
    template <class SC, class CF, class I> void solve(
      int dim, CF &costfunc, I &iterator, int *rowsol, bool use_epsilon);
    template <class SC, class CF, class I> void solve(
      int dim, int dim2, CF &costfunc, I &iterator, int *rowsol, 
      bool use_epsilon);
    template <class SC, class CF> SC cost(
      int dim, CF &costfunc, int *rowsol);
    template <class SC, class CF> SC cost(
      int dim, int dim2, CF &costfunc, int *rowsol);
  }
}
\end{verbatim}
The GPU version of the interface is as follows:
\begin{verbatim}
namespace lap
{
  namespace cuda
  {
    template <class SC, class TC, class CF, class I> void solve(
      int dim, CF &costfunc, I &iterator, int *rowsol, bool use_epsilon);
    template <class SC, class TC, class CF, class I> void solve(
      int dim, int dim2, CF &costfunc, I &iterator, int *rowsol, 
      bool use_epsilon);
    template <class SC, class TC, class CF> SC cost(
      int dim, CF &costfunc, int *rowsol, cudaStream_t stream);
    template <class SC, class TC, class CF> SC cost(
      int dim, int dim2, CF &costfunc, int *rowsol, cudaStream_t stream);
  }
}
\end{verbatim}

Please refer to the same for additional helper classes that can be used for the low-level interface.

%% !TeX encoding = UTF-8 Unicode
% !TeX root = manual.tex
% !TeX spellcheck = en_GB


\chapter{Introduction}
These packages provide functions for the computation of the joint spectral radius of a finite set of matrices using the modified invariant polytope algorithm
as well as functions for the work with multiple, multivariate, stationary subdivision schemes.
Some functions work also for symbolic matrices/subdivision schemes.

\section{System requirements}

In order to use the packages, you need at least Matlab R2016b.

The \texttt{sequence}- and the \texttt{subdivision}-package depend on the Matlab \texttt{Symbolic Math Toolbox} and the  
\texttt{Signal Processing Toolbox} but also run without the latter.
The \texttt{sequence}- and the \texttt{tjsr}-package depend on the Matlab \texttt{Parallel Computing Toolbox} but should run without it.
If these toolboxes are not installed, they can be installed as described in the Matlab documentation.

All other external, necessary toolboxes and functions are included in this package. These are
the \emph{TTEST}s~test-suite~v0.3, the \emph{SeDuMi} solver~v1.32 and the \emph{JSR-Toolbox}~v1.2b.

\section{Installation}
In order to install the \texttt{t}-packages do the following steps:

\begin{enumerate}

\item This package works with the \emph{Gurobi} solver and the Matlab \verb|linprog| solver, despite much slower with the latter since some optimizations of the modified invariant polytope algorithm are not possible to be realized with \verb|linprog|.

\item If you have not installed the \emph{Gurobi} solver yet and you want to use it, 
you should install it.
As of March~2019, the installation works as follows for academic users:
\begin{itemize}\label{install_gurobi}
    \item Download the Gurobi solver from 
    \href{http://www.gurobi.com/}{{\itshape http://www.gurobi.com/}} and
    extract the archive in a folder of your choice. 
    Do not choose a path which has blanks, i.e.\ instead of e.g.\ \texttt{/Gurobi 8/} use \texttt{/Gurobi\_8/}.
    
    \item Obtain a free academic licence. For that, open a shell, and execute in the
    \texttt{bin} subdirectory where you extracted the \texttt{Gurobi} files 
    the command which you find on the Gurobi page under the link 
    \emph{Free Academic License}. 
    The command looks like this 
    "\texttt{grbgetkey 1eb9501e-4e90-13e2-a19f-02e454bb2c50}".
    
    If this command fails, add "\texttt{./}" in front of the command, 
    i.e.\ instead of the above command, type 
    "\texttt{./grbgetkey 1eb9501e-4c91-11e9-a19f-02e454ff9c50}"
    If this command still fails, make sure you are connected to your universities network.
    
    If you are asked questions during the exectution of the command, always use the proposed default values, i.e.\ just press \emph{Enter}.
    
    \item At last, in Matlab run \verb|gurobi_setup| and \verb|savepath| afterwards.
        
\end{itemize}

\item \label{install_copy}
Copy the content of this archive 
(i.e.\ the folders \verb|TTEST| and \verb|ttoolboxes|)
into the directory of your choice.
In Matlab run the file \verb|setupt|, in the folder \texttt{ttoolboxes}.
This file adds all packages to the Matlab path and runs a self-test of all included functions.
If the test fails, you may run 
\texttt{runtests('testcell')},
\texttt{runtests('testdouble')},
\texttt{runtests('testm')},
\texttt{runtests('testsequence')},
\texttt{\justify runtests('testsubdivison')},
\texttt{runtests('testtjsr')},
\texttt{runtests('testtmisc')} or
\texttt{runtests('testTTEST')}
to test the individual packages.
\end{enumerate}

The Toolbox has been tested on several architectures (Windows, Linux, Mac) 
and Matlab versions (R2016b, R2017a, R2017b, R2018a, R2018b, R2019a).
If you encounter any problem, please contact the author at 
\begin{center}
\href{mailto:tommsch@gmx.at}{tommsch@gmx.at}.
\end{center}


%%%%%%%%%%%%%%%%%%%
% EOF
%%%%%%%%%%%%%%%%%%%
%\input{related.tex}
%\input{algorithm.tex}
%\input{large.tex}
%\input{results.tex}
%\input{conclusion.tex}
% don't really want to have these...
%%% Waki 2007/05/03
%% Modify table on functions which are used in SparsePOP.

\section*{Appendix: MATLAB functions used in the function \\ 
sparsePOP.m} 
\label{functions}

In Table \ref{prog1} and \ref{prog2}, we give a brief description of utility programs in sparsePOP.
%\begin{center}
\hspace*{-2.0cm}
\begin{table}
\begin{tabular}{|l|p{10cm}|} \hline
addBoundToPOP.m & Add bounds to $y_{\salpha}$ $(\balpha \in \widetilde{\FC})$. \\
boundToIneqPolySys.m & Include lower and upper bounds, {\sf lbd} and {\sf ubd}, in {\sf ineqPolySys}. \\
checkPOP.m &  Verify input data {\sf objPoly}, {\sf ineqPolySys}, {\sf lbd} and {\sf ubd} describing a POP in the sparsePOP format. \\
% chodalStruct.m  & Generate a sequence of maximal cliques \\
%                                 & which satisfies the chordal graph property. \\
%                                 & This program is called in genSpRealPoly.m. \\
conver2.m& Convert a POP into one with positive variables of \\
& lower bound $0$ and free variables\\
deleteVar.m & Substitute $c$ into $x_i$ if a POP has the constraint $x_i = c$ for some constant value $c$\\
evalPolynomials.m  & Evaluate a polynomial or a polynomial system. \\
% & at $\x \in \Real^n$. \\ 
genApproxSolution.m  & Compute an approximate solution of POP using the optimal value obtained from the SDP relaxation. \\
genBasisIndices.m  & Form {\sf basisIndices} used for {\sf basisSupports} from {\sf clique}. \\ 
genBasisSupports.m  & Generate {\sf basisSupports} based on {\sf basisIndices} and {\sf relaxOrder}. \\ 
% genBallInEqSys.m &  Generate ball inequality constraints. \\
% genBMI.m &  Generate a minimization problem over a bilinear \\
% 		&   matrix inequality. \\ 
genClique.m  & Generate {\sf clique} based the correlative sparsity pattern of objective and constraint polynomials of a POP.\\
% genSimplexSupport.m  & Generating a set of supports. \\ 
% genSpRealPoly.m &  Generating a real valued sparse polynomials \\
%                                       & using a chordal graph structure. \\
getSDPinfo.m & Get information of an SDP obtained by the SDP relaxation\\
% infeasibility.m &  Check the infeasibility of polynomial inequality constraints. \\ 
% lexicoCompare.m &  Compare two vectors according to the lexico graphical order. \\
% lexicoSort.m &  Sorting lexicographically. \\ 
%listupAllSupports.m & List up all support vectors used in the PSDP. \\ 
make\_mexdata.m & Convert the SparsePOP format into input arguments of mexconv1.cpp and mexconv2.cpp\\
%mexconv1.cpp & \\
%mexconv2.cpp & \\
monomialSort.m & Sort monomials. \\ 
multiplyPolynomials.m & Product of two polynomials\\
perturbObjPoly.m & Perturb objective polynomial with a given small number. \\
% printSolution.m & Print detailed information on an optimal solution after \\ 
% & executing the sparsePOP.m.\\
% POPexamples.m &  Some examples of POPs. \\ 
plusPolynomials.m  & Addition of two polynomials. \\ 
PSDPtoLSDP.m & Generate an SDP relaxation problem in the SDPA  sparse format. \\ 
% readGMS.m & Read a POP in GMS format and convert it into a POP \\
% & in the sparsePOP format.\\\end{tabular}
\hline
\end{tabular}
\label{prog1}
\caption{The list of programs used in SparsePOP  \ (1 of 2)}
\end{table}

\hspace*{-2.0cm}
\begin{table}
\begin{tabular}{|l|p{10cm}|} 
\hline
reduceSupSets.m & Remove redundant monomials from PSDP (\ref{PolySDP0}).\\
relax1EqTo2Ineqs.m & Convert every equality constraint into two inequality constraints.\\
saveOriginalPOP.m & Save the original POP before being modified.\\
scalingPOP.m & Scale {\sf objPoly}, {\sf inEqPolySys}, {\sf lbd} and {\sf ubd} of a POP. \\ 
% such that        \\   &       $ \max\{ |lbd(j)|, |ubd(j)| \}  = 1$
% and that the maximum absolute value of\\ & coefficients of 
% all monomials of each
% polynomial  are one. \\
SDPAtoSeDuMi.m &  Convert SDPA sparse format data into SeDuMi format data. \\ 
separateSpecialMonomial.m & Find the complementarity constraints in a POP\\
setParameter.m & Set default values for parameters not specified by the user.\\
simplifyPolynomial.m  & Simplify a polynomial or a polynomial system. \\ 
SOCPtoSDP.m & Convert a second-order cone constraint into an SDP constraint.\\
%smallExamples.m &  Small size examples of POPs over cones for checking. \\
solveBySeDuMi.m & Solve an SDP obtained from a POP by SeDuMi.\\
%solveGloballib.m & A driver to solve a set of problems in Globallib. \\
% sparsePOP.m & Main part of the program. \\
%sparsePOPmain.m & \\
%sparsePOPmainMex.m & \\
substituteEq.m & Substitute complementarity constraints into a POP.\\
%testBMI.m &  Solving BMI examples by SDP relaxations. \\
%                     & This program calls SDPAtoSeDuMi.m. \\
%testPSDPtoLSDP.m &  Solves POPs by SDP relaxations. \\
%                     & This program calls SDPAtoSeDuMi.m. \\
writeBasisIndices.m &  Print {\sf basisIndices}  used for {\sf basisSupports}.\\
writeBasisSupports.m & Print {\sf basisSupports}.\\
writeClique.m & Print {\sf clique}.\\
writeParameters.m & Print {\sf param}. \\
writePolynomials.m &  Print a polynomial or a polynomial system in the sparsePOP format. \\ 
writePOP.m & Print a POP expressed in the sparsePOP format.\\
writeResults.m & Print an approximate optimal solution and errors.\\
writeSDPAformatData.m  & Print SDPA sparse format data on an LSDP. \\
writeSeDuMiInputData.m  & Print SeDuMi format data on an LSDP. \\ 
\hline
\end{tabular}
\label{prog2}
\caption{The list of programs used in SparsePOP  \ (2 of 2)}
\end{table}
%\end{center}
 


%\bibliographystyle{ACM-Reference-Format}
%\bibliography{refs}

\end{document}
