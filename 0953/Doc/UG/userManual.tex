\documentclass{article}

\usepackage{amsmath,amssymb,amsfonts}
\usepackage{graphicx}
\usepackage{geometry}
\usepackage{url}
\usepackage{hyperref}

\title{PDHSEQR User's Guide}
\author{Robert Granat\thanks{
Department of Computing Science and HPC2N,
Ume{\aa} University,
SE-901~87 Ume{\aa}, Sweden.
Email: \texttt{\{granat,bokg,myshao\}@cs.umu.se}\,.},
Bo K{\aa}gstr\"om$^*$,
Daniel Kressner\thanks{
MATHICSE ANCHP,
EPF Lausanne,
CH-1015 Lausanne, Switzerland.
Email: \texttt{daniel.kressner@epfl.ch}\,.},
and Meiyue Shao$^{*,\dagger}$}

\begin{document}
\maketitle

\section{Introduction}
PDHSEQR is a parallel ScaLAPACK-style library for solving nonsymmetric
standard eigenvalue problems.
The library is written in Fortran~90 and targets distributed memory HPC systems.
Using the small-bulge multishift QR
algorithm with aggressive early deflation,
it computes the real Schur decomposition $H=ZTZ^T$ of an upper Hessenberg
matrix~$H\in\mathbb{R}^{n\times n}$, such that~$Z$ is orthogonal and~$T$ is
quasi-upper triangular.
This document concerns the usage of PDHSEQR and is a supplement to the
article~\cite{GKKS2013}.
For the description of the algorithm and implementation, we refer
to~\cite{GKKS2013} and the references therein (especially,
\cite{BBM2002a,BBM2002b,GKK2009,GKK2010,KKS2012}).

\section{Installation}
In the following, an installation guide is provided.
It is assumed that the user is working in a Unix-like system.

\subsection{Prerequisites}
To build the library, the following software is required.
\begin{itemize}
\item A Fortran 90/95 compiler.
\item The MPI library, e.g., OpenMPI or MPICH.
\item An optimized BLAS library, e.g., ATLAS or OpenBLAS.
\item The LAPACK library.
\item The ScaLAPACK library (including BLACS and PBLAS).
\end{itemize}

\subsection{How to compile the library}
\begin{paragraph}
{Download location.}~\\
The software version published by ACM TOMS can be downloaded from CALGO~\cite{CALGO}.
The latest version of the source code (with bug fixes) as well as the
associated documents are available on the PDHSEQR homepage~\cite{home}.
\end{paragraph}

\begin{paragraph}
{Files in the tar-ball.}~\\
The following command unpacks the tar-ball and creates a directory
\texttt{pdhseqr/}, which is the root directory of the library.\\
\medskip
\fbox{\hbox to\textwidth{$\strut$\texttt{~tar xzfv pdhseqr.tar.gz}}}\\
\medskip
Inside the root directory, there are several files and directories:\\
\medskip
\fbox{\hbox to\textwidth{$\strut$\texttt{~EXAMPLES/~MAKE\_INC/~Makefile%
~make.inc~README~SRC/~TESTING/~TOOLS/~TUNING/~ug.pdf}}}\\
\medskip
Below is an overview of these items.
\begin{itemize}
\item \texttt{EXAMPLES/} This directory contains two simple drivers.
\item \texttt{MAKE\_INC/} This directory contains several templates of
\texttt{make.inc} for GNU, Intel, and PathScale compilers.
\item \texttt{Makefile} The Makefile for building the library.
This file does \emph{not} need to be modified.
\item \texttt{make.inc}
\emph{This is the only file which requires modifications when building the
library}.
It contains compiler settings and external libraries for the Makefile.
The user is required to modify this file according to the target computational
environment before compiling the library.
Several templates of this file are provided in the directory
\texttt{MAKE\_INC}.
\item \texttt{README}
A shorter version of this document containing a quick installation guide.
\item \texttt{SRC/}
This directory contains source code for all computational routines of the
library.
\item \texttt{TESTING/} This directory contains testing examples.
\item \texttt{TOOLS/} This directory contains several auxiliary routines
(e.g., random number/matrix generators, input/output routines).
\item \texttt{TUNING/} This directory contains auto-tuning scripts.
\item \texttt{ug.pdf}
The User's Guide of PDHSEQR (i.e., this document).
\end{itemize}
\end{paragraph}

\begin{paragraph}
{Build the library.}~\\
Once \texttt{make.inc} is properly modified according to the computational
environment, the library can be built by\\
\medskip
\fbox{\hbox to\textwidth{$\strut$\texttt{~make all}}}\\
\medskip
in the root directory of PDHSEQR.
This generates the library archive \texttt{libpdhseqr.a} in the root
directory, two examples in \texttt{EXAMPLES/}, and test programs in
\texttt{TESTING/}.
The script \texttt{quicktest.sh} in \texttt{TESTING}, which performs twelve
quick tests, needs to be run after the compilation.
Hopefully the following result will be displayed on the screen:\\
\fbox{\vtop{\hbox to\textwidth{$\strut$%
\texttt{~ \%           7 out of           7 tests passed!}}
\hbox to\textwidth{$\strut$%
\texttt{~ \%           5 out of           5 tests passed!}}}}\\
This means that the Schur decomposition has been successfully computed for
seven random matrices and five benchmark matrices, indicating that the
compilation has been successful.
We recommend that the script \texttt{runquick.sh} in \texttt{TESTING} is also
run once to make sure that the parallel code works properly.
You may need to modify the MPI execution command in this script according to
your system (e.g., \texttt{mpirun}, \texttt{mpiexec}, etc.).
If everything works out, 20 lines of information summarizing the 120 tests
will be displayed and written to the file \texttt{summary.txt}.
We also provide \texttt{runall.sh} with many large test cases in the same
directory.
(Running this set of tests may take \emph{very long}!)
\end{paragraph}


\section{Using the package}
\subsection{ScaLAPACK data layout convention}
\begin{figure}\centering
\includegraphics[scale=0.75]{layout.0}
\caption{The 2D block-cyclic data layout across a $2\times3$ processor grid.
For example, processor $(0,0)$ owns all highlighted blocks.
Picture from~\cite{GKKS2013}.}
\label{fig:layout}
\end{figure}

In ScaLAPACK, the $p=p_rp_c$ processors are usually arranged into a
$p_r\times p_c$ grid.
Matrices are distributed across the rectangular processor grid in a \emph{2D
block-cyclic layout} with block size $m_b\times n_b$ (see Figure~\ref{fig:layout} for an example).
The information regarding the data layout is stored in an \emph{array
descriptor} so that the mapping between the entries of the global matrix and their
corresponding locations in the memory hierarchy can be established.
We adopt ScaLAPACK's data layout convention and require that the $n\times
n$ input matrices $H$ and $Z$ have identical data layout with square data
blocks (i.e., $m_b=n_b$). The processor grid, however, does not need to be square.
A distributed matrix~$H$ is referenced by two arrays
\texttt{H} (local matrix entries) and \texttt{DESCH} (array descriptor).
A typical setting of \texttt{DESCH} is listed below.
\begin{itemize}
\item \texttt{DESCH(1)}:
Type of the matrix.
In our case, \texttt{DESCH(1) = 1} since $H$ is stored as a dense matrix.
\item \texttt{DESCH(2)}:
The handle of the BLACS context.
\item \texttt{DESCH(3)}, \texttt{DESCH(4)}:
The size of $H$, i.e., \texttt{DESCH(5) = DESCH(6) = }$n$.
\item \texttt{DESCH(5)}, \texttt{DESCH(6)}:
Blocking factors $m_b$ and $n_b$.
We require that \texttt{DESCH(5) = DESCH(6)}.
\item \texttt{DESCH(7)}, \texttt{DESCH(8)}:
The process row and column that contain $h_{11}$.
Usually, \texttt{DESCH(7) = DESCH(8) = 0}.
\item \texttt{DESCH(9)}:
Leading dimension of the local part of~$H$ on the current processor.
This value needs to be at least one, even if the local part is empty.
\end{itemize}

\subsection{Calling sequence}
The main functionality of this package is to compute the real Schur
decomposition of an upper Hessenberg matrix using the routine
\texttt{PDHSEQR}.
The
ScaLAPACK routine \texttt{PDGEHRD} can be used to transform a general square matrix to Hessenberg form, see the test programs in \texttt{TESTING/} for
examples.
The interface of \texttt{PDHSEQR} displayed below follows the convention of
LAPACK/ScaLAPACK routines~\cite{LAPACK,ScaLAPACK}. \\
\newsavebox{\verbboxpdhseqr}
\begin{lrbox}{\verbboxpdhseqr}
\begin{minipage}{\textwidth}
\begin{verbatim}
      SUBROUTINE PDHSEQR( JOB, COMPZ, N, ILO, IHI, H, DESCH, WR, WI, Z,
     $                    DESCZ, WORK, LWORK, IWORK, LIWORK, INFO )
*
*     .. Scalar Arguments ..
      INTEGER            IHI, ILO, INFO, LWORK, LIWORK, N
      CHARACTER          COMPZ, JOB
*     ..
*     .. Array Arguments ..
      INTEGER            DESCH( * ) , DESCZ( * ), IWORK( * )
      DOUBLE PRECISION   H( * ), WI( N ), WORK( * ), WR( N ), Z( * )
\end{verbatim}
\end{minipage}
\end{lrbox}
\newsavebox{\verbboxdhseqr}
\begin{lrbox}{\verbboxdhseqr}
\begin{minipage}{\textwidth}
\begin{verbatim}
      SUBROUTINE DHSEQR( JOB, COMPZ, N, ILO, IHI, H, LDH, WR, WI, Z,
     $                   LDZ, WORK, LWORK, INFO )
\end{verbatim}
\end{minipage}
\end{lrbox}
\newsavebox{\verbboxpdlahqr}
\begin{lrbox}{\verbboxpdlahqr}
\begin{minipage}{\textwidth}
\begin{verbatim}
      SUBROUTINE PDLAHQR( WANTT, WANTZ, N, ILO, IHI, A, DESCA, WR, WI,
     $                    ILOZ, IHIZ, Z, DESCZ, WORK, LWORK, IWORK,
     $                    ILWORK, INFO )
\end{verbatim}
\end{minipage}
\end{lrbox}
\medskip
%\begin{figure*}
\noindent
\fbox{\usebox{\verbboxpdhseqr}}
\medskip

\noindent For comparison, the (nearly identical) interface of the LAPACK routine \texttt{DHSEQR}.\\
\medskip
\noindent
\fbox{\usebox{\verbboxdhseqr}}
\medskip
Also, the interface of the ScaLAPACK auxiliary routine
\texttt{PDLAHQR} is similar.\\
\medskip
\noindent
\fbox{\usebox{\verbboxpdlahqr}}
\medskip
%\end{figure*}
Therefore, it may not require much effort to switch existing code calling \texttt{PDLAHQR} to
\texttt{PDHSEQR}.

An example for calling \texttt{PDHSEQR} is provided in the test program
(\texttt{TESTING/driver.f}).
We advice that \texttt{PDHSEQR} is called twice---the first call for
performing a workspace query (by setting \texttt{LWORK = -1}) and the second
call for actually performing the computation.

Below is a detailed list of the arguments.
\begin{itemize}
\item \texttt{JOB}: (global input) \texttt{CHARACTER*1}.\\
\texttt{JOB = 'E'}: Compute eigenvalues only;\\
\texttt{JOB = 'S'}: Compute eigenvalues and the Schur form $T$.

\item \texttt{COMPZ} (global input) \texttt{CHARACTER*1}.\\
\texttt{COMPZ = 'N'}: Schur vectors (i.e., $Z$) are not computed;\\
\texttt{COMPZ = 'I'}: $Z$ is initialized to the identity matrix and both~$H$
and~$Z$ are returned;\\
\texttt{COMPZ = 'V'}: $Z$ must contain an orthogonal matrix~$Q$ on entry, and
the product $QZ$ is returned.

\item \texttt{N}: (global input) \texttt{INTEGER}.\\
The order of the Hessenberg matrix~$H$ (and~$Z$).

\item \texttt{ILO}, \texttt{IHI}: (global input) \texttt{INTEGER}.\\
It is assumed that~$H$ is already upper triangular in rows and columns
(\texttt{1:ILO-1}) and (\texttt{IHI+1:N}).
They are normally set by a previous call to \texttt{PDGEBAL}, and then passed
to \texttt{PDGEHRD} when the matrix output by \texttt{PDGEBAL} is reduced to
Hessenberg form.
Otherwise \texttt{ILO = 1} and \texttt{IHI = N} should be used, which
guarantees that $Z$ is orthogonal on exit of \texttt{PDHSEQR}.

\item \texttt{H}: (global input/output) \texttt{DOUBLE PRECISION} array of
dimension \texttt{(DESCH(9),*)}.\\
\texttt{DESCH}: (global and local input) \texttt{INTEGER} array descriptor of
dimension~9.\\
\texttt{H} and \texttt{DESCH} define the distributed matrix~$H$.\\
On entry, \texttt{H} contains the upper Hessenberg matrix~$H$.\\
On exit, if \texttt{JOB = 'S'}, $H$ is quasi-upper triangular in rows and
columns (\texttt{ILO:IHI}), with $1\times1$ and $2\times2$ blocks on the main
diagonal.
The $2\times2$ diagonal blocks (corresponding to complex conjugate pairs of
eigenvalues) are returned in standard form, with $h_{ii}=h_{i+1,i+1}$ and
$h_{i+1,i}h_{i,i+1}<0$.
If \texttt{INFO = 0} and \texttt{JOB = 'E'}, the contents of~$H$ are
unspecified on exit.

\item \texttt{WR}, \texttt{WI}: (global output) \texttt{DOUBLE PRECISION}
array of dimension \texttt{N}.\\
The eigenvalues of \texttt{H(ILO:IHI,IHO:IHI)} are stored in
\texttt{WR(ILO:IHI)} and \texttt{WI(ILO:IHI)} where \texttt{WR} contains the
real parts and \texttt{WI} contains the imaginary parts, respectively.\\
If two eigenvalues are computed as a complex conjugate pair, they are stored
in consecutive elements of \texttt{WR} and \texttt{WI}, say the \texttt{i}-th
and (\texttt{i+1})-th, with \texttt{WI(i) > 0} and \texttt{WI(i+1) < 0}.\\
If \texttt{JOB = 'S'}, the eigenvalues are stored in the same order as on the
diagonal of the Schur form returned in~$H$.

\item \texttt{Z}: (global input/output) \texttt{DOUBLE PRECISION} array of
dimension \texttt{(DESCZ(9),*)}.\\
\texttt{DESCZ}: (global and local input) \texttt{INTEGER} array descriptor of
dimension~9.\\
\texttt{Z} and \texttt{DESCZ} define the distributed matrix~$Z$.\\
If \texttt{COMPZ = 'V'}, on entry \texttt{Z} must contain the current
matrix~$Z$ of accumulated transformations from, e.g., \texttt{PDGEHRD},
and on exit \texttt{Z} has been updated.\\
If \texttt{COMPZ = 'N'}, \texttt{Z} is not referenced.\\
If \texttt{COMPZ = 'I'}, on entry \texttt{Z} does not need be set and on exit,
if INFO = 0, \texttt{Z} contains the orthogonal matrix~$Z$ of the Schur
vectors of~$H$.

\item \texttt{WORK}: (local workspace) \texttt{DOUBLE PRECISION} array of
dimension \texttt{LWORK}.\\
\texttt{LWORK}: (local input) \texttt{INTEGER}.\\
In case \texttt{LWORK = -1}, a workspace query will be performed and on exit,
\texttt{WORK(1)} is set to the required length of the double precision
workspace.
No computation is performed in this case.

\item \texttt{IWORK}: (local workspace) \texttt{INTEGER} array of dimension
\texttt{LIWORK}.\\
\texttt{LIWORK}: (local input) \texttt{INTEGER}.\\
In case \texttt{LIWORK = -1}, a workspace query will be performed and on exit,
\texttt{IWORK(1)} is set to the required length of the integer workspace.
No computation is performed in this case.

\item \texttt{INFO}: (global output) \texttt{INTEGER}.\\
If \texttt{INFO = 0}, \texttt{PDHSEQR} returns successfully.\\
If \texttt{INFO < 0}, let $i=-\texttt{INFO}$, then the $i$-th argument had an
illegal value.\\
(See below for exceptions with $i=7777$ or $i=8888$.)\\
If \texttt{INFO > 0}, then \texttt{PDHSEQR} failed to compute all of the
eigenvalues.
(This is a rare case.)
Elements \texttt{(1:ILO-1)} and \texttt{(INFO+1:N)} of \texttt{WR} and
\texttt{WI} contain the eigenvalues which have been successfully computed.
Let~$U$ be the orthogonal matrix logically produced in the computation
(regardless of \texttt{COMPZ}, i.e., no matter whether it is explicitly
formulated or not).
Then on exit,
\[
\begin{cases}
H_{in}U=U^TH_{out},~Z=U,
& \text{if \texttt{INFO > 0}, \texttt{COMPZ = 'I'},}\\
H_{in}U=U^TH_{out},~Z_{out}=Z_{in}U,
& \text{if \texttt{INFO > 0}, \texttt{COMPZ = 'V'}.}
\end{cases}
\]
If \texttt{INFO = 7777} or \texttt{INFO = 8888}, please send a bug report to
the authors.
\end{itemize}

\subsection{Example programs}
We provide two simple examples in the directory \texttt{EXAMPLES/}.
The program 
\texttt{example1.f} generates a $500\times500$ random matrix and computes
its Schur decomposition, while \texttt{example2.f} reads the benchmark
matrix OLM500%
\footnote{Downloaded from
\url{ftp://math.nist.gov/pub/MatrixMarket2/NEP/olmstead/olm500.mtx.gz}.}
in the Matrix Market format~\cite{MatrixMarket}.

To compute eigenvalues of other matrices, the following lines of the example
program need to be adjusted:
\begin{itemize}
\item Line~40: Matrix size and the block factor.
\item Lines~50--51: Make sure to provide sufficient memory.
\item Line~222: Replace the matrix generator
\texttt{PDMATGEN2}/\texttt{PQRRMMM} by your own matrix.
\end{itemize}

\section{Tuning of parameters$^\star$}
The instructions below are intended for experienced users.
Other users may want to skip reading this section.

In \texttt{SRC/pilaenvx.f} and \texttt{SRC/piparmq.f}, there are several
machine-dependent parameters, see~\cite{GKKS2013} for details.
On contemporary architectures, we expect that most of the default values provided in the source code
yield reasonable performance.
However, for \texttt{PILAENVX(ISPEC=12, 14, 23)} some fine tuning might be
helpful.
The package offers two scripts in the directory
\texttt{TUNING/} that aim at tuning these three parameters.

Before starting the tuning procedure, you first need to choose a frequently
used block factor ($n_b$) and modify the corresponding value in
\texttt{tune1.in}, \texttt{tune2\_1.in}, \texttt{tune2\_2.in}, and
\texttt{tune2\_3.in}.
You also need to adjust the MPI execution command according to your system in
the scripts \texttt{tune1.sh} and \texttt{tune2.sh}.

\begin{itemize}
\item The first script \texttt{tune1.sh} searches suitable settings for
\texttt{PILAENVX(ISPEC=12, 23)}.
It performs the tests described in \texttt{tune1.in} on $1\times1$,
$2\times2$, $4\times4$, $8\times8$ processor grids, and analyzes the collected
data by the code \texttt{tune1.f}.
This procedure usually takes 1--4 hours.
When completed, it reports suggestions on the parameters in the file
\texttt{suggestion1.txt}.
You should then modify the constants \texttt{NMIN} (Line~$193$ in
\texttt{SRC/piparmq.f}) and \texttt{NTHRESH} (Line~$648$ in
\texttt{SRC/pilaenvx.f}) in accordance with these suggestions.

\item The second script \texttt{tune2.sh} searches suitable settings for
\texttt{PILAENVX(ISPEC=14)}.
This set of tests should only be done after running \texttt{tune1.sh}
and modifying
the parameters in \texttt{SRC/pilaenvx.f}, \texttt{SRC/piparmq.f}, and
\texttt{TUNING/piparmq.f} correspondingly.
Then the library should be compiled again with the new settings:\\
\medskip
\fbox{\hbox to 0.9\textwidth{$\strut$%
\texttt{~make clean; make all; make tuning}}}\\
%\medskip
Finally, the script \texttt{tune2.sh} is executed.
This set of tests takes a long time (up to 1--2 days).
If all tests are completed successfully, \texttt{tune2.f} computes
and reports the suggested settings for the parameters in
\texttt{suggestions2.txt}.
You should then update \texttt{SRC/piparmq.f} (Line~197) and rebuild the
library.
This completes the tuning procedure.

If some of the tests are interrupted, due to an error, it may not be necessary to rerun
the whole set of tests.
You can also manually collect the execution times for each test into
\texttt{summary2.txt} and apply \texttt{tune2.f} to determine the parameter suggestions.
\end{itemize}

It is possible to run both tuning scripts on other processor
grids (besides the default $2\times2$, $4\times4$, $8\times8$).
This, however, requires to not only adjust the scripts \texttt{tune*.sh} but also the
programs \texttt{tune*.f} correspondingly.


\section{Terms of Usage}
Use of the ACM Algorithm is subject to the ACM Software Copyright and License
Agreement~\cite{ACMSoftware}.
Furthermore, any use of the PDHSEQR library should be acknowledged by
citing the corresponding paper~\cite{GKKS2013}. Depending on the context, the
citation of the papers~\cite{GKK2009,GKK2010,KKS2012} is also encouraged.

\section{Summary}
The use of the software library PDHSEQR is presented in this document.
Currently only the double precision version is provided; a complex version of
the software (\texttt{PZHSEQR}) is planned for future releases.
Comments, suggestions, and bug reports from users are welcome.
The latest version of the software and documents are always available from the
website of PDHSEQR~\cite{home}.

\section*{Acknowledgements}
The authors are grateful to Bj\"orn Adlerborn and Lars Karlsson for
discussions and comments, and to {\AA}ke Sandgren for support at High
Performance Computing Center North (HPC2N).
They also thank David Guerrero, Rodney James, Julien Langou, Jack Poulson,
Jose Roman, as well as anonymous users from IBM for helpful feedback.

\begin{thebibliography}{10}

\bibitem{ACMSoftware}
{ACM} software license agreement.
\newblock See \url{http://www.acm.org/publications/policies/softwarecrnotice}.

\bibitem{CALGO}
Collected algorithms of the {ACM}.
\newblock See \url{http://calgo.acm.org/}.

\bibitem{home}
{PDHSEQR} homepage.
\newblock See \url{http://www8.cs.umu.se/~myshao/software/pdhseqr/}.

\bibitem{LAPACK}
E.~Anderson, Z.~Bai, C.~H. Bischof, S.~Blackford, J.~W. Demmel, J.~J. Dongarra,
  J.~J.~D. Croz, A.~Greenbaum, S.~J. Hammarling, A.~McKenney, and D.~C.
  Sorensen.
\newblock {\em {LAPACK} User's Guide, 3rd Edition}.
\newblock Society for Industrial and Applied Mathematics, Philadelphia, PA,
  USA, 1999.

\bibitem{MatrixMarket}
Z.~Bai, D.~Day, J.~W. Demmel, and J.~J. Dongarra.
\newblock A test matrix collection for non-{Hermitian} eigenvalue problems
  (release 1.0).
\newblock Technical Report CS-97-355, Department of Computer Science,
  University of Tennessee, 1997.
\newblock Also available online from http://math.nist.gov/MatrixMarket.

\bibitem{ScaLAPACK}
L.~S. Blackford, J.~Choi, A.~Cleary, E.~D'Azevedo, J.~W. Demmel, I.~Dhillon,
  J.~J. Dongarra, S.~Hammarling, G.~Henry, A.~Petitet, K.~Stanley, D.~Walker,
  and R.~C. Whaley.
\newblock {\em {ScaLAPACK} User's Guide}.
\newblock Society for Industrial and Applied Mathematics, Philadelphia, PA,
  USA, 1997.

\bibitem{BBM2002a}
K.~Braman, R.~Byers, and R.~Mathias.
\newblock The multishift {QR} algorithm. {Part I}: Maintaining well-focused
  shifts and level 3 performance.
\newblock {\em {SIAM} J. Matrix Anal. Appl.}, 23(4):929--947, 2002.

\bibitem{BBM2002b}
K.~Braman, R.~Byers, and R.~Mathias.
\newblock The multishift {QR} algorithm. {Part II}: Aggressive early deflation.
\newblock {\em {SIAM} J. Matrix Anal. Appl.}, 23(4):948--973, 2002.

\bibitem{GKK2009}
R.~Granat, B.~K{\aa}gstr\"om, and D.~Kressner.
\newblock Parallel eigenvalue reordering in real {Schur} forms.
\newblock {\em Concurrency and Computat.: Pract. Exper.}, 21(9):1225--1250,
  2009.

\bibitem{GKK2010}
R.~Granat, B.~K{\aa}gstr\"om, and D.~Kressner.
\newblock A novel parallel {QR} algorithm for hybrid distributed memory {HPC}
  systems.
\newblock {\em {SIAM} J. Sci. Comput.}, 32(4):2345--2378, 2010.

\bibitem{GKKS2013}
R.~Granat, B.~K{\aa}gstr\"om, D.~Kressner, and M.~Shao.
\newblock Algorithm xxx: Parallel library software for the multishift {QR} algorithm with
  aggressive early deflation.
\newblock {\em ACM Trans. Math. Software}.
\newblock (to appear).

\bibitem{KKS2012}
B.~K{\aa}gstr\"om, D.~Kressner, and M.~Shao.
\newblock On aggressive early deflation in parallel variants of the {QR}
  algorithm.
\newblock In K.~J\'onasson, editor, {\em Applied Parallel and Scientific
  Computing (PARA 2010)}, volume 7133 of {\em Lecture Notes in Comput. Sci.},
  pages 1--10, Berlin, 2012. Spring{\-}er-Ver{\-}lag.

\end{thebibliography}

\end{document}
