\documentclass[a4paper]{article}

%\VignetteIndexEntry{Getting Started with Fitting Fuzzy Linear Regression Models in R}
%\VignettePackage{fuzzyreg}
%\VignetteDepends{FuzzyNumbers}

\title{Getting Started with Fitting Fuzzy Linear Regression Models in \texttt{R}}
\author{Pavel \v{S}krab\'{a}nek and Nat\'{a}lia Mart\'{i}nkov\'{a}}

\begin{document}
\SweaveOpts{concordance=TRUE}
\bibliographystyle{plain}
\maketitle

@
Fuzzy regression provides an alternative to statistical regression when the
model is indefinite, the relationships between model parameters are vague, sample
size is low or when the data are hierarchically structured. The fuzzy regression is thus
applicable in cases, where the data structure prevents statistical analysis.

Here, we explain the implementation of fuzzy linear regression methods in the \texttt{R} \cite{ref:R}
package \texttt{fuzzyreg} \cite{ref:fuzzyreg}. The Chapter 1: Quick start guides the user
through the direct steps necessary to obtain a fuzzy
regression model from crisp (not fuzzy) data. Followed by the Chapter~\ref{chapter:interpretation}, the user will
be able to infer and interpret the fuzzy regression model.

\tableofcontents

%%%%%%%%%%%%%%%%%%%%%%%%%%%%%%%%%%%%%%%%%%%%%%%%%%%%%%%%%%%%%%%%%%%%%%%%%%%%%%%%%
\section{Quick start}
%%%%%%%%%%%%%%%%%%%%%%%%%%%%%%%%%%%%%%%%%%%%%%%%%%%%%%%%%%%%%%%%%%%%%%%%%%%%%%%%%

To install, load the package and access help, run the following \texttt{R} code:
<<eval=FALSE>>=
install.packages("fuzzyreg", dependencies = TRUE)
require(fuzzyreg)
help(package = "fuzzyreg")
<<echo=FALSE>>=
if(!require(fuzzyreg)) install.packages("fuzzyreg", dependencies = TRUE)
@

Next, load the example data and run the fuzzy linear regression
using a wrapper function \texttt{fuzzylm} that simulates the established functionality
of other regression functions and uses the \texttt{formula} and \texttt{data} arguments:

<<fuz>>=
data(fuzzydat)
f = fuzzylm(y ~ x, data = fuzzydat$lee)
@
The result shows the coefficients of the fuzzy linear regression in form of non-symmetric
triangular fuzzy numbers.
<<>>=
print(f)
@
We can next plot the regression fit, using shading to indicate the degree of membership
of the model predictions.
<<pl, eval=FALSE>>=
plot(f, res = 20, col = "lightblue", main = "PLRLS")
@

\begin{figure}[h]
	\centering
<<fig=TRUE, echo=FALSE, height=5>>=
<<fuz>>
<<pl>>
@
	\caption{Fuzzy linear regression using the PLRLS method \cite{ref:LeeTanaka1999}.}
	\label{fig:plrls}
\end{figure}

Shading visualizes the degree of membership to the triangular fuzzy number (TFN) with a gradient from lightblue to white,
indicating the decrease of the degree of membership from 1 to 0.
The central tendency (thick line) in combination with the left and the right spreads
determine a support interval (dotted lines) of possible values, i.e. values with non-zero membership
degrees, of model predictions. The left and the right spreads determine the lower
and upper boundary of the interval, respectively, where the degree of membership equals to 0.
We can display the model with the \texttt{summary} function.

<<>>=
summary(f)
@

%%%%%%%%%%%%%%%%%%%%%%%%%%%%%%%%%%%%%%%%%%%%%%%%%%%%%%%%%%%%%%%%%%%%%%%%%%%%%%%%%
\section{Triangular fuzzy numbers}
%%%%%%%%%%%%%%%%%%%%%%%%%%%%%%%%%%%%%%%%%%%%%%%%%%%%%%%%%%%%%%%%%%%%%%%%%%%%%%%%%

The package \texttt{FuzzyNumbers} \cite{ref:FuzzyNumbers} provides an excellent
introduction into fuzzy numbers and offers a great flexibility in designing the fuzzy
numbers. Here, we implement a special case of fuzzy numbers, the triangular fuzzy numbers.

A fuzzy real number $\tilde{A}$ is a fuzzy set defined on the set of real numbers.
Each real value number $x$ belongs to the fuzzy set $\tilde{A}$, with a
degree of membership that can range from 0 to 1. The degrees of membership of
$x$ are defined by the membership function $\mu_{\tilde{A}}(x):x\to[0,1]$,
where $\mu_{\tilde{A}}(x^*)=0$ means that the value of $x^*$ is not included in the fuzzy
number $\tilde{A}$ while $\mu_{\tilde{A}}(x^*)=1$ means that $x^*$ is positively
comprehended in $\tilde{A}$ (Figure~\ref{fig:TFN}).

In \texttt{FuzzyNumbers}, the fuzzy number is defined using side functions. In
\texttt{fuzzyreg}, we simplify the input of the TFNs as a vector
of length 3. The first element of the vector specifies the central value $x_c$, where the degree of
membership is equal to 1, $\mu(x_c)=1$. The second element is the left spread, which is
the distance from the central value to a value $x_l$ where $\mu(x_l)=0$ and $x_l<x_c$.
The left spread is thus equal to $x_c-x_l$. The third element of the TFN is the right
spread, i.e. the distance from the central value to a value $x_r$ where $\mu(x_r)=0$ and
$x_r>x_c$. The right spread is equal to $x_r-x_c$. The central value $x_c$ of the TFN is
its core, and the interval $(x_l,x_r)$ is the support of the TFN.

<<TFNfig, echo=FALSE, eval=FALSE>>=
plot(1, type = "n", xlim = c(0,3), ylim=c(0,1.1), xlab = expression(italic("x")),
ylab = "degree of membership", las = 1)
points(matrix(c(.5,1.5,2.5,1,1,1), ncol=2), pch=19)
text(c(.5,1.5,2.5),1, labels=as.expression(lapply(LETTERS[1:3], function(x) bquote(italic(.(x))))),pos=2)
segments(.5,0,.5,1,lty=2)
segments(1.5,0,1.5,1,lty=2)
segments(2.5,0,2.5,1, lty=2)
segments(.7,0,1.5,1)
segments(1.9,0,1.5,1)
segments(2,0,2.5,1)
segments(3,0,2.5,1)
@

\begin{figure}[h]
	\centering
<<fig=TRUE, echo=FALSE, height = 3.5>>=
<<TFNfig>>
@
	\caption{Triangular fuzzy numbers.}
	\label{fig:TFN}
\end{figure}

The crisp number $a=0.5$ can be written as a TFN $A$ with spreads equal
to 0 (Figure~\ref{fig:TFN}):
<<A, eval=FALSE>>=
A = c(0.5, 0, 0)
@
The non-symmetric TFN $B$ and the symmetric TFN $C$ are then:
<<B, eval = FALSE>>=
B = c(1.5, 0.8, 0.4)
C = c(2.5, 0.5, 0.5)
@

%%%%%%%%%%%%%%%%%%%%%%%%%%%%%%%%%%%%%%%%%%%%%%%%%%%%%%%%%%%%%%%%%%%%%%%%%%%%%%%%%
\subsection{Converting crisp numbers to TFNs}
%%%%%%%%%%%%%%%%%%%%%%%%%%%%%%%%%%%%%%%%%%%%%%%%%%%%%%%%%%%%%%%%%%%%%%%%%%%%%%%%%

When the collected data do not contain spreads, we can directly apply the PLRLS method to model
the relationship between the variables in the fuzzy set framework (Chapter 1: Quick start). 
For other methods, the spreads must be imputed.

\begin{itemize}
\item{The TFN might have spreads equal to 0 as shown for $A$ in
Figure~\ref{fig:TFN}.}
\item{Small random spreads might be generated, e.g. using \texttt{abs(runif(n) * 1e-6)}, where
\texttt{n} is the number of the observations.}
\item{Spreads might represent a known measurement error of the device used to collect
the data.}
\item{Spreads might be calculated as a statistic from the data.}
\end{itemize}

Note that the spreads must always be equal to or greater than zero.

%%%%%%%%%%%%%%%%%%%%%%%%%%%%%%%%%%%%%%%%%%%%%%%%%%%%%%%%%%%%%%%%%%%%%%%%%%%%%%%%%
\subsection{Converting TFN from class \texttt{FuzzyNumber}}
%%%%%%%%%%%%%%%%%%%%%%%%%%%%%%%%%%%%%%%%%%%%%%%%%%%%%%%%%%%%%%%%%%%%%%%%%%%%%%%%%

The conversion from an object of the class \texttt{FuzzyNumber} to TFN used in
\texttt{fuzzyreg} requires adjusting the core and the support values of the \texttt{FuzzyNumber}
object to the central value and the spreads. For example, let's define a
trapezoidal fuzzy number $B_1$
that is identical with TFN $B$ displayed in Figure~\ref{fig:TFN}, but that is an object of
class \texttt{FuzzyNumber}.

<<>>=
require(FuzzyNumbers)
B1 = FuzzyNumber(0.7, 1.5, 1.5, 1.9,
		left = function(x) x,
		right = function(x) 1 - x,
		lower = function(a) a,
		upper = function(a) 1 - a)
B1
@

The core of $B_1$ will be equal to the central value of the TFN if the \texttt{FuzzyNumber} object is a
TFN. However, the \texttt{FuzzyNumbers} package considers TFNs as a special case of trapezoidal
fuzzy numbers. The core of $B_1$ thus represents the interval, where $\mu_{B_1}(x^*)=1$, and
the support is the interval, where $\mu_{B_1}(x^*)>0$. We can use these values to construct
the TFN $B$.

<<>>=
xc = core(B1)[1]
l = xc - supp(B1)[1]
r = supp(B1)[2] - xc
c(xc, l, r)
@
<<echo=FALSE>>=
detach("package:FuzzyNumbers", unload=TRUE)
require(fuzzyreg)
@

When the trapezoidal fuzzy number has the core wider than one point, we need to approximate 
a TFN. The simplest method calculates the mean of the core as
\texttt{mean(core(B1))}.

We can also defuzzify the fuzzy number and approximate
the central value with the \texttt{expectedValue()} function. However, the expected value is a
midpoint of the expected interval of a fuzzy number derived from integrating the side functions.
The expected values will not have the degree of membership equal to 1 for non-symmetric
fuzzy numbers. Constructing the mean of the core might be a more appropriate method to
obtain the central value of the TFN for most applications.

Fuzzy numbers with non-linear side functions may have large support intervals, for which
the above conversion algorithm might skew the TFN. The function
\texttt{trapezoidalApproximation()} can first provide a suitable approximation of the
fuzzy number with non-linear side functions, for which the core and the support values
will suitably reflect the central value and the spreads used in \texttt{fuzzyreg}.

%%%%%%%%%%%%%%%%%%%%%%%%%%%%%%%%%%%%%%%%%%%%%%%%%%%%%%%%%%%%%%%%%%%%%%%%%%%%%%%%%
\section{Methods for fitting fuzzy regression models}
%%%%%%%%%%%%%%%%%%%%%%%%%%%%%%%%%%%%%%%%%%%%%%%%%%%%%%%%%%%%%%%%%%%%%%%%%%%%%%%%%

Methods implemented in \texttt{fuzzyreg 0.4} fit fuzzy linear models (Table~\ref{tab:methods}).

\begin{table}[h]
	\caption{Methods for fitting fuzzy regression with \texttt{fuzzyreg}.
	$m$ - number of allowed independent
	variables $x$; $x$, $y$, $\hat{y}$ - type of expected number for independent,
	dependent variables and predictions; s - symmetric; ns - non-symmetric.}
	\label{tab:methods}
	\begin{center}
	\begin{tabular}{lccccc}
		\hline
		Method	&	$m$			&	$x$		&	$y$		&	$\hat{y}$ & Reference\\ \hline
		PLRLS	&	$\infty$	&	crisp	&	crips	&	nsTFN     & \cite{ref:LeeTanaka1999}\\
		PLR		&	$\infty$	&	crisp	&	sTFN	&	sTFN      & \cite{ref:Tanaka1989}\\
		OPLR	&	$\infty$	&	crisp	&	sTFN	&	sTFN      & \cite{ref:Hung2006}\\
		FLS		&	1			&	crisp	&	nsTFN	&	nsTFN     & \cite{ref:Diamond1988}\\
		MOFLR	&	$\infty$	&	sTFN	&	sTFN	&	sTFN      & \cite{ref:Nasrabadi2005}\\
		\hline
	\end{tabular}
	\end{center}
\end{table}

Methods that require symmetric TFNs handle input specifying one spread, but in methods
expecting non-symmetric TFN input, both spreads must be defined even in cases when the
data contain symmetric TFNs.

A possibilistic linear regression (PLR) is a paradigm of whole family of possibilistic-based fuzzy
estimators. It was proposed for crisp observations of the explanatory variables and symmetric fuzzy
observations of the response variable. \texttt{fuzzyreg} uses the min problem implementation 
\cite{ref:Tanaka1989} that estimates the regression coefficients in such a way that the spreads for the model
response are minimal needed to include all observed data. Consequently, the outliers in the data will
increase spreads in the estimated coefficients.

The possibilistic linear regression combined with the least squares (PLRLS) method
\cite{ref:LeeTanaka1999} fits the model prediction spreads and the central tendency with the
possibilistic and the least squares approach, respectively. The input data represent crisp numbers and
the model predicts the response in form of a non-symmetric TFN. Local outliers in the data strongly
influence the spreads, so a good practice is to remove them prior to the analysis.

The method by Hung and Yang \cite{ref:Hung2006} expands PLR \cite{ref:Tanaka1989} by adding an omission
approach for detecting outliers (OPLR). We implemented a version that identifies a single outlier in the
data located outside of the Tukey's fences. The input data include crisp explanatory variables and the
response variable in form of a symmetric TFN.

Fuzzy least squares (FLS) method \cite{ref:Diamond1988} supports a simple FLR for a non-symmetric TFN
explanatory as well as a response variable. This probabilistic-based method (FLS calculates the fuzzy
regression coefficients using least squares) is relatively robust against outliers compared to the
possibilistic-based methods.

A multi-objective fuzzy linear regression (MOFLR) method estimates the fuzzy regression coefficients with
a possibilistic approach from symmetric TFN input data \cite{ref:Nasrabadi2005}. Given a specific weight,
the method determines a trade-off between outlier penalization and data fitting that enables the user to
fine-tune outlier handling in the analysis.


%%%%%%%%%%%%%%%%%%%%%%%%%%%%%%%%%%%%%%%%%%%%%%%%%%%%%%%%%%%%%%%%%%%%%%%%%%%%%%%%%
\section{Running a fuzzy linear regression}
%%%%%%%%%%%%%%%%%%%%%%%%%%%%%%%%%%%%%%%%%%%%%%%%%%%%%%%%%%%%%%%%%%%%%%%%%%%%%%%%%

The TFN definition used in \texttt{fuzzyreg} enables an easy setup of the regression model
using the well-established syntax for regression analyses in \texttt{R}. The model is set up
from a \texttt{data.frame} that contains all observations for the dependent variable and
the independent variables. The \texttt{data.frame} must contain columns with the
respective spreads for all variables that are TFNs.

The example data from Nasrabadi et al. \cite{ref:Nasrabadi2005} contain symmetric TFNs.
The spreads for the independent
variable $x$ are in the column \texttt{xl} and all values are equal to 0. The spreads for the
dependent variable $y$ are in the column \texttt{yl}.

<<>>=
fuzzydat$nas
@

Note that the data contain only one column with spreads per variable. This is an accepted format
for symmetric TFNs, because the values can be recycled for the left and right spreads.

The \texttt{formula} argument used to invoke a fuzzy regression with the \texttt{fuzzylm()} function
will relate $y \sim x$. The columns \texttt{x} and \texttt{y}
contain the central values of the variables. The spreads are not included in the \texttt{formula}.
To calculate the fuzzy regression from TFNs, list the column names with the spreads as a character
vector in the respective arguments of the \texttt{fuzzylm} function.

<<>>=
f2 = fuzzylm(formula = y ~ x, data = fuzzydat$nas,
             fuzzy.left.x = "xl",
             fuzzy.left.y = "yl", method = "moflr")
@

Calls to methods that analyse non-symmetric TFNs must include both arguments for the left and right
spreads, respectively. The arguments specifying spreads for the dependent variable are
\texttt{fuzzy.left.y} and \texttt{fuzzy.right.y}. However, if we wish to analyse symmetric TFNs
using a method for non-symmetric TFNs,
both argumets might call the same column with the values for the spreads.

<<>>=
f3 = fuzzylm(y ~ x, data = fuzzydat$nas,
             fuzzy.left.y = "yl",
             fuzzy.right.y = "yl", method = "fls")
@

As the spreads are included in the model using the column names, the function cannot check
whether the provided information is correct. The issue gains importance when developing a multiple
fuzzy regression model. The user must ascertain that the order of the column names for
the spreads corresponds to the order of the variables in the \texttt{formula} argument.

The fuzzy regression models can be used to predict new data within the range of data used to
infer the model with the \texttt{predict} function. The reason for disabling extrapolations from
the fuzzy regression models lies in
the non-negligible risk that the support boundaries for the TFN might intersect the central tendency.
The predicted TFNs outside of the range of data might not be defined. The predicted values
will replace the original variable values in the \texttt{fuzzylm} data structure.

<<>>=
pred2 = predict(f2)
pred2$y
@



%%%%%%%%%%%%%%%%%%%%%%%%%%%%%%%%%%%%%%%%%%%%%%%%%%%%%%%%%%%%%%%%%%%%%%%%%%%%%%%%%
\subsection{Comparing fuzzy regression models}
%%%%%%%%%%%%%%%%%%%%%%%%%%%%%%%%%%%%%%%%%%%%%%%%%%%%%%%%%%%%%%%%%%%%%%%%%%%%%%%%%

The choice of the method to estimate the parameters of the fuzzy regression model
is data-driven (Table~\ref{tab:methods}). Following the application of the suitable
methods, fuzzy regression models can be compared according to the sum of differences
between membership values of the observed and predicted membership functions
\cite{ref:KimBishu1998}.

In Figure~\ref{fig:TEF}, the points represent central values of the observations and
whiskers indicate their spreads. The shaded area shows the model predictions with the
degree of membership greater than zero. We can
compare the models numerically using the total error of fit $\sum{E}$ \cite{ref:KimBishu1998}
with the \texttt{TEF()} function:

<<figTEF, echo=FALSE, eval=FALSE>>=
par(mfrow=c(1,2))
plot(f2, res=20, col.fuzzy="lightblue", main = "f2 - MOFLR")
plot(f3, res=20, col.fuzzy = "lightblue", main = "f3 - FLS")
@


\setkeys{Gin}{width=\textwidth}
\begin{figure}[h]
  \begin{center}
<<fig=TRUE, echo=FALSE, height=3.5,fig.show='hold'>>=
<<figTEF>>
@
\end{center}
\caption{Comparison of fuzzy regression models fitted with the MOFLR and FLS methods. The FLS
   method has lower total error of fit $\sum{E}$ and thus better reflects the observations.}
\label{fig:TEF}
\end{figure}
\setkeys{Gin}{width=.8\textwidth}


<<>>=
TEF(f2)
TEF(f3)
@

Lower values of $\sum{E}$ mean that the predicted TFNs fit better with the observed TFNs.




%%%%%%%%%%%%%%%%%%%%%%%%%%%%%%%%%%%%%%%%%%%%%%%%%%%%%%%%%%%%%%%%%%%%%%%%%%%%%%%%%
\subsection{Interpreting the fuzzy regression model}
%%%%%%%%%%%%%%%%%%%%%%%%%%%%%%%%%%%%%%%%%%%%%%%%%%%%%%%%%%%%%%%%%%%%%%%%%%%%%%%%%
\label{chapter:interpretation}

When comparing the fuzzy linear regression and a statistical linear regression models, we can observe
that while the fuzzy linear regression shows something akin to a confidence interval,
the interval differs from the confidence intervals derived from a statistical linear 
regression model (Figure~\ref{fig:compare}).

<<regfig, eval=FALSE, echo=F>>=
par(mfrow=c(1,2))
f4 = fuzzylm(y~x, fuzzydat$lee, method="plrls")
plot(f4, res=20, col.fuzzy="lightblue", main="PLRLS")
f1 = lm(y~x, fuzzydat$lee)
newx = seq(min(fuzzydat$lee$x), max(fuzzydat$lee$y), length.out = 500)
conf1 = predict(f1, newdata=data.frame(x=newx),
                interval="confidence")
plot(fuzzydat$lee$x, fuzzydat$lee$y, xlab = "x", ylab = "y", main="LS")
abline(f1)
lines(newx, conf1[,2], lty=2)
lines(newx, conf1[,3], lty=2)
@

\setkeys{Gin}{width=\textwidth}
\begin{figure}[h]
  \begin{center}
<<fig=TRUE, echo=FALSE, height=3.5,fig.show='hold'>>=
<<regfig>>
@
\end{center}
\caption{Fuzzy and statistical linear regression. Fuzzy regression displays the central value and
the support boundaries determined by the left and right spread (PLRLS - left panel). 
Statistical least-squares regression
shows the 95\% confidence interval for the regression fit (LS - right panel).}
\label{fig:compare}
\end{figure}
\setkeys{Gin}{width=.8\textwidth}

The confidence interval from a statistical regression model shows the certainty that the modeled
relationship fits within. We are 95\% certain that the true relationship between the variables
is as displayed.

On the other hand, the support of the fuzzy regression model prediction shows the range of possible
values. The dependent variable can reach any value from the set, but the values more distant from the
central tendency will
have smaller degree of membership. We can imagine the values closer
to the boundaries as vaguely disappearing from the set as if they bleached out (gradient towards white
in the fuzzy regression model plots).

%%%%%%%%%%%%%%%%%%%%%%%%%%%%%%%%%%%%%%%%%%%%%%%%%%%%%%%%%%%%%%%%%%%%%%%%%%%%%%%%%
\section{How to cite}
%%%%%%%%%%%%%%%%%%%%%%%%%%%%%%%%%%%%%%%%%%%%%%%%%%%%%%%%%%%%%%%%%%%%%%%%%%%%%%%%%

To cite \texttt{fuzzyreg}, include the reference to the software and the used method.

\begin{quote}
\v{S}krab\'{a}nek P. and Mart\'{i}nkov\'{a} N. 2021. Algorithm 1017: fuzzyreg: An R Package for 
Fuzzy Linear Regression Models. ACM Trans. Math. Softw. 47: 29. doi: 10.1145/3451389.
\end{quote}

The references for the specific methods are accessible through the method help, e.g.
the default fuzzy linear regression method PLRLS:
<<eval=FALSE>>=
?plrls
@

%%%%%%%%%%%%%%%%%%%%%%%%%%%%%%%%%%%%%%%%%%%%%%%%%%%%%%%%%%%%%%%%%%%%%%%%%%%%%%%%%
%%%%%%%%%%%%%%%%%%%%%%%%%%%%%%%%%%%%%%%%%%%%%%%%%%%%%%%%%%%%%%%%%%%%%%%%%%%%%%%%%
\bibliography{references}
\end{document}
