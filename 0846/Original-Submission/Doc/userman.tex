\documentclass[acmtocl]{acmtrans2m}

\usepackage{amsfonts}

%\usepackage{natbib}

%\acmVolume{0}
%\acmNumber{0}
%\acmYear{0}
%\acmMonth{0}

\newcommand{\BibTeX}{{\rm B\kern-.05em{\sc i\kern-.025em b}\kern-.08em
    T\kern-.1667em\lower.7ex\hbox{E}\kern-.125emX}}

\markboth{Tangan Gao, T.Y. Li and Mengnien Wu}{A Software Package
for Mixed Volume Computation}

\title{ALgorithm xxx: MixedVol: User Manual}

\author{ TANGAN GAO \\  California State University, Long Beach \\
         T. Y. LI \\ Michigan State University \\
         MENGNIEN WU \\ Tamkang University}
\newcommand{\calA}{{\mathcal S}}
\newcommand{\hcalA}{\hat{\mathcal S}}
\newcommand{\calQ}{{\mathcal Q}}
\newcommand{\hcalQ}{\hat{\mathcal Q}}
\newcommand{\calM}{{\mathcal M}}
\newcommand{\calC}{{\mathcal C}}
\newcommand{\calD}{{\mathcal D}}
\newcommand{\calT}{{\mathcal T}}
\newcommand{\calP}{{\mathcal P}}
\newcommand{\calF}{{\mathcal F}}
\newcommand{\calE}{{\mathcal E}}
\newcommand{\ba}{{\mathbf a}}
\newcommand{\bb}{{\mathbf b}}
\newcommand{\bc}{{\mathbf c}}
\newcommand{\bff}{{\mathbf f}}
\newcommand{\bs}{{\mathbf s}}
\newcommand{\bq}{{\mathbf q}}
\newcommand{\bu}{{\mathbf u}}
\newcommand{\bx}{{\mathbf x}}
\newcommand{\by}{{\mathbf y}}
\newcommand{\bz}{{\mathbf z}}
\newcommand{\conv}{{\rm conv}}
\newcommand{\Vol}{{\rm Vol}}

\begin{document}

\section{ORGANIZATION AND USAGE}


A makefile is provided to compile the codes in this package. After unpacking
this software,
simply execute
the command
\begin{verbatim}
         build
\end{verbatim}
in the main directory MixedVol
to generate the executable code ``mixedvol''.

\subsection{Call Sequence and Output}

The software package MixedVol is easy to use.
To calculate, for example, the mixed volume
of the support $\calA=(\calA_1,\calA_2,\calA_3)$, where
\begin{eqnarray*}
\calA_1&=&\{
        (0,0,0), (1,0,0),(0,1,0),(0,0,1),(1,1,1)
        \}, \\
\calA_2&=&\{
        (0,0,0), (2,2,2)
        \}, \\
\calA_3&=&\{
        (0,0,0), (3,0,0),(0,3,0),(0,0,3)
        \},
\end{eqnarray*}
a file should be created to contain the support data
in the following format:
\begin{verbatim}
         3
         5
         2
         4
         0  0  0
         1  0  0
         0  1  0
         0  0  1
         1  1  1
         0  0  0
         2  2  2
         0  0  0
         3  0  0
         0  3  0
         0  0  3
\end{verbatim}
The number on the first line is the dimension of the support. For this support it is 3.
The numbers from line two to line four give the numbers of points in the
individual support
$\calA_1$, $\calA_2$, and $\calA_3$.
They are 5, 2 and 4 points, respectively.
The next five lines contain the points in $\calA_1$ with each point listed on one line.
The two lines after those contain the points in $\calA_2$. The last four lines
are the points in $\calA_3$.
Call this file MySupport, say,  and
execute the command
\begin{verbatim}
         mixedvol -s MySupport
\end{verbatim}
It will produce the mixed volume of the support (namely 18) and will output it on screen.
The flag ``{\tt -s}'' means the input file
consists of a support.
Note that for an $n$-dimensional support $\calA=(\calA_1,\dots,\calA_n)$, all information about
$\calA_1,\dots,\calA_n$ must be listed in the file.
The algorithm will automatically determine the semi-mixed structure of the support
when it exists.

To calculate, for example, the mixed volume of the following system of 3 polynomials
in 3 variables,
\begin{eqnarray*}
p_1(x_1,x_2,x_3)&=& (x_1^2+x_2^2+x_3^2-1)(x_1-0.5)(x_2-x_1^2), \\
p_2(x_1,x_2,x_3)&=& (x_1^2+x_2^2+x_3^2-1)(x_2-0.5)(x_3-x_1^3), \\
p_3(x_1,x_2,x_3)&=& (x_1^2+x_2^2+x_3^2-1)(x_3-0.5)(x_3-x_1^2)(x_2-x_1^2),
\end{eqnarray*}
a file should be created first to define the polynomials of the
system in the following format:
\begin{verbatim}
         {
          (x1^2+x2^2+x3^2-1)*(x1-0.5)*(x2-x1^2);
          (x1^2+x2^2+x3^2-1)*(x2-0.5)*(x3-x1^3);
          (x1^2+x2^2+x3^2-1)*(x3-0.5)*(x3-x1^2)*(x2-x1^2);
         }
\end{verbatim}
The file starts with a ``$\{$'' and ends with a ``$\}$''. Each polynomial ends with a ``;''.
Call this file MyPolynomials, say, and execute the command
\begin{verbatim}
         mixedvol -p MyPolynomials
\end{verbatim}
It will produce the mixed volume of the polynomial system (namely 119)
and will output it on screen.
The flag ``{\tt -p}'' means the input file consists of a polynomial system.

\subsection{General Description of the Modules}

To calculate the mixed volume of a support
$\calA=(\calA_1,\dots,\calA_n)$, $\calA_i\subset\mathbb Z^n$ for
$i=1,\dots,n$, the main module MixedVolDriver
calls two modules Pre4MV and MixedVol.

The module Pre4MV determines all non-vertex points of each
support $\calA_1,\dots,\calA_n$.  It
permanently removes them from the mixed volume computation
because they don't contribute to the mixed volume of the support,
resulting in a smaller support.
Then Pre4MV determines the semi-mixed structure of the smaller support
and re-arranges the support.

From the output of Pre4MV, if the support $\calA$ has
the semi-mixed structure $(\calA^{(1)},\dots,\calA^{(r)})$ of type $(k_1,\dots,k_r)$,
i.e., each $\calA^{(i)}$ is repeated $k_i$ times for $i=1,\dots,r$,
the module MixedVol then computes all mixed cells of type $(k_1,\dots,k_r)$
by searching through the lower hull of the randomly lifted support
$(\hat{\calA}^{(1)},\dots,\hat{\calA}^{(r)})$, where
$$
\hat{\calA}^{(i)}=\{(a,\omega_i(a)) \,|\, a\in\calA^{(i)} \mbox{ \rm and } \omega_i(a) \,
\mbox{ \rm is a random real number}\},
\quad i=1,\dots, r.
$$
A mixed cell $(C_1,\dots,C_r)$ of type $(k_1,\dots,k_r)$, where for each
$i=1,\dots,r$, $C_i$ consists of $k_i+1$ affinely independent points of $\calA^{(i)}$,
is defined to have the property
$$
\dim(\conv(C_1+\cdots+C_r))=\dim(\conv(C_1))+\cdots+\dim(\conv(C_r))=k_1+\cdots+k_r=n.
$$
And the mixed volume of the support $(\calA^{(1)},\dots,\calA^{(r)})$ of type $(k_1,\dots,k_r)$
can be assembled from the volumes of all the mixed cells of type
$(k_1,\dots,k_r)$ induced by
$(\hat{\calA}^{(1)},\dots,\hat{\calA}^{(r)})$.

To calculate the mixed volume of the support of a system of $n$ polynomials
in $n$ variables, the main module MixedVolDriver first
calls the module PolynomialSystemReader, which
reads the polynomial system from an input file
and generates its support.
With the support of the polynomial system available,
the main module  MixedVolDriver then
calls the modules Pre4MV and MixedVol
to calculate the mixed volume of the polynomial system.
The module PolynomialSystemReader is developed in \cite{LiLi}.
The mixed cells together with the lifting function
$\omega=(\omega_1,\dots,\omega_r)$ of the support are crucial
for solving the polynomial system
by the polyhedral homotopy continuation method.

In this release of the package MixedVol, the module PolynomialSystemReader
together with its dependencies are grouped into the sub-directory ``SRC/PolyReader'',
the module Pre4MV as well as its dependencies are located in the sub-directory
``SRC/PreProcess'',
and the module MixedVol along with its dependencies are in the sub-directory ``SRC/MixedVol''.

\bibliographystyle{acmtrans}

\begin{thebibliography}{}


\bibitem[Li and Li (2001)]{LiLi}
Li, T.Y. and Li, X. 2001.
Finding mixed cells in the mixed volume computation.
{\it Foundation of Comput. Math.} {\bf 1}, 161--181.
Software package available at
http://www.math.msu.edu/$\sim$li.


\end{thebibliography}
\end{document}
