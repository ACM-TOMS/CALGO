\documentclass[acmtoms]{acmtrans2m}
\usepackage{amssymb}
\usepackage{array}
\usepackage{alltt}
\usepackage[reqno]{amsmath}
\usepackage{alltt}
\usepackage{fancyvrb}
\usepackage{graphicx}
\usepackage{epsfig}
\usepackage{color}
\usepackage{subfigure}
\markboth{M. Mehra and K. S. Patel}{}
\title{ User Manual for the paper titled ``Algorithm xxx: A Suite of Compact Finite Difference Schemes"}
\author{MANI MEHRA and KULDIP SINGH PATEL, Indian Institute of Technology Delhi, India.}
\begin{document}
\begin{bottomstuff}
Author's address: M. Mehra (mmehra@maths.iitd.ac.in) and K. S. Patel
(kuldip@maths.iitd.ac.in), Indian Institute of Technology Delhi, Hauz
Khas, New Delhi-110 016
\end{bottomstuff}
\maketitle
Our MATLAB routine have one MATLAB function and three folders.
\begin{itemize}
\item \verb#test.m#: This MATLAB function computes the error between analytic differentiation and corresponding compact finite difference approximation for a given function.
\item \verb#cfdm_periodic#: This folder has the MATLAB files for computing the compact finite difference approximations for periodic boundary condition in one and two dimensions.
\item \verb#cfdm_dirichlet#: This folder has the MATLAB files for computing the compact finite difference approximations for dirichlet boundary condition in one and two dimensions.
\item \verb#cfdm_Neumann#: This folder has the MATLAB files for computing the compact finite difference approximations for Neumann boundary in one dimension.
\end{itemize}
@@@@@@@@@@@@@@@@@@@@@@@@@@@@@@@@@@@@@@@@\\
 The function \verb#test.m# computes the error between analytic differentiation and corresponding compact finite difference approximation for a given function.
\begin{alltt}
>>test
\end{alltt}
@@@@@@@@@@@@@@@@@@@@@@@@@@@@@@@@@@@@@@@@@\\
\textbf{Compact finite difference schemes for periodic boundary in one dimension}\\
 In this case, \verb#N# is the number of grid points, \verb#p# is the order of accuracy desired, \verb#x_l# is the left end of the interval and \verb#x_r# is the right end of the interval. Differentiation matrix \verb#D# is order of \verb#N#. \verb#f(x)# is a function defined as $f=(f_{1},f_{2},...,f_{N})^T$.
\section{} \verb#compact_first_periodic.m#\\
 The function \verb#compact_first_periodic.m# computes a differentiation matrix for first derivative approximation. The calling command for this function is
\begin{alltt}
>>[D]=compact_first_periodic(N,p,x_l,x_r)
\end{alltt}
Permissible values for \verb#p# are $4$, $6$, $8$ and $10$. Then first derivative approximation $(f')$ for any function \verb#f# can be written as
\[
f'\approx D*f.
\]
\section{} \verb#compact_second_periodic.m#\\
 The function \verb#compact_second_periodic.m# computes a differentiation matrix for second derivative approximation. The calling command for this function is
\begin{alltt}
>>[D]=compact_second_periodic(N,p,x_l,x_r)
\end{alltt}
Permissible values for $p$ are $4$, $6$, $8$ and $10$. Second derivative approximation $(f'')$ for any function \verb#f# can be written as
\[
f''\approx D*f.
\]
\section{} \verb#compact_third_periodic.m#\\
 The function \verb#compact_third_periodic.m# computes a differentiation matrix for third derivative approximation. The calling command for this function is
\begin{alltt}
>>[D]=compact_third_periodic(N,p,x_l,x_r)
\end{alltt}
Permissible values for \verb#p# is $6$. Third derivative approximation $(f''')$ for any function \verb#f# can be written as
\[
f'''\approx D*f.
\]
\section{} \verb#compact_fourth_periodic.m#\\
 The function \verb#compact_fourth_periodic.m# computes a differentiation matrix for fourth derivative approximation. The calling command for this function is
\begin{alltt}
>>[D]=compact_fourth_periodic(N,p,x_l,x_r)
\end{alltt}
Permissible values for \verb#p# are $4$ and $6$. Fourth derivative approximation $(f'''')$ for any function \verb#f# can be written as
\[
f''''\approx D*f.
\]
@@@@@@@@@@@@@@@@@@@@@@@@@@@@@@@@@@@@@@@@@\\
\textbf{Compact finite difference schemes for periodic boundary in two dimensions}\\
 In this case, \verb#N# is the number of grid points, \verb#p# is the order of accuracy desired, \verb#x_l# and \verb#x_r#  are the left end and the right end of the interval in $x$ direction, \verb#y_l# and \verb#y_r# are the left end and right end of the interval in $y$ direction. Differentiation matrix \verb#D# is order of \verb#N#$^2$ and permissible values for \verb#p# are $4$, $6$, $8$ and $10$. If $f(x,y)$ is a function defined as $f=(f_{1,1},f_{2,1},...,f_{N,1},f_{1,2},f_{2,2},...,f_{N,2},...,f_{1,N},f_{2,N},...,f_{N,N})^T$.
\section{} \verb#compact_first_periodic_2dx.m#\\
 The function \verb#compact_first_periodic_2dx.m# computes a differentiation matrix for first order partial derivative $(\frac{\partial}{\partial x})$ approximation. The calling command for this function is
\begin{alltt}
>>[D]=compact_first_periodic_2dx(N,p,x_l,x_r)
\end{alltt}
First order partial derivative approximation $\left(\frac{\partial f}{\partial x}\right)$ for any function \verb#f# can be written as
\[
\frac{\partial f}{\partial x}\approx D*f.
\]
\section{} \verb#compact_first_periodic_2dy.m#\\
 The function \verb#compact_first_periodic_2dy.m# computes a differentiation matrix for first order partial derivative approximation with respect to $y$ variable $\left(\frac{\partial f}{\partial y}\right)$. The calling command for this function is
\begin{alltt}
>>[D]=compact_first_periodic_2dy(N,p,y_l,y_r)
\end{alltt}
 First order partial derivative approximation $\left(\frac{\partial f}{\partial y}\right)$ for any function \verb#f# can be written as
\[
\left(\frac{\partial f}{\partial y}\right)\approx D*f.
\]
\section{} \verb#compact_second_periodic_2dxx.m#\\
 The function \verb#compact_second_periodic_2dxx.m# computes a differentiation matrix for second order partial derivative approximation. The calling command for this function is
\begin{alltt}
>>[D]=compact_second_periodic_2dxx(N,p,x_l,x_r)
\end{alltt}
Second order partial derivative approximation $\left(\frac{\partial^2 f}{\partial x^2}\right)$ for any function \verb#f# can be written as
\[
\left(\frac{\partial^2 f}{\partial x^2}\right)\approx D*f.
\]
\section{} \verb#compact_second_periodic_2dyy.m#\\
 The function \verb#compact_second_periodic_2dyy.m# computes a differentiation matrix for second order partial derivative approximation. The calling command for this function is
\begin{alltt}
>>[D]=compact_second_periodic_2dyy(N,p,y_l,y_r)
\end{alltt}
 Second order partial derivative approximation $\left(\frac{\partial^2 f}{\partial y^2}\right)$ for any function \verb#f# can be written as
\[
\left(\frac{\partial^2 f}{\partial y^2}\right)\approx D*f.
\]
\section{} \verb#compact_mixed_periodic_2dxy.m#\\
 The function \verb#compact_mixed_periodic_2dxy.m# computes a differentiation matrix for mixed derivative approximation. The calling command for this function is
\begin{alltt}
>>[D]=compact_mixed_periodic_2dxy(N,p,x_l,x_r,y_l,y_r)
\end{alltt}
 Mixed derivative approximation $\left(\frac{\partial^2 f}{\partial x \partial y}\right)$ for any function \verb#f# can be written as
\[
\left(\frac{\partial^2 f}{\partial x \partial y}\right)\approx D*f.
\]
@@@@@@@@@@@@@@@@@@@@@@@@@@@@@@@@@@@@@@@@@\\
\textbf{Compact finite difference schemes for Dirichlet boundary in one dimension}\\
 In this case, \verb#N# is the number of grid points, \verb#p# is the order of accuracy desired, \verb#x_l# is the left end of the interval and \verb#x_r# is the right end of the interval. Differentiation matrix \verb#D# is of order \verb#N# and permissible values for \verb#p# are $4$ and $6$. $f(x)$ is a function defined as $f=(f_{1},f_{2},...,f_{N})^T$.
\section{} \verb#first_compact_dirichlet.m#\\
 The function \verb#first_compact_dirichlet.m# computes a differentiation matrix for first derivative approximation in case of Dirichlet boundary conditions. The calling command for this function is
\begin{alltt}
>>[D]=first_compact_dirichlet(N,p,x_l,x_r)
\end{alltt}
First derivative approximation ($f'$) for any function \verb#f# can be written as
\[
f'\approx D*f.
\]
\section{} \verb#second_compact_dirichlet.m#\\
 The function \verb#second_compact_dirichlet.m# computes a differentiation matrix for second derivative approximation in case of Dirichlet boundary conditions. The calling command for this function is
\begin{alltt}
>>[D]=second_compact_dirichlet(N,p,x_l,x_r)
\end{alltt}
Second derivative approximation ($f''$) for any function \verb#f# can be written as
\[
f''\approx D*f.
\]
@@@@@@@@@@@@@@@@@@@@@@@@@@@@@@@@@@@@@@@@\\
\textbf{Compact finite difference schemes for Dirichlet boundary in two dimensions}\\
 In this case, \verb#N# is the number of grid points, \verb#p# is the order of accuracy desired, \verb#x_l# and \verb#x_r#  are the left end and the right end of the interval in $x$ direction, \verb#y_l# and \verb#y_r# are the left end and right end of the interval in $y$ direction. In this case, differentiation matrix \verb#D# is order of \verb#N#$^2$, permissible values for \verb#p# are $4$ and $6$ and $f(x,y)$ is a function in two variable defined as $f=(f_{1,1},f_{2,1},...,f_{N,1},f_{1,2},f_{2,2},...,f_{N,2},...,f_{1,N},f_{2,N},...,f_{N,N})^T$.
\section{} \verb#first_compact_dirichlet_2dx.m#\\
 The function \verb#first_compact_dirichlet_2dx.m# computes a differentiation matrix for first order partial derivative $(\frac{\partial}{\partial x})$ approximation. The calling command for this function is
\begin{alltt}
>>[D]=first_compact_dirichlet_2dx(N,p,x_l,x_r)
\end{alltt}
First order partial derivative approximation $\left(\frac{\partial f}{\partial x}\right)$ for any function \verb#f# can be written as
\[
\frac{\partial f}{\partial x}\approx D*f.
\]
\section{} \verb#first_compact_dirichlet_2dy.m#\\
 The function \verb#first_compact_dirichlet_2dy.m# computes a differentiation matrix for first order partial derivative approximation with respect to $y$ variable $\left(\frac{\partial f}{\partial y}\right)$. The calling command for this function is
\begin{alltt}
>>[D]=first_compact_dirichlet_2dy(N,p,y_l,y_r)
\end{alltt}
First order partial derivative approximation $\left(\frac{\partial f}{\partial y}\right)$ for any function \verb#f# can be written as
\[
\left(\frac{\partial f}{\partial y}\right)\approx D*f.
\]
\section{} \verb#second_compact_dirichlet_2dxx.m#\\
 The function \verb#second_compact_dirichlet_2dxx.m# computes a differentiation matrix for second order partial derivative approximation. The calling command for this function is
\begin{alltt}
>>[D]=second_compact_dirichlet_2dxx(N,p,x_l,x_r)
\end{alltt}
Second order partial derivative approximation $\left(\frac{\partial^2 f}{\partial x^2}\right)$ for any function \verb#f# can be written as
\[
\left(\frac{\partial^2 f}{\partial x^2}\right)\approx D*f.
\]
\section{} \verb#second_compact_dirichlet_2dyy.m#\\
 The function \verb#compact_second_periodic_2dyy.m# computes a differentiation matrix for second order partial derivative approximation. The calling command for this function is
\begin{alltt}
>>[D]=second_compact_dirichlet_2dyy(N,p,y_l,y_r)
\end{alltt}
Second order partial derivative approximation $\left(\frac{\partial^2 f}{\partial y^2}\right)$ for any function \verb#f# can be written as
\[
\left(\frac{\partial^2 f}{\partial y^2}\right)\approx D*f.
\]
\section{} \verb#mixed_compact_dirichlet_2dxy.m#\\
The function \verb#mixed_compact_dirichlet_2dxy.m# computes a differentiation matrix for mixed derivative approximation in case of Dirichlet boundary conditions. The calling command for this function is
\begin{alltt}
>>[D]=mixed_compact_dirichlet_2dxy(N,p,x_l,x_r,y_l,y_r)
\end{alltt}
Mixed derivative approximation for any function \verb#f# can be written as
\[
\frac{\partial^2 f}{\partial x \partial y}\approx D*f.
\]
@@@@@@@@@@@@@@@@@@@@@@@@@@@@@@@@@@\\
\textbf{Compact finite difference schemes for Neumann boundary in one dimension}\\
\section{} \verb#compact_second_neumann.m#\\
 The function \verb#compact_second_neumann.m# computes a differentiation matrix for second derivative approximation in case of Neumann boundary conditions. The calling command for this function is
\begin{alltt}
>>[D,K]=compact_second_neumann(N,u_left,u_right,x_l,x_r)
\end{alltt}
where \verb#D# is of order \verb#N#, \verb#K# is a \verb#N# $\times$ $1$ vector, \verb#u_left# is the left Neumann boundary condition and \verb#u_right# is the right Neumann boundary condition. Second derivative approximation ($f''$) for any function \verb#f# can be written as
\[
f''\approx D*f+K.
\]
\end{document} 