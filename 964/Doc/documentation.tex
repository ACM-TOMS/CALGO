\documentclass[a4paper,10pt]{article}
\usepackage[utf8]{inputenc}
\usepackage{verbatim}

%opening
\title{Documentation for sample code}
\author{Adri\'an Lozano-Dur\'an and Guillem Borrell}

\begin{document}

\maketitle

\section{Folder information}

This package contains the next folders:
\begin{enumerate}
 \item \emph{Src}:  source files for the sample codes. Serial and Coarrays.
 \item \emph{Doc}:  documentation (this file).
 \item \emph{data}: input data required to execute the example.
\end{enumerate}

\section{Compilation}

\noindent To compile the code with GNU Fortran go to the
\emph{Src/fortran\_serial/} folder and run in a linux terminal:
%
\begin{verbatim}
  make genus
\end{verbatim}
%
\noindent To compile with Intel\textsuperscript{\textregistered}
Fortran Compiler edit the \emph{F90} variable in the \emph{Makefile}
and use \emph{F90=ifort}. Modify the flags \emph{F90FLAGS} and
\emph{LFLAGS} accordingly. To compile the Fortran Coarrays version go
to \emph{Src/fortran\_coarrray/} and run \verb|make genus_coarray|.

\section{Usage}
%
\noindent To run the code, in a linux terminal:
\begin{verbatim}
  ./genus <file-name>
\end{verbatim}
where file-name contains the raw data.

\section{Example}

\noindent To run the example:
\begin{verbatim}
  ./genus ../../data/random.128.dat
\end{verbatim}
The results should match those from case \emph{Random2} in Ref. [1],
table III.

For the Fortran Coarrays version with OpenCoarray - GNU
Fortran\footnote{IMPORTANT NOTE: By the time this document was
  generated, the \emph{critical constructs} cause an internal compiler
  error with OpenCoarray version 1.0.3. In that case, comment all the
  \emph{critical constructs} and the code should work. Note that this
  change will not affect the output of the program, but could lead to
  I/O bottlenecks when the number of images used is large. Intel
  Fortran Compiler at version 2016 16.0.0 20150815 is free from this
  issue.}
\begin{verbatim}
  mpiexec -n <N> ./genus_coarrays ../../data/random.128.dat
\end{verbatim}
where N is the number of processing images.

For the Fortran Coarrays version with
Intel\textsuperscript{\textregistered} Fortran Compiler
\begin{verbatim}
  ./genus_coarrays ../../data/random.128.dat
\end{verbatim}
Use the environment variable \emph{FOR\_COARRAY\_NUM\_IMAGES} to set
the number of processing images.

\section{References}
[1] Adri\'an Lozano-Dur\'an and Guillem Borrell `\emph{An efficient
  algorithm to compute the genus of discrete surfaces and applications
  to turbulent flows}' ACM TOMS, 2015.

\end{document}
