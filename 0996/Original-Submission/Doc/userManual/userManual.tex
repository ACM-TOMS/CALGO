%%%%%%%%%%%%%%%%%%%%%%%%%%%%%%%%%%%%%
%
%file name = latextop.tex
%
%LATEXTOP TEX%%%%%%%%%%%%%%%%%%%%%%%%%%
%\documentclass[12pt, a4j, landscape]{jarticle}
\documentclass[12pt]{article}
\usepackage[dvips]{epsfig}
\usepackage{amsthm,amsmath,amssymb}
%\usepackage{ascmac}
%\usepackage{latexsym}
\usepackage{latexsym}
% \usepackage{times}
%\usepackage[dvipdfmx]{graphicx}
\usepackage{graphics}
\usepackage{url}
\usepackage{placeins}
\usepackage{hyperref}
\usepackage{color}
\usepackage{bm}
\usepackage{comment}
\usepackage{statex}


\setlength{\evensidemargin}{0cm}
\setlength{\oddsidemargin}{0cm}
%\setlength{\textwidth}{26cm}
%\setlength{\textheight}{18cm}
%\setlength{\textwidth}{6.25in}
\setlength{\textwidth}{6.45in}
%\setlength{\textheight}{8.5in}
\setlength{\textheight}{9.2in}
% \setlength{\topmargin}{-1.8cm}
% \setlength{\topmargin}{-1.0cm}
\setlength{\topmargin}{0in}
\setlength{\headheight}{0in}
\setlength{\headsep}{0in}
\setlength{\topskip}{0in} 


%%%%% Definition of Color ---> %%%%%
\def\TWhite{\textcolor{white}}
\def\TBlack{\textcolor{black}}
\def\TGray{\textcolor{gray}}
\def\TBlue{\textcolor{blue}}
\def\TGreen{\textcolor{green}}
\def\TYellow{\textcolor{yellow}}
\def\TRed{\textcolor{red}}
\def\TOrange{\textcolor{orange}}
\def\TViolet{\textcolor{violet}}
\def\TPurple{\textcolor{purple}} 
\def\TBrown{\textcolor{brown}}

\definecolor{blue}{rgb}{0,0,0.9}
\definecolor{red}{rgb}{0.9,0,0}
\definecolor{green}{rgb}{0,0.9,0}
\definecolor{violet}{rgb}{0.5804,0.0000,0.8275}  
\newcommand{\blue}[1]{\begin{color}{blue}#1\end{color}}
\newcommand{\magenta}[1]{\begin{color}{magenta}#1\end{color}}
\newcommand{\red}[1]{\begin{color}{red}#1\end{color}}
\newcommand{\green}[1]{\begin{color}{green}#1\end{color}}
\newcommand{\violet}[1]{\begin{color}{violet}#1\end{color}}
%%%%% <--- Definition of Color %%%%%


\def\@themcountersep{}
\def\thesection{\arabic{section}}

\newtheorem{THEO}{Theorem}[section]
\newtheorem{ALGo}[THEO]{Algorithm}
\newtheorem{CONJ}[THEO]{Conjecture}
\newtheorem{COND}[THEO]{Condition}
\newtheorem{CORO}[THEO]{Corollary}
\newtheorem{DEFI}[THEO]{Definition}
\newtheorem{EXAMP}[THEO]{Example}
\newtheorem{FACT}[THEO]{Fact}
\newtheorem{HYPO}[THEO]{Hypothesis}
\newtheorem{LEMM}[THEO]{Lemma}
\newtheorem{PROB}[THEO]{Problem}
\newtheorem{PROP}[THEO]{Proposition}
\newtheorem{REMA}[THEO]{Remark}
\newcommand{\theo}{\begin{THEO}}
\newcommand{\algo}{\begin{ALGo} \rm}
\newcommand{\cond}{\begin{COND}}
\newcommand{\conj}{\begin{CONJ}}
\newcommand{\coro}{\begin{CORO}}
\newcommand{\defi}{\begin{DEFI} \rm}
\newcommand{\examp}{\begin{EXAMP} \rm}
\newcommand{\fact}{\begin{FACT}}
\newcommand{\hypo}{\begin{HYPO} \rm}
\newcommand{\lemm}{\begin{LEMM}}
\newcommand{\prob}{\begin{PROB} \rm}
\newcommand{\prop}{\begin{PROP}}
\newcommand{\rema}{\begin{REMA} \rm}
\newcommand{\etheo}{\end{THEO}}
\newcommand{\ealgo}{\end{ALGo}}
\newcommand{\econd}{\end{COND}}
\newcommand{\econj}{\end{CONJ}}
\newcommand{\ecoro}{\end{CORO}}
\newcommand{\edefi}{\end{DEFI}}
\newcommand{\eexamp}{\end{EXAMP}}
\newcommand{\efact}{\end{FACT}}
\newcommand{\ehypo}{\end{HYPO}}
\newcommand{\elemm}{\end{LEMM}}
\newcommand{\eprob}{\end{PROB}}
\newcommand{\eprop}{\end{PROP}}
\newcommand{\erema}{\end{REMA}}
\def\br{\hfill\break}
\def\mAth{\mathsurround=0pt}
\def\eqalign#1{\,\vcenter{\openup1\jot \mAth
   \ialign{\strut\hfil$\displaystyle{##}$&$\displaystyle{{}##}$\hfil
     \crcr#1\crcr}}\,}

%
%file name = bflatex.tex
%
\def\0{\mbox{\bf 0}}
\def\1{\mbox{\bf 1}}
\def\2{\mbox{\bf 2}}
\def\3{\mbox{\bf 3}}
\def\4{\mbox{\bf 4}}
\def\5{\mbox{\bf 5}}
\def\6{\mbox{\bf 6}}
\def\7{\mbox{\bf 7}}
\def\8{\mbox{\bf 8}}
\def\9{\mbox{\bf 9}}
\def\a{\mbox{\boldmath $a$}}
\def\b{\mbox{\boldmath $b$}}
%\def\c{\mbox{\boldmath $c$}}
\def\cc{\mbox{\boldmath $c$}}
\def\d{\mbox{\boldmath $d$}}
\def\e{\mbox{\boldmath $e$}}
\def\f{\mbox{\boldmath $f$}}
\def\g{\mbox{\boldmath $g$}}
\def\h{\mbox{\boldmath $h$}}
\def\i{\mbox{\boldmath $i$}}
\def\j{\mbox{\boldmath $j$}}
\def\k{\mbox{\boldmath $k$}}
\def\l{\mbox{\boldmath $l$}}
\def\m{\mbox{\boldmath $m$}}
\def\n{\mbox{\boldmath $n$}}
\def\o{\mbox{\boldmath $o$}}
\def\p{\mbox{\boldmath $p$}}
\def\q{\mbox{\boldmath $q$}}
\def\r{\mbox{\boldmath $r$}}
\def\s{\mbox{\boldmath $s$}}
\def\t{\mbox{\boldmath $t$}}
\def\u{\mbox{\boldmath $u$}}
\def\v{\mbox{\boldmath $v$}}
\def\w{\mbox{\boldmath $w$}}
\def\x{\mbox{\boldmath $x$}}
\def\y{\mbox{\boldmath $y$}}
\def\z{\mbox{\boldmath $z$}}
\def\A{\mbox{\boldmath $A$}}
\def\B{\mbox{\boldmath $B$}}
\def\C{\mbox{\boldmath $C$}}
\def\D{\mbox{\boldmath $D$}}
\def\E{\mbox{\boldmath $E$}}
\def\F{\mbox{\boldmath $F$}}
\def\G{\mbox{\boldmath $G$}}
\def\H{\mbox{\boldmath $H$}}
\def\I{\mbox{\boldmath $I$}}
\def\J{\mbox{\boldmath $J$}}
\def\K{\mbox{\boldmath $K$}}
\def\L{\mbox{\boldmath $L$}}
\def\M{\mbox{\boldmath $M$}}
\def\N{\mbox{\boldmath $N$}}
\def\O{\mbox{\boldmath $O$}}
\def\P{\mbox{\boldmath $P$}}
\def\Q{\mbox{\boldmath $Q$}}
\def\R{\mbox{\boldmath $R$}}
\def\S{\mbox{\boldmath $S$}}
\def\T{\mbox{\boldmath $T$}}
\def\U{\mbox{\boldmath $U$}}
\def\V{\mbox{\boldmath $V$}}
\def\W{\mbox{\boldmath $W$}}
\def\X{\mbox{\boldmath $X$}}
\def\Y{\mbox{\boldmath $Y$}}
\def\Z{\mbox{\boldmath $Z$}}
\def\AC{\mbox{$\cal A$}}
\def\BC{\mbox{$\cal B$}}
\def\CC{\mbox{$\cal C$}}
\def\DC{\mbox{$\cal D$}}
\def\EC{\mbox{$\cal E$}}
\def\FC{\mbox{$\cal F$}}
\def\GC{\mbox{$\cal G$}}
\def\HC{\mbox{$\cal H$}}
\def\IC{\mbox{$\cal I$}}
\def\JC{\mbox{$\cal J$}}
\def\KC{\mbox{$\cal K$}}
\def\LC{\mbox{$\cal L$}}
\def\MC{\mbox{$\cal M$}}
\def\NC{\mbox{$\cal N$}}
\def\OC{\mbox{$\cal O$}}
\def\PC{\mbox{$\cal P$}}
\def\QC{\mbox{$\cal Q$}}
\def\RC{\mbox{$\cal R$}}
\def\SC{\mbox{$\cal S$}}
\def\TC{\mbox{$\cal T$}}
\def\UC{\mbox{$\cal U$}}
\def\VC{\mbox{$\cal V$}}
\def\WC{\mbox{$\cal W$}}
\def\XC{\mbox{$\cal X$}}
\def\YC{\mbox{$\cal Y$}}
\def\ZC{\mbox{$\cal Z$}}
\def\ACs{\mbox{\tiny $\cal A$}}
\def\BCs{\mbox{\tiny $\cal B$}}
\def\CCs{\mbox{\tiny $\cal C$}}
\def\DCs{\mbox{\tiny $\cal D$}}
\def\ECs{\mbox{\tiny $\cal E$}}
\def\FCs{\mbox{\tiny $\cal F$}}
\def\GCs{\mbox{\tiny $\cal G$}}
\def\HCs{\mbox{\tiny $\cal H$}}
\def\ICs{\mbox{\tiny $\cal I$}}
\def\JCs{\mbox{\tiny $\cal J$}}
\def\KCs{\mbox{\tiny $\cal K$}}
\def\LCs{\mbox{\tiny $\cal L$}}
\def\MCs{\mbox{\tiny $\cal M$}}
\def\NCs{\mbox{\tiny $\cal N$}}
\def\OCs{\mbox{\tiny $\cal O$}}
\def\PCs{\mbox{\tiny $\cal P$}}
\def\QCs{\mbox{\tiny $\cal Q$}}
\def\RCs{\mbox{\tiny $\cal R$}}
\def\SCs{\mbox{\tiny $\cal S$}}
\def\TCs{\mbox{\tiny $\cal T$}}
\def\UCs{\mbox{\tiny $\cal U$}}
\def\VCs{\mbox{\tiny $\cal V$}}
\def\WCs{\mbox{\tiny $\cal W$}}
\def\XCs{\mbox{\tiny $\cal X$}}
\def\YCs{\mbox{\tiny $\cal Y$}}
\def\ZCs{\mbox{\tiny $\cal Z$}}

\def\inprod#1#2{\langle#1, \, #2\rangle}

\def\Real{\mbox{$\mathbb{R}$}}
\def\Integer{\mbox{$\mathbb{Z}$}}
\def\balpha{\mbox{\boldmath $\alpha$}}
\def\bbeta{\mbox{\boldmath $\beta$}}
\def\bgamma{\mbox{\boldmath $\gamma$}}

\def\bdelta{\mbox{\boldmath $\delta$}}

\def\bepsilon{\mbox{\boldmath $\epsilon$}}

\def\salpha{\mbox{\scriptsize \boldmath $\alpha$}}
\def\sbeta{\mbox{\scriptsize \boldmath $\beta$}}
\def\sgamma{\mbox{\scriptsize \boldmath $\gamma$}}
\def\s0{\mbox{\scriptsize \boldmath $0$}}

\def\sdelta{\mbox{\scriptsize \boldmath $\delta$}}

\def\sepsilon{\mbox{\scriptsize \boldmath $\epsilon$}}


\def\blambda{\mbox{\boldmath $\lambda$}}
\def\bLambda{\mbox{\boldmath $\Lambda$}}
\def\hLambda{\mbox{$\widehat{\bLambda}$}}
\def\bmu{\mbox{\boldmath $\mu$}}
\def\bvarphi{\mbox{\boldmath $\varphi$}}
\def\bpsi{\mbox{\boldmath $\psi$}}
\def\FCB{\mbox{\boldmath $\cal F$}}
\def\barSigma{\mbox{$\overline{\Sigma}$}}
\def\barAC{\mbox{$\overline{\AC}$}}
\def\tAC{\mbox{$\widetilde{\AC}$}}
\def\tBC{\mbox{$\widetilde{\BC}$}}
\def\tG{\mbox{$\widetilde{\G}$}}
\def\tGC{\mbox{$\widetilde{\GC}$}}
\def\sGC{\mbox{\scriptsize ${\GC}$}}
\def\hGC{\mbox{$\widehat{\GC}$}}
\def\tV{\mbox{$\widetilde{V}$}}
\def\hV{\mbox{$\widehat{V}$}}
\def\tbV{\mbox{$\widetilde{\V}$}}
\def\hbV{\mbox{$\widehat{\V}$}}
\def\tw{\mbox{$\widetilde{w}$}}
\def\hw{\mbox{$\widehat{w}$}}
\def\bbW{\mbox{$\overline{\W}$}}
\def\boldSigma{\mbox{\boldmath $\Sigma$}}
\def\boldXi{\mbox{\boldmath $\Xi$}}
\def\OmegaD{\mbox{$\AC \times \AC$}}
\def\sOmegaD{\mbox{\scriptsize $\AC \times \AC$}}
\def\Ibinary{\mbox{$I_{\mbox{\scriptsize bin}}$}}
\def\Ibox{\mbox{$I_{\mbox{\scriptsize box}}$}}


\def\Real{\mathbb{R}}
\def\coneK{\mathbb{K}}
\def\coneJ{\mathbb{J}}
\def\spaceL{\mathbb{L}}
\def\spaceM{\mathbb{M}}
\def\spaceV{\mathbb{V}}
\def\SymMat{\mathbb{S}}
\def\SymC{\mathbb{C}}
\def\SymN{\mathbb{N}}
\def\Integer{\mathbb{Z}}


% \def\IM{\mbox{$\I \hspace{-0.6mm}$mat}}
\def\IM{\mbox{$\A$}}
\def\barIM{\mbox{$\overline{\A}$}}
\def\barG{\mbox{$\overline{G}$}}
\def\barE{\mbox{$\overline{E}$}}

% \def\GMS{GAMS-like }
\def\GMS{GAMS scalar }

\def\matBP{BBCPOP}

%----------------------------------------------------------------

%\reporttitle{User Manual for {\bf BP}: a {\bf B}isection  and   {\bf P}rojection Method
%	 \\
%	for Polynomial Optimization Problems with Binary and Box Constraint }
%\reportauthors{Naoki Ito, Sunyoung Kim, \\ Masakazu Kojima, Akiko Takeda \\
%and Kim-Chuan Toh}
%\reportnumber{4??}
%\publishmonthyear{February}{2018}

\begin{document}


\thispagestyle{empty}

\noindent
%\parbox[t]{1.2cm}{B-490}
\parbox[t]{16.5cm}
{ 
User Manual for {\bf \matBP}:  
A Sparse Doubly Nonnegative Relaxation
of {\bf P}olynomial {\bf O}ptimization {\bf P}roblems
with {\bf B}inary, {\bf B}ox and {\bf C}omplementarity Constraints
 \\~\\
$\mbox{Naoki Ito}^{\star}$,
$\mbox{Sunyoung Kim}^{\dagger}$,
$\mbox{Masakazu Kojima}^{\ddagger}$, 
$\mbox{Akiko Takeda}^{\mathsection}$, 
$\mbox{Kim-Chuan Toh}^{\mathparagraph}$ \\
 \begin{center}
March 2018
\end{center}
} 

\vspace{1cm}


\begin{abstract}  
\noindent
\matBP\ proposed in \cite{ITO2018} is a MATLAB implementation of a hierarchy of sparse doubly nonnegative (DNN) relaxations 
of a class of polynomial optimization (minimization) problems (POPs) with binary, box and complementarity constraints. 
Given a POP in the class and a relaxation order (or a hierarchy level), 
\matBP\ constructs a simple conic optimization problem (COP), 
which serves as a DNN relaxation of the POP, and then solves the COP by 
applying the bisection and projection (BP) method \cite{KIM2013,KIM2016}.
The software package {\bf \matBP}, this manual, and
a  test set of POPs are available at https://sites.google.com/site/bbcpop1/.
\end{abstract}


%\vspace{1cm}
  
\noindent
{\bf Key words. } % \vspace{0.1cm} \\
Polynomial optimization problems,  Doubly nonnegative relaxation,
Bisection and projection method, Large-scale problems,
MATLAB software package.

\noindent
{\bf AMS Classification. } 
% 90C20, 90C25, 90C26
90C20,  	%Quadratic programming
90C22,  	%Semidefinite programming
90C25, 	%Convex programming
90C26.  	%Nonconvex programming, global optimization


\vspace{1cm}


\noindent
\parbox[t]{0.5cm}{$\star$}
\parbox[t]{14.9cm}{Department of Mathematical Informatics,
        			The University of Tokyo, Tokyo 113-8656, Japan. 
        			This work was supported by Grant-in-Aid for JSPS Research Fellowship JP17J07365.
			({\tt naoki\_ito{@}mist.i.u-tokyo.ac.jp}).
}

\medskip

\noindent
\parbox[t]{0.5cm}{$\dagger$}
\parbox[t]{14.9cm}{Department of Mathematics, Ewha W. University,
52 Ewhayeodae-gil, Sudaemoon-gu, Seoul 03760 Korea. 
 The research was supported
by NRF 2017-R1A2B2005119. 
({\tt skim@ewha.ac.kr}).
}

\medskip

\noindent
\parbox[t]{0.5cm}{$\ddagger$}
\parbox[t]{14.9cm}{Department of Industrial and Systems Engineering,
Chuo University, Tokyo 112-8551 Japan.
This research was supported by Grant-in-Aid for Scientific Research (A) 26242027.
({\tt kojimamasakazu@mac.com}).
}


\medskip

\noindent
\parbox[t]{0.5cm}{$\mathsection$}
\parbox[t]{14.9cm}{ Department of Mathematical Analysis and Statistical Inference, 
The Institute of Statistical Mathematics, 
10-3 Midori-cho, Tachikawa, Tokyo 190-8562, Japan
           The work of this author was supported by Grant-in-Aid for Scientific Research (C), 15K00031.
    ({\tt atakeda{@}ism.ac.jp}).
}

\medskip

\noindent
\parbox[t]{0.5cm}{$\mathparagraph$}
\parbox[t]{14.9cm}{Department of Mathematics and Institute of Operations Research and Analytics, 
	National University of
         Singapore, 10 Lower Kent Ridge Road, Singapore 119076.  
 ({\tt mattohkc@nus.edu.sg}).
}

\newpage



%\input{sect1.tex}
\section{Introduction}

\matBP\ proposed in \cite{ITO2018} is a MATLAB software package to compute tight lower bounds for 
global optimal values of a class of polynomial optimization (minimization) problems (POPs) 
with binary, box and complementarity constraints. 
As shown in Figure~\ref{fig:structure}, \matBP.m constructs a hierarchy of sparse DNN relaxations of a POP in the class 
and solves them by applying the bisection and projection method 
\cite{KIM2013,ARIMA2017,KIM2016} and the accelerated proximal gradient (APG) method \cite{BECK2009}. 
These two methods were described as BP Algorithm and APGR Algorithm (an enhanced version of the APG method) in \cite{ITO2018}, respectively. 

\begin{figure}[hbt]
\label{fig:structure}
\begin{center}
\includegraphics[width=13cm,height=5.cm]{BBCPOPstructure.pdf} 
\caption{The structure of the main function BBCPOP.m}
\end{center}
\end{figure}

Let $f_0$ be a real valued polynomial  defined on the $n$-dimensional Euclidean space $\Real^n$, 
$\Ibinary$ and $\Ibox$ a partition of $N\equiv\{1,2,\ldots,n\}$, {\it i.e.}, $\Ibinary\cup\Ibox = N$ and 
$\Ibinary\cap\Ibox=\emptyset$, and $\CC$ a family of subsets of $N$. 
Each POP in the class is described as
\begin{eqnarray}
\zeta^* & = & \min_{\x} \left\{ f_0(\x) \Big|
\begin{array}{l}
x_i\in\{0,1\} \ (i\in\Ibinary) \ \mbox{(box constraint)}, \\
x_j\in [0,1] \ (j\in\Ibox) \ \mbox{(binary constraint)}, \\
\displaystyle \prod_{j\in C} x_j = 0 \ (C \in \CC) \ \mbox{(complementarity constraint)} 
\end{array}
 \right\}. \label{eq:POP0} %\nonumber \\ 
%& = &   \min_{\x} \{ f_0(\x) \mid \x \in H \}, 
%  \label{eq:POP0} 
\end{eqnarray}

As an illustrative example, we consider the following POP with 
five variables $ x_1,x_2, x_3, x_4$ and $x_5$: 
 \begin{equation}
\left. 
 \begin{array}{ll}
	\mbox{minimize} & f_0(\x) \equiv 0.5x_1 -1.8x_3 -2.2x_5 +3x_3^2 +x_1x_2x_4 + 1.3x_2x_4x_5      \\
	\mbox{subject to} &  x_2x_3 = 0,\  x_3x_4 = 0, \  x_1, x_2 \in \{0,1\}, \  x_3, x_4, x_5 \in [0,1]  
	\end{array}
\right\} 
\label{ex:EX1}
\end{equation}
In this case, $\Ibinary = \{1,2\}$, $\Ibox = \{3,4,5\}$ and $\CC = \left\{ \{2,3\},\{3,4\} \right\}$.  
We can easily compute the optimal solution $\x^*=(0,0,0.3,0,1)$ and the optimal value $\zeta^*=-2.47$. 

The input of \matBP.m consists of the data for POP \eqref{eq:POP0}, the relaxation order $\omega$ which determines 
the hierarchy level of DNN relaxation to be constructed, and parameters which control the execution of 
\matBP.m. With the input data, \matBP.m constructs a simple conic optimization problem (COP):
\begin{eqnarray}
  y_0^* & = &  \max_{\scriptsize y_0,\Y_1, \Y_2}\{y_0 \mid  \Q_0 - y_0 \H_0 = \Y_1+ \Y_2, \ \Y_1 \in \coneK_1^*, 
\ \Y_2 \in \coneK_2^*\}
\label{eq:generalDualCOP0}, % \\
\end{eqnarray}
which serves as a DNN relaxation of POP~\eqref{eq:POP0} (hence $y_0^* \leq \zeta^*$). Here $\H_0$ denotes 
a constant vector in a linear space $\spaceV$ ($=$ the Cartesian product of symmetric matrix spaces) 
endowed with an inner product, % $\inprod{\cdot}{\cdot}$, 
$\coneK_1, \coneK_2 \subset \spaceV$ closed convex cones, and $y_0 \in\Real$, $\Y_1 \in \spaceV, \ \Y_2 \in \spaceV$ 
variables. The output of  \matBP.m is
 an approximate optimal solution $(y_0,\Y_1,\Y_2)$ of~\eqref{eq:generalDualCOP0} such that 
$y_0 \leq y_0^*$; hence $y_0 \leq \zeta^*$ is guaranteed. 

The degree of POP~\eqref{eq:POP0}  is defined as 
$ %\begin{eqnarray*}
\max\{ \mbox{deg}f_0, \ |C| \ (C \in \CC) \}, 
$ %\end{eqnarray*}
where $|C|$ denotes the number of elements in $C$ $(C \in \CC)$. 
The relaxation order $\omega$ needs to be a positive integer not less than 
the half of the degree of POP~\eqref{eq:POP0}. %\mbox{deg(POP\eqref{eq:POP0})}. 
Hence the minimum relaxation order that can be taken is  % given by 
\begin{eqnarray*}
\omega_{\min} = \mbox{the smallest positive integer not less than the half of the degree of POP~\eqref{eq:POP0}}. 
\end{eqnarray*} 
%In the case of POP~\eqref{ex:EX1}, $\omega_{\min}=2$. 
As we take a larger $\omega$, we can expect a tighter lower bound 
$y_0$ ($=$ the approximate optimal value of \eqref{eq:generalDualCOP0})  for $\zeta^*$, but 
COP~\eqref{eq:generalDualCOP0} to be solved by the BP Algorithm becomes larger, so that 
longer execution time is required. 
For many application problems in practice, taking $\omega_{\min}$ for the relaxation order $\omega$ is sufficient 
to obtain a tight lower bound for $\zeta^*$. 

In the above example~\eqref{ex:EX1}, we see that deg$f_0 = 3$, $|\{2,3\}| = 2$ and $|\{3,4\}| = 2$. 
Hence $\omega_{\min}= 2 \leq \omega$. 

In Section 2, we present how to describe POP \eqref{eq:POP0} for the input of the main function \matBP.m. In Section 3, 
we illustrate the execution of \matBP.m and present the details on the output of \matBP.m. Section 4 lists the parameters 
which control the execution of \matBP.m, and Section 5 some main functions contained in the \matBP\ software package.


\section{Representing polynomial optimization problems}

\label{Representation}

POP~\eqref{eq:POP0} is described by
\begin{eqnarray*}
{\sf objPoly}, \ {\sf I01} \ \mbox{ and } {\sf Icomp},
\end{eqnarray*}
which are input arguments of  \matBP.m. 
Let term$f_0$ denote the number of terms of $f_0$. 
We employ a simplified SparsePOP format to describe the objective function of POP~\eqref{eq:POP0}: 
\begin{center}
\begin{tabular}{rcll}
{\sf objPoly.supports}		& = & a set of supports of $f_0(\x)$, % \\
				% &	& & 
\ term$f_0$ $\times n$ matrix. \\
{\sf objPoly.coef}			& = & coefficients, % \\ 
				% &	& & a 
\ column vector of dimension term$f_0$. \\
\end{tabular}
\end{center}
For the original SparsePOP format, see \cite{SPARSEPOP_UG}. 
The functions simplifyPoly.m, 
addPoly.m and multiplyPoly.m in the directory polyTools/  can be used  % are useful to 
when describing an objective function $f_0(\x)$  % in the form of \eqref{POP2}
in the simplified SparsePOP format. 


The partition $\Ibinary\cup\Ibox$ of $N\equiv\{1,2,\ldots,n\}$ is described by the $n$-dimensional 
row vector  {\sf I01} such that 
\begin{eqnarray*}
 {\sf I01}_j = \left\{ \begin{array}{ll}
{\sf true}  & \mbox{if } j \in \Ibinary, \\ 
{\sf false} & \mbox{otherwise, {\it i.e.},  }  j \in \Ibox. 
\end{array} \right. 
\end{eqnarray*}
The family $\CC$ of subsets of $N$, which represents the complementarity condition in POP~\eqref{eq:POP0}, 
is represented by  {\sf Icomp}. Suppose that $\CC$ consists of $m$ nonempty subsets $C_1,\ldots,C_m$ of $N$. 
Then we set {\sf Icomp} to be the $m \times n$ matrix such that 
\begin{eqnarray*}
{\sf Icomp}_{ij} & = &
 \left\{ \begin{array}{ll}
{\sf true} & \mbox{if } j \in C_i, \\ 
{\sf false} & \mbox{otherwise, {\it i.e.},  }  j \not\in C_i
\end{array} \right. 
\end{eqnarray*}
$(i=1,\ldots,m, \ j=1,\ldots,n)$. If $\CC = \emptyset$, set ${\sf Icomp} = [\ ]$. 

The three input arguments {\sf objPoly},  {\sf I01} \ \mbox{ and } {\sf Icomp} for POP~\eqref{ex:EX1} 
are described as follows:
\begin{verbatim}
function [objPoly, I01, Icomp] = example1;
       objPoly.supports   = ...
                         [1 0 0 0 0;
                          0 0 1 0 0;
                          0 0 0 0 1;
                          0 0 2 0 0;
                          1 1 0 1 0;
                          0 1 0 1 1];
         objPoly.coef = [0.5; -1.8; -2.2; 3; 1; 1.3];	 
         Icomp = logical([0 1 1 0 0; 0 0 1 1 0]);
         I01 = logical([1 1 0 0 0]);
end 
\end{verbatim}

\section{Executing \matBP}

\label{sample}

A lower bound for the optimal value of POP~\eqref{ex:EX1} can be computed by \matBP.m\ as follows:
\begin{verbatim}
>>[objPoly, I01, Icomp] = example1;
>>relaxOrder = 2; params = [];
>>[sol, info] = BBCPOP(objPoly, I01, Icomp, relaxOrder, params);
\end{verbatim}

The following is shown on the screen  as the \matBP.m terminates at 17 iterations.



\begin{verbatim}
Original
iter=17:y0=-2.469903e+00,LBv=-2.470066e+00,[LB,UB]=[-2.470066e+00,-2.469903e+00]
Scaled
iter=17:y0=-9.058401e-01,LBv=-9.058999e-01,[LB,UB]=[-9.058999e-01,-9.058401e-01]

     LBv           y0            UB           UB-LBv   relnormX iter b_yes
 0  -1.000000e+21 +5.800000e+00 +5.800000e+00 9.80e+00 
 1  -4.536533e+01 +5.800000e+00 +5.800000e+00 9.80e+00 8.59e-01  101 90
 2  -1.867610e+01 +9.000000e-01 +9.000000e-01 4.90e+00 4.31e-01   56 1
 3  -6.974229e+00 -1.550000e+00 -1.550000e+00 2.45e+00 9.58e-02  218 90
 4  -2.775000e+00 -2.775000e+00 -1.550000e+00 1.22e+00 0.00e+00   17 2
 5  -2.775000e+00 -2.162500e+00 -2.162500e+00 6.12e-01 3.06e-02  301 88
 6  -2.476112e+00 -2.468750e+00 -2.162500e+00 3.14e-01 1.21e-04  350 88
 7  -2.476112e+00 -2.319306e+00 -2.319306e+00 1.57e-01 1.48e-02  274 90
 8  -2.476112e+00 -2.397709e+00 -2.397709e+00 7.84e-02 7.06e-03  180 90
 9  -2.476112e+00 -2.436911e+00 -2.436911e+00 3.92e-02 3.22e-03  301 88
10  -2.476112e+00 -2.456512e+00 -2.456512e+00 1.96e-02 1.31e-03  294 90
11  -2.476112e+00 -2.466312e+00 -2.466312e+00 9.80e-03 3.58e-04  301 88
12  -2.471212e+00 -2.471212e+00 -2.466312e+00 4.90e-03 0.00e+00   10 2
13  -2.471212e+00 -2.468762e+00 -2.468762e+00 2.45e-03 1.20e-04  301 88
14  -2.470066e+00 -2.469987e+00 -2.468762e+00 1.30e-03 1.29e-06  339 97
15  -2.470066e+00 -2.469414e+00 -2.469414e+00 6.52e-04 5.69e-05  339 97
16  -2.470066e+00 -2.469740e+00 -2.469740e+00 3.26e-04 2.53e-05  339 97
17  -2.470066e+00 -2.469903e+00 -2.469903e+00 1.63e-04 9.45e-06  301 99
 timeBP (Excecution time) = 1.98,  termcodeBP = 2
\end{verbatim}
The value  $ -2.470039e$+$00$ is the approximate optimal value of \eqref{ex:EX1} obtained by BP Algorithm, which is a valid lower bound 
of the optimal value $\zeta^* = -2.47$ of POP~\eqref{ex:EX1}. 

The output {\sf sol} is a structure containing the solution information for~\eqref{eq:generalDualCOP0}:
\begin{verbatim}
    y0init: 2.1272
       LBv: -2.4700
        LB: -2.4700
        UB: -2.4700
        Y1: [1×226 double]
        Y2: [1×226 double]
\end{verbatim}
Here {\sf y0init} corresponds to the initial value of $y_0^m$,  and   
{\sf LBv, LB, UB, Y1, Y2} to the terminal values of $y^{\ell v}_0$, $y^{\ell}$, $y^u_0$, 
$\hat{\Y}_1$ and $\hat{\Y}_2$, respectively,  in BP Algorithm in \cite{ITO2018}.
The output {\sf info} is a structure  
for some of the execution information of BP.m (BP Algorithm) and APGR.m 
(APGR Algorithm):
\begin{verbatim}
        iterBP: 17
        timeBP: 1.5884
    termcodeBP: 2
      iterAPGR: 3545
\end{verbatim}
Here {\sf iterBP} denotes the number of iteration of BP.m, {\sf timeBP} its execution time, 
{\sf termcodeBP} its termination code and {\sf iterAPGR} the total number of iterations of APGR.m. 
BP.m stops when 
\begin{description}
\item{(i) } the difference of the upper bound $y_0^u$ and the lower bound $y_0^\ell$ becomes 
 smaller than the prescribed parameter 
{\sf params.delta},
\item{(ii) } $(y_0^u-y_0^\ell)/\max\{1.0,|y_0^\ell|, |y_0^u|\}$ is smaller than the prescribed parameter
{\sf params.delta2}
\item{(iii) } the reduction of the length of the interval $[y_0^u-y_0^\ell]$ gets smaller than 
$\max\{{\sf params.delta},$ $ {\sf params.delta2}\}$.
\item{(iv) } the iteration of BP.m exceeds the prescribed parameter {\sf params.maxiterBP}, or 
\item{(v) } the execution time exceeds the prescribed parameter {\sf params.maxtimeBP}. 
\end{description}
These terminations are associated with the termination code 1,2,3,0,-1, respectively. 



\section{Parameters}
\label{PARAM}

In addition to {\sf objPoly}, {\sf I01}, {\sf Icomp} and {\sf relaxOrder} for  describing a POP,  
% in the SparsePOP format, 
the MATLAB function \matBP.m has {\sf params} as an input argument. It is a structure consisting of 
many parameters that control the performance of the function. Table \ref{tableOfparam} 
shows the list of parameters used in \matBP.m. The default values of parameters are given in the 
MATLAB function defaultParamBP.m. They can be modified if necessary. 

\begin{table}[htp]
\caption{The fields of {\sf params}, default values and possible values}
\begin{center}
\begin{tabular}{|l|l|l|} \hline
field of {\sf params} & default &  usage, possible values\\
\hline
\multicolumn{3}{|c|}{Parameters for DNN relaxation} \\ 
\hline 
{\sf sparseSW}   & 1 & Set 0 for dense DNN relaxation. \\
                           &    & Set 1 for sparse DNN relaxation. \\
\hline
\multicolumn{3}{|c|}{Parameters for BP and APGR Algorithms} \\ 
\hline
{\sf maxtimeBP }  & 20000 & the maximum execution time\\
\hline
{\sf maxiterBP} & 40 & the maximum iteration  for BP Algorithm \\
\hline
{\sf maxiterAPGR}  & 20000 & the maximum iteration for APGR Algorithm \\
\hline
{\sf delta1}  & 1e-4 & the relative tolerance for BP Algorithm\\
                             &    & Set $0$ if {\sf delta} is used. \\
\hline
{\sf delta}  & 0 & the absolute tolerance  for BP Algorithm \\
                         &    & Set  ${\sf delta} \in (0,1)$ if $\zeta^*$ is integer. \\
                         &    & Set $0$ otherwise. \\
\hline
{\sf printyes}  & 2 & print level, 0,1,2 or 3 \\  
\hline                        
{\sf UbdObjVal}   & $f_0(\0)$   & the upper bound %\\
%                  &             & 
for the optimal value $\zeta^*$ of POP~\eqref{eq:POP0} \\
\hline
{\sf UbdIX}          &         & $= \rho$ given in Assumption (A1) of \cite{ITO2018}.\\
                     &         & To specify this parameter, see Sections 2.3 and 4.1 of \cite{ITO2018}. \\
\hline
\end{tabular}
\end{center}
\label{tableOfparam}
\end{table}


%% Skim 5/15/07
\section{Description of main and principal subfunctions}
%The main MATLAB functions and important subfunctions}
\label{mainFunctions}

The main function and principal subfunctions are described in terms of input and output
arguments in this section. 

\subsection{The MATLAB functions sparsePOP.m, SDPrelaxation.m, and SDPrelaxationMex.m} 


The main function sparsePOP.m, its principal subfunctions SDPrelaxation.m 
% Skim 5/15/07
and SDPrelaxationMex.m  shown in Figure~\ref{structure} 
have the following function declarations: 
\begin{verbatim}
function [param,SDPobjValue,POP,cpuTime,SeDuMiInfo,SDPinfo] = ...
    sparsePOP(objPoly,ineqPolySys,lbd,ubd,param); 

function [param,SDPobjValue,POP,cpuTime,SeDuMiInfo,SDPinfo] = ...
    SDPrelaxation(param,objPoly,ineqPolySys,lbd,ubd);

function [param,SDPobjValue,POP,cpuTime,SeDuMiInfo,SDPinfo] = ...
    SDPrelaxationMex(param,objPoly,ineqPolySys,lbd,ubd);
\end{verbatim}
respectively. These three functions have the same input and output 
arguments. Among the input arguments, {\sf param} contains a set of parameters
% Skim 5/15/07 
% Kojima 5/18/07 
% for choosing certain algorithmic procedures. % their behavior.
% More 
whose detailed description is included in Section \ref{PARAM}. 
The other input arguments, if all of them are specified,   describe a POP 
in the SparsePOP format, as presented in Section~4.2. 

Although sparsePOP.m is defined with 5 input arguments, using 1 or 2 input arguments
is also possible as mentioned in Section~\ref{sample}. 
%the following usages with 1 or 2 input arguments are also possible: 
\begin{itemize}
\item ${\bf >>}$ {\sf sparsePOP('example1.gms')} for solving a POP described in the GAMS scalar format 
with the default {\sf param}. 
\item ${\bf >>}$ {\sf sparsePOP('example1.gms',param)} for solving a POP described in the GAMS scalar 
format with the user-specified {\sf param}. 
\item ${\bf >>}$ {\sf sparsePOP('example1')} for solving a POP described in the SparsePOP  format with 
the default {\sf param}. 
\item ${\bf >>}$ {\sf sparsePOP('example1',param)} for solving a POP described in the SparsePOP format 
with the user-specified {\sf param}. 
\end{itemize}
%These have been illustrated in Section~\ref{sample}. 

If the SDPrelaxation.m or the SDPrelaxationMex.m is to be utilized directly,
%On the other hand,
 either a set of the 5 input arguments 
\[
\mbox{ {\sf param}, {\sf objPoly}, {\sf ineqPolySys}, {\sf lbd} and {\sf ubd} } 
\]
or a set of the 3 input arguments 
\[
\mbox{ {\sf param}, {\sf objPoly} and {\sf ineqPolySys}} 
\]
needs to be specified. %if the user utilize the sparsePOPmain.m or the sparsePOPmainMex.m directly. 
%In the latter  3 input arguments case, 
In the case that 3 input arguments are prescribed,
the functions SDPrelaxation.m and SDPrelaxationMex.m assign the default values 
% Skim 5/24/07
{\sf lbd}$(i) = -1.0$e+$10$ and  {\sf ubd}$(i) = +1.0$e+$10$ $(i=1,2,\ldots,n)$, which implies that 
$-\infty < x_i < \infty$ $(i=1,2,\ldots,n)$.  

For the output arguments, user-specified or default values for the parameters are stored  in {\sf param}.
{\sf SDPobjValue} contains a lower bound for the optimal objective value of the POP (\ref{POP0}). 
% Skim 5/15/07
\mbox{For every feasible solution $\x$ of the POP (\ref{POP0})},
\begin{eqnarray}
\mbox{{\sf SDPobjValue}}  \leq f_0(\x) \label{lowerBound} 
\end{eqnarray}
holds. %\mbox{for every feasible solution $\x$ of (\ref{POP0})}. 
The output argument {\sf POP} has four components: 
\begin{itemize}
\item {\sf POP.xVect}: a candidate $\x^{\omega}$ of an optimal solution of  the POP (\ref{POP0}). 
\item {\sf POP.objValue}: 
the objective function value $f_0(\x^{\omega})$ at $\x^{\omega} = $ {\sf POP.xVect}.
\item {\sf POP.absError}: an  absolute feasibility error at  $\x^{\omega}$. 
\item {\sf POP.scaledError}: a scaled feasibility error $\x^{\omega}$. 
\end{itemize}
(Recall that the relaxation order $\omega = $ {\sf param.relaxOrder} determines the 
quality of the SDP relaxation of the POP). % to be solved). 
Here the absolute feasibility error 
at $\x^{\omega}$ is given by 
\begin{eqnarray*}
\min \left\{ \min\{ f_i(\x^{\omega}), \ 0 \} \ (i=1,2,\ldots,\ell), \ - |f_j(\x^{\omega})| \ (j=\ell+1,\ldots,m) \right\}, 
\end{eqnarray*}
and the scaled feasibility error  
is given by 
\begin{eqnarray*}
\min \left\{ \min\{ f_i(\x^{\omega})/\sigma_i(\x^{\omega}), \ 0 \} \ (i=1,2,\ldots,\ell), \ - |f_j(\x^{\omega})|/\sigma_j(\x^{\omega}) \ (j=\ell+1,\ldots,m) \right\}, 
\end{eqnarray*}
where $\sigma_i(\x^{\omega})$ denotes the maximum of the absolute values 
of all monomials 
of $f_i(\x)$ evaluated at $\x^{\omega}$ if the maximum is greater than $1$, 
or $\sigma_i(\x^{\omega}) = 1$ otherwise $(i=1,2,\ldots,m)$. 
Note that both errors are always nonpositive. 
The relative error in the objective value at $\x^{\omega}$ in the output of sparsePOP.m, 
which has been illustrated in Section 4,  is computed as 
\[
\mbox{{\sf   relative obj error}} = \frac{{\sf POP.objValue} - {\sf SDPobjValue}}{\max \{ 1, |{\sf POP.objValue}| \}}. 
\]
If {\sf POP.scaledError} $\leq 0$ is close to $0$, say $-1.0$e-$6 \leq $  {\sf POP.scaledError} $\leq 0$, 
we may regard that $\x^{\omega}$ is feasible approximately. If, in addition, 
{\sf   relative obj error} $ \geq 0$ is close to $0$, say $0 \leq $ {\sf   relative obj error} $\leq 1.0$e-$6$, 
% Skim 5/15/07 provides
$\x^{\omega}$ is an approximate optimal solution of the POP (\ref{POP0}). 

The output argument {\sf cpuTime} shows various cpu times consumed by 
% Skim 5/24/07
%the execution of
 sparsePOP.m:
\begin{itemize}
\item  {\sf cpuTime.conversion}: 
the cpu time consumed to convert the POP into its SDP relaxation. 
\item {\sf cpuTime.SeDuMi}: the cpu time consumed by SeDuMi to solve the SDP.
\item {\sf cpuTime.Total}: the cpu time for the entire process.
\end{itemize}

The output argument {\sf SDPinfo} has information of the SDP relaxation problem solved 
by SeDuMi.
\begin{itemize}
\item {\sf SDPinfo.rowSizeA}:
the number of rows  of the coefficient matrix $\A$ of the SDP.  
\item {\sf SDPinfo.colSizeA}:
the number of  columns of the coefficient matrix $\A$. % of the primal SDP 
\item {\sf SDPinfo.nonzeroInA}: the number of nonzeros of the 
coefficient matrix $A$.
\item {\sf SDPinfo.noOfLPvariables}:
the number of LP variables of the SDP.
\item {\sf SDPinfo.noOfFRvariables}: the number of free variables of the SDP. 
\item {\sf   SDPinfo.SDPblock}:
the row vector of the sizes of SDP blocks.
\end{itemize}

Finally, the output argument 
{\sf SeDuMiInfo} contains {\sf SeDuMiInfo.numerr}, {\sf SeDuMiInfo.pinf},
and {\sf SeDuMi.dinf}, which
are equivalent to info.numerr, info.pinf, and info.dinf in SeDuMi output.
See \cite{STRUM99} for the details.

\subsection{The MATLAB function readGMS.m} 

The MATLAB function readGMS.m has the following function declaration:
\begin{verbatim}
function [objPoly,ineqPolySys,lbd,ubd] = readGMS(fileName,symbolicMath);
\end{verbatim}
% Skim 5/15/07
The first argument {\sf fileName}, a string in MATLAB,
is the name of the file where a problem is described
in the \GMS format. It must have the extension .gms such as 'example1.gms'. 
The second input argument {\sf symbolicMath} is set to be $1$ by default, assuming that 
the Symbolic Math Tool is available. It should be set to $0$ if it is not available. 
The output of {\sf objPoly}, {\sf ineqPolySys}, {\sf lbd}, and {\sf ubd} 
is a POP data in  the SparsePOP format, and can be passed to SDPrelaxation.m
or SDPrelaxationMex.m.


\subsection{The MATLAB function printSolution.m} 

The function printSolution.m for printing the results has
the following function declaration.
\begin{verbatim}
function printSolution(fileId,printLevel,dataFileName,param,SDPobjValue,...
        POP,cpuTime,SeDuMiInfo,SDPinfo);
\end{verbatim}
The meaning of each input argument is as follows.
\begin{itemize}
\item {\sf fileId}: fileId where output is printed. If fileId is $1$,
then the result is displayed on the screen (i.e., the standard output).
If the result in a file is desired,
% Skim 5/15/07 
the file should be open in writable mode before specifying it in fileId.
\item {\sf printLevel}: a larger value of printLevel gives more detailed description
of the result. Default is $2$.
\item {\sf dataFileName}: the name of the problem solved.
\end{itemize}
The rest of the input arguments, i.e.,
{\sf param},
{\sf SDPobjValue},
{\sf POP},
{\sf cpuTime},
{\sf SeDuMiInfo}, and 
{\sf SDPinfo} should be the output of 
SDPrelaxation.m and SDPrelaxationMex.m.




\section{Some functions}

\label{mainFunctions}

We list some functions contained in the \matBP\ software package. Important subfunctions 
of \matBP.m are listed in 
Section~5.1, and functions for some test instances are listed in Section~5.2. 

\subsection{Main subfunctions of \matBP.m}



\begin{description}
\item{relaxation/BBCPOPtoDNN.m} constructs COP~\eqref{eq:generalDualCOP0}, 
which serves as a DNN relaxation of~\eqref{eq:POP0}. 
\item{solver/solveDNN.m } solves COP~\eqref{eq:generalDualCOP0} by applying BP Algorithm and APGR Algorithm.
\item{solver/BP/BP.m } is an implementation of BP Algorithm in \cite{ITO2018}.
\item{solver/BP/APGR.m} is an implementation of APGR Algorithm in \cite{ITO2018}.
\end{description}

\subsection{Functions for POP instances}

All the functions below output {\sf objPoly}, {\sf I01}, {\sf Icomp}, {\sf relaxOrder} $=\omega_{min}$ and {\sf params}, which 
can be used as input of \matBP.m.
\begin{description}
\item{instances/POPrandom/genPOPdense.m } generates a dense POP instance with binary, box and complementarity constraints.
\begin{verbatim}
>> degree=3; nDim=5; isBin=true; addComplement=false; 
>> [objPoly,I01,Icomp,relaxOrder,params] = ... 
        genPOPdense(degree,nDim,isBin,addComplement); 
\end{verbatim}
\item{instances/POPrandom/genPOParrow.m } generates a sparse POP instance (whose objective function $f_0(\x)$ has 
an arrow type sparsity pattern Hessian matrix) with binary, box and complementarity constraints.
\begin{verbatim}
>> degree=4; a=10; b=2; c=2; l=3; isBin=0; addComplement=true; 
>> [objPoly, I01, Icomp, relaxOrder, params] = ...
        genPOParrow(degree,a,b,c,l,isBin,addComplement);
\end{verbatim}
\item{instances/POPrandom/genPOPchordal.m } generates a sparse POP instance (whose objective function $f_0(\x)$ has 
a sparse Hessian matrix characterized by a chordal graph) with binary, box and complementarity constraints.
\begin{verbatim}
>> degree=3; nDim=100; radiorange=0.1; isBin=1; addComplement=false;
>> [objPoly, I01, Icomp, relaxOrder, params] = ...
        genPOPchordal(degree,nDim,radiorange,isBin,addComplement);
\end{verbatim}
\item{instances/QAP/qapreadBP.m} generates a QOP with box and complementarity constraints which is induced from a Lagrangian relaxation of a QAP instance from QAPLIB \cite{QAPLIB}.
\begin{verbatim}
>> instance='chr12a'; lambda=10000; 
>> [objPoly, Icomp, I01, relaxOrder, params] = ...
        qapreadBP(instance,lambda);
\end{verbatim}
\item{instances/BIQ/biqreadBP.m} generates a QOP with box and complementarity constraints which is induced from a Lagrangian 
relaxation of a binary/box constrained QOP instance from  BIQMAC \cite{BIQMAC}.
\begin{verbatim}
>> instance='bqp100-1'; lambda=10000;
>> [objPoly, I01, Icomp, relaxOrder, params] = ...
        biqreadBP(instance,lambda);
\end{verbatim}
\end{description}




\bibliographystyle{plain}


\begin{thebibliography}{1}

\bibitem{ARIMA2017}
N.~Arima, S.~Kim, M.~Kojima, and K.C. Toh.
\newblock A robust \mbox{Lagrangian-DNN} method for a class of quadratic
  optimization problems.
\newblock {\em Comput. Optim. Appl.}, 66(3):453--479, 2017.

\bibitem{BECK2009}
A.~Beck and M.~Teboulle.
\newblock A fast iterative shrinkage-thresholding algorithm for linear inverse
  problems.
\newblock {\em SIAM J. Imaging Sci.}, 2:183--202, 2009.

\bibitem{QAPLIB}
P.~Hahn and M.~Anjos.
\newblock {QAPLIB} -- a quadratic assignment problem library.
\newblock http://www.seas.upenn.edu/qaplib.

\bibitem{ITO2018}
N.~Ito, S.~Kim, M.~Kojima, A.~Takeda, and K.~C. Toh.
\newblock {BBCPOP}: {A} sparse doubly nonnegative relaxation of {P}olynomial
  {O}ptimization {P}roblems with {B}inary, {B}ox and {C}omplementarity
  constraints.
\newblock Research {R}port {B}-48?, {T}okyo {I}nstitute of {T}echnology,
  {D}epartment of {M}athematical and {C}omputing {S}ciences, {T}okyo
  {I}nstitute of {T}echnology, {O}h-{O}kayama, {M}eguro-ku, {T}okyo 152-8552,
  March 2018.

\bibitem{KIM2016}
S.~Kim, M.~Kojima, and K.~C. Toh.
\newblock Doubly nonnegative relaxations for quadratic and polynomial
  optimization problems with binary and box constraints.
\newblock Research {R}port {B}-483, {T}okyo {I}nstitute of {T}echnology,
  {D}epartment of {M}athematical and {C}omputing {S}ciences, {O}h-{O}kayama,
  {M}eguro-ku, T{}okyo 152-8552, July 2016.

\bibitem{KIM2013}
S.~Kim, M.~Kojima, and K.~C. Toh.
\newblock A {L}agrangian-{DNN} relaxation: a fast method for computing tight
  lower bounds for a class of quadratic optimization problems.
\newblock {\em Math. {P}rogram.}, 156:161--187, 2016.

\bibitem{SPARSEPOP_UG}
H.~Waki, S.~Kim, M.~Kojima, M.~Muramatsu, H.~Sugimoto, and M.~Yamashita.
\newblock {\em User Manual for SparsePOP: a Sparse Semidefinite Programming
  Relaxation of Polynomial Optimization Problems}.
\newblock https://sourceforge.net/projects/sparsepop/?source={typ\_redirect},
  August 2009.

\bibitem{BIQMAC}
A.~Wiegele.
\newblock Biq mac library.
\newblock http://www.biqmac.uni-klu.ac.at/biqmaclib.html, 2007.

\end{thebibliography}



\end{document}

