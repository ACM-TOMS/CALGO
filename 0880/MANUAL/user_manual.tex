\documentclass[acmtoms]{acmtrans2m}
\usepackage{algorithm}
\usepackage{algpseudocode}
\usepackage{algcompatible}
\usepackage{graphicx}
\usepackage{epsf}
\usepackage{latexsym}
\usepackage{amsfonts}
\usepackage{alltt}
% this is optional to print the date/time of compilation
\usepackage[long,12hr]{datetime}

%
\acmVolume{V}
\acmNumber{N}
\acmYear{YY}
\acmMonth{M}

\markboth{O.~A.~Marques, C.~V\"omel, J.~.W.~Demmel, and B.~N.~Parlett}
{Algorithm xxx: A Testing Infrastructure for Symmetric Tridiagonal
  Eigensolvers. Usage of {\tt stetester}}

\title{Algorithm xxx: A Testing Infrastructure for Symmetric Tridiagonal
  Eigensolvers. Usage of {\tt stetester}}

\author{
OSNI~A.~MARQUES
and
CHRISTOF V\"OMEL \\Lawrence Berkeley National Laboratory
\and
JAMES~W.~DEMMEL
and 
BERESFORD~N.~PARLETT \\University of California, Berkeley
}


\begin{abstract}
LAPACK is often mentioned as a positive example of a software library 
that encapsulates complex, robust, and widely used numerical algorithms
for a wide range of applications. 
At installation time, the user has the option of running a (limited) 
number of test cases to verify the integrity of the installation process. 
On the algorithm developer's side, however, more exhaustive tests are usually 
performed to study algorithm behavior
on a variety of problem settings and also computer architectures. 
In this process, difficult test cases need to be found that reflect
particular challenges of an application or push algorithms to extreme
behavior. These tests are then assembled into a comprehensive collection,
therefore making it possible for any 
new or competing algorithm to be stressed in a similar way. 

This document describes the supported macros of the testing infrastrucure
and gives examples of input and output files.
\end{abstract}

\category{G.1.3}{Numerical Analysis}{Numerical Linear Algebra}

\terms{Algorithms, Design}

\keywords{Eigenvalues, eigenvectors, symmetric matrix,
LAPACK, accuracy, performance, test matrices,
numerical software, design, implementation, testing.}

\begin{document}

\setcounter{page}{1}

\begin{bottomstuff}
Osni~A.~Marques and Christof V\"omel:
Lawrence Berkeley National Laboratory, 
Computational Research Division, 1 Cyclotron
Road, MS 50F-1650, Berkeley, CA 94720, USA, {OAMarques,CVoemel}@lbl.gov.\\
James W. Demmel and Beresford N. Parlett:
Mathematics Department and Computer Science Division,
University of California, Berkeley, CA 94720, USA. 
demmel@cs.berkeley.edu;parlett@math.berkeley.edu\\
This work was partly supported by a grant from the
National Science Foundation (Cooperative Agreement no. ACI-9619020), 
and by the Director, Office of Computational and Technology Research, 
Division of Mathematical, Information, and Computational Sciences of 
the U.S. Department of Energy under contract No. DE-AC03-76SF00098.
(\copyright)
\end{bottomstuff}

\maketitle


%%%%%%%%%%%%%%%%%%%%%%%%%%%%%%%%%%%%%%%%%%%%%%%%%%%%%%%%%%%%%%%%%%%%%%%%%%%%%%
\section{Usage of the testing infrastructure {\tt stetester}} 
\label{sec:appendix}

This document serves as a reference for use of  {\tt stetester}.
Section~\ref{sec:stetestermacros} contains the supported macros.
Section~\ref{sec:testmatsoverview} shows some of the available test
matrices.
In Section~\ref{sec:sampleinput}, a sample input file is given
that illustrates how to use {\tt stetester}. Section~\ref{sec:sampleoutput}
shows a generated output file in Matlab format 
that allows for easy post-processing and plotting of test data.

%%%%%%%%%%%%%%%%%%%%%%%%%%%%%%%%%%%%%%%%%%%%%%%%%%%%%%%%%%%%%%%%%%%%%%%%%%%%%%
\subsection{Supported macros} 
\label{sec:stetestermacros}

Tables~\ref{tbl:macros_1},
\ref{tbl:macros_2}, and \ref{tbl:macros_3}
contain all currently supported macros in alphabetical order. 

\begin{table}[htbp]
\protect \caption{Key words for {\tt stetester}, part 1.} 
         \label{tbl:macros_1} 
\begin{center}
\begin{tabular}{lll} \hline
Key word & argument & purpose \\ \hline\hline
{\tt CALLST}  & {\it list} & 
              \begin{minipage}[t]{3.4in}
              Defines the subroutines to be tested. Possible
              entries in {\it list} are:
              \begin{tabbing}
              {\tt STEQRV}~~~~ \= (calls {\tt steqr} with COMPZ=`V') \\
              {\tt STEVXA}     \> (calls {\tt stevx} with RANGE=`A') \\
              {\tt STEVXI}     \> (calls {\tt stevx} with RANGE=`I') \\
              {\tt STEVXV}     \> (calls {\tt stevx} with RANGE=`V') \\
              {\tt STEDCI}     \> (calls {\tt stedc} with COMPZ=`I') \\
              {\tt STEGRA}     \> (calls {\tt stegr} with RANGE=`A') \\
              {\tt STEGRI}     \> (calls {\tt stegr} with RANGE=`I') \\
              {\tt STEGRV}     \> (calls {\tt stegr} with RANGE=`V') \\
              {\tt ALL}        \> (performs all tests above)
              \end{tabbing}
              \end{minipage} \\ \hline
{\tt DUMP}    & {\it list} & 
              \begin{minipage}[t]{3.4in}
              Defines data to be written into files. Possible
              entries in {\it list} are:
              \begin{tabbing}
              {\tt T}~~~~ \= (writes the tridiagonal matrix as
                             triplets $i,t_{i,i},t_{i,i+1}$  \\
                          \> in file {\it stetester.out.T}) \\
              {\tt W}     \> (writes the eigenvalues in file
                             {\it stetester.out.W}) \\
              {\tt Z}     \> (writes the eigenvectors in file 
                             {\it stetester.out.Z}) \\
              {\tt LOG}   \> (writes timings, residuals and orthogonality \\
                          \> level in file {\it stetester.out.log}) \\
              {\tt T.M}   \> (writes the tridiagonal matrix in Matlab 
                             format \\
                          \> in file {\it stetester.out.m}) \\
              {\tt W.M}   \> (writes the eigenvalues in Matlab
                             format \\
                          \> in file {\it stetester.out.m}) \\
              {\tt Z.M}   \> (writes the eigenvectors in Matlab
                             format \\
                          \> in file {\it stetester.out.m})
              \end{tabbing}
              \end{minipage} \\ \hline
{\tt ECOND}   & {\it int} &
              \begin{minipage}[t]{3.4in}
              Sets the condition number for types 1 to 4 in Table 
              \ref{tbl:dist_types}. Possible values of {\it int} are: \\
              1, then $k = \frac{1}{\sqrt{ulp}}$, default \\
              2, then $k = \frac{1}{n\times\sqrt{ulp}}$ \\
              3, then $k = \frac{1}{10\times n\times\sqrt{ulp}}$ \\
              4, then $k = \frac{1}{ulp}$ \\
              5, then $k = \frac{1}{n\times ulp}$ \\
              6, then $k = \frac{1}{10\times n\times ulp}$ 
              \end{minipage} \\ \hline
{\tt EDIST}   & {\it int} & 
              \begin{minipage}[t]{3.4in}
              Sets the random distribution to be used in type 6
              in Table \ref{tbl:dist_types}. Possible values of 
              {\it int} are: \\ 
              1, for uniform distribution (-1,1), default \\
              2, for uniform distribution ( 0,1) \\
              3, for normal distribution  ( 0,1)
              \end{minipage} \\ \hline
{\tt ESIGN}   & {\it int} & 
              \begin{minipage}[t]{3.4in}
              Assigns (random) signs to the eigenvalues defined in 
              Table \ref{tbl:dist_types}. Possible values of {\it int} are: \\
              0, then the eigenvalues will not be negative, default \\
              1, then the eigenvalues can be positive, negative or zero
              \end{minipage} \\ \hline
\end{tabular}
\end{center}
\end{table}

\begin{table}[htbp]
\protect \caption{Key words for {\tt stetester}, part 2.} 
         \label{tbl:macros_2} 
\begin{center}
\begin{tabular}{lll} \hline
Key word & argument & purpose \\ \hline\hline
{\tt EIGVAL}  & & 
              \begin{minipage}[t]{3.4in}
              Defines the built-in eigenvalue distributions to be used in 
              the generation of test matrices. The next two lines must
              set integers \\[2mm]
              \hbox{\hspace{0.25in}}
              \parbox{2.0in}{\tt
              {\it etype}$_1$~ {\it etype}$_2$~ {\it etype}$_3$ ... \\
              {\it esize}$_1$~ {\it esize}$_2$~ {\it esize}$_3$ ... } 
              \\[2mm] 
              where {\it etype} is a list of types (see Table 
              \ref{tbl:dist_types}) and {\it esize} is a list of dimensions.
              A negative {\it etype} reverses the eigenvalue distribution.
              For example, {\it etype}$=-1$ results in $\lambda_i=
              \frac{1}{k},~ i=1,2,\dots n-1,~\lambda_n=1$. {\it esize} can 
              also be defined with {\tt NMIN[:NINC]:NMAX}, where
              {\tt NMIN} ($>0$) is the minimum dimension, 
              {\tt NMAX} ($\ge${\tt NMIN}) is the maximum dimension, and
              {\tt NINC} ($>0$) is the increment.
              \end{minipage} \\ \hline
{\tt EIGVALF} & {\it string} & 
              \begin{minipage}[t]{3.4in}
              Defines a file containing an eigenvalue distribution to 
              be used in the generation of a tridiagonal matrix. 
              The file defined
              by {\it string} should contain only one entry per 
              line as follows \\[2mm]
              \hbox{\hspace{0.25in}}
              \parbox{1.0in}{
              $n\\ \lambda_1\\ \vdots\\ \lambda_n$} \\[1mm]
              \end{minipage} \\ \hline
{\tt EIGVI}   & & 
              \begin{minipage}[t]{3.4in}
              Defines indices of the smallest and largest eigenvalues 
              to be computed. The next two lines must define pairs of
              integers \\[2mm]
              \hbox{\hspace{0.25in}}
              \parbox{2.0in}{\tt
              IL$_1$ IL$_2$ ... \\
              IU$_1$ IU$_2$ ... 
              } \\[2mm]
              with 1 $\leq$ {\tt IL}$_i$ $\leq${\tt IU}$_i$. These indices 
              are used only in the tests where RANGE=`I'.
              \end{minipage} \\ \hline
{\tt EIGVV}   & & 
              \begin{minipage}[t]{3.4in}
              Defines lower and upper bounds of intervals to be searched
              for eigenvalues. The next two lines must define pairs of
              values \\[2mm]
              \hbox{\hspace{0.25in}}
              \parbox{2.0in}{\tt
              VL$_1$ VL$_2$ ... \\
              VU$_1$ VU$_2$ ...
              } \\[2mm]
              with VL$_i$ $\leq$ VU$_i$. These indices are used only
              in the tests where RANGE=`V'.
              \end{minipage} \\ \hline
{\tt GLUED}   & & 
              \begin{minipage}[t]{3.4in}
              Defines glued matrices. 
              The next fours lines must set \\[2mm]
              \hbox{\hspace{0.25in}}
              \parbox{2.8in}{\tt
              {\it gform}$_1$~ {\it gform}$_2$~ ...
              {\it gform}$_{k-1}$~ {\it gform}$_k$ \\
              {\it gtype}$_1$~ {\it gtype}$_2$~ ...
              {\it gtype}$_{k-1}$~ {\it gtype}$_k$ \\
              {\it gsize}$_1$~ {\it gsize}$_2$~ ...
              {\it gsize}$_{k-1}$~ {\it gsize}$_k$ \\
              $\gamma_1$~ $\gamma_2$~ ... $\gamma_{k-1}$ 
              } \\[2mm]
              where the integers {\it gform}, {\it gtype} and {\it gsize}
              define, respectively, how the matrix is generated (1 for 
              built-in eigenvalue distribution, 2 for built-in tridiagonal
              matrix), its type (accordingly to Tables \ref{tbl:dist_types} 
              and \ref{tbl:mtrx_types}) and its dimension. The real value
              $\gamma$ (real) is the glue factor.
              \end{minipage} \\ \hline
\end{tabular}
\end{center}
\end{table}

\begin{table}[htbp]
\protect \caption{Key words for {\tt stetester}, part 3.} 
         \label{tbl:macros_3} 
\begin{center}
\begin{tabular}{lll} \hline
Key word & argument & purpose \\ \hline\hline
{\tt HBANDA}  & $k$ &
              \begin{minipage}[t]{3.4in}
              Sets the halfbandwidth of the symmetric matrix $A$ to be 
              generated and then tridiagonalized; {\it k} must be an 
              integer between 1 and 100, which corresponds to
              $\max(1,(kn)/100))$ subdiagonals, where $n$ is the
              dimension of the matrix. By default $k=100$.
              \end{minipage} \\ \hline
{\tt HBANDR}  & $k$ &
              \begin{minipage}[t]{3.4in}
              Sets the halfbandwidth of the matrices used in the tests
              for numerical orthogonality and residual norm; 
              {\it k} must be an 
              integer between 0 and 100, then $\max(1,(kn)/100))$ 
              subdiagonals of those matrices are computed. If $k=0$ 
              the tests are not performed and the corresponding
              results are simply set to 0. By default, $k=100$.
              \end{minipage} \\ \hline
{\tt ISEED}   & $k_1 ~ k_2 ~ k_3 ~ k_4$ & 
              \begin{minipage}[t]{3.4in}
              Sets the (initial) seed of the random number generator. Each
              (integer) $k$ should lie between 0 and 4095 inclusive and 
              $k_4$ should be odd. The default is $k_i=5-i$.
              \end{minipage} \\ \hline
{\tt MATRIX}  & &
              \begin{minipage}[t]{3.4in}
              Defines built-in tridiagonal matrices to be used in the 
              tests. The next two lines must set integers \\[2mm]
              \hbox{\hspace{0.25in}}
              \parbox{2.0in}{\tt
              {\it mtype}$_1$ {\it mtype}$_2$ {\it mtype}$_3$ ...  \\
              {\it msize}$_1$ {\it msize}$_2$ {\it msize}$_3$ ...} \\[2mm]
              where {\it mtype} (integer) is a list of built-in tridiagonal 
              matrices (see Table \ref{tbl:mtrx_types}), 
              and {\it msize} (integer) is a list of dimensions. {\it msize} can 
              also be defined with {\tt NMIN[:NINC]:NMAX}, where
              {\tt NMIN} ($>0$) is the minimum dimension, 
              {\tt NMAX} ($\ge${\tt NMIN}) is the maximum dimension, and
              {\tt NINC} ($>0$) is the increment.
              \end{minipage} \\ \hline
{\tt MATRIXF} & {\it string} &
              \begin{minipage}[t]{3.4in}
              Defines a file containing a tridiagonal matrix, where 
              {\it string} is a file name. This file should contain \\[2mm]
              \hbox{\hspace{0.25in}}
              $ \begin{array}{ccccccc}
              n \\ 
              1 & & d_1 & & e_1 \\
              2 & & d_2 & & e_2 \\ & & \vdots \\
              n & & d_n & & 0.0 \end{array}
              $ \\[2mm]
              which will then be used to generate a tridiagonal
              matrix with diagonal entries set to $d_i$ and 
              offdiagonals set to $e_i$.
              \end{minipage} \\ \hline
{\tt NRILIU}  & {\it k} &
              \begin{minipage}[t]{3.4in}
              Defines the number of {\it k} random indices of the smallest
              and largest eigenvalues to be computed. These indices
              are used only in the tests where RANGE=`I'.
              \end{minipage} \\ \hline
{\tt NRVLVU}  & {\it k} &
              \begin{minipage}[t]{3.4in}
              Defines the number of {\it k} random lower and upper bounds
              of intervals to be searched for eigenvalues. These i
              intervals are used only in the
              tests where RANGE=`V'.
              \end{minipage} \\ \hline
{\tt END}     &   &
              \begin{minipage}[t]{3.4in}
              End of data (subsequent lines are ignored).
              \end{minipage} \\ \hline
\end{tabular}
\end{center}
\end{table}

\clearpage
%%%%%%%%%%%%%%%%%%%%%%%%%%%%%%%%%%%%%%%%%%%%%%%%%%%%%%%%%%%%%%%%%%%%%%%%%%%%%%
\subsection{Examples of available test matrices} 
\label{sec:testmatsoverview}

Tables~\ref{tbl:mtrx_types}  and \ref{tbl:dist_types}   
list some of the test matrices available as part of {\tt stetester}.

\begin{table}[htbp]
\protect \caption{Built-in matrices with distinguishing performance-relevant
                  features.} 
         \label{tbl:mtrx_types} 
\begin{center}
\begin{tabular}{cl} \hline
type & description \\ \hline\hline
  0  & zero matrix \\
  1  & identity matrix \\
  2  & (1,2,1) tridiagonal matrix \\
  3  & Wilkinson-type tridiagonal matrix \\
  4  & Clement-type tridiagonal matrix \\ 
  5  & Legendre-type  tridiagonal matrix\\ 
  6  & Laguerre-type  tridiagonal matrix\\ 
  7  & Hermite-type  tridiagonal matrix\\ 
\hline
\end{tabular}
\end{center}
\end{table}

\begin{table}[htbp]
\protect \caption{LAPACK-style test matrices with a given eigenvalue 
distribution. For distributions 1-5,
the parameter $k$ can be chosen as $ulp^{-1}$ like in the LAPACK
tester but other choices are also possible, see the options for
parameter {\tt ECOND} in Table~\ref{tbl:macros_1}.
For distribution 6, see parameter {\tt EDIST} in Table~\ref{tbl:macros_1}.
} 
         \label{tbl:dist_types} 
\begin{center}
\begin{tabular}{cl} \hline
type & description \\ \hline\hline
  1  & $\lambda_1=1,~ \lambda_i=\frac{1}{k},~ i=2,3,\dots n$ \\
  2  & $\lambda_i=1,~ i=1,2,\dots n-1,~ \lambda_n=\frac{1}{k}$ \\
  3  & $\lambda_i=k^{-(\frac{i-1}{n-1})},~ i=1,2,\dots n$ \\
  4  & $\lambda_i=1-(\frac{i-1}{n-1})(1-\frac{1}{k}),~ i=1,2,\dots n$ \\
  5  & $n$ random numbers in the range $(\frac{1}{k},1)$,
       their logarithms \\
     & are uniformly distributed \\
  6  & $n$ random numbers from a specified distribution \\
  7  & $\lambda_i=ulp\times i,~ i=1,2,\dots n-1,~ \lambda_n=1$ \\
  8  & $\lambda_1=ulp,~ \lambda_i=1+\sqrt{ulp}\times i,~ 
       i=2,3,\dots n-1, \lambda_n =2$ \\
  9  & $\lambda_1=1,~ \lambda_i=\lambda_{i-1}+100\times ulp,
       i=2,\dots n$ \\
\hline
\end{tabular}
\end{center}
\end{table}

%%%%%%%%%%%%%%%%%%%%%%%%%%%%%%%%%%%%%%%%%%%%%%%%%%%%%%%%%%%%%%%%%%%%%%%%%%%%%%
\subsection{A sample input file} 
\label{sec:sampleinput}

Table~\ref{tbl:inputfile} contains a sample input file. After the test
matrices have been specified, the algorithms to be tested (`ALL') 
and the output format are selected. 

\begin{table}[htbp]
\protect \caption{A sample input file for {\tt stetester}.} 
         \label{tbl:inputfile} 
\begin{center}
\begin{minipage}[t]{4.5in}
{\small
\begin{verbatim}
%------------------------------------------------------------------
% This is a simple input file for STETESTER. 
%------------------------------------------------------------------
%
EIGVAL               % Sets built-in eigenvalue distributions
        3            % Distribution 3, EIG(i)=COND**(-(i-1)/(N-1))
       10  15        % Dimensions of the matrices to be generated
%
MATRIX               % Sets built-in matrices
        2   3        % Matrix type 2 and 3
       20:2:25       % Dimension of the matrices to be generated
%
GLUED                % Sets glued matrices
     1     2     1   % If 1, set eigenvalues; if 2, set matrix 
     1     2     3   % Eigenvalue distribution or matrix type
    10    11    12   % Dimensions
 0.001 0.002         % Glue factors
%
EIGVALF DATA/T_0010.eig % Eigenvalues read from file 'T10.eig'
%
MATRIXF DATA/T_0010.dat % Matrix read from file 'T10.dat'
%
% Tests to be performed. Note that 'ALL' is equivalent to
%
%   "STEQRV"  (calls STEQR with COMPZ='V'),
%   "STEVXA"  (calls STEVX with RANGE='A'),
%   "STEVXI"  (calls STEVX with RANGE='I'),
%   "STEVXV"  (calls STEVX with RANGE='V'),
%   "STEDCI"  (calls STEDC with COMPZ='I'),
%   "STEGRA"  (calls STEGR with RANGE='A'),
%   "STEGRI"  (calls STEGR with RANGE='I'),
%   "STEGRV"  (calls STEGR with RANGE='V'),
%
% Also note that no interval was specified (by means of EIGVI, 
% EIGVV, NRILIU or NRVLVU) so in spite of 'ALL' some tests 
% will be skipped. 
%
CALLST  ALL 
%
% Halfbandwidth of the symmetric matrix to be generated and then
% tridiagonalized. This can save time for big matrices.
%
HBANDA  100
%
% Dump results in different formats (including Matlab)
%
DUMP    LOG  T  W  Z  T.M  W.M  Z.M
%
END
\end{verbatim}
}
\end{minipage}
\end{center}
\end{table}


%%%%%%%%%%%%%%%%%%%%%%%%%%%%%%%%%%%%%%%%%%%%%%%%%%%%%%%%%%%%%%%%%%%%%%%%%%%%%%
\subsection{A sample output file in Matlab format} 
\label{sec:sampleoutput}

Table~\ref{tbl:outputfile} contains a sample output file in Matlab format
of a tridiagonal matrix with diagonal $D$ and offdiagonal $E$.
Printed are the computed eigenpairs $W, Z$ from two different 
computations with the same matrix, 
first using QR and then MRRR.
%first using QR  (`STEQR(COMPZ=I)') and subsequently  
%MRRR (`STEGR(RANGE=A)').

\begin{table}[htbp]
\protect \caption{A sample output file generated by {\tt stetester}.
                  The data is printed in Matlab format and stored
                  with a name whose trailing part identifies the test
                  that has been executed} 
         \label{tbl:outputfile} 
\begin{center}
\begin{minipage}[t]{4.5in}
{\scriptsize
\begin{verbatim}
% Case:    1 ################################################################
N =    5;
N_001 = N;
D = zeros(N,1); E = zeros(N,1);
D(    1)= 6.364984420732012E-002; E(    1)=-2.438638589637976E-001; 
D(    2)= 9.364644735979822E-001; E(    2)=-4.811682261688812E-003; 
D(    3)= 1.093556433149412E-002; E(    3)= 3.729709837873370E-005; 
D(    4)= 1.218230276935389E-004; E(    4)= 5.319506124657539E-006; 
D(    5)= 2.722043635334067E-007; E(    5)= 0.000000000000000E+000; 
D_001 = D; E_001 = E; clear D E;
% QR algorithm STEQR(COMPZ=I) ================================================
M =    5;
W = zeros(M,1);
W(    1)= 1.490116120469489E-008; W(    2)= 1.348699152530776E-006; 
W(    3)= 1.220703124999231E-004; W(    4)= 1.104854345603982E-002; 
W(    5)= 1.000000000000000E+000; 
W_001_1 = W; clear W;
Z = zeros(N,M);
Z(    1,    1)= 1.307984066973555E-001; Z(    2,    1)= 3.413911473056844E-002; 
Z(    3,    1)= 1.518458390297948E-002; Z(    4,    1)=-4.786418109812126E-002; 
Z(    5,    1)= 9.895477483327149E-001; Z(    1,    2)= 9.521524057428332E-001; 
Z(    2,    2)= 2.485118884672883E-001; Z(    3,    2)= 1.094543724694096E-001; 
Z(    4,    2)=-2.781623695716281E-002; Z(    5,    2)=-1.374541190267801E-001; 
Z(    1,    3)= 3.316346817614305E-002; Z(    2,    3)= 8.639251903968444E-003; 
Z(    3,    3)= 4.003930825830011E-004; Z(    4,    3)= 9.984606995074429E-001; 
Z(    5,    3)= 4.360755587691128E-002; Z(    1,    4)= 1.080661771201757E-001; 
Z(    2,    4)= 2.330981518906979E-002; Z(    3,    4)=-9.938646010435163E-001; 
Z(    4,    4)=-3.392442840664115E-003; Z(    5,    4)=-1.633388614144078E-006; 
Z(    1,    5)=-2.520306703255168E-001; Z(    2,    5)= 9.677077957618335E-001; 
Z(    3,    5)=-4.707784724629880E-003; Z(    4,    5)=-1.756081031362012E-007; 
Z(    5,    5)=-9.341486344518498E-013; 
Z_001_1 = Z; clear Z;
M_001_1 = M; clear M;
% MRRR algorithm STEGR(RANGE=A) ==============================================
M =    5;
W = zeros(M,1);
W(    1)= 1.490116120299405E-008; W(    2)= 1.348699152440647E-006; 
W(    3)= 1.220703124999230E-004; W(    4)= 1.104854345603979E-002; 
W(    5)= 9.999999999999978E-001; 
W_001_6 = W; clear W;
Z = zeros(N,M);
Z(    1,    1)= 1.307984067061942E-001; Z(    2,    1)= 3.413911473287538E-002; 
Z(    3,    1)= 1.518458390399541E-002; Z(    4,    1)=-4.786418109837592E-002; 
Z(    5,    1)= 9.895477483314390E-001; Z(    1,    2)= 9.521524057416197E-001; 
Z(    2,    2)= 2.485118884669718E-001; Z(    3,    2)= 1.094543724692678E-001; 
Z(    4,    2)=-2.781623695669255E-002; Z(    5,    2)=-1.374541190359646E-001; 
Z(    1,    3)= 3.316346817611779E-002; Z(    2,    3)= 8.639251903961875E-003; 
Z(    3,    3)= 4.003930825800720E-004; Z(    4,    3)= 9.984606995074433E-001; 
Z(    5,    3)= 4.360755587691132E-002; Z(    1,    4)=-1.080661771201750E-001; 
Z(    2,    4)=-2.330981518906962E-002; Z(    3,    4)= 9.938646010435160E-001; 
Z(    4,    4)= 3.392442840664119E-003; Z(    5,    4)= 1.633388614144083E-006; 
Z(    1,    5)=-2.520306703255170E-001; Z(    2,    5)= 9.677077957618334E-001; 
Z(    3,    5)=-4.707784724629883E-003; Z(    4,    5)=-1.756081031362015E-007; 
Z(    5,    5)=-9.341486344518528E-013; 
Z_001_6 = Z; clear Z;
M_001_6 = M; clear M;
clear N;
\end{verbatim}
}
\end{minipage}
\end{center}
\end{table}

\end{document}


