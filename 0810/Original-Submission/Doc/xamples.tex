%% This document created by Scientific Word (R) Version 3.5

\documentclass[12pt]{amsart}%
\usepackage{graphicx}
\usepackage{amscd}
\usepackage{amsmath}%
\usepackage{amsfonts}%
\usepackage{amssymb}
%TCIDATA{OutputFilter=latex2.dll}
%TCIDATA{CSTFile=amsartci.cst}
%TCIDATA{Created=Tue Mar 09 09:05:39 1999}
%TCIDATA{LastRevised=Wednesday, April 04, 2001 12:27:55}
%TCIDATA{<META NAME="GraphicsSave" CONTENT="32">}
%TCIDATA{<META NAME="DocumentShell" CONTENT="Journal Articles\amslatex">}
%TCIDATA{Language=British English}
\newtheorem{theorem}{Theorem}
\theoremstyle{plain}
\newtheorem{acknowledgement}{Acknowledgement}
\newtheorem{algorithm}{Algorithm}
\newtheorem{axiom}{Axiom}
\newtheorem{case}{Case}
\newtheorem{claim}{Claim}
\newtheorem{conclusion}{Conclusion}
\newtheorem{condition}{Condition}
\newtheorem{conjecture}{Conjecture}
\newtheorem{corollary}{Corollary}
\newtheorem{criterion}{Criterion}
\newtheorem{definition}{Definition}
\newtheorem{example}{Example}
\newtheorem{exercise}{Exercise}
\newtheorem{lemma}{Lemma}
\newtheorem{notation}{Notation}
\newtheorem{problem}{Problem}
\newtheorem{proposition}{Proposition}
\newtheorem{remark}{Remark}
\newtheorem{solution}{Solution}
\newtheorem{summary}{Summary}
\numberwithin{equation}{section}
\newcommand{\thmref}[1]{Theorem~\ref{#1}}
\newcommand{\secref}[1]{\S\ref{#1}}
\newcommand{\lemref}[1]{Lemma~\ref{#1}}
\setlength{\oddsidemargin}{0.0in}
\setlength{\evensidemargin}{0.0in}
\setlength{\textwidth}{6.5in}
\setlength{\textheight}{8.5in}
\setlength{\headsep}{0.25in}
\setlength{\headheight}{0.0in}


\begin{document}
\title[SLEIGN2]{SLEIGN2\\Commentary on the individual examples in xamples.f}
\author{P.B. Bailey}
\address{P.B. Bailey, c/o Department of Mathematical Sciences, Northern Illinois
University, DeKalb, IL 60155-2888, USA}
\email{70621.3674@compuserve.com}
\author{W.N. Everitt}
\address{W.N. Everitt, School of Mathematics and Statistics, University of Birmingham,
Edgbaston, Birmingham B15 2TT, England, UK}
\email{w.n.everitt@bham.ac.uk}
\author{A. Zettl}
\address{A. Zettl, Department of Mathematical Sciences, Northern Illinois University,
DeKalb, IL 60155-2888, USA}
\email{zettl@math.niu.edu}
\date{01 March 2001 (File: xamples.tex)\quad\AmS  -\LaTeX  {}; prepared in
\textsl{Scientific Word\/}{}}
\maketitle


\section{Introduction}

The examples in this commentary have been chosen to illustrate the
capabilities and limitations of the program SLEIGN2. Many of the examples have
been chosen from special cases of the well known and well studied ``special
functions'' of mathematical analysis. All possible cases of endpoint
classification are represented; all types of self-adjoint boundary conditions
are included, \textit{i.e.} regular or singular, and separated or coupled. In
the limit-circle case examples are given for which the endpoints may be
oscillatory or non-oscillatory.

For a general account of both the analytical and numerical properties of the
SLEIGN2 code see the paper by Bailey, Everitt and Zettl \cite{BEZ3}.

For all 32 examples in this commentary the following data have been entered:

(i) the Sturm-Liouville differential equation and associated interval on the
real line $\mathbb{R}$

(ii) the range of any parameters in the differential equation; this serves to
remind the reader that numerical values for these parameters have to be
entered in some of the examples given in xamples.x, for any such example to run

(iii) the endpoint classification of the differential equation, in the
relevant Hilbert function space $L^{2}((a,b);w)$

(iv) the boundary condition functions $u,v$ required for any LCNO or LCO endpoint

(v) comments on any particular features of the numbered example.

The data in items (i) to (iv) above can also be found in the file xamples.f,
but this search requires scrolling through the file as the data items, for any
particular example, are located in different sections.

For some of these examples it is possible to give explicit information on the
spectrum of associated boundary value problems; this can take the form of
providing explicit formulas for eigenvalues against which the program
calculated results can be compared.

In all cases of limit-circle endpoints, boundary condition functions $u$ and
$v$ have been entered as part of the example data. In the case of limit-circle
non-oscillatory endpoints we use the convention that the boundary condition
function $u$ determines the principal or Friedrichs boundary condition.

On selecting a numbered example in the file xamples.x, the differential
equation is displayed in Fortran, and details of the endpoint classification
given. If information on the form of the boundary condition functions $u$ and
$v$ is required then the user should scroll separately through the file
xamples.f to the appropriate numbered part of the $u,v$ section or refer to
the information in this commentary.

Some regular and weakly regular problems can be more successfully run using
the limit-circle non-oscillatory (LCNO) algorithm; details are given below for
some of the examples.

It should be noted that for limit-circle oscillatory problems it is sometimes
difficult to compute numerically more than a few of the eigenvalues. This is
due, at least in part, to the rapid growth of the eigenvalues in both the
positive and the negative directions; but particularly in the negative direction.

The Laguerre problem, Example 22, has a discrete spectrum and for one
particular boundary condition the eigenvalues are known explicitly, leading to
the classical Laguerre orthogonal polynomials. In this case numerical values
to confirm the details of the spectrum can also be obtained by use of the
Liouville transformation; this leads to the Laguerre/Liouville Example 23, for
which the program is successful over a wide range of boundary conditions.

The Liouville transformation has also been applied to the Jacobi equation,
Example 16, to yield the Jacobi/Liouville Example 24.

The Liouville transformation is sometimes useful in other cases to put a
Sturm-Liouville differential equation into a form more suitable for numerical
computation; see in particular the Bessel Example 2.

\textbf{Parameters.} The reader is reminded that many of the examples involve
the choice of one or more parameters; the range of these parameters is given
when the numbered differential equation is displayed in xamples.x; if a choice
of parameter is made outside of the stated range the program may abort.

\section{Remarks on the individual examples.}

\begin{enumerate}
\item \textbf{Classical Legendre equation} (see \cite[Chapter IV]{T})
\[
-\left(  \left(  1-x^{2}\right)  y^{\prime}(x)\right)  ^{\prime}+\tfrac{1}%
{4}y(x)=\lambda y(x)\;\text{for all}\;x\in(-1,+1).
\]

Endpoint classification in $L^{2}(-1,+1)$:%
\[%
\begin{tabular}
[c]{cc}\hline
Endpoint & Classification\\\hline
$-1$ & LCNO\\
$+1$ & LCNO\\\hline
\end{tabular}
\]
$\quad$

For both endpoints the boundary condition functions $u,v$ are given by (note
that $u$ and $v$ are solutions of the Legendre equation for $\lambda=1/4$)%
\[
u(x)=1\quad\quad v(x)=\frac{1}{2}\ln\left(  \frac{1+x}{1-x}\right)
\;\text{for all}\;x\in(-1,+1).
\]

\begin{itemize}
\item[(i)] The Legendre polynomials are obtained by taking the principal
(Friedrichs) boundary condition at both endpoints $\pm1:$ enter
$A1=1,A2=0,\;B1=1,B2=0;$ \textit{i.e.} take the boundary condition function
$u$ at $\pm1$; eigenvalues: $\lambda_{n}=(n+1/2)^{2}\ ;$ $n=0,1,2,\cdots;$
eigenfunctions: Legendre polynomials $P_{n}(x)$.

\item[(ii)] Enter $A1=0,\;A2=1,\;B1=0,\;B2=1,$ \textit{i.e.} use the boundary
condition function $v$ at $\pm1$; eigenvalues: $\mu_{n};$ $n=0,1,2,\cdots$ but
no explicit formula is available; eigenfunctions are logarithmically unbounded
at $\pm1$.

\item[(iii)] Observe that $\mu_{n}<\lambda_{n}<\mu_{n+1};$ $n=0,1,2\cdots$.
\end{itemize}

\item \textbf{The Bessel equation }(see \cite[Chapter IV]{T})
\[
-y^{\prime\prime}(x)+\left(  \nu^{2}-1/4\right)  x^{-2}y(x)=\lambda
y(x)\;\text{for all}\;x\in(0,+\infty)
\]
with the parameter $\nu\in\lbrack0,+\infty).$ This is the Liouville form of
the classical Bessel equation.

Endpoint classification in $L^{2}(0,+\infty)$:%
\[%
\begin{tabular}
[c]{ccc}\hline
Endpoint & Parameter $\nu$ & Classification\\\hline
$0$ & For $\nu=1/2$ & R\\
$0$ & For all $\nu\in\lbrack0,1)$ but $\nu\neq1/2$ & LCNO\\
$0$ & For all $\nu\in\lbrack1,\infty)$ & LP\\\hline
$+\infty$ & For all $\nu\in\lbrack0,\infty)$ & LP\\\hline
\end{tabular}
\]

For endpoint $0$ and $\nu\in(0,1)$ but $\nu\neq1/2,$ the LCNO boundary
condition functions $u,v$ are determined by, for all$\;x\in(0,+\infty),$
\[%
\begin{tabular}
[c]{ccc}\hline
Parameter & $u$ & $v$\\\hline
$\nu\in(0,1)$ but $\nu\neq1/2$ & $x^{\nu+1/2}$ & $x^{-\nu+1/2}$\\
$\nu=0$ & $x^{1/2}$ & $x^{1/2}\ln(x)$\\\hline
\end{tabular}
\
\]

(a) Problems on $(0,1]$ with $y(1)=0$:

\noindent For $0\leq\nu<1,\nu\neq\frac{1}{2}:$ the Friedrichs case:
$A1=1,A2=0$ yields the classical Fourier-Bessel series; here $\lambda
_{n}=j_{\nu,n}^{2}$ where $\{j_{\nu,n}:n=0,1,2,\ldots\}$ are the zeros
(positive) of the Bessel function $J_{\nu}(\cdot).$

\noindent For $\nu\geq1;$ LP at $0$ so that there is a unique boundary value
problem with $\lambda_{n}=j_{\nu,n}^{2}$ as before.

(b) Problems on $[1,\infty)$ all have continuous spectrum on $[0,\infty)$:

\noindent For Dirichlet and Neumann boundary conditions there are no eigenvalues.

\noindent For $A1=A2=1$ at $1$ there is one isolated negative eigenvalue.

(c) Problems on $(0,\infty)$ all have continuous spectrum on $[0,\infty)$:

\noindent For $\nu\geq1$ there are no eigenvalues.

\noindent For $0\leq\nu<1$ the Friedrichs case is given by $A1=1,A2=0;$ there
are no eigenvalues.

\noindent For $\nu=0.45$ and $A1=10,A2=-1$ there is one isolated eigenvalue
near to the value $-175.57.$\noindent

\item \textbf{The Halvorsen equation}%
\[
-y^{\prime\prime}(x)=\lambda x^{-4}\exp(-2/x)y(x)\;\text{for all }%
x\in(0,+\infty)
\]

The endpoint classification in the weighted space $L^{2}((0,+\infty;x^{-4}%
\exp(-2/x))$:%
\[%
\begin{tabular}
[c]{cc}\hline
Endpoint & Classification\\\hline
$0$ & WR\\
$+\infty$ & LCNO
\end{tabular}
\]

For the endpoints $0$ and $+\infty$ in the WR and LCNO classification the
boundary condition functions $u,v$ are determined by%
\[%
\begin{tabular}
[c]{ccc}\hline
Endpoint & $u$ & $v$\\\hline
$0$ & $x$ & $1$\\
$+\infty$ & $1$ & $x$\\\hline
\end{tabular}
\]

Since this equation is WR at $0$ and LCNO at $+\infty$ the spectrum is
discrete and bounded below for all boundary conditions. However, this example
illustrates that even a R or WR endpoint can cause difficulties for
computation. The program fails on R at $0$; is successful for WR at $0$; is
successful for LCNO at $0.$

At $0$, the principal boundary condition entry is $A1=1,\ A2=0$; at $\infty$
with $u(x)=1,\ v(x)=x$ the principal boundary condition entry is also
$A1=1,\;A2=0,$ but note the interchange of the definitions of $u$ and $v$ at
these two endpoints.

\item \textbf{The Boyd equation}%
\[
-y^{\prime\prime}(x)-x^{-1}y(x)=\lambda y(x)\;\text{for all}\;x\in
(-\infty,0)\cup(0,+\infty).
\]

Endpoint classification in $L^{2}(-\infty,0)\cup$ $L^{2}(0,+\infty)$:%
\[%
\begin{tabular}
[c]{cc}\hline
Endpoint & Classification\\\hline
$-\infty$ & LP\\
$0-$ & LCNO\\
$0+$ & LCNO\\
$+\infty$ & LP\\\hline
\end{tabular}
\]

For both endpoints $0-$ and $0+$%
\[
u(x)=x\quad\quad v(x)=x\ln(\left|  x\right|  )\;\text{for all}\;x\in
(-\infty,0)\cup(0,+\infty).
\]

This equation arises in a model studying eddies in the atmosphere; see
\cite{B}. There is no explicit formula for the eigenvalues of any particular
boundary condition; eigenfunctions can be given in terms of Whittaker
functions; see \cite[Example 3]{BEZ}.

\item \textbf{The regularized Boyd equation}%
\[
-(p(x)y^{\prime}(x))^{\prime}+q(x)y(x)=\lambda w(x)y(x)\;\text{for all}%
\;x\in(-\infty,0)\cup(0,+\infty)
\]
where%
\[
p(x)=r(x)^{2}\quad q(x)=-r(x)^{2}\left(  \ln(\left|  x\right|  \right)
^{2}\quad w(x)=r(x)^{2}%
\]
with%
\[
r(x)=\exp\left(  -(x\ln(\left|  x\right|  )-x)\right)  \;\text{for all}%
\;x\in(-\infty,0)\cup(0,+\infty).
\]

Endpoint classification in $L^{2}(-\infty,0)\cup$ $L^{2}(0,+\infty)$:%
\[%
\begin{tabular}
[c]{cc}\hline
Endpoint & Classification\\\hline
$-\infty$ & LP\\
$0-$ & WR\\
$0+$ & WR\\
$+\infty$ & LP\\\hline
\end{tabular}
\]

This is a WR form of Example 4; the singularity at zero has been regularized
using quasi-derivatives. There is a close relationship between the examples 4
and 5; in particular they have the same eigenvalues - see \cite{AEZ}. For a
general discussion of regularization using non-principal solutions see
\cite{NZ}. For numerical results see \cite[Example 3]{BEZ}.

\item \textbf{The Sears-Titchmarsh equation}%
\[
-(xy^{\prime}(x))^{\prime}-xy(x)=\lambda x^{-1}y(x)\;\text{for all}%
\;x\in(0,+\infty).
\]

Endpoint classification in $L^{2}(0,\infty)$:%
\[%
\begin{tabular}
[c]{cc}\hline
Endpoint & Classification\\\hline
$0$ & LP\\
$+\infty$ & LCO\\\hline
\end{tabular}
\]

For the endpoint $+\infty$%
\[
u(x)=x^{-1/2}\left(  \cos(x)+\sin(x)\right)  \quad v(x)=x^{-1/2}\left(
\cos(x)-\sin(x)\right)  \;\text{for all}\;x\in(0,+\infty).
\]

This differential equation has one LP and one LCO endpoint. For details of
boundary value problems on $[1,\infty)$ see \cite[Example 4]{BEZ}. The
equation was studied originally in \cite[Chapter IV]{T}; but see \cite{ST}.

For problems on $[1,\infty)$ the spectrum is simple and discrete but unbounded
both above and below.

Numerical results are given in \cite[Example 4]{BEZ}.

\item \textbf{The BEZ equation}%
\[
-(xy^{\prime}(x))^{\prime}-x^{-1}y(x)=\lambda y(x)\;\text{for all}%
\;x\in(-\infty,0)\cup(0,+\infty).
\]

Endpoint classification in $L^{2}(-\infty,0)\cup$ $L^{2}(0,+\infty)$:%
\[%
\begin{tabular}
[c]{cc}\hline
Endpoint & Classification\\\hline
$-\infty$ & LP\\
$0-$ & LCO\\
$0+$ & LCO\\
$+\infty$ & LP\\\hline
\end{tabular}
\]

For both endpoints $0-$ and $0+$:%
\[
u(x)=\cos\left(  \ln(\left|  x\right|  )\right)  \quad\quad v(x)=\sin\left(
\ln(\left|  x\right|  )\right)  \;\text{for all}\;x\in(-\infty,0)\cup
(0,+\infty).
\]

This example is similar to the differential equation of Example 6. On the
interval $(0,1]$ there is a singularity at $0$ in LCO; the equation is R at 1.

For numerical results see \cite[Example 5]{BEZ}.

\item \textbf{The Laplace tidal wave equation}%
\[
-(x^{-1}y^{\prime}(x))^{\prime}+\left(  kx^{-2}+k^{2}x^{-1}\right)
y(x)=\lambda y(x)\;\text{for all}\;x\in(0,+\infty)
\]
where the parameter $k\in(-\infty,0)\cup(0,+\infty)$

Endpoint classification in $L^{2}(0,\infty)$:%
\[%
\begin{tabular}
[c]{cc}\hline
Endpoint & Classification\\\hline
$0$ & LCNO\\
$+\infty$ & LP\\\hline
\end{tabular}
\]

For the endpoint $0$:%
\[
u(x)=x^{2}\quad\quad v(x)=x-k^{-1}\;\text{for all}\;(0,+\infty).
\]

This equation is a particular case of the more general equation with this
name; for details and references see \cite{H}.

There are no representations for solutions of this differential equation in
terms of the well-known special functions. Thus to determine boundary
conditions at the LCNO endpoint $0$ use has to be made of maximal domain
functions; see the $u,\ v$ functions given above. Numerical results for some
boundary value problems and certain values of the parameter $k,$ are given in
\cite[Example 8]{BEZ}.

\item \textbf{The Latzko equation}%
\[
-((1-x^{7})y^{\prime}(x))^{\prime}=\lambda x^{7}y(x)\;\text{for all}%
\;x\in(0,1].
\]

Endpoint classification in $L^{2}(0,1]$:%
\[%
\begin{tabular}
[c]{cc}\hline
Endpoint & Classification\\\hline
$0$ & WR\\
$1$ & LCNO\\\hline
\end{tabular}
\]

For the endpoint $1$:%
\[
u(x)=1\quad\quad v(x)=-\ln(1-x)\;\text{for all}\;(0,1).
\]

This differential equation has a long and celebrated history; see \cite[Pages
43 to 45]{F}. There is a LCNO singularity at the endpoint $1$ which requires
the use of maximal domain functions; see the $u,\ v$ functions given above.
The endpoint $0$ is WR due to the fact that $w(0)=0$.

This example is similar in some respects to the Legendre equation of Example 1 above.

For numerical results see \cite[Example 7]{BEZ}.

\item \textbf{A weakly regular equation}%
\[
-(x^{1/2}y^{\prime}(x))^{\prime}=\lambda x^{-1/2}y(x)\;\text{for all}%
\;x\in(0,+\infty).
\]

Endpoint classification in $L^{2}(0,1]$:%
\[%
\begin{tabular}
[c]{cc}\hline
Endpoint & Classification\\\hline
$0$ & WR\\
$+\infty$ & LP\\\hline
\end{tabular}
\]

This is a devised example to illustrate the computational difficulties of
weakly regular problems.

The differential equation gives $p(0)=0$ and $w(0)=\infty$ but nevertheless
$0$ is a regular endpoint in the Lebesgue integral sense; however $0$ has to
be classified as weakly regular in the computational sense.

The Liouville normal form of this equation is the Fourier equation, see
Example 21 below; thus numerical results for this WR problem can be checked
against numerical results from (i) a R problem, (ii) the roots of
trigonometrical equations, and (iii) a LCNO problem (see below).

There are explicit solutions of this equation given by%
\[
\cos(2x^{1/2}\surd\lambda)\ \ ;\ \sin(2x^{1/2}\surd\lambda)/\surd\lambda.
\]

If $0$ is treated as a LCNO endpoint then $u,\ v$ boundary condition functions
are%
\[
u(x)=2x^{1/2}\quad\quad\ v(x)=1.
\]

The regular Dirichlet condition$\;y(0)=0$ is equivalent to the singular
condition $[y,u](0)=0$. Similarly the regular Neumann condition $(py^{\prime
})(0)=0$ is equivalent to the singular condition $[y,\ v](0)=0$.

The following indicated boundary value problems have the given explicit
formulae for the eigenvalues:
\begin{align*}
y(0)  &  =0\text{{ or }}[y,\ u](0)=0,\text{{ and }}y(1)=0\text{ gives}\\
\;\;\lambda_{n}  &  =((n+1)\pi)^{2}/4\ (n=0,1,...)
\end{align*}%
\begin{align*}
(py^{\prime})(0)  &  =0\text{{ or }}[y,v](0)=0,\text{{ and }}(py^{\prime
})(1)=0\text{ gives}\\
\;\;\lambda_{n}  &  =\left(  (n+\tfrac{1}{2})\pi\right)  ^{2}/4\ (n=0,1,...).
\end{align*}

\item \textbf{The Plum equation}%
\[
-(y^{\prime}(x))^{\prime}+100\cos^{2}(x)y(x)=\lambda y(x)\;\text{for
all}\;x\in(-\infty,+\infty).
\]

Endpoint classification in $L^{2}(-\infty,+\infty)$:%
\[%
\begin{tabular}
[c]{cc}\hline
Endpoint & Classification\\\hline
$-\infty$ & LP\\
$+\infty$ & LP\\\hline
\end{tabular}
\]

Plum \cite{P} computed the first seven eigenvalues for periodic eigenvalues on
the interval $[0,\pi],$\textit{i.e.}%
\[
y(0)=y(\pi)\quad\quad\quad y^{\prime}(0)=y^{\prime}(\pi),
\]
using a numerical homotopy method together with interval arithmetic, and
obtained rigorous bounds for these seven computed eigenvalues. In double
precision the SLEIGN2 computed eigenvalues are in good agreement with these
guaranteed bounds.

\item \textbf{The Mathieu equation}%
\[
-y^{\prime\prime}(x)+2k\cos(2x)y(x)=\lambda y(x)\;\text{for all}\;x\in
(-\infty,+\infty)
\]
where that parameter $k\in(-\infty,0)\cup(0,+\infty).$

Endpoint classification in $L^{2}(-\infty,+\infty)$:%
\[%
\begin{tabular}
[c]{cc}\hline
Endpoint & Classification\\\hline
$-\infty$ & LP\\
$+\infty$ & LP\\\hline
\end{tabular}
\]

The classical Mathieu equation has a celebrated history and voluminous
literature. There are no eigenvalues for this problem on $(-\infty,+\infty)$.
There may be one negative eigenvalue of the problem on $[0,\infty)$ depending
on the boundary condition at the endpoint $0$. The continuous (essential)
spectrum is the same for the whole line or half-line problems and consists of
an infinite number of disjoint closed intervals. The endpoints of these - and
thus the spectrum of the problem - can be characterized in terms of periodic
and semi-periodic eigenvalues of Sturm-Liouville problems on the compact
interval $[0,2\pi]$; these can be computed with SLEIGN2.

The above remarks also apply to the general Sturm-Liouville equation with
periodic coefficients of the same period; the so-called Hill's equation.

Of special interest is the starting point of the continuous spectrum - this is
also the oscillation number of the equation. For the Mathieu equation
($p=1,q=\cos(x),w=1$) on both the whole line and the half line it is
approximately -0.378; this result may be obtained by computing the first
eigenvalue $\lambda_{0}$ of the periodic problem on the interval $[0,2\pi]$.

\item \textbf{The hydrogen atom equation}

It is convenient to take this equation in two forms:%
\begin{equation}
-y^{\prime\prime}(x)+(kx^{-1}+hx^{-2})y(x)=\lambda y(x)\;\text{for all}%
\;x\in(0,+\infty) \tag{1}%
\end{equation}
where the two independent parameters $h\in\lbrack-1/4,+\infty)$ and
$k\in\mathbb{R},$ and%
\begin{equation}
-y^{\prime\prime}(x)+(kx^{-1}+hx^{-2}+1)y(x)=\lambda y(x)\;\text{for
all}\;x\in(0,+\infty) \tag{2}%
\end{equation}
where the two independent parameters $h\in(-\infty,-1/4)$ and $k\in\mathbb{R}.$

Note that form (2) is introduced as a device to aid the numerical computations
in the difficult LCO case; it forces the boundary value problem to have a
non-negative eigenvalue.

Endpoint classification, for both forms (1) and (2), in $L^{2}(0,+\infty)$:%
\[%
\begin{tabular}
[c]{cccc}\hline
Endpoint & Form & Parameters & Classification\\\hline
$0$ & 1 & $h=k=0$ & R\\
$0$ & 1 & $h=0,k\in\mathbb{R}\,\backslash\,\{0\}$ & LCNO\\
$0$ & 1 & $-1/4\leq h<3/4,h\neq0,k\in\mathbb{R}$ & LCNO\\
$0$ & 1 & $h\geq3/4,k\in\mathbb{R}$ & LP\\
$0$ & 2 & $h<-1/4,k\in\mathbb{R}$ & LCO\\\hline
$+\infty$ & 1 and 2 & $h,k\in\mathbb{R}$ & LP\\\hline
\end{tabular}
\]

This is the two parameter version of the classical one-dimensional equation
for quantum modelling of the hydrogen atom; see \cite[Section 10]{J}.

For form (1) and all $h,k$ there are no positive eigenvalues; form (2) is best
considered in the single LCO case when some eigenvalues are positive; in form
(1) there is a continuous spectrum on $[0,\infty)$; in form (2) there is a
continuous spectrum on $[1,\infty).$

If $k=0$ the equation reduces to Bessel, see Example 2 above with $h=\nu
^{2}-1/4$.

\noindent\textbf{Results for form (1)}

In all cases below $\rho$ is defined by
\[
\rho:=(h+1/4)^{1/2}\text{ for }h\geq-1/4.
\]

\begin{enumerate}
\item For $h\geq3/4$ and $k\geq0$ no boundary conditions are required; there
is at most one negative eigenvalue and $\lambda=0$ may be an eigenvalue; for
$h\geq3/4$ and $k<0$ there are infinitely many negative eigenvalues given by
\[
\lambda_{n}=\frac{-k^{2}}{(2n+2\rho+1)^{2}},\ \rho=(h+1/4)^{1/2}%
>0,\ n=0,1,2,3,\ldots
\]
and $\lambda=0$ is not an eigenvalue.

\item For $h=0$ and $k\in\mathbb{R}\,\backslash\,\{0\}$ a boundary condition
is required at $0$ for which
\[
u(x)=x\quad\quad\ v(x)=1+k\,x\ln(x).
\]
For some computed eigenvalues see \cite{BEZ} and \cite[Section 10]{J}.

\item For $-1/4<h<3/4,$ \textit{i.e.} $0<\rho<1,$ and $h\neq0,$ \textit{i.e.}
$\rho\not =1/2$, then a boundary condition is required at $0$ for which, for
all $x\in(0,+\infty),$%
\[
u(x)=x^{\frac{1}{2}+\rho}\quad\quad v(x)=x^{\frac{1}{2}-\rho}+\frac{k}%
{1-2\rho}x^{\frac{3}{2}-\rho};
\]

the following results hold for the non-Friedrichs boundary condition
$[y,v](0)=0$, \textit{i.e.} $A1=0,A2=1$:

\begin{itemize}
\item[(i)] $k>0,\ 0<\rho<1/2$ there are no negative eigenvalues

\item[(ii)] $k>0,\ 1/2<\rho<1$ there is exactly one negative eigenvalue given
by
\[
\lambda_{0}=\frac{-k^{2}}{(2\rho-1)^{2}}%
\]

\item[(iii)] if $k<0,\ 0<\rho<1/2$ there are infinitely many negative
eigenvalues given by
\[
\lambda_{n}=\frac{-k^{2}}{(2n-2\rho+1)^{2}},\ n=0,1,2,3,\ldots
\]

\item[(iv)] if $k<0,\ 1/2<\rho<1$ there are infinitely many negative
eigenvalues given by
\[
\lambda_{n}=\frac{-k^{2}}{(2n-2\rho+3)^{2}}\,,\ n=0,1,2,3,\ldots
\]

\item[(v)] for $k=0$ and $(A1)(A2)<0$ there is exactly one negative eigenvalue
given by:
\[
\lambda_{0}=\frac{4\left(  A1\right)  \Gamma(1+\rho)}{\left(  A2\right)
\Gamma(1-\rho)^{1/\rho}}.
\]
\end{itemize}

\item For $h=-1/4,\ k\in R$ , the LCNO classification at $0$ prevails and a
boundary condition is required for which, for all $x\in(0,+\infty),$%
\[
u(x)=x^{1/2}+kx^{3/2}\quad\quad v(x)=2x^{1/2}+\left(  x^{1/2}+kx^{3/2}\right)
\ln(x).
\]
\noindent For $k=0$ and $(A1)(A2)<0$ there is exactly one negative eigenvalue
given by:
\[
\lambda_{0}=-c\exp(2(A1)/A2),\quad c=4\exp(4-2\gamma)
\]
where $\gamma$ is Euler's constant: $\gamma=0.5772156649\ldots$.

\noindent\noindent\textbf{Results for form (2)}

\item For $h<-1/4,\ k\in R$, the equation is LCO at $0$ (recall that we added
$1$ to the coefficient $q(\cdot)$ for this case, thus moving the start of the
continuous spectrum from $0$ to $1$) for which, defining%
\[
\sigma:=(-h-1/4)^{1/2},
\]
then, for all $x\in(0,+\infty),$%
\begin{align*}
u(x)  &  =x^{1/2}\left[  (1-(4h)^{-1}kx)\cos(\sigma\ln(x))+k\sigma
x\sin(\sigma\ln(x))/2\right] \\
v(x)  &  =x^{1/2}\left[  (1-(4h)^{-1}kx)\sin(\sigma\ln(x))+k\sigma
x\cos(\sigma\ln(x))/2\right]  ;
\end{align*}

\begin{itemize}
\item[(i)] when $k=0$ this equation reduces to the Krall equation Example 20
below (but note that the notation is different).

\item[(ii)] When $k\not =0$ explicit formulas for the eigenvalues are not
available; however we report here on the qualitative properties of the
spectrum for any boundary condition at $0$:

$(\alpha)$ for all $k\in R$ there are infinitely many negative eigenvalues
tending exponentially to $-\infty$

$(\beta)$ for $k>0$ there are only a finite number of eigenvalues in any
bounded interval, in particular they do not accumulate at $1$

$(\gamma)$ for $k\leq0$ the eigenvalues accumulate also at $1$.

$(\delta)$ for $k=0$ and $(A1)(A2)<0$ there is exactly one negative eigenvalue
given by:
\[
\lambda_{0}=\frac{4\left(  A1\right)  \Gamma(1+\rho)}{\left(  A2\right)
\Gamma(1-\rho)^{1/\rho}}.
\]
\end{itemize}

Most of these results are due to J\"{o}rgens, see \cite[Section 10]{J}; a few
new results were established by the authors.
\end{enumerate}

\item \textbf{The Marletta equation}%
\[
-y^{\prime\prime}(x)+\frac{3(x-31)}{4(x+1)(x+4)^{2}}y(x)=\lambda
y(x)\;\text{for all}\;x\in\lbrack0,+\infty).
\]

Endpoint classification in $L^{2}(0,+\infty)$:%
\[%
\begin{tabular}
[c]{cc}\hline
Endpoint & Classification\\\hline
$0$ & R\\
$+\infty$ & LP\\\hline
\end{tabular}
\]
Since $q(x)\rightarrow0$ as $x\rightarrow\infty$ the continuous spectrum
consists of $[0,\infty)$ and every negative number is an eigenvalue for some
boundary condition at $0.$

For the boundary condition $A1=5,A2=8$ there is a negative eigenvalue
$\lambda_{0}$ near $-1.185.$ However the equation with $\lambda=0$ has a
solution
\[
y(x)=\frac{1-x^{2}}{(1+x/4)^{5/2}}\text{ for all }x\in\lbrack0,\infty)
\]
that satisfies this boundary condition which is NOT in $L^{2}(0,\infty)$ but
is ``nearly'' in this space. This solution deceives SLEIGN and SLEIGN2 in
single precision, and SLEDGE in double precision, into reporting $\lambda=0$
as a second eigenvalue; in double precision SLEIGN and SLEIGN2 correctly
report that $\lambda_{0}$ is the only eigenvalue, and SLEIGN2 reports the
start of the continuous spectrum at $0.$

Additional details of this example are to be found in the Marletta
certification report on SLEIGN (not SLEIGN2) \cite{M}.

\item \textbf{The harmonic oscillator equation}%
\[
-y^{\prime\prime}(x)+x^{2}y(x)=\lambda y(x)\;\text{for all}\;x\in
(-\infty,+\infty).
\]

Endpoint classification in $L^{2}(-\infty,+\infty)$:%
\[%
\begin{tabular}
[c]{cc}\hline
Endpoint & Classification\\\hline
$-\infty$ & LP\\
$+\infty$ & LP\\\hline
\end{tabular}
\
\]

This is another classical equation; it is also the Liouville normal form of
the differential equation for the Hermite orthogonal polynomials. On the whole
real line the boundary value problem requires no boundary conditions at the
endpoints of $\pm\infty$. Thus there is a unique self-adjoint extension with
discrete spectrum given by :
\[
\{\lambda_{n}=2n+1;\;n=0,1,2,...\}.
\]
For a classical treatment see \cite[Chapter IV, Section 2]{T}.

\item \textbf{The Jacobi equation}%
\[
-\left(  (1-x)^{\alpha+1}(1+x)^{\beta+1}y^{\prime}(x)\right)  ^{\prime
}=\lambda(1-x)^{\alpha}(1+x)^{\beta}y(x)\;\text{for all}\;x\in(-1,+1)
\]
where the parameters $\alpha,\beta\in(-\infty,+\infty).$

Endpoint classification in the weighted space $L^{2}((-1,+1);(1-x)^{\alpha
}(1+x)^{\beta}))$:%
\[%
\begin{tabular}
[c]{ccc}\hline
Endpoint & Parameter & Classification\\\hline
$-1$ & $\beta\leq-1$ & LP\\
$-1$ & $-1<\beta<0$ & WR\\
$-1$ & $0\leq\beta<1$ & LCNO\\
$-1$ & $1\leq\beta$ & LP\\\hline
\end{tabular}
\quad\quad%
\begin{tabular}
[c]{ccc}\hline
Endpoint & Parameter & Classification\\\hline
$+1$ & $\alpha\leq-1$ & LP\\
$+1$ & $-1<\alpha<0$ & WR\\
$+1$ & $0\leq\alpha<1$ & LCNO\\
$+1$ & $1\leq\alpha$ & LP\\\hline
\end{tabular}
\]

For the endpoint $-1$ and for the WR and LCNO cases the boundary condition
functions $u,v$ are determined by%
\[%
\begin{tabular}
[c]{ccc}\hline
Parameter & $u$ & $v$\\\hline
$-1<\beta<0$ & $(1+x)^{-\beta}$ & $1$\\
$\beta=0$ & $1$ & $\ln\left(  \dfrac{1+x}{1-x}\right)  $\\
$0<\beta<1$ & $1$ & $(1+x)^{-\beta}$\\\hline
\end{tabular}
.
\]

For the endpoint $+1$ and for the WR and LCNO cases the boundary condition
functions $u,v$ are determined by%
\[%
\begin{tabular}
[c]{ccc}\hline
Parameter & $u$ & $v$\\\hline
$-1<\alpha<0$ & $(1-x)^{-\alpha}$ & $1$\\
$\alpha=0$ & $1$ & $\ln\left(  \dfrac{1+x}{1-x}\right)  $\\
$0<\alpha<1$ & $1$ & $(1-x)^{-\alpha}$\\\hline
\end{tabular}
.
\]

To obtain the classical Jacobi orthogonal polynomials it is necessary to take
$-1<\alpha,\,\beta$; then note the required boundary conditions:

Endpoint $-1$:%
\[%
\begin{tabular}
[c]{cc}\hline
Parameter & Boundary condition\\\hline
$-1<\beta<0$ & $(py^{\prime})(-1)=0\;$or$\;[y,v](-1)=0$\\
$0\leq\beta<1$ & $[y,u](-1)=0$\\\hline
\end{tabular}
\]

Endpoint $+1$:%
\[%
\begin{tabular}
[c]{cc}\hline
Parameter & Boundary condition\\\hline
$-1<\alpha<0$ & $(py^{\prime})(+1)=0\;$or$\;[y,v](+1)=0$\\
$0\leq\alpha<1$ & $[y,u](+1)=0$\\\hline
\end{tabular}
\]
\newline For the classical Jacobi orthogonal polynomials the eigenvalues are
given by:
\[
\lambda_{n}=n(n+\alpha+\beta+1)\;\text{for}\;n=0,1,2,\ldots
\]
and this explicit formula can be used to give an independent check on the
accuracy of the results from the SLEIGN2 code.

It is interesting to note that the required boundary condition for these
Jacobi polynomials is the Friedrichs condition in the LCNO case but not in the
WR case.

\item \textbf{The rotation Morse oscillator equation}%
\[
-y^{\prime\prime}(x)+(2x^{-2}-2000(2e(x)-e(x)^{2}))y(x)=\lambda
y(x)\;\text{for all}\;x\in(0,+\infty)
\]
where%
\[
e(x)=\exp(-1.7(x-1.3))\;\text{for all}\;x\in(0,+\infty).
\]

Endpoint classification in the space $L^{2}(0,+\infty)$%
\[%
\begin{tabular}
[c]{cc}\hline
Endpoint & Classification\\\hline
$0$ & LP\\
$+\infty$ & LP\\\hline
\end{tabular}
\
\]

This classical problem has continuous spectrum on $[0,\infty)$ and exactly 26
negative eigenvalues. Enter NUMEIG1 = 0, NUMEIG2 = 28 and observe the 26
eigenvalues and the start of the continuous spectrum at 0.

\item \textbf{The Dunsch equation}%
\[
-\left(  (1-x^{2})y^{\prime}(x)\right)  ^{\prime}+\left(  \frac{2\alpha^{2}%
}{(1+x)}+\frac{2\beta^{2}}{(1-x)}\right)  y(x)=\lambda y(x)\;\text{for
all}\;x\in(-1,+1)
\]
where the independent parameters $\alpha,\beta\in\lbrack0,+\infty).$

Boundary value problems for this differential equation are discussed in
\cite[Chapter VIII, Pages 1515-20]{DS}.

Endpoint classification in the space $L^{2}(-1,+1)$ for $-1$:%
\[%
\begin{tabular}
[c]{cc}\hline
Parameter & Classification\\\hline
$0\leq\alpha<1/2$ & LCNO\\
$1/2\leq\alpha$ & LP\\\hline
\end{tabular}
\]

Endpoint classification in the space $L^{2}(-1,+1)$ for $+1$:%
\[%
\begin{tabular}
[c]{cc}\hline
Parameter & Classification\\\hline
$0\leq\beta<1/2$ & LCNO\\
$1/2\leq\beta$ & LP\\\hline
\end{tabular}
\]

For the LCNO cases the boundary condition functions $u,v$ are given by%
\[%
\begin{tabular}
[c]{cccc}\hline
Endpoint & Parameter & $u$ & $v$\\\hline
$-1$ & $\alpha=0$ & $1.0$ & $\dfrac{1}{2}\ln\left(  \dfrac{1+x}{1-x}\right)
$\\
$-1$ & $0<\alpha<1/2$ & $(1+x)^{\alpha}$ & $(1+x)^{-\alpha}$\\
$+1$ & $\beta=0$ & $1.0$ & $\dfrac{1}{2}\ln\left(  \dfrac{1+x}{1-x}\right)
$\\
$+1$ & $0<\beta<1/2$ & $(1-x)^{\beta}$ & $(1-x)^{-\beta}$\\\hline
\end{tabular}
\]

Note that these $u$ and $v$ are not solutions of the differential equation but
maximal domain functions. In \cite[Page 1519]{DS} it is stated that the
boundary value problem determined by the boundary conditions
\[
\lbrack y,u](-1)=0=[y,u](1)
\]
has eigenvalues given by the explicit formula
\[
\lambda_{n}=(n+\alpha+\beta+1)(n+\alpha+\beta)\;\text{for}\;n=0,1,2,\ldots
\]

\item \textbf{The Donsch equation}%
\[
-\left(  (1-x^{2})y^{\prime}(x)\right)  ^{\prime}+\left(  \frac{-2\gamma^{2}%
}{(1+x)}+\frac{2\beta^{2}}{(1-x)}\right)  y(x)=\lambda y(x)\;\text{for
all}\;x\in(-1,+1)
\]
where the independent parameters $\gamma,\beta\in\lbrack0,+\infty).$

Endpoint classification in the space $L^{2}(-1,+1)$ for $-1$:%
\[%
\begin{tabular}
[c]{cc}\hline
Parameter & Classification\\\hline
$\gamma=0$ & LCNO\\
$0<\gamma$ & LCO\\\hline
\end{tabular}
\]

Endpoint classification in the space $L^{2}(-1,+1)$ for $+1$:%
\[%
\begin{tabular}
[c]{cc}\hline
Parameter & Classification\\\hline
$0\leq\beta<1/2$ & LCNO\\
$1/2\leq\beta$ & LP\\\hline
\end{tabular}
\]

For these LCNO/LCO cases the boundary condition functions $u,v$ are given by%
\[%
\begin{tabular}
[c]{cccc}\hline
Endpoint & Parameter & $u$ & $v$\\\hline
$-1$ & $\gamma=0$ & $1$ & $\dfrac{1}{2}\ln\left(  \dfrac{1+x}{1-x}\right)  $\\
$-1$ & $0<\gamma$ & $\cos(\gamma\ln(1+x))$ & $\sin(\gamma\ln(1+x))$\\
$+1$ & $\beta=0$ & $1$ & $\dfrac{1}{2}\ln\left(  \dfrac{1+x}{1-x}\right)  $\\
$+1$ & $0<\beta<1/2$ & $(1-x)^{\beta}$ & $(1-x)^{-\beta}$\\\hline
\end{tabular}
\
\]

This is a modification of Example 18 above which illustrates an LCNO/LCO mix
obtained by replacing $\alpha$ with $i\gamma$; this changes the singularity at
$-1$ from LCNO to LCO.

Again these $u$ and $v$ are not solutions of the differential equation but
maximal domain functions.

\item \textbf{The Krall equation}%
\[
-y^{\prime\prime}(x)+(1-(k^{2}+1/4)x^{-2})y(x)=\lambda y(x)\;\text{for
all}\;x\in(0,+\infty)
\]
where the parameter $k\in(0,+\infty).$

Endpoint classification in the space $L^{2}(0,+\infty)$:%
\[%
\begin{tabular}
[c]{cc}\hline
Endpoint & Classification\\\hline
$0$ & LCO\\
$+\infty$ & LP\\\hline
\end{tabular}
\]

This example should be seen as a special case of the Bessel Example 2 above;
solutions can be obtained in terms of the modified Bessel functions.

To help with the computations for this example the spectrum is translated by a
term $+1$; this simple device is used for numerical convenience.

For problems with separated boundary conditions at endpoints $0$ and $\infty$
there is a continuous spectrum on $[1,\infty)$ with a discrete (and simple)
spectrum on $(-\infty,1)$. This discrete spectrum has cluster points at both
$-\infty$ and $1$.

For the LCO endpoint at $0$ the boundary condition functions are given by%
\[
u(x)=x^{1/2}\cos(k\ln(x))\quad\quad v(x)=x^{1/2}\sin(k\ln(x)).
\]

For the boundary value problem with boundary condition $[y,u](0)=0$ the
eigenvalues are given explicitly by:

(i) suppose $\Gamma(1+i)=\alpha+i\beta$ and $\mu>0$ satisfies $\tan\left(
\ln(\frac{1}{2}\mu)\right)  =-\alpha/\beta$

(ii) $\theta=\operatorname{Im}(\log(\Gamma(1+i)))$

(iii) $\ln(\frac{1}{2}\mu)=\tfrac{1}{2}\pi+\theta+s\pi\;$for$\;\;s=0,\pm1,\pm2,...$

(iv) $\mu_{s}^{2}=\left(  2\exp(\theta+\frac{1}{2}\pi)\right)  ^{2}\exp
(2s\pi)\;\,s=0,\pm1,\pm2,...$

\noindent then the eigenvalues are $\lambda_{n}=-\mu_{-(n+1)}^{2}%
+1\ (n=0,\pm1,\pm2,...)$.

SLEIGN2 can compute only six of these eigenvalues in a normal UNIX server,
even in double precision, $\lambda_{-3}$ to $\lambda_{2}$; other eigenvalues
are, numerically, too close to 1 or too close to $-\infty$. Here we list these
SLEIGN2 computed eigenvalues in double precision in a normal UNIX server and
compare them with the same eigenvalues computed from the transcendental
equation; for the problem on $(0,\infty)$ with $k=1$ and $A1=1.0,\ A2=0.0$.
\[%
\begin{array}
[c]{cccc}%
\text{NUMEIG} & \text{eig from SLEIGN2} & \text{eig from trans. equ.} &
\text{iflag}\\
-3 & -276,562.5 & -14,519,130 & 4\\
-2 & -27,114.48 & -27,114.67 & 2\\
-1 & -49.62697 & -49.63318 & 2\\
0 & 0.9054452 & 0.9054454 & 1\\
1 & 0.9998234 & 0.9998234 & 1\\
2 & 0.9999997 & 0.9999997 & 3
\end{array}
.
\]

\item \textbf{The Fourier equation}%
\[
-y^{\prime\prime}(x)=\lambda y(x)\;\text{for all}\;x\in(-\infty,+\infty)
\]

Endpoint classification in $L^{2}(-\infty,+\infty)$:%
\[%
\begin{tabular}
[c]{cc}\hline
Endpoint & Classification\\\hline
$-\infty$ & LP\\
$+\infty$ & LP\\\hline
\end{tabular}
\]

This is a simple constant coefficient equation whose eigenvalues, for any
self-adjoint boundary condition, can be characterized in terms of a
transcendental equation involving only trigonometric functions.

\item \textbf{The Laguerre equation}%
\[
-(x^{\alpha+1}\exp(-x)y^{\prime}(x))^{\prime}=\lambda x^{\alpha}%
\exp(-x)y(x)\;\text{for all}\;x\in(0,+\infty)
\]
where the parameter $\alpha\in(-\infty,+\infty).$

Endpoint classification in the weighted space $L^{2}((0,+\infty);x^{\alpha
}\exp(-x))$:%
\[%
\begin{tabular}
[c]{ccc}\hline
Endpoint & Parameter & Classification\\\hline
$0$ & $\alpha\leq-1$ & LP\\
$0$ & $-1<\alpha<0$ & WR\\
$0$ & $0\leq\alpha<1$ & LCNO\\
$0$ & $1\leq\alpha$ & LP\\
$+\infty$ & $\alpha\in(-\infty,+\infty)$ & LP\\\hline
\end{tabular}
\]

For these WR/LCNO cases the boundary condition functions $u,v$ are given by:%
\[%
\begin{tabular}
[c]{cccc}\hline
Endpoint & Parameter & $u$ & $v$\\\hline
$0$ & $-1<\alpha<0$ & $x^{-\alpha}$ & $1$\\
$0$ & $\alpha=0$ & $1$ & $\ln(x)$\\
$0$ & $0<\alpha<1$ & $1$ & $x^{-\alpha}$\\\hline
\end{tabular}
\]

This is the classical form of the differential equation which for parameter
$\alpha>-1$ produces the classical Laguerre polynomials as eigenfunctions; for
the boundary condition $[y,1](0)=0$ at $0$, when required, the eigenvalues are
then (remarkably!) independent of $\alpha$ and given by $\lambda
_{n}=n\;(n=0,1,2,...)$; see \cite[Chapter 22, Section 22.6]{AB}.

SLEIGN2 does not compute eigenvalues well with this differential equation on
$(0,\infty)$, with the code in a UNIX server; this appears to be due to
numerical problems resulting from the exponentially small coefficients;
however, see Example 23 below.

\item \textbf{The Laguerre/Liouville equation}%
\[
-y^{\prime\prime}(x)+\left(  \frac{\alpha^{2}-1/4}{x^{2}}-\frac{\alpha+1}%
{2}+\frac{x^{2}}{16}\right)  y(x)=\lambda y(x)\;\text{for all}\;x\in
(0,+\infty)
\]
where the parameter $\alpha\in(-\infty,+\infty).$

Endpoint classification in the space $L^{2}(0,+\infty)$:%
\[%
\begin{tabular}
[c]{ccc}\hline
Endpoint & Parameter & Classification\\\hline
$0$ & $\alpha\leq-1$ & LP\\
$0$ & $-1<\alpha<1,$ but $\alpha^{2}\neq1/4$ & LCNO\\
$0$ & $\alpha^{2}=1/4$ & R\\
$0$ & $1\leq\alpha$ & LP\\
$+\infty$ & $\alpha\in(-\infty,+\infty)$ & LP\\\hline
\end{tabular}
\]

For these WR/LCNO cases the boundary condition functions $u,v$ are given by:%
\[%
\begin{tabular}
[c]{cccc}\hline
Endpoint & Parameter & $u$ & $v$\\\hline
$0$ & $-1<\alpha<0$ but $\alpha\neq-1/2$ & $x^{\frac{1}{2}-\alpha}$ &
$x^{\frac{1}{2}+\alpha}$\\
$0$ & $\alpha=-1/2$ & $x$ & $1$\\
$0$ & $\alpha=0$ & $x^{1/2}$ & $x^{1/2}\ln(x)$\\
$0$ & $0<\alpha<1$ but $\alpha\neq1/2$ & $x^{\frac{1}{2}+\alpha}$ &
$x^{\frac{1}{2}-\alpha}$\\
$0$ & $\alpha=1/2$ & $x$ & $1$\\\hline
\end{tabular}
\]

This is the Liouville normal form of the Laguerre equation; the two forms are
unitarily equivalent so that the spectrum and the eigenfunctions of equivalent
boundary value problems are identical. This Liouville form is more suitable
for eigenvalue computations in contrast to the previous example.

The Laguerre polynomials are produced as eigenfunctions only when $\alpha>-1$.
For $\alpha\geq1$ the LP condition holds at $0$. For $0\leq\alpha<1$ the
appropriate boundary condition is the Friedrichs condition: $[y,u](0)=0;$ for
$-1<\alpha<0$ use the non-Friedrichs condition: $[y,\ v](0)=0$. In all these
cases $\lambda_{n}=n$ for $n=0,1,2,...$.

\item \textbf{The Jacobi/Liouville equation}%
\[
-y^{\prime\prime}(x)+q(x)y(x)=\lambda y(x)\;\text{for all}\;x\in(-\pi
/2,+\pi/2)
\]
where the coefficient $q$ is given by, for all$\;x\in(-\pi/2,+\pi/2),$%
\[
q(x)=\frac{\beta^{2}-1/4}{4\tan^{2}((x+\pi)/2)}+\frac{\alpha^{2}-1/4}%
{4\tan^{2}((x-\pi)/2)}-\frac{4\alpha\beta+4\beta+4\alpha+3}{8}.
\]
Here the parameters $\alpha,\beta\in(-\infty+,\infty).$

Endpoint classification in the space $L^{2}(-\pi/2,+\pi/2)$:%
\[%
\begin{tabular}
[c]{ccc}\hline
Endpoint & Parameter & Classification\\\hline
$-\pi/2$ & $\beta\leq-1$ & LP\\
$-\pi/2$ & $-1<\beta<1$ but $\beta^{2}\neq1/4$ & LCNO\\
$-\pi/2$ & $\beta^{2}=1/4$ & R\\
$-\pi/2$ & $1\leq\beta$ & LP\\\hline
\end{tabular}
\]%
\[%
\begin{tabular}
[c]{ccc}\hline
Endpoint & Parameter & Classification\\\hline
$+\pi/2$ & $\alpha\leq-1$ & LP\\
$+\pi/2$ & $-1<\alpha<1$ but $\alpha^{2}\neq1/4$ & LCNO\\
$+\pi/2$ & $\alpha^{2}=1/4$ & R\\
$+\pi/2$ & $1\leq\alpha$ & LP\\\hline
\end{tabular}
\]

For the endpoint $-\pi/2$ and for LCNO cases the boundary condition functions
$u,v$ are determined by, here $b(x)=2\tan^{-1}(1)+x$ for all $x\in(-\pi
/2,+\pi/2),$%
\[%
\begin{tabular}
[c]{ccc}\hline
Parameter & $u$ & $v$\\\hline
$-1<\beta<0$ & $b(x)^{\frac{1}{2}-\beta}$ & $b(x)^{\frac{1}{2}+\beta}$\\
$\beta=0$ & $\sqrt{b(x)}$ & $\sqrt{b(x)}\ln(b(x))$\\
$0<\beta<1$ & $b(x)^{\frac{1}{2}+\beta}$ & $b(x)^{\frac{1}{2}-\beta}$\\\hline
\end{tabular}
\]

For the endpoint $+\pi/2$ and for LCNO cases the boundary condition functions
$u,v$ are determined by, here $a(x)=2\tan^{-1}(1)-x$ for all $x\in(-\pi
/2,+\pi/2),$%
\[%
\begin{tabular}
[c]{ccc}\hline
Parameter & $u$ & $v$\\\hline
$-1<\alpha<0$ & $a(x)^{\frac{1}{2}-\alpha}$ & $a(x)^{\frac{1}{2}+\alpha}$\\
$\alpha=0$ & $\sqrt{a(x)}$ & $\sqrt{a(x)}\ln(a(x))$\\
$0<\alpha<1$ & $a(x)^{\frac{1}{2}+\alpha}$ & $a(x)^{\frac{1}{2}-\alpha}%
$\\\hline
\end{tabular}
\]

This is the Liouville normal form of the Jacobi equation of Example 16.

The classical Jacobi orthogonal polynomials are produced only when both
$\alpha,\beta>-1.$ For $\alpha,\beta>+1$ the LP condition holds and no
boundary condition is required to give the polynomials. If $-1<\alpha,\beta<1$
then the LCNO condition holds and boundary conditions are required to produce
the Jacobi polynomials; these conditions are as follows:

Endpoint $-\pi/2$%
\[%
\begin{tabular}
[c]{cc}\hline
Parameter & Boundary condition\\\hline
$-1<\beta<0$ & $[y,v](-\pi/2)=0$\\
$0\leq\beta<1$ & $[y,u](-\pi/2)=0$\\\hline
\end{tabular}
\]

Endpoint $+\pi/2$%
\[%
\begin{tabular}
[c]{cc}\hline
Parameter & Boundary condition\\\hline
$-1<\alpha<0$ & $[y,v](+\pi/2)=0$\\
$0\leq\alpha<1$ & $[y,u](+\pi/2)=0$\\\hline
\end{tabular}
\]

Recall from Example 16 for the classical orthogonal Jacobi polynomials the
eigenvalues are given explicitly by:%
\[
\lambda_{n}=n(n+\alpha+\beta+1)\;\text{for}\;n=0,1,2,\ldots
\]

\item \textbf{The Meissner equation}%
\[
-y^{\prime\prime}(x)=\lambda w(x)y(x)\;\text{for all}\;x\in(-\infty,+\infty)
\]
where the weight coefficient $w$ is defined by%
\begin{align*}
w(x)  &  =1\;\text{for all}\;x\in(-\infty,0]\\
&  =9\;\text{for all}\;x\in(0,+\infty).
\end{align*}

Endpoint classification in the space $L^{2}(-\infty,+\infty)$:%
\[%
\begin{tabular}
[c]{cc}\hline
Endpoint & Classification\\\hline
$-\infty$ & LP\\
$+\infty$ & LP\\\hline
\end{tabular}
\]

This equation arose in a model of a one dimensional crystal. For this constant
coefficient equation with a weight function which has a jump discontinuity the
eigenvalues can be characterized as roots of a transcendental equation
involving only trigonometrical and inverse trigonometrical functions. There
are infinitely many simple eigenvalues and infinitely many double ones for the
periodic case; they are given by:

\textbf{Periodic boundary conditions on} $(-1/2,+1/2)$\textbf{, }\textit{i.e.}%
\[
y(-1/2)=y(+1/2)\quad\quad y^{\prime}(-1/2)=y^{\prime}(+1/2).
\]

We have $\lambda_{0}=0$ and for $n=0,1,2,\ldots$
\[
\lambda_{4n+1}=(2m\pi+\alpha)^{2};\ \ \lambda_{4n+2}=(2(n+1)\pi-\alpha))^{2};
\]%
\[
\ \lambda_{4n+3}\ =\ \ \lambda_{4n+4}=(2(n+1)\pi))^{2}.
\]
where $\alpha\,=\,\cos^{-1}(-7/8)$

\textbf{Semi-periodic boundary conditions on }$($\textbf{ }$-1/2,+1/2)$%
\textbf{, }\textit{i.e.}%
\[
y(-1/2)=-y(+1/2)\quad\quad y^{\prime}(-1/2)=-y^{\prime}(+1/2).
\]

With $\beta=\cos^{-1}((1+\sqrt{(}33))/16)$ and $\gamma=\cos^{-1}((1-\sqrt
{(}33))/16)$ these are all simple and given by, for $n=0,1,2,\ldots$
\[
\lambda_{4n}\,=\,(2n\pi\,+\,\beta)^{2};\ \lambda_{4n+1}\,=\,(2n\pi
\,+\,\gamma)^{2};\ \
\]%
\[
\lambda_{4n+2}\,=\,(2(n+1)\pi\,-\,\gamma)^{2};\text{ }\lambda_{4n+3}%
\,=\,(2(n+1)\pi\,-\,\beta)^{2}.
\]
See \cite{E} and \cite{Hoc}.

\item \textbf{The Lohner equation}%
\[
-y^{\prime\prime}(x)-1000xy(x)=\lambda y(x)\;\text{for all}\;x\in
(-\infty,+\infty)
\]

Endpoint classification in the space $L^{2}(-\infty,+\infty)$:%
\[%
\begin{tabular}
[c]{cc}\hline
Endpoint & Classification\\\hline
$-\infty$ & LP\\
$+\infty$ & LP\\\hline
\end{tabular}
\]

In \cite{L} Lohner computed the Dirichlet eigenvalues of index (in SLEIGN2
notation) 0, 9, 49 and 99 using interval arithmetic and obtained rigorous
bounds. In double precision SLEIGN2 computed eigenvalues are in good agreement
with these guaranteed bounds.

\item \textbf{The J\"{o}rgens equation}%
\[
-y^{\prime\prime}(x)+(\exp(2x)/4-k\exp(x))y(x)=\lambda y(x)\;\text{for
all}\;x\in(-\infty,+\infty)
\]
where the parameter $k\in(-\infty,+\infty).$

Endpoint classification in the space $L^{2}(-\infty,+\infty),$ for all
$k\in(-\infty,+\infty)$:%
\[%
\begin{tabular}
[c]{cc}\hline
Endpoint & Classification\\\hline
$-\infty$ & LP\\
$+\infty$ & LP\\\hline
\end{tabular}
\]

This is a remarkable example from J\"{o}rgens and SLEIGN2 obtains excellent
results. Details of this problem are given in \cite[Part II, Section 10]{J}.
For all $k\in(-\infty,+\infty)$ the boundary value problem on the interval
$(-\infty,+\infty)$ has a continuous spectrum on $[0,+\infty)$; for $k\leq1/2$
there are no eigenvalues; for $h=0,1,2,3,\ldots$ and then $k$ chosen by
$h<k-1/2\leq h+1,$ there are exactly $h+1$ eigenvalues and these are all below
the continuous spectrum; these eigenvalues are given explicitly by
\[
\lambda_{n}=-(k-1/2-n)^{2},\quad n=0,1,2,3,\ldots,h.
\]

\item \textbf{The Behnke-Goerisch equation}%
\[
-y^{\prime\prime}(x)+k\cos^{2}(x)y(x)=\lambda y(x)\;\text{for all}%
\;x\in(-\infty,+\infty)
\]
where the parameter $k\in(-\infty,+\infty),$

Endpoint classification in the space $L^{2}(-\infty,+\infty),$ for all
$k\in(-\infty,+\infty)$:%
\[%
\begin{tabular}
[c]{cc}\hline
Endpoint & Classification\\\hline
$-\infty$ & LP\\
$+\infty$ & LP\\\hline
\end{tabular}
\]

This is a form of the Mathieu equation. In \cite{BG} these authors computed a
number of Neumann eigenvalues of this problem using interval arithmetic with
rigorous bounds. In double precision SLEIGN2 computed eigenvalues are in good
agreement with these guaranteed bounds.

\item \textbf{The Whittaker equation}%
\[
-y^{\prime\prime}(x)+\left(  \frac{1}{4}+\frac{k^{2}-1}{x^{2}}\right)
y(x)=\lambda\frac{1}{x}y(x)\;\text{for all}\;x\in(0,+\infty)
\]
where the parameter $k\in\lbrack1,+\infty).$

Endpoint classification in the space $L^{2}(0,+\infty),$ for all $k\in
\lbrack1,+\infty)$:%
\[%
\begin{tabular}
[c]{cc}\hline
Endpoint & Classification\\\hline
$0$ & LP\\
$+\infty$ & LP\\\hline
\end{tabular}
\]

This equation is studied in \cite[Part II, Section 10]{J}. There it is shown
that the LP case holds at $+\infty$ and also at $0$ for $k\geq1$. The spectrum
is discrete and is given explicitly by:
\[
\lambda_{n}=n+(k+1)/2,\quad n=0,1,2,3,\ldots.
\]

\item \textbf{The Littlewood-McLeod equation}%
\[
-y^{\prime\prime}(x)+x\sin(x)y(x)=\lambda y(x)\;\text{for all}\;x\in
\lbrack0,+\infty).
\]

Endpoint classification in the space $L^{2}(0,+\infty)$:%
\[%
\begin{tabular}
[c]{cc}\hline
Endpoint & Classification\\\hline
$0$ & R\\
$+\infty$ & LP\\\hline
\end{tabular}
\]

The spectral analysis of this differential equation is considered in
\cite{JEL} and \cite{JBM}; the equation is R at $0$ and LP at $\pm\infty.$ All
self-adjoint operators in $L^{2}[0,\infty)$ have a simple, discrete spectrum
$\{\lambda_{n}:n=0,\pm1,\pm2,\ldots\}$ that is unbounded both above and below,
\textit{i.e.}%
\[
\lim_{n\rightarrow-\infty}\lambda_{n}=-\infty\quad\quad\quad\lim
_{n\rightarrow+\infty}\lambda_{n}=+\infty.
\]
Every eigenfunction has infinitely many zeros in $(0,\infty).$

SLEIGN2, and other codes, fail to compute the eigenvalues for this type of LP
oscillatory problem. However there is qualitative information to be obtained
by considering regular problems on $[0,X]$ with, say, Dirichlet boundary
conditions $y(0)=Y(X)=0.$

\item \textbf{The Morse equation}%
\[
-y^{\prime\prime}(x)+(9\exp(-2x)-18\exp(-x))y(x)=\lambda y(x)\;\text{for
all}\;x\in(-\infty,+\infty)
\]

Endpoint classification in the space $L^{2}(-\infty,+\infty)$%
\[%
\begin{tabular}
[c]{cc}\hline
Endpoint & Classification\\\hline
$-\infty$ & LP\\
$+\infty$ & LP\\\hline
\end{tabular}
\]

This differential equation is studied in \cite[Example 6]{PBB1}; the spectrum
has exactly three negative, simple eigenvalues, and a continuous spectrum on
$[0,\infty);$ the eigenvalues are given explicitly by%
\[
\lambda_{n}=-(n-2.5)^{2}\;\text{for}\;n=0,1,2.
\]

\item \textbf{The Heun equation}%
\[
-(py^{\prime})^{\prime}+qy=\lambda wy\;\text{on}\;(0,1)
\]
where the coefficients $p,q,w$ are given explicitly by, for all $x\in(0,1),$%
\[%
\begin{array}
[c]{l}%
p(x)=x^{c}(1-x)^{d}(x+s)^{e}\\
q(x)=abx^{c}(1-x)^{d-1}(x+s)^{e-1}\\
w(x)=x^{c-1}(1-x)^{d-1}(x+s)^{e-1}.
\end{array}
\]
The parameters $a,b,c,d,e$ and $s$ are all real numbers and satisfy the
following two conditions%
\[
(i)\;s>0\;\text{and}\;c\geq1,d\geq1,a\geq b
\]
and%
\[
(ii)\;a+b+1-c-d-e=0.
\]
From these conditions it follows that%
\[
a\geq1,b\geq1,e\geq1\;\text{and}\;a+b-d\geq1.
\]

The differential equation above is a special case of the general Heun equation%
\[
\frac{d^{2}w(z)}{dz^{2}}+\left(  \frac{\gamma}{z}+\frac{\delta}{z-1}%
+\frac{\varepsilon}{z-a}\right)  \frac{dw(z)}{dz}+\frac{\alpha\beta
z-q}{z(z-1)(z-a)}w(z)=0
\]
with the general parameters $\alpha,\beta,\gamma,\delta,\varepsilon$ replaced
by the real numbers $a,b,c,d,e,$ $a$ replaced by $-s,$ and $q$ replaced by the
spectral parameter $\lambda.$ For general information concerning the Heun
equation see the compendium \cite{AR1}; for the special form of the Heun
equation considered here, and for the connection with confluence of
singularities and applications, see the recent paper \cite{LS1}.

We note that the coefficients of the Sturm-Liouville differential equation
above satisfy the conditions

\begin{itemize}
\item[$(i)$] $q,w\in C[0,1]$ and $w(x)>0$ for all $x\in(0,1)$

\item[$(ii)$] $p^{-1}\in L_{\text{loc}}^{1}(0,1),p(x)>0$ for all $x\in(0,1)$

\item[$(iii)$] $p^{-1}\notin L^{1}(0,1/2]$ and $p^{-1}\notin L^{1}[1/2,1).$
\end{itemize}

Thus both endpoints $0$ and $1$ are singular for the differential equation.
Analysis shows that the endpoint classification for this equation is%
\[%
\begin{tabular}
[c]{ccc}\hline
Endpoint & Parameter & Classification\\\hline
$0$ & $c\in\lbrack1,2)$ & LCNO\\
$0$ & $c\in\lbrack2,+\infty)$ & LP\\
$1$ & $d\in\lbrack1,2)$ & LCNO\\
$1$ & $d\in\lbrack2,+\infty)$ & LP\\\hline
\end{tabular}
\ \
\]

For the endpoint $0$ and for LCNO cases the boundary condition functions $u,v$
are determined by:%
\[%
\begin{tabular}
[c]{ccc}\hline
Parameter & $u$ & $v$\\\hline
$c=1$ & $1$ & $\ln(x)$\\
$1<c<2$ & $1$ & $x^{1-c}$\\\hline
\end{tabular}
\]

For the endpoint $1$ and for LCNO cases the boundary condition functions $u,v$
are determined by:%
\[%
\begin{tabular}
[c]{ccc}\hline
Parameter & $u$ & $v$\\\hline
$d=1$ & $1$ & $\ln(1-x)$\\
$1<d<2$ & $1$ & $(1-x)^{1-d}$\\\hline
\end{tabular}
\]

Further it may be shown that the spectrum of any self-adjoint problem on
$(0,1),$ with the parameters $a,b,c,d,e$ and $s$ satisfying the above
conditions, and considered in the space $L^{2}((0,1);w)$ with either separated
or coupled boundary conditions, is bounded below and discrete. For the
analytic properties, and proofs of the spectral properties of this Heun
differential equation, see the paper \cite{BEHZ}.

In xamples.f, under Example 32, the user can enter a choice for the parameters
$a,b,c,d,e$ and $s,$ subject to the conditions above being satisfied, and
obtain numerical results for the eigenvalues.
\end{enumerate}

\begin{thebibliography}{9}                                                                                                %

\bibitem {AB}M. Abramowitz and I.A. Stegun, \emph{Handbook of mathematical
functions}, Dover Publications, Inc., New York, 1972.

\bibitem {AEZ}F.V. Atkinson, W.N. Everitt and A. Zettl, \emph{Regularization
of a Sturm-Liouville problem with an interior singularity using
quasi-derivatives} Diff. and Int. Equations 1 (1988), 213-222.

\bibitem {PBB1}P.B. Bailey, \emph{SLEIGN}:\emph{ An eigenvalue-eigenfunction
code for Sturm-Liouville problems.} Report Sand77-2044, Sandia National
Laboratory, New Mexico, USA: 1978.

\bibitem {BEHZ}P.B. Bailey, W.N. Everitt, D.B. Hinton and A. Zettl, \emph{Some
spectral properties of the Heun differential equation.} (Submitted for
publication: 2001.)

\bibitem {BEZ}P.B. Bailey, W.N. Everitt and A. Zettl, \emph{Computing
eigenvalues of singular Sturm-Liouville problems}, Results in Mathematics,
\textbf{20 }(1991), 391-423.

\bibitem {BEZ2}P.B. Bailey, W.N. Everitt and A. Zettl, \emph{Regular and
singular Sturm-Liouville problems with coupled boundary conditions}, Proc.
Royal Soc. Edinburgh (A) \textbf{126} (1996), 505-514.

\bibitem {BEZ3}P.B. Bailey, W.N. Everitt and A. Zettl, \emph{The SLEIGN2
Sturm-Liouville code,} ACM Trans. Math. Software. (To appear.)

\bibitem {BEWS}P.B. Bailey, W.N. Everitt, J. Weidmann and A. Zettl,
\emph{Regular approximation of singular Sturm-Liouville problems}. Results in
Mathematics \textbf{23} (1993), 3-22.

\bibitem {BG}H. Behnke and F. Goerisch, \emph{Inclusions for eigenvalues of
self-adjoint problems}, J. Herzberger (Hrsg.): Topics in Validated
Computation, North Holland Elsevier, Amsterdam, (1994).

\bibitem {B}J.P. Boyd, \emph{Sturm-Liouville eigenvalue problems with an
interior pole}, J. Math. Physics, \textbf{22 }(1981), 1575-1590.

\bibitem {DS}N. Dunford and J.T. Schwartz, \emph{Linear Operators, part II},
Interscience Publishers, New York, 1963.

\bibitem {E}M.S.P. Eastham, \emph{The spectral theory of periodic differential
equations}, Scottish Academic Press, Edinburgh and London, 1973.

\bibitem {EGZ}W.N. Everitt, J. Gunson and A. Zettl, \emph{Some comments on
Sturm-Liouville eigenvalue problems with interior singularities}, J. Appl.
Math. Phys. (ZAMP) \textbf{38} (1987), 813-838.

\bibitem {F}G. Fichera, \emph{Numerical and quantitative analysis}, Pitman
Press, London, 1978.

\bibitem {FP1}C.T. Fulton and S. Pruess. `Mathematical software for
Sturm-Liouville problems.' \emph{NSF Final Report for Grants DMS88 and
DMS88-00839; }1993.

\bibitem {H}M.S. Homer, \emph{Boundary value problems for the Laplace tidal
wave equation}, Proc. Roy. Soc. of London (A) \textbf{428} (1990), 157-180.

\bibitem {Hoc}H. Hochstadt, \emph{A special Hill's equation with discontinuous
coefficients}, Amer. Math. Monthly, \textbf{70 }(1963), 18-26.

\bibitem {J}K. J\"{o}rgens, \emph{Spectral theory of second-order ordinary
differential operators}, Lecture Notes: Series no.2, Matematisk Institut,
Aarhus Universitet, 1962/63.

\bibitem {K}A.M. Krall, \emph{Boundary value problems for an eigenvalue
problem with a singular potential}, J. Diff. Equations, \textbf{45} (1982), 128-138.

\bibitem {LS1}W. Lay and S. Yu. Slavyanov, \emph{Heun's equation with nearby
singularities.} Proc. R. Soc. Lond. A \textbf{455} (1999), 4374-4261.

\bibitem {JEL}J.E. Littlewood, \emph{On linear differential equations of the
second order with a strongly oscillating coefficient of }$y$, J. London Math.
Soc. \textbf{41} (1966), 627-638.

\bibitem {L}R.J. Lohner, \emph{Verified solution of eigenvalue problems in
ordinary differential equations}. (Personal communication, 1995.)

\bibitem {M}M. Marletta, \emph{Numerical tests of the SLEIGN software for
Sturm-Liouville problems}, ACM TOMS, \textbf{17 }(1991), 501-503.

\bibitem {JBM}J.B. McLeod, \emph{Some examples of wildly oscillating
potentials}, J. London Math. Soc. \textbf{43} (1968), 647-654.

\bibitem {PMM}P.M. Morse, \emph{Diatomic molecules according to the wave
mechanics}; II: \emph{Vibration levels. }Phys. Rev. \textbf{34} (1929), 57-61.

\bibitem {NZ}H.-D. Niessen and A. Zettl, \emph{Singular Sturm-Liouville
problems; the Friedrichs extension and comparison of eigenvalues}, Proc.
London Math. Soc. \textbf{64} (1992), 545-578.

\bibitem {JDP}J.D. Pryce. \emph{A test package for Sturm-Liouville solvers.}
ACM Trans. Math. Software \textbf{25} (1999), 21-57.

\bibitem {P}M. Plum, \emph{Eigenvalue inclusions for second-order ordinary
differential operators by a numerical homotopy method}, ZAMP, \textbf{41
}(1990), 205-226.

\bibitem {AR1}A. Ronveaux, \emph{Heun differential equations.} (Oxford
University Press; 1995.)

\bibitem {ST}D.B. Sears and E.C. Titchmarsh, \emph{Some eigenfunction
formulae}, Quart. J. Math. Oxford (2) \textbf{1} (1950), 165-175.

\bibitem {T}E.C. Titchmarsh, \emph{Eigenfunction expansions associated with
second- order differential equations}: \textbf{I}. Clarendon Press, Oxford; 1962.

\bibitem {W}G.N. Watson, \emph{A treatise on the theory of Bessel functions},
Cambridge University Press, Cambridge, England, 1958.

\bibitem {Z}A. Zettl, \emph{Computing continuous spectrum}, Proc.
International Symposium, Trends and Developments in Ordinary Differential
Equations, Y. Alavi and P. Hsieh editors, World Scientific, (1994), 393-406.
\end{thebibliography}
\end{document}
