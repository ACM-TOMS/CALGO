%% This document created by Scientific Word (R) Version 3.5

\documentclass[12pt]{amsart}%
\usepackage{graphicx}
\usepackage{amscd}
\usepackage{amsmath}%
\usepackage{amsfonts}%
\usepackage{amssymb}
%TCIDATA{OutputFilter=latex2.dll}
%TCIDATA{CSTFile=amsartci.cst}
%TCIDATA{Created=Tue Sep 07 11:15:04 1999}
%TCIDATA{LastRevised=Sunday, April 01, 2001 11:19:55}
%TCIDATA{<META NAME="GraphicsSave" CONTENT="32">}
%TCIDATA{<META NAME="DocumentShell" CONTENT="Journal Articles\amsnumb">}
%TCIDATA{Language=British English}
\newtheorem{theorem}{Theorem}
\theoremstyle{plain}
\newtheorem{acknowledgement}{Acknowledgement}
\newtheorem{algorithm}{Algorithm}[section]
\newtheorem{axiom}{Axiom}[section]
\newtheorem{case}{Case}[section]
\newtheorem{claim}{Claim}[section]
\newtheorem{conclusion}{Conclusion}[section]
\newtheorem{condition}{Condition}[section]
\newtheorem{conjecture}{Conjecture}[section]
\newtheorem{corollary}{Corollary}[section]
\newtheorem{criterion}{Criterion}[section]
\newtheorem{definition}{Definition}[section]
\newtheorem{example}{Example}[section]
\newtheorem{exercise}{Exercise}[section]
\newtheorem{lemma}{Lemma}[section]
\newtheorem{notation}{Notation}[section]
\newtheorem{problem}{Problem}[section]
\newtheorem{proposition}{Proposition}[section]
\newtheorem{remark}{Remark}[section]
\newtheorem{solution}{Solution}[section]
\newtheorem{summary}{Summary}[section]
\numberwithin{equation}{section}
\numberwithin{theorem}{section}
\newcommand{\thmref}[1]{Theorem~\ref{#1}}
\newcommand{\secref}[1]{\S\ref{#1}}
\newcommand{\lemref}[1]{Lemma~\ref{#1}}
\setlength{\oddsidemargin}{0.0in}
\setlength{\evensidemargin}{0.0in}
\setlength{\textwidth}{6.5in}
\setlength{\textheight}{8.5in}
\setlength{\headsep}{0.25in}
\setlength{\headheight}{0.0in}


\begin{document}
\title[HELP file]{Sleign2: the HELP file}
\author{P.B. Bailey}
\author{W.N. Everitt}
\author{A. Zettl}
\address{Department of Mathematical Sciences, Northern Illinois University, DeKalb, IL
60115-2888, USA}
\date{01 March 2001 (File: help.tex)}
\maketitle


This copy of the HELP data has been written in AMS-LaTeX in view of the
considerable amount of mathematical formulae involved in the text. The user
should note that when HELP is accessed in MAKEPQW and/or DRIVE the
corresponding data is given in Fortran notation. The best use of this HELP
data can be made by printing out a copy of this AMS-LaTeX file to have
available when the code files are in use.

HELP may be called at any point where the program halts and displays (h?), by
pressing ``h
%TCIMACRO{\TEXTsymbol{<}}%
%BeginExpansion
$<$%
%EndExpansion
ENTER%
%TCIMACRO{\TEXTsymbol{>}}%
%BeginExpansion
$>$%
%EndExpansion
''. To RETURN from HELP, press ``r
%TCIMACRO{\TEXTsymbol{<}}%
%BeginExpansion
$<$%
%EndExpansion
ENTER%
%TCIMACRO{\TEXTsymbol{>}}%
%BeginExpansion
$>$%
%EndExpansion
''. To QUIT at any program halt, press ``q
%TCIMACRO{\TEXTsymbol{<}}%
%BeginExpansion
$<$%
%EndExpansion
ENTER%
%TCIMACRO{\TEXTsymbol{>}}%
%BeginExpansion
$>$%
%EndExpansion
''.

This AMS-LaTeX file is supplied as a separate text file within the SLEIGN2
package; it can be accessed on-line in both the MAKEPQW (if used) and DRIVE files.

HELP contains information to aid the user in entering data: on the coefficient
functions $p,q,w;$ on the self-adjoint, separated and coupled, regular and
singular, boundary conditions; on the limit-circle boundary condition
functions $u,v$ at the endpoint $a$ and $U,V$ at the endpoint $b$ of the
interval $(a,b)$; on the endpoint classifications of the differential
equation; on DEFAULT entry; on eigenvalue indexes; on IFLAG information; and
on the general use of the program SLEIGN2.

\medskip

The 17 sections of HELP are:

H1: Overview of HELP. H2: File name entry. H3: The differential equation. H4:
endpoint classification. H5: DEFAULT entry. H6: Self-adjoint limit-circle
boundary conditions. H7: General self-adjoint boundary conditions. H8:
Recording the results. H9: Type and choice of interval. H10: Entry of
endpoints. H11: endpoint values of $p,q,w$. H12: Initial value problems. H13:
Indexing of eigenvalues. H14: Entry of eigenvalue index, initial guess, and
tolerance. H15: IFLAG information. H16: Plotting. H17: Indexing of eigenvalues.

HELP can be accessed at each point in MAKEPQW and DRIVE where the user is
asked for input, by pressing ``h
%TCIMACRO{\TEXTsymbol{<}}%
%BeginExpansion
$<$%
%EndExpansion
ENTER%
%TCIMACRO{\TEXTsymbol{>}}%
%BeginExpansion
$>$%
%EndExpansion
''; this places the user at the appropriate HELP section. Once in HELP, the
user can scroll the further HELP sections by repeatedly pressing ``h
%TCIMACRO{\TEXTsymbol{<}}%
%BeginExpansion
$<$%
%EndExpansion
ENTER%
%TCIMACRO{\TEXTsymbol{>}}%
%BeginExpansion
$>$%
%EndExpansion
'', or jump to a specific HELP section Hn (n =1,2,...17) by typing ``hn
%TCIMACRO{\TEXTsymbol{<}}%
%BeginExpansion
$<$%
%EndExpansion
ENTER%
%TCIMACRO{\TEXTsymbol{>}}%
%BeginExpansion
$>$%
%EndExpansion
''; to return to the place in the program from which HELP is called, press
``r
%TCIMACRO{\TEXTsymbol{<}}%
%BeginExpansion
$<$%
%EndExpansion
ENTER%
%TCIMACRO{\TEXTsymbol{>}}%
%BeginExpansion
$>$%
%EndExpansion
''.

\medskip

H2: File name entry.

MAKEPQW is used to create a Fortran file containing the coefficients $p,q,w$
defining the differential equation, and the boundary condition functions $u,v$
and $U,V$ if required. The file must be given a NEW filename which is
acceptable to your Fortran compiler. For example, it might be called bessel.f
or bessel.for depending upon your compiler.

The same naming considerations apply if the Fortran file is prepared other
than with the use of MAKEPQW.

\medskip

H3: The differential equation.

The prompt ``Input $p$ (or $q$ or $w$) ='' requests you to type in a Fortran
expression defining the function $p$, which is one of the three coefficient
functions defining the Sturm-Liouville differential equation%
\begin{equation}
-(py^{\prime})^{\prime}+qy=\lambda wy \tag{*}%
\end{equation}
to be considered on some interval $(a,b)$ of the real line. The actual
interval used in a particular problem can be chosen later, and may be either
the whole interval $(a,b)$ where the coefficient functions $p,q,w$ are
defined, or on any sub-interval $(a^{\prime},b^{\prime})$ of $(a,b)$;
$a=\infty$ and/or $b=+\infty$ are allowable choices for the endpoints.

The coefficient functions $p,q,w$ of the differential equation may be chosen
arbitrarily but must satisfy the following conditions:

(1) $p,q,w$ are real-valued throughout $(a,b)$.

(2) $p,q,w$ are piece-wise continuous and defined throughout the interior of
the interval $(a,b)$.

(3) $p$ and $w$ are strictly positive in $(a,b)$.

\noindent For better error analysis in the numerical procedures, condition (2)
above is often replaced with

(2$^{\prime}$) $p,q,w$ are four times continuously differentiable on $(a,b)$.

\noindent The behaviour of $p,q,w$ near the endpoints $a$ and $b$ is critical
to the classification of the differential equation (see H4 and H11).

\medskip

H4: endpoint classification.

The correct classification of the endpoints $a$ and $b$ is essential to the
working of the SLEIGN2 program. To classify the endpoints, it is convenient to
choose a point $c$ in $(a,b)$; \textit{i.e.} $a<c<b$. \quad\quad$\quad
\;\;$Subject to the general conditions on the coefficient functions $p,q,w$
(see H3):

(1) a is REGULAR (say R) if $-\infty<a,$ $p,q,w$ are piece-wise continuous on
$[a,c]$, and $p(a)>0,w(a)>0.$

(2) $a$ is WEAKLY REGULAR (say WR) if $-\infty<a,$ $a$ is not R and%
\[
\int_{a}^{c}\left\{  p^{-1}+\left|  q\right|  +w\right\}  <+\infty.
\]

If endpoint $a$ is neither R nor WR, then $a$ is SINGULAR; that is, either
$-\infty=a,$ or $-\infty<a$ and%
\[
\int_{a}^{c}\left\{  p^{-1}+\left|  q\right|  +w\right\}  =+\infty.
\]

(3) The SINGULAR endpoint $a$ is LIMIT-CIRCLE NON-OSCILLATORY (say LCNO) if
for some real value of the spectral parameter $\lambda$ ALL real-valued
solutions $y$ of the differential equation%
\[
-(py^{\prime})^{\prime}+qy=\lambda wy\;\text{on}\;(a,c]
\]

satisfy the conditions:%
\[
\int_{a}^{c}w\left|  y\right|  ^{2}<+\infty
\]

\noindent and $y$ has at most a finite number of zeros in $(a,c].$

(4) The SINGULAR endpoint a is LIMIT-CIRCLE OSCILLATORY (say LCO) if for some
real $\lambda$ ALL real-valued solutions of the differential equation (*)
satisfy the conditions:%
\[
\int_{a}^{c}w\left|  y\right|  ^{2}<+\infty
\]

\noindent and $y$ has an infinite number of zeros in $(a,c].$

(5) The SINGULAR endpoint a is LIMIT POINT (say LP) if for some real $\lambda$
at least one solution of the differential equation $(\ast)$ satisfies the
condition:%
\[
\int_{a}^{c}w\left|  y\right|  ^{2}=+\infty.
\]

There is a similar classification of the endpoint $b$ into one of the five
distinct cases R, WR, LCNO, LCO, LP.

Although the classification of singular endpoints invokes a real value of the
parameter $\lambda$, this classification is invariant in $\lambda$; all real
choices of $\lambda$ lead to the same classification.

In determining the classification of singular endpoints for the differential
equation $(\ast)$, it is often convenient to start with the choice $\lambda=0$
in attempting to find solutions (particularly when $q=0$ on $(a,b)$); however,
see example 7 below.

See H6 on the use of maximal domain functions to determine the classification
at singular endpoints.

EXAMPLES:

1. $-y^{\prime\prime}=\lambda y$ is R at both endpoints of $(a,b)$ when $a$
and $b$ are finite.

2. $-y^{\prime\prime}=\lambda y$ on $(-\infty,+\infty)$ is LP at both endpoints.

3. $-\left(  x^{1/2}y^{\prime}(x)\right)  ^{\prime}=\lambda x^{-1/2}y(x)$ for
all$\;x\in(0,+\infty)$ is WR at $0$ and LP at $+\infty$ (take $\lambda=0$ in
$(\ast)$). See the file xamples.f, \#10 (Weakly Regular).

4. $-((1-x^{2})y^{\prime}(x))^{\prime}=\lambda y(x)\;$for all$\;x\in(-1,+1)$
is LCNO at both ends (take $\lambda=0$ in $(\ast)$). See xamples.f, \#1 (Legendre).

5. $-y^{\prime\prime}(x)+Cx^{-2}y(x)=\lambda y(x)\;$for all$\;x\in(0,+\infty)$
is LP at $+\infty;$ however at $0$ the equation illustrates a number of
different classifications(take $\lambda=0$ in $(\ast)$) as follows:

LP $C\geq3/4$ ; LCNO for $-1/4\leq C<3/4$ (but $C\neq0$); LCO for $C<-1/4.$

6. $-(xy^{\prime}(x))^{\prime}-x^{-1}y(x)=\lambda y(x)\;$for all$\;x\in
(0,\infty)$ is LCO at $0$ and LP $+\infty$ (take $\lambda=0$ in $(\ast)$ with
solutions $\cos(\ln(x))$ and $\sin(\ln(x))$. See xamples.f, \#7 (BEZ).

7. $-(xy^{\prime}(x))^{\prime}-xy(x)=\lambda x^{-1}y(x)\;$for all$\;x\in
(0,+\infty)$ is LP at $0$ and LCO at $+\infty$ (take $\lambda=-1/4$ in
$(\ast)$ with solutions $x^{-1/2}\cos(x)$ and $x^{-1/2}\sin(x)$). See
xamples.f, \#6 (Sears-Titchmarsh).

8. $-y^{\prime\prime}(x)+x\sin(x)y(x)=\lambda y(x)$ on $(0,+\infty)$ is R at
$0$ and LP at $+\infty.$ See xamples.f \#30 (Littlewood-McLeod).

\medskip

H5: DEFAULT entry.

The complete range of problems for which SLEIGN2 is applicable can only be
reached by appropriate entries under endpoint classification and boundary
conditions. However, there is a DEFAULT application which requires no detailed
entry of endpoint classification or boundary conditions, subject to:

1) The DEFAULT application CANNOT be used at a LCO endpoint.

2) If an endpoint $a$ is R, then the Dirichlet boundary condition $y(a)=0$ is
automatically used.

3) If an endpoint $a$ is WR, then the following boundary condition is
automatically applied:

(i) if $p(a)=0$, and both $q$ and $w$ are bounded near $a$, then the Dirichlet
boundary condition $y(a)=0$ is used

(ii) if $p(a)>0$, and $q$ and/or $w$ are not bounded near $a$, then the
Neumann boundary condition $y^{\prime}(a)=0$ is used.

If $p(a)=0$, and $q$ and/or $w$ are not bounded near $a$, then no reliable
information, in general, can be given on the DEFAULT boundary condition.

4) If an endpoint is LCNO, then in most cases the principal or Friedrichs
boundary condition is applied (see H6).

5) If an endpoint is LP, then the normal LP procedure is applied (see H7(1.)).

If you choose the DEFAULT condition, then no entry is required for the $u,v$
and $U,V$ boundary condition functions.

\medskip

H6: Limit-circle (LC) boundary conditions.

At an endpoint $a$, the LC type separated boundary condition is of the form
(similar remarks throughout apply to the endpoint $b$ with $U,V$ being
boundary condition functions at $b$)%
\begin{equation}
A1[y,u](a)+A2[y,v](a)=0 \tag{**}%
\end{equation}

\noindent where $y$ is a solution of the differential equation%
\begin{equation}
-(py^{\prime})^{\prime}+qy=\lambda wy\;\text{on}\;(a,b) \tag{*}%
\end{equation}

Here $A1,A2$ are real numbers, not both zero; $u$ and $v$ are boundary
condition functions at $a$; and for real-valued $y$ and $u$ the form
$[y,u](\cdot)$ is defined by%
\[
\lbrack y,u](x)=y(x)(pu^{\prime})(x)-(py^{\prime})(x)u(x)\;\text{for
all}\;x\in(a,b).
\]

If neither endpoint is LP then there are also self-adjoint coupled boundary
conditions. These have a canonical form given by%
\[
Y(b)=\exp(i\alpha)\mathbf{K}Y(a)
\]

\noindent where

(i) $\mathbf{K}$ is a real $2\times2$ matrix with $\det(\mathbf{K})=1$

(ii) the parameter $\alpha$ is restricted to $\alpha\in(-\pi,\pi]$

(iii) $Y$ is the solution column vector $Y(a)=[y(a),(py^{\prime})(a)]^{T}$ at
a regular R endpoint $a$, and $Y$ is the ``singular solution vector''
$Y(a)=[[y,u](a),[y,v](a)]^{T}$ at a singular LC endpoint $a$. Similarly at the
right endpoint $b$ with $U,V.$

The object of this section is to provide help in choosing appropriate
functions $u$ and $v$ in $(\ast\ast)$ (or in choosing $U,V$) given the
differential equation $(\ast)$. Full details of the boundary conditions for
$(\ast)$ are discussed in H7; here it is sufficient to say that the
limit-circle type boundary condition $(\ast\ast)$ must be applied at any
endpoint in the LCNO, LCO classification, but can also be used in the R, WR
classification subject to the appropriate choice of $u$ and $v,$ and $U$ and $V.$

Let $(\ast)$ be R, WR, LCNO, or LCO at endpoint $a$ and choose $c$ in
$(a,b).$. Then \textit{either} $u$ and $v$ are a pair of linearly independent
real solutions of $(\ast)$ on $(a,c]$ for any chosen real value of $\lambda$,
\textit{or }$u$ and $v$ are a pair of real-valued maximal domain functions
defined on $(a,c]$ satisfying $[u,v](a)\neq0.$

The maximal domain $D(a,b)$ is defined by%
\[%
\begin{array}
[c]{lll}%
D(a,b)= & \{f:(a,b)\rightarrow\mathbb{C}: & (i)\;f\text{ and }pf^{\prime}\in
AC_{\text{loc}}(a,b)\\
&  & (ii)\;f\text{ and }w^{-1}(-(pf^{\prime})^{\prime}+qf)\in L^{2}%
((a,b);w)\}.
\end{array}
\]
The domains $D(a,c]$ and $D[c,b)$ are the restrictions of the functions
$D(a,b)$ to the sub-intervals. It is known that for all $f,g\in D(a,c]$ the
limit%
\[
\lbrack f,g](a)=\lim_{x\rightarrow a}[f,g](x)
\]

\noindent exists and is finite. If $(\ast)$ is LCNO or LCO at $a$, then all
solutions of (*) belong to $D(a,c]$ for all values of $\lambda.$ The boundary
condition $(\ast\ast)$ is essential in the LCNO and LCO cases but can also be
used with advantage in some R and WR cases.

In the R, WR, and LCNO cases, but not in the LCO case, the boundary condition
functions can always be chosen so that%
\[
\lim_{x\rightarrow a}\frac{u(x)}{v(x)}=0
\]
and it is recommended that this normalisation be effected, but this is not
essential; this normalization has been entered in the examples given below. In
this case, the boundary condition $[y,u](a)=0$ (\textit{i.e. }$A1=1,A2=0$ in
$(\ast\ast)$) is called the principal or Friedrichs boundary condition at $a$.

In the case when endpoints $a$ and $b$ are, independently, in the R, WR, LCNO,
or LCO classification, it may be that symmetry or other reasons permit one set
of boundary condition functions to be used at both end-points (see xamples.f,
\#1 (Legendre)). In other cases, different pairs must be chosen for each
endpoint (see xamples.f: \#16 (Jacobi), \#18 (Dunsch), and \#19 (Donsch)).

Note that a solution pair $u,v$ is always a maximal domain pair, but not
necessarily vice versa. EXAMPLES:

1. $-y^{\prime\prime}=\lambda y$ on $[0,\pi]$ is R at $0$ and R at $\pi.$ At
$0$, with $\lambda=0$, a solution pair is
\[
u(x)=x,v(x)=1\;\text{for all}\;x\in\lbrack0,\pi].
\]
At $\pi,$with $\lambda=1$, a solution pair is%
\[
U(x)=\sin(x),V(x)=\cos(x)\;\text{for all}\;x\in\lbrack0,\pi].
\]

2. $-(x^{1/2}y^{\prime}(x))^{\prime}=\lambda x^{-1/2}y(x)$ on $(0,1]$ is WR at
$0$ and R at $1.$ (The general solutions of this equation are $u(x)=\cos
(2x^{1/2}\sqrt{\lambda})$ and $v(x)=\sin(2x^{1/2}\sqrt{\lambda})$.)

At $0$, $\lambda=0$, a solution pair is%
\[
u(x)=2x^{1/2},v(x)=1.
\]
At $1$, with $\lambda=\pi^{2}/4$, a solution pair is%
\[
U(x)=\sin(\pi x^{1/2}),V(x)=\cos(\pi x^{1/2}).
\]
Also at 1, with $\lambda=0$, a solution pair is%
\[
U(x)=2(1-x^{1/2}),V(x)=1.
\]

See also xamples.f, \#10 (Weakly Regular).

3. $-((1-x^{2})y^{\prime}(x))^{\prime}=\lambda y(x)$ on $(-1,+1)$ is LCNO at
both endpoints.

At both $\pm1$, $\lambda=0$, a solution pair is%
\[
u(x)=1,v(x)=\ln((1+x)/(1-x))/2
\]

At $+1$, a maximal domain pair is $U(x)=1,V(x)=\ln(1-x)$. At $-1$, a maximal
domain pair is $u(x)=1,v(x)=\ln(1+x).$

See also xamples.f, \#1 (Legendre).

4. $-y^{\prime\prime}(x)-(4x^{2})^{-1}y(x)=\lambda y(x)$ on $(0,+\infty)$ is
LCNO at $0$ and LP at $+\infty$.

At $0$, a maximal domain pair is%
\[
u(x)=x^{1/2},v(x)=x^{1/2}\ln(x).
\]

See also xamples.f, \#2 (Bessel).

5. $-y^{\prime\prime}(x)-5(4x^{2})^{-1}y(x)=\lambda y(x)$ on $(0,+\infty)$ LCO
at $0$ and LP at $+\infty.$

At $0$, $\lambda=0$, a solution pair is%
\[
u(x)=x^{1/2}\cos(\ln(x)),v(x)=\sin(\ln(x)).
\]

See also xamples.f, \#20 (Krall).

6. $-y^{\prime\prime}(x)=\lambda y(x)$ on $(0,+\infty)$ is LCNO at 0 and LP at +infinity.

At $0$, a maximal domain pair is%
\[
u(x)=x,v(x)=1-x\ln(x).
\]

See also xamples.f, \#4(Boyd).

7. $-(x^{-1}y^{\prime}(x))^{\prime}+(kx^{-2}+k^{2}x^{-1})y(x)=\lambda y(x)$ on
$(0,1]$ with $k$ real and $k\neq0$, is LCNO at 0 and R at 1.

At $0$, a maximal domain pair is%
\[
u(x)=x^{2},v(x)=x-k^{-1}%
\]

See also xamples.f, \#8 (Laplace Tidal Wave).

\medskip

H7: General self-adjoint boundary conditions.

Boundary conditions for Sturm-Liouville boundary value problems%
\begin{equation}
-(py^{\prime})^{\prime}+qy=\lambda wy\;\text{on}\;(a,b) \tag{*}%
\end{equation}

\noindent are \textit{either }SEPARATED, with at most one condition at
endpoint $a$ and at most one condition at endpoint $b$,

\noindent\textit{or }COUPLED, when both $a$ and $b$ are, independently, in one
of the end-point classifications R, WR, LCNO, LCO, in which case two
independent boundary conditions are required which link the solution values
near $a$ to those near $b$. The SLEIGN2 program allows for all self-adjoint
boundary conditions; separated self-adjoint conditions and all cases of
coupled self-adjoint conditions.

Separated Conditions: the boundary conditions to be selected depend upon the
classification of the differential equation at the endpoint, say, $a$:

1. If the endpoint a is LP, then no boundary condition is required or allowed.

2. If the endpoint $a$ is R or WR, then a separated self-adjoint boundary
condition is of the form%
\[
A1y(a)+A2(py^{\prime})(a)=0
\]

\noindent where $A1,A2$ are real constants the user must choose, not both zero.

3. If the endpoint a is LCNO or LCO, then a separated boundary condition is of
the form%
\[
A1[y,u](a)+A2[y,v](a)=0
\]

\noindent where $A1,A2$ are real constants the user must choose, not both
zero; here $u,v$ are the pair of boundary condition functions the user has
previously selected when the input Fortran file was being prepared with makepqw.f.

4. If the endpoint a is LCNO and the boundary condition pair $u,v$ has been
chosen so that%
\[
\lim_{x\rightarrow a}\frac{u(x)}{v(x)}=0
\]

\noindent(which is always possible), then $A1=1,A2=0$ (\textit{i.e.
}$[y,u](a)=0$) gives the principal (Friedrichs) boundary condition at $a.$

5. If $a$ is R or WR and boundary condition functions $u,v$ have been entered
in the Fortran input file, then 3 and 4 above apply to entering separated
boundary conditions at such an endpoint; the boundary conditions in this form
are equivalent to the point-wise conditions in 2 (subject to care in choosing
$A1,A2$). This singular form of a regular boundary condition may be
particularly effective in the WR case if the boundary condition form in 2
leads to numerical difficulties. Conditions 2,3,4,5 apply similarly at
endpoint $b$ (with $U,V$ as the boundary condition functions at $b$).

6. If $a$ is R, WR, LCNO, or LCO and $b$ is LP, then only a separated
condition at $a$ is required and allowed (or instead at $b$ if $a$ and $b$ are interchanged).

7. If both endpoints $a$ and $b$ are LP, then no boundary conditions are
required or allowed.

The indexing of eigenvalues for boundary value problems with separated
conditions is discussed in H13.

Coupled Conditions:

8. Coupled regular self-adjoint boundary conditions on $(a,b)$ apply only when
both endpoints $a$ and $b$ are R or WR.

\medskip

H8: Recording the results.

If you choose to have a record kept of the results, then the following
information is stored in a file with the name you select:

1. The file name.

2. The header line prompted for (up to 32 characters of your choice).

3. The interval $(a,b)$ selected by the user.

For SEPARATED boundary conditions:

4. The endpoint classification.

5. A summary of coefficient information at WR, LCNO, LCO endpoints.

6. The boundary condition constants $(A1,A2),(B1,B2)$ if entered.

7. (NUMEIG,EIG,TOL) or (NUMEIG1,NUMEIG2,TOL), as entered.

For COUPLED boundary conditions:

8. The boundary condition parameter $\alpha$ and the coupling matrix
$\mathbf{K}$, see H6.

For ALL self-adjoint boundary conditions:

9. The computed eigenvalue, EIG, and its estimated accuracy, TOL.

10. IFLAG reported (see H15).

\medskip

H9: Type and choice of interval.

You may enter any interval $(a,b)$ for which the coefficients $p,q,w$ are well
defined by your Fortran statements in the input file, provided that $(a,b)$
contains no interior singularities.

\medskip

H10: Entry of endpoints.

Endpoints a and b should generally be entered as real numbers to an
appropriate number of decimal places.

\medskip

H11: endpoint values of $p,q,w.$

The program SLEIGN2 needs to know whether the coefficient functions $p,q,w$ as
defined by the Fortran expressions entered in the input file, can be evaluated
numerically without running into difficulty. If, for example, either $q$ or
$w$ is unbounded at $a$, or $p(a)=0$, then SLEIGN2 needs to know this
information so that $a$ is not chosen for functional evaluation.

\medskip

H12: Initial value problems.

The initial value problem facility for Sturm-Liouville problems%
\begin{equation}
-(py^{\prime})^{\prime}+qy=\lambda wy\;\text{on}\;(a,b) \tag{*}%
\end{equation}

\noindent allows for the computation of a solution of $(\ast)$ with a
user-chosen value $\lambda$ and any one of the following initial conditions:

1. From endpoint $a$ of any classification except LP towards endpoint $b$ of
any classification

2. From endpoint $b$ of any classification except LP back towards endpoint $a$
of any classification

3. From endpoints $a$ and $b$ of any classifications except LP towards an
interior point of $(a,b)$ selected by the program.

Initial values at $a$ are of the form $y(a)=\alpha1,(py^{\prime})(a)=\alpha2$
when $a$ is R or WR; and $[y,u](a)=\alpha1,[y,v](a)=\alpha2$ when $a$ is LCNO
or LCO.

Initial values at $b$ are of the form $y(b)=\beta1,(py^{\prime})(b)=\beta2$
when $b$ is R or WR; and $[y,u](b)=\beta1,[y,v](b)=\beta2$ when $b$ is LCNO or LCO.

In $(\ast)$, $\lambda$ is a user-chosen real number; while in the above
initial values, $(\alpha1,\alpha2)$ and $(\beta1,\beta2)$ are user-chosen
pairs of real numbers not both zero.

In the initial value case 3 above when the interval $(a,b)$ is finite, the
interior point selected by the program is generally near the midpoint of
$(a,b)$; when $(a,b)$ is infinite, no general rule can be given. Also if,
given $(\alpha1,\alpha2)$ and $(\beta1,\beta2)$, the $\lambda$ chosen is an
eigenvalue of the associated boundary value problem, the computed solution may
not be the corresponding eigenfunction -- the signs of the computed solutions
on either side of the interior point may be opposite.

The output for a solution of an initial value problem is in the form of stored
numerical data which can be plotted on the screen (see H16), or printed out in
graphical form if graphics software is available.

\medskip

H13: Indexing of eigenvalues.

The indexing of eigenvalues is an automatic facility in SLEIGN2. The following
general results hold for the separated boundary condition problem (see H7):

1. If neither endpoint $a$ or $b$ is LP or LCO, then the spectrum of the
eigenvalue problem is discrete (eigenvalues only), simple (eigenvalues all of
multiplicity 1), and bounded below with a single cluster point at $+\infty.$
The eigenvalues are indexed as $\{\lambda_{n}:n=0,1,2,...\}$, where
$\lambda_{n}<\lambda_{n+1}$ $(n=0,1,2,..)$, $\lim_{n\rightarrow+\infty}%
\lambda_{n}=+\infty$; and if $\{\psi_{n}:n=0,1,2,...\}$ are the corresponding
eigenfunctions, then $\psi_{n}$ has exactly $n$ zeros in the open interval $(a,b).$

2. If neither endpoint $a$ or $b$ is LP but at least one endpoint is LCO, then
the spectrum is discrete and simple as for 1, but with cluster points at both
$\pm\infty.$ The eigenvalues are indexed as $\{\lambda_{n}:n=0,\pm
1,\pm2,...\}$, where $\lambda_{n}<\lambda_{n+1}$ $($for $n<n+1$ and
$n=0,\pm1,\pm2,..),$ with $\lambda_{0}$ the smallest non-negative eigenvalue;
$\lim_{n\rightarrow-\infty}\lambda_{n}=-\infty$ and $\lim_{n\rightarrow
+\infty}\lambda_{n}=+\infty$; and if $\{\psi_{n}:n=0,\pm1,\pm2,...\}$ are the
corresponding eigenfunctions, then every $\psi_{n}$ has infinitely many zeros
in $(a,b).$

3. If one or both endpoints is LP, then there can be one or more intervals of
continuous spectrum for the boundary value problem in addition to some
(necessarily simple) eigenvalues. For these essentially more difficult
problems, SLEIGN2 can be used as an investigative tool to give qualitative and
possibly quantitative information on the spectrum.

For example, if a problem has continuous spectrum starting at a real number
$L$, then there may be no eigenvalues below $L$, any finite number of
eigenvalues below $L$, or an infinite (but countable) number of eigenvalues
below $L.$ SLEIGN2 can be used to compute $L$ (see the paper BEWZ on the
SLEIGN2 home page for an algorithm to compute $L$), and to determine the
number of these eigenvalues and compute them. In this respect, see xamples.f:
\#13 (Hydrogen Atom), \#17 (Morse Oscillator), \#21 (Fourier), and \#27
(J\"{o}rgens) as examples of success; and \#2 (Mathieu), \#14 (Marletta), and
\#28 (Behnke-Goerisch) as examples of failure.

The problem need not have a continuous spectrum, in which case if its discrete
spectrum is bounded below, then the eigenvalues are indexed and the
eigenfunctions have zero counts as in 1. If, on the other hand, the discrete
spectrum is unbounded below, then all the eigenfunctions have infinitely many
zeros in the open interval $(a,b)$. SLEIGN2 can, in principle, compute these
eigenvalues if neither endpoint is LP although this is a computationally
difficult problem. Note however, that SLEIGN2 has no algorithm when the
spectrum is discrete, unbounded above and below and one endpoint is LP, as in
xamples.f \#30.

In respect to the five classes of endpoints, the following identified examples
from xamples.f illustrate the spectral property of these boundary value problems:

1. Neither endpoint is LP or LCO:

\#1 (Legendre), \#2 (Bessel) with $-1/4<C<3/4,$ \#4 (Boyd), \#5 (Latzko).

2. Neither endpoint is LP, but at least one is LCO:

\#6 (Sears-Titchmarsh), \#7 (BEZ), \#19 (Donsch)

3. At least one endpoint is LP:

\#13 (Hydrogen Atom), \#14 (Marletta), \#20 (Krall), \#21 (Fourier) on [0,infinity).

\medskip

H14: Entry of eigenvalue index, initial guess, and tolerance.

For all self-adjoint boundary condition problems (see H7), SLEIGN2 calls for
input information options to compute \textit{either}

1. a single eigenvalue, \textit{or}

2. a series of eigenvalues.

\noindent In each case indexing of eigenvalues is called for (see H13).

1 above asks for data triples NUMEIG, EIG, TOL separated by commas. Here
NUMEIG is the integer index of the desired eigenvalue; NUMEIG can be negative
only when the problem is LCO at one or both endpoints. EIG allows for the
entry of an initial guess for the requested eigenvalue (if an especially good
one is available), or can be set to 0 in which case an initial guess is
generated by SLEIGN2 itself. TOL is the desired accuracy of the computed
eigenvalue. It is an absolute accuracy if the magnitude of the eigenvalue is 1
or less, and is a relative accuracy otherwise. Typical values for TOL might be
0.001 for moderate accuracy and 0.0000001 for high accuracy in single
precision. If TOL is set to 0, the maximum achievable accuracy is requested.

If the input data list is truncated with a ``\thinspace/\thinspace'' after
NUMEIG or EIG, then the remaining elements default to 0.

2 above asks for data triples NUMEIG1, NUMEIG2, TOL separated by commas. Here
NUMEIG1 and NUMEIG2 are the first and last integer indices of the sequence of
desired eigenvalues, with NUMEIG1
%TCIMACRO{\TEXTsymbol{<}}%
%BeginExpansion
$<$%
%EndExpansion
NUMEIG2; they can be negative only when the problem is LCO at one or both
endpoints. TOL is the desired accuracy of the computed eigenvalues. It is an
absolute accuracy if the magnitude of an eigenvalue is 1 or less, and is a
relative accuracy otherwise. Typical values for TOL might be 0.001 for
moderate accuracy and 0.0000001 for high accuracy in single precision. If TOL
is set to 0, the maximum achievable accuracy is requested.

If the input data list is truncated with a ``\thinspace/\thinspace'' after
NUMEIG2, then TOL defaults to 0.

For COUPLED self-adjoint boundary condition problems (see H7 and H17), SLEIGN2
also reports which eigenvalues are double. Double eigenvalues can occur only
for coupled boundary conditions with the parameter $\alpha=0$ or $\pi.$

\medskip

H15: IFLAG information.

All results are reported by SLEIGN2 with a flag identification. There are four
values of IFLAG:

1. The computed eigenvalue has an estimated accuracy within the

tolerance requested.

2. The computed eigenvalue does not have an estimated accuracy within the
tolerance requested, but is the best the program could obtain.

3. There seems to be no eigenvalue of index equal to NUMEIG.

4. The program has been unable to compute the requested eigenvalue.

\medskip

H16: Plotting.

After computing a single eigenvalue (see H14(1.)), but not a sequence of
eigenvalues (see H14(2.)), the eigenfunction can be plotted for separated
conditions and for coupled ones with $\alpha=0,\pi.$ If this data is desired,
respond $y$ when asked and SLEIGN2 will then compute eigenfunction data and
store them for subsequent use.

The user can ask that the eigenfunction data be in the form of either points
$(x,y)$ for $x\in(a,b)$, or points $(t,y)$ for $t$ in the standardized
interval $(-1,+1)$ mapped onto from $(a,b)$; the $t$-choice can be especially
helpful when the original interval is infinite. Additionally, the user can ask
for a plot of the so-called Pr\"{u}fer angle, in the $x$- or $t$-variables.

In both forms, once the choice has been made of the function to be plotted, a
crude plot is displayed on the monitor screen and the user is asked whether or
not to save the computed plot points in a file.

\medskip

H17: Indexing of eigenvalues for coupled self-adjoint problems.

The indexing of eigenvalues is an automatic facility in SLEIGN2. The following
general result holds for coupled boundary condition problems (see H7):

The spectrum of the eigenvalue problem is discrete (eigenvalues only). In
general the spectrum is not simple, but no eigenvalue exceeds multiplicity
$2$. The eigenvalues are indexed as $\{\lambda_{n}:n=0,1,2,...\}$ where
$\lambda_{n}<\lambda_{n+1}$ for $n=0,1,2,...,$and $\lim_{n\rightarrow+\infty
}\lambda_{n}=+\infty$ if neither endpoint is LCO. If one or both endpoints are
LCO the eigenvalues cluster at both $-\infty$ and at $+\infty,$ and all
eigenfunctions have infinitely many zeros.

If neither endpoint is LCO and $\alpha=0,\pi$, then the $n$-th eigenfunction
has $n-1,n,n+1$ zeros in the half open interval $[a,b)$ (also in $(a,b]$). All
three possibilities occur. Recall that in the case of double eigenvalues,
although the $n$-th eigenvalue is well defined there is an ambiguity about
which solution is declared as the $n$-th eigenfunction. If $\alpha\neq0,\pi$,
then the eigenfunction is non-real and has no zero in $(a,b)$, but each of the
real and imaginary parts of the $n$-th eigenfunction have the same zero
properties as mentioned above when $\alpha=0,\pi.$

The following identified examples from xamples.f are of special interest:

\#11 (Plum) on $[0,\pi],$ \#21 (Fourier) on $[0,\pi],$ \#25 (Meissner) on $[-1/2,+1/2].$
\end{document}
