%% This document created by Scientific Word (R) Version 3.5

\documentclass[12pt]{amsart}%
\usepackage{graphicx}
\usepackage{amscd}
\usepackage{amsmath}%
\usepackage{amsfonts}%
\usepackage{amssymb}
%TCIDATA{OutputFilter=latex2.dll}
%TCIDATA{CSTFile=amsartci.cst}
%TCIDATA{Created=Mon Sep 06 21:20:17 1999}
%TCIDATA{LastRevised=Sunday, April 01, 2001 11:12:39}
%TCIDATA{<META NAME="GraphicsSave" CONTENT="32">}
%TCIDATA{<META NAME="DocumentShell" CONTENT="Journal Articles\amsnumb">}
%TCIDATA{Language=British English}
\newtheorem{theorem}{Theorem}
\theoremstyle{plain}
\newtheorem{acknowledgement}{Acknowledgement}
\newtheorem{algorithm}{Algorithm}[section]
\newtheorem{axiom}{Axiom}[section]
\newtheorem{case}{Case}[section]
\newtheorem{claim}{Claim}[section]
\newtheorem{conclusion}{Conclusion}[section]
\newtheorem{condition}{Condition}[section]
\newtheorem{conjecture}{Conjecture}[section]
\newtheorem{corollary}{Corollary}[section]
\newtheorem{criterion}{Criterion}[section]
\newtheorem{definition}{Definition}[section]
\newtheorem{example}{Example}[section]
\newtheorem{exercise}{Exercise}[section]
\newtheorem{lemma}{Lemma}[section]
\newtheorem{notation}{Notation}[section]
\newtheorem{problem}{Problem}[section]
\newtheorem{proposition}{Proposition}[section]
\newtheorem{remark}{Remark}[section]
\newtheorem{solution}{Solution}[section]
\newtheorem{summary}{Summary}[section]
\numberwithin{equation}{section}
\numberwithin{theorem}{section}
\newcommand{\thmref}[1]{Theorem~\ref{#1}}
\newcommand{\secref}[1]{\S\ref{#1}}
\newcommand{\lemref}[1]{Lemma~\ref{#1}}
\setlength{\oddsidemargin}{0.0in}
\setlength{\evensidemargin}{0.0in}
\setlength{\textwidth}{6.5in}
\setlength{\textheight}{9.0in}
\setlength{\headsep}{0.25in}
\setlength{\headheight}{0.0in}


\begin{document}
\title[Introduction to SLEIGN2]{Introduction to SLEIGN2}
\author{P.B. Bailey}
\author{W.N. Everitt}
\author{A. Zettl}
\address{Department of Mathematical Sciences, Northern Illinois University, DeKalb, IL
60115-2888, USA}
\date{01 March 2001 (File: intro.tex).}
\maketitle


The main purpose of this code is to compute eigenvalues and eigenfunctions of
regular and singular self-adjoint Sturm-Liouville problems (SLP) and to
approximate the continuous spectrum in the singular case. For a general
description of the analytical and numerical properties of the SLEIGN2 code see
\cite{BEZ} and \cite{BEZ3}.

These problems consist of a second order linear differential equation
\[
-(py^{\prime})^{\prime}+qy=\lambda wy\;\text{on}\;(a,b)
\]
together with boundary conditions (BC). The nature of the BC depends on the
regular or singular classification of the end points a and b. For both cases
the BC fall into two major classes: separated and coupled. The former are two
separate conditions, one at each endpoint; the latter are two coupled
conditions linking the values of the solution near the two endpoints, e.g.
periodic and semi-periodic boundary conditions. SLEIGN2 seems to be the only
general purpose code available for arbitrary self-adjoint BC, separated or
coupled, and for both regular and singular problems.

A number $\lambda$ for which there is a nontrivial solution satisfying the BC
is called an eigenvalue and such a solution is a (corresponding)
eigenfunction. If one or both endpoints are LP (see below or section 4 of HELP
for a definition) there may be points $\lambda$ in the spectrum in addition to
eigenvalues i.e. there may be continuous spectrum.

In the theory of SLP the coefficients $1/p$ and $q$ and the weight function
$w$ are assumed to be real-valued and locally Lebesgue integrable on $(a,b).$

To meet the needs of numerical computing techniques we make the stronger assumptions:

(i) The interval $(a,b)$ of R may be bounded or unbounded

(ii) $p,q$ and $w$ are real-valued functions on $(a,b)$

(iii) $p,q$ and $w$ are piecewise continuous on $(a,b)$

(iv) $p$ and $w$ are strictly positive on $(a,b)$.

For better error analysis in the numerical procedures, condition (iii) above
is replaced with

(iii)$^{\prime}$ $p,q$ and $w$ are four times continuously differentiable on
$(a,b)$.

To study a SLP using operator theory one associates a self-adjoint operator in
the weighted Hilbert space of square-integrable functions, with respect to the
weight $w$, on $(a,b)$, with each SLP in such a way that the spectrum of the
problem is the spectrum of the operator. In the case of a regular problem the
spectrum consists entirely of eigenvalues and these are bounded below (when
$p>0$ and $w>0$). This is still so for the case when each endpoint is either
regular (R) or singular limit-circle nonoscillatory (LCNO). In case one
endpoint is limit-circle oscillatory (LCO) and the other is not limit-point
(LP) then there are still only eigenvalues in the spectrum but these are not
bounded below. (The spectrum is never bounded above.)

If one or both endpoints is LP the spectrum may be extremely complicated.
There may be no eigenvalues, finitely many, or infinitely many. Some may be
embedded in the continuous spectrum. For $p=1,w=1,q(x)=\sin(x)$ on
$(-\infty,+\infty)$ there are no eigenvalues and the continuous spectrum
consists of the union of an infinite number of disjoint compact intervals.
(SLEIGN2 can be used to approximate this spectrum - see example 12 in the file
xamples.tex and the references quoted there.)

See the HELP file help.tex for a definition of the terms regular (R),
limit-circle (LC), limit-circle non-oscillatory (LCNO), limit-circle
oscillatory (LCO), limit-point (LP).

SLP problems are classified into various classes based on the classification
of the endpoints and on whether the boundary conditions are separated (S) or
coupled (C). We have the following categories:

1. R/R, S

2. R/R, C

3. R/LCNO or LCNO/R, S

4. R/LCNO or LCNO/R, C

5. R/LCO or LCO/R , S

6. R/LCO or LCO/R , C

7. LCNO/LCO or LCO/LCNO or LCO/LCO, S

8. LCNO/LCO or LCO/LCNO or LCO/LCO, C

9. LP/R or LP/LCNO or LP/LCO or R/LP or LCNO/LP or LCO/LP

10. LP/LP

For 9 there is only a separated condition at the non-LP endpoint and for 10
there are no boundary conditions at either end.

There are only two other major general purpose codes for computing eigenvalues
and eigenfunctions of Sturm-Liouville problems : SLEDGE (Fulton and Pruess)
and the earlier code SLEIGN (Bailey \textit{et al).} (There was another code,
written by Pryce and Marletta, available from the NAG library but the current
NAG library no longer includes such a code.)

SLEDGE uses a method based on piecewise constant approximations of the
coefficients of the differential equation; SLEIGN and SLEIGN2 both use the
Pr\"{u}fer transformation.

Both SLEIGN and SLEDGE are designed for separated regular boundary conditions
and both have a mechanism to automatically handle endpoints which are either
regular or singular but non-oscillatory. In the latter case if an endpoint is
LCNO the Friedrichs condition is usually, but not always, the one chosen by
the code. SLEDGE can also determine the LP/LC classification but only in a
restricted number of cases (the method requires that the coefficients $p,q$
and $w$ are analytic on $(a,b)$ and depends on the Frobenius method for series
solutions at regular singular endpoints); SLEIGN and SLEIGN2 do not have such
a facility but, for the sake of generality, rely on the user input of this information.

SLEIGN2 is the only general purpose code in existence which can, in principle,
handle arbitrary self-adjoint, separated or coupled, regular or singular,
boundary conditions, see the HELP file help.tex and \cite{BEZ3}. Problems with
coupled boundary conditions, see\cite{BEZ2}, at singular endpoints are
difficult to handle numerically, especially if one or both of the endpoints is LCO.

In addition to the above mentioned capabilities to compute eigenvalues and
eigenfunctions SLEIGN2 also computes the solution of an initial value problem
with the users choice of $\lambda$ and either a regular or singular initial condition.

When combined with the algorithm established in \cite{BEWZ}, SLEIGN2 can be
used to approximate the continuous spectrum.

An important feature of the SLEIGN2 program is its user friendly interface
contained in makepqw.f and drive.f. Users who want to bypass this interface
and use their own driver may wish to look at a sample driver; two such drivers
are provided, see (7) and (8) below.

The whole package consists of the following files:

1. A brief ``readme'' file readme.txt with basic information on how to run the code.

2. This introduction file intro.tex.

3. makepqw.f - This is an interactive Fortran file to input the coefficient
functions $p,q,w$ and, if necessary, the functions $u,v$ which define the
singular boundary conditions

4. drive.f - This is an interactive Fortran file containing the driver, a HELP
file help.tex, and a ``user friendly'' interface.

5. sleign2.f - The main code for the computation of eigenvalues and eigenfunctions.

6. xamples.f - A Fortran file with 32 examples ready to run. These examples
were chosen to illustrate various features of the code.

7. sepdr.f - A sample driver for separated regular or singular boundary conditions.

8. coupdr.f - A sample driver for coupled regular or singular boundary conditions.

9. xamples.tex - A LaTeX file containing information about the 32 examples.

10. help.tex - A LaTeX file with information about endpoint classifications,
boundary conditions etc. It is a separate text file and it can also be
accessed from both makepqw.f and drive.f.

11. autoinput.txt- A file describing an ``automatic'' method for using SLEIGN2
which avoids the user friendly ``question and answer'' format; this is
recommended for experienced users only.

When an eigenfunction has been computed it is stored. If it is real-valued, it
can be examined:

(i) by printing out the numerical data

(ii) by using the discrete graph plotter in the program

(iii) by using a local graph plotter.

Full details are given in HELP; see the file help.tex.

All six of the Fortran files in the SLEIGN2 package are in single precision.

\begin{thebibliography}{9}                                                                                                %

\bibitem {BEZ}P.B. Bailey, W.N. Everitt and A. Zettl, \emph{Computing
eigenvalues of singular Sturm-Liouville problems}, Results in Mathematics,
\textbf{20 }(1991), 391-423.

\bibitem {BEZ2}P.B. Bailey, W.N. Everitt and A. Zettl, \emph{Regular and
singular Sturm-Liouville problems with coupled boundary conditions}, Proc.
Royal Soc. Edinburgh (A) \textbf{126} (1996), 505-514.

\bibitem {BEZ3}P.B. Bailey, W.N. Everitt and A. Zettl, \emph{The SLEIGN2
Sturm-Liouville code.} (To appear in ACM Trans. Math. Software).

\bibitem {BEWZ}P.B. Bailey, W.N. Everitt, J. Weidmann and A. Zettl,
\emph{Regular approximation of singular Sturm-Liouville problems}. Results in
Mathematics \textbf{23} (1993), 3-22.
\end{thebibliography}
\end{document}
