\section{Parameters}
\label{PARAM}
In addition to {\sf objPoly}, {\sf ineqPolySys}, {\sf lbd} and {\sf ubd} for  describing a POP in the
SparsePOP format, the MATLAB
function sparsePOP.m has {\sf param} as an input argument. It is a structure consisting of
many parameters that control the performance of the function. Table \ref{tableOfparam}
shows the list of parameters defined in SparsePOP. The default values of all parameters are given in the
MATLAB function defaultParameters.m. They can be modified if necessary.

\begin{table}[htdp]
\caption{The fields of {\sf param}, default values and possible values}
\begin{center}
\begin{tabular}{c|c|c}
field of {\sf param} & default values &  possible values\\
\hline
% 1. Parameters to control the basic relaxation scheme.
{\sf relaxOrder} & $\omega_{\max}$ & a positive integer not less than $\omega_{\max}$\\
{\sf sparseSW} & 1 & 0 and 1\\
%\hline
{\sf multiCliquesFactor} & 1 & 0, 1 and 'objPoly.dimVar' \\
%\hline
% 2. Switch to handle numerical difficulties.
{\sf scalingSW} & 1 & 0 and 1\\
%\hline
{\sf boundSW} & 1 & 0, 1 and 2\\
%\hline
{\sf eqTolerance} & 0.0 & 0.0 and a positive real number $>$  1.0e-10\\
%\hline
{\sf perturbation} & 0.0 & 0.0 and a positive real number $>$ 1.0e-10 \\
%\hline
{\sf reduceMomentMatSW} & 1 & 0 and 1\\
%\hline
{\sf complementaritySW} & 1 & 0 and 1\\
%\hline
% 3. Parameters for SDP solvers.
{\sf SeDuMiSW} & 1 & 0 and 1\\
%\hline
{\sf SeDuMiOutFile} & 0 & 0, 1 and a file name\\
%\hline
{\sf SeDuMiEpsilon} & 1.0e-9 & a positive real value\\
%\hline
{\sf sdpaDataFile} & {\sf ' '} & {\sf ' '} and a file name with the extension .dat-s\\
%\hline
% 4. Parameters for printing numerical results.
{\sf detailedInfFile} & {\sf ' '} & {\sf ' '} and a file name\\
%\hline
{\sf printFileName} & 1 & 1 and a file name\\
%\hline
{\sf printLevel} & {\sf [}2,2{\sf ]} & 0,1 or 2 for both elements \\
%\hline
% 5. Parameters to use Symbolic Math Toolbox and C++ subrouties
{\sf symbolicMath} & 1& 0 and 1\\
%\hline
{\sf mex} & 1 & 0 and 1\\
%\hline
\hline
\end{tabular}
\end{center}
\label{tableOfparam}
\end{table}
These parameters can be divided into five categories:
\begin{enumerate}
\item  Parameters for controlling the basic relaxation scheme.
\item  Switches for techniques to reduce numerical difficulties.
\item  Parameters for  SDP solvers.
\item  Parameters for printing.
\item  Parameters for Symbolic Math Toolbox and C++ subroutines.
\end{enumerate}

% Skim 5/15/07
\subsection{Parameters to choose the relaxation method}

%We first need to specify
The relaxation order should be specified as 
\begin{eqnarray*}
& & \mbox{{\sf param.relaxOrder} }  =\omega \geq \omega_{\max} = \max \{ \omega_k : k =0,1,\ldots,m \}
\end{eqnarray*}
% {\sf param.relaxOrder} $=\omega \geq \omega_{\max}$
to execute the sparse or dense SDP relaxation for a POP,
where 
\begin{eqnarray*}
% & & \omega_{\max} = \max \{ \omega_k : k =0,1,\ldots,m \}, \\
& & \omega_k = \lceil \deg(f_k(\x)) / 2 \rceil \ (k=0,1,2,\ldots,m). 
\end{eqnarray*}
% $\omega_{\max}$ is defined by (\ref{omegaMax}) for a POP of the form (\ref{POP0}).
The default is {\sf param.relaxOrder} $= \omega_{\max}$. If the accuracy of the obtained approximate optimal solution
% Skim 5/15/07
is not satisfactory,  increasing the relaxation order {\sf param.relaxOrder} $=\omega \geq \omega_{\max}$ is
one way to obtain more accurate optimal solution.
We should mention, however, that increasing the relaxation order may take more cpu time and
numerical difficulty might occur while solving the SDP relaxation problem.

The type of SDP relaxation can be chosen by setting
{\sf  param.sparseSW} $=0$ for the dense relaxation
based on  \cite{LAS01}, or $=1$ for the sparse relaxation based on \cite{WAKI04}.
The sparsity of
the sparse SDP relaxation problem is varied by {\sf param.multiCliquesFactor}.
Suppose that {\sf param.sparseSW}$= 1$; otherwise this parameter is not relevant.
The purpose of this parameter is to strengthen the sparse relaxation
by taking the union of some of the maximal cliques $C_{\ell}$ $(\ell=1,2,\ldots,p)$
of a chordal extension $G(N,E')$ of the
csp graph induced from the POP (\ref{POP0}) for $\widetilde{C}_k$ $(k=1,2,\ldots,m)$.
Let $\rho_{\max}$ denote the maximum over  $\sharp C_{\ell}$ $(\ell=1,2,\ldots,p)$,
where $\sharp C_{\ell}$ denotes the cardinality of $C_{\ell}$.
Recall that  $F_k = \{ i : x_i \mbox{ appears in } f_k(x) \geq 0 \}$.
Let $J_k = \{ \ell : F_k \subset C_{\ell} \}$.
Take one clique from $C_{\ell} \ (\ell \in J_k)$ for $\widetilde{C}_k$.
Add another clique
 from $C_{\ell} \ (\ell \in J_k)$ to $\widetilde{C}_k$ if
 $\sharp \left(\widetilde{C}_k \bigcup C_{\ell}\right)$ does not exceed
 {\sf param.multiCliquesFactor} $\times \rho_{\max}$.
 Repeat this procedure to obtain the union $\widetilde{C}_k$ of some cliques from $C_{\ell}
\ (\ell \in J_k)$.
If {\sf param.multiCliquesFactor} $ = 0$, then $\widetilde{C}_k$ consists of a single clique
$C_{\ell}$ for some $\ell \in J_k$.
If {\sf param.multiCliquesFactor}$=${\sf objPoly.dimVar}, then $\widetilde{C}_k$ consists
of the union of all $C_{\ell} \ (\ell \in J_k)$. The
default value is $1$, which means that
the cardinality of $\widetilde{C}_k$  is bounded by $\rho_{\max}$. If the accuracy of the obtained
% Skim 5/15/07
approximate optimal solution is not satisfactory,  sparsePOP.m can be executed again with
the choice of either {\sf param.multiCliquesFactor}$=${\sf objPoly.dimVar} or {\sf  param.sparseSW} $=0$
before increasing the relaxation order {\sf param.relaxOrder} $=\omega$.

% Skim 5/15/07
\subsection{Switches for techniques to handle numerical difficulties}

Because the POP (\ref{POP0}) is basically a hard optimization problem,
% Skim 5/15/07
numerical difficulties are often unavoidable while solving its SDP relaxation,
and/or an inaccurate approximate solution might be obtained.
The switches described in this subsection  are intended to prevent numerical
difficulties from occuring, and improve the accuracy of an obtained solution.

With {\sf param.scalingSW}$=1$, the objective polynomial,
constraint polynomials, lower and upper bounds are
 scaled such that the maximum of $\{ |\mbox{lower bound of $x_j$}|,
 | \mbox{upper bound of $x_j$} | \} $ $= 1$ $(j \in J)$
 and that the maximum absolute value of the coefficients of all monomials in each
 polynomial is $1$, where $J$ denotes the set of indices $j$ for which
 the variable $x_j$ has finite lower and upper bounds;
 $-1.0$e+$10 < $ lbd$(j) \leq $ ubd$(j) < 1.0$e+$10$.
This scaling technique is very effective to improve the numerical stability
when solving the resulting SDP relaxation. The default is  {\sf param.scalingSW}  $=1$.

Appropriate bounds are added  for all linearized
variables $y_{\salpha}$ $(\balpha \in \widetilde{\FC})$ if {\sf param.boundSW} $= 1$.
If {\sf param.boundSW}$=2$, some redundant bounds of variables $y_{\salpha}$ are
% Skim 5/15/07 generated
removed from the added  bounds for all $y_{\salpha}$. Otherwise, no
bounds are added to $y_{\salpha}$. The default is  {\sf param.boundSW}$= 2$.
In particular, when every variable $x_j$
is scaled such that lbd$(j) = 0$ and
ubd$(j)=1$ $(j=1,2,\ldots,n)$, the bounds
$0 \leq y_{\salpha} \leq 1$ $(\balpha \in \widetilde{\FC})$
are added. Empirically, we know such a bounding is very effective to improve
the numerical stability in solving the SDP relaxation. Therefore,
% Skim 5/15/07
our recommendation is to
modify a POP so that every variable $x_j$ is nonnegative and
has a finite positive upper bound; then the desired scaling and bounding of
variables $y_{\salpha}$ $(\balpha \in \widetilde{\FC})$ are performed in
 sparsePOP.m by taking {\sf param.scalingSW} $=1$
and {\sf param.boundSW} $= 1 \mbox{ or } 2$.

The parameter {\sf param.eqTolerance} is used to
convert  every equality constraint into two inequality constraints;
if $1.0$e-$10 < $ {\sf param.eqTolerance},
then  each equality constraint is replaced by
$f(x) = 0$ by $ f(x) \geq$ {\sf $-$param.eqTolerance}  and $ -f(x) \geq$ {\sf $-$param.eqTolerance}.
When SeDuMi displays  numerical difficulty while solving
the SDP relaxation of a POP with equality constraints,
this technique with $1.0$e-$3 \leq $ {\sf param.eqTolerance} $\leq 1.0$e-$7$
often provides a more stable SDP relaxation problem that can be solved by
SeDuMi. The default is {\sf param.eqTolerance} $=0$, {\it i.e.},
% Skim 5/15/07
no conversion of the equality constraints is specified.

Perturbing the objective polynomial to compute an optimal
solution of a POP with multiple optimal solutions is described in
Section 5. See also Section 5.1 of \cite{WAKI04}.
The parameter {\sf param.perturbation} is used for this purpose.
If $1.0$e-$10 < $ {\sf param.perturbation}, then
% Skim 5/15/07
the objective polynomial $f_0(x)$ is modified to  $f_0(\x) + \p^T \x$,
where $0 \leq p_i \leq$ {\sf param.perturbation}. Otherwise,
% Skim 5/15/07
no perturbation is performed. The default value for {\sf param.perturbation} is $0.0$, {\it i.e.},
no perturbation of the objective polynomial is desired.

% Skim 5/15/07
%When the SDP relaxation is too large to be solved,
The parameter
{\sf  param.reduceMomentMatSW} is intended for SDP relaxations too large to solve.
%% Waki: 2007/12/03 PSDP --> SDP relaxation problem
If {\sf  param.}{\sf reduceMomentMatSW} $=1$, then  sparsePOP.m
eliminates redundant elements
of $\AC_{\omega}^{C_{\ell}}$ $(\ell=1,2,\ldots,p)$
in the %PSDP (\ref{PolySDP0})
SDP relaxation problem
using the method proposed in the paper \cite{KOJ03a}.
See also \cite{WAKI04}.

When the complementarity condition exists
in the constraints of a POP, % to be solved,
we can set {\sf param.}{\sf complementaritySW} $ = 1$.
Suppose that $x_i x_j = 0$ appears as an equality constraint.
%% Waki: 2007/12/03 PSDP --> SDP relaxation problem
Then, any variable $y_{\alpha}$
corresponding to a monomial $\x^{\salpha}$ such that $\alpha_i \geq 1$
and $\alpha_j \geq 1$ is set to zero and eliminated from the %PSDP (\ref{PolySDP0})
SDP relaxation problem.
The default is  {\sf param.complementaritySW} $ = 0$.

\subsection{Parameters for  SDP solvers}


The function sparsePOP.m can provide three kinds of output for the SDP
relaxation problem:
information on the problem itself such as
the size and the nonzero elements of the constraint matrix of the problem,
data on an approximate optimal solution of the problem obtained by SeDuMi,
and SDPA sparse format data of the problem.

% Skim 5/15/07
For infomation on the problem and data on the obtained optimal solution,
SeDuMi should be called from the function
SDPrelaxation.m or SDPrelaxationMex.m
by setting {\sf param.SeDuMiSW}$=1$.
Users can increase or decrease the desired accuracy of the optimal value and
optimal solutions of an SDP relaxation problem by providing a smaller or larger
value for {\sf param.SeDuMiEpsilon} that corresponds to the parameter
 {\sf pars.eps} in SeDuMi.
The parameter {\sf param.SeDuMiOutFile} is used in connection with
the parameter {\sf param.SeDuMiSW}$=1$.
The default value of {\sf param.SeDuMiOutFile}$=1$ is used to display the output from SeDuMi
on the screen. If  the name of
a file such as {\sf param.SeDuMiOutFile} $=$ 'SeDuMi.out' is assigned,
the output from SeDuMi is written in the file. Information from SeDuMi is
not displayed if  {\sf param.SeDuMiOutFile} $=0$.
%display no information from SeDuMi.
The value $0$ for {\sf param.SeDuMiSW} is for just printing  information
on the problem  without solving  the SDP relaxation problem.
The default value for  {\sf param.SeDuMiSW} is $1$.

%% Waki 2007/12/06
% Skim 12/22/07
In SparsePOP,  SeDuMi \cite{STRUM99} is used for solving SDPs  because
SeDuMi seems to have better numerical stability than other SDP solvers.  
The use of   an iterative method (a variant of CG method) for solving ill-conditional linear systems in SeDuMi
leads to more accurate optimal solutions than other SDP solvers. % for SDP problems with free variables.
Also, SparsePOP can provide SDPA sparse format data for (\ref{POP0}) and a file that contains
 necessary information to extract an approximated solution of (\ref{POP0}).

% Skim 5/15/07
SDPA sparse  format data of the SDP relaxation problem can be also obtained by
assigning the name of a file for SDPA sparse format data to
the parameter {\sf param.sdpaDataFile}, for example,  {\sf param.sdpaDataFile} $=$ 'test.dat-s'.
With the SDPA sparse format input file 'test.dat-s',
 the SDP relaxation problem can be solved later by using some software packages
such as SDPA \cite{YAMASHITA03} and SDPT3 \cite{SDPT3}.
%% Waki: 2007/12/03 [] --> ''
The default is  {\sf param.sdpaDataFile} %$= [ \ ]$,
$ = $ '', {\it i.e.}, no SDPA sparse format  data
is created.

%% Waki 2007/12/06
% Skim 12/22/07
 If  the name of a file for SDPA sparse format data is provided, SparsePOP also generates the file containing necessary information for extracting an approximated solution of (\ref{POP0}) from the SDP relaxation problem. %We can construct  an approximated solution of (\ref{POP0}) from the solution of the SDP relaxation %problem by %using information of this file.
The file has the extension ``info" and  the structure of the file is as follows:
\begin{table}[htdp]
%\begin{center}
\begin{tabular}{ccc}
%$s$&  & & \\
%$c$&  & & \\
$k_1$&$a_1$&$b_1$\\
$k_2$&$a_2$&$b_2$\\
$\vdots$&&\\
$k_j$&$a_j$&$b_j$\\
$\vdots$&&\\
$k_n$&$a_n$&$b_n$
\end{tabular}
%\end{center}
\label{structureOfinfo}
\end{table}%
%% Waki 2007/12/06
Here $k_1, \ldots, k_n$ are integers, and $a_1,\ldots, a_n$ and $b_1,\ldots, b_n$ are real numbers. After solving the SDP relaxation problem  in SDPA sparse format by an SDP solver,
 if $k_j = -1$ for some $j$, set $x_{j}=b_j$. Otherwise, set $x_{j} = a_j y_{k_j}+b_j$, where $y_{k_j}$ is the $k_j$th element of the optimal solution of the primal of SDP relaxation problem in SDPA sparse 
format. Then, $\x$ is an approximated solution for (\ref{POP0}) computed by SparsePOP.

\subsection{Parameters for printing numerical results}

% Skim 5/15/07
Whether we have {\sf param.SeDuMiSW}$=1$ or $0$, we can store
detailed information of the POP and its SDP relaxation in a file specified using
 {\sf param.detailedInfFile}; for example,  {\sf param.detailedInfFile} $=$ 'details.out'.
 %% Waki: 2007/12/03 [] --> ''
The default is  {\sf param.detailedInfFile} $=$ '', {\it i.e.}, no detailed information is printed.

{\sf param.printFileName} is the parameter for displaying the computational results, % of SeDuMi,
such as {\sf param}, {\sf SDPinfo}, and {\sf POP}. That is, {\sf param.printFileName}$=1$
is for displaying the results  on screen, and {\sf param.printFileName}$=0$ prevents them
from displaying.
In addition,
the name of a file to {\sf param.printFileName}  can be  assigned to print the results in the file, such as
{\sf param.printFileName}$=$'result.out'.

% Skim 5/15/07
The default value 2 for {\sf param.printLevel(1)} is used to display the computational result
with an approximate optimal solution of the POP on screen. Setting {\sf param.printLevel(1)}$=1$
 stops displaying an approximate optimal solution of the POP on the screen.
%set {\sf param.printLevel(1)}$=1$.
The value $0$ for {\sf param.printLevel(1)} displays no  computational result on the screen.
The default value 2 for {\sf param.printLevel(2)} is used to write the computational result with
an approximate optimal solution of the POP in a file whose name is defined by
{\sf param.printFileName}. Setting {\sf param.printLevel(2)}$=1$ prevents
printing an approximate optimal solution on the file.
If {\sf param.printLevel(2)}$=0$, no information is written in the file.

\subsection{Parameters to use Symbolic Math Toolbox and C++ subroutines}

% Skim 5/15/07
The parameter {\sf param.symbolicMath} indicates whether
Symbolic Math Toolbox can be utilized for reading a POP in the GAMS scalar format, or not.
Setting {\sf param.symbolicMath}$=1$ means that functions of Symbolic Math Toolbox can
be used. See also the last paragraph of Section 3.1.  C++ subroutines can be used by
setting the parameter {\sf param.mex}$=1$.
