\section{Representation of polynomial optimization problems}
\label{Representation}

% Skim 5/15/07
Polynomials in the objective
function and constraints of a POP can be described in two ways to be read by SparsePOP. 
%The polynomials can be represented   in two ways. 
If the \GMS format is chosen,  
data in the \GMS format  is converted into data in the SparsePOP 
format by the function readGMS.m. Or, we can directly
%The other is to directly 
describe the objective and constraint polynomials in terms 
of the SparsePOP format. 
As an illustrative example, we consider an inequality-equality constrained POP with three 
variables $x_1, \ x_2$ and $x_3$: 
\begin{equation}
\left. 
 \begin{array}{ll}
	\mbox{minimize} & -2x_1 +3x_2 -2x_3                     \\
	\mbox{subject to} & 6x_1^2  + 3x_2^2-2x_2x_3 + 3 x_3^2 -
		17x_1 + 8 x_2 - 14 x_3 \geq -19, \\
		& x_1 + 2 x_2 + x_3 \leq 5, \\
		& 5 x_2 + 3 x_3 \leq 7, \\
	 &  0 \leq x_1 \leq 2, \ 0 \leq x_2 \leq  2.  
	\end{array}
\right\} 
\label{Bex314}
\end{equation}

\subsection{The GAMS scalar format}

% Skim 5/15/07
The \GMS format describing the POP (\ref{Bex314})  is: 
\begin{verbatim}
* example1.gms
* This file contains the GAMS scalar format description of the problem 
*
* minimize objvar = -2*x1 +3*x2 -2*x3
* subject to 
*       x1^2 + 3*x2^2 -2*x2*x3 +3*x3^2 -17*x1 +8*x2 -14*x3 >= -19, 
*       x1 + 2*x2 + x3 <= 5, 
*       0 <= x1 <= 2, 0 <= x2 <= 1. 
*
* To solve this problem by sparsePOP.m:
* >> param.relaxOrder = 3;
* >> sparsePOP('example1.gms',param); 
* 
* This problem is also described in terms of the SparsePOP format 
* in the file example1.m.  See Section 3 of the manual.
*
* To obtain a tight bound for the optimal objective value by the function
* sparsePOP.m, set the parameter param.relaxOrder = 3.

* The description consists of 5 parts except comment lines
* starting the character '*'. The 5 parts are:
* < List of the names of variables >
* < List of the names of nonnegative variables >
* < List of the names of constraints >
* < The description of constraints >
* < Lower and upper bounds of variables  >

* < List of the names of variables >
Variables  x1,x2,x3,objvar;
* 'objvar' represents the value of the objective function.

* < List of the names of nonnegative variables >
Positive Variables x1, x2;

* < List of the names of constraints >
Equations  e1,e2,e3,e4;

* < The description of constraints >
* Each line should start with the name of a constraint in the list of names
* of constraints,  followed by '.. '. The symbols '*', '+', '-', '^', '=G='
* (not less than), '=E=' (equal to) and '=L=' (not larger than) can be used 
* in addition  to the variables in the list of the names of variables and real 
* numbers. One constraint can be described in more than one lines; 
* for example, 
* e2..    - 17*x1 + 8*x2 - 14*x3 +6*x1^2 + 3*x2^2 - 2*x2*x3 + 3*x3^2 =G= -19;
* is equivalent to
* e2..     - 17*x1 + 8*x2 - 14*x3 +6*x1^2
*                + 3*x2^2 - 2*x2*x3 + 3*x3^2 =G= -19;
* Note that the first letter of a line can not be '*' except comment lines.

* minimize objvar = -2*x1 +3*x2 -2*x3
e1..    2*x1 - 3*x2 + 2*x3 + objvar =E= 0;

* 6*x1^2 + 3*x2^2 -2*x2*x3 +3*x3^2 -17*x1 +8*x2 -14*x3 >= -19
e2..    - 17*x1 + 8*x2 - 14*x3 +6*x1^2 + 3*x2^2 - 2*x2*x3 + 3*x3^2 =G= -19;

* x1 + 2*x2 + x3 <= 5
e3..    x1 + 2*x2 + x3 =L= 5;

* 5*x2 + 2*x3 = 7
e4..    5*x2 + 2*x3 =E= 7;

* < Lower and upper bounds on variables  >
* Each line should contain exactly one bound; 
* For 0.5 <= x3 <= 2, we set 
* x3.lo = 0.5; 
* x3.up = 2; 
* A line such that 'x3.lo = 0.5; x3.up = 2;' is not allowed.

* x1 <= 2
x1.up = 2;

* x2 <= 1
x2.up = 1;

* end of example1.gms
\end{verbatim}

% Skim 5/15/07
Many examples of the GAMS scaler format of POPs can be found in the directory 
\[
\mbox{example/GMSformat/}, 
\]
which are from \cite{GLOBAL}; we have added and/or modified lower and upper bounds of some 
of the problems. 

If polynomials in the \GMS format include parentheses, then
% Skim 5/15/07
 {\sf param.symbolicMath} $= 1$ should be set.   The value $1$, in sparsePOP.m, is sent 
% via  the second input argument {\sf symbolicMath } of readGMS.m 
 to  the function readGMS.m where symbolic expansion takes place using the Symbolic Math 
Toolbox. If the Symbolic Math Toolbox 
is not available, %parentheses need to be expanded manually 
expanded polynomials should be prepared in the \GMS format and 
%before applying the readGMS.m or the sparsePOP.m. 
set {\sf param.symbolicMath} $= 0$. 

% Skim 6/5/07
We note that the  \GMS format is different from the GAMS format.
The \GMS format is produced by the program ``convert" from the GAMS  format. In the \GMS format
that can be used in SparsePOP,
the right-hand side of an inequality or an equality should be a single constant, 
``objvar" is reserved as a keyword to represent the value of the objective function,
 only multivariate polynomials are handled, and
no integer constraint is allowed. As seen in example1.gms, ``Variables", ``Positive variables",
``Equations" can not appear more than once. For more details, we refer to the 
examples included in SparsePOP and \cite{GAMS}.

%GAMS scaler format is (very)  different from GAMS format.
%The GAMS scaler format is produced by the program `convert' from GAMS  format.
%Unfortunately, I couldn't find the exact definition of GAMS scalar  format,
%but probably this is the definition; what `convert' produces is called
%the GAMS scalar format.
%As it is processed by a program, it is natural that the right-hand  
%side is just a constant.
%Also, there's only one Variables keyword is allowed. This is true  
%even in GAMS format.
%Although not explicitly written, it seems that multiple lines are  
%always allowed, e.g.,
%Equations e1, e2,e3,
 %                  e4,e5;
%As far as I searched, I couldn't find the special keyword objvar.
%Instead, GAMS scalar format uses
%Model m;
%Solve m using SparsePOP minimizing objvar;
%There are many other keywords which we are ignoring.
%It is written in the `convert' manual that real variables are named x1
%to xn where n is the number of variables. Similarly, equations are e1  to em.
%This is because`convert' eliminates information on the source of the  
%problem
%as much as possible. Obviously we don't need to keep this rule.


\subsection{The SparsePOP format}

Alternatively, a POP 
can be described directly using the SparsePOP format. A polynomial class 
is defined for this purpose as follows: 
\begin{center}
\begin{tabular}{rcrll}
{\sf poly.typeCone} 	& = 	& 1 & if $f(\x) \in \Real[\x]$ is used as an objective function, \\
                            & =    & 1 & if $f(\x) \in \Real[\x]$ is used as an inequality constraint $f(\x) \geq 0$, \\ 
                            & =    & -1  & if $f(\x) \in \Real[\x]$ is used as an equality constraint $f(\x) = 0$. \\ 
			%		&	& 2  if $f(\x) \in \QC^{\ell} [\x]$. \\
			%		& 	& 3  if $f(\x) \in \SC^{\ell}[\x]$. \\
%poly.sizeCone	&= 	& 1  & if  $f(\x) \in \Real[\x]$. \\
%			%		&	&$\ell$ if $f(\x) \in \QC^{\ell} [\x]$. \\
%			%		&	&$\ell$  if $f(\x) \in \SC^{\ell}[\x]$. \\
{\sf poly.degree	}	& = 	& & the degree of $f(\x)$. \\
{\sf poly.dimVar	}	& =	& & the dimension of the variable vector $\x$. \\
{\sf poly.noTerms}  		& = 	& & the number of terms of $f(\x)$. \\
{\sf poly.supports}		& = 	& & a set of supports of $f(\x)$, \\
				&	& & a poly.noTerms $\times$ poly.dimVar matrix. \\
%\end{tabular}
%\end{center}
%
%\begin{center}
%\begin{tabular}{rcll}
{\sf poly.coef}			& = 	& & coefficients, \\ 
% Skim 5/15/07
				&	& & a column vector of poly.noTerms dimension. \\
%				&	& & if poly.typeCone = 1 and poly.sizeCone = 1, \\
			%	& 	& a poly.noTerms $\times$ $\ell$ matrix \\
			%	&	& if poly.typeCone = 2 and poly.sizeCone = $\ell$. \\
			%	& 	& a poly.noTerms $\times$ $\ell^2$ matrix \\
			%	&	& if poly.typeCone = 3 and poly.sizeCone = $\ell$, \\
			%	&	& where each $\ell \times \ell$ coefficient matrix is described \\
			%	&	& as a $\ell^2$ row vector. 
\end{tabular}
\end{center}
% Skim 5/15/07
The name {\sf objPoly} is   for the objective polynomial function $f_0(\x)$ and 
{\sf ineqPolySys$\{j\}$} $(j=1,2,\ldots,m)$ for the polynomials $f_j(\x)$ $(j=1,2,\ldots,m)$ of the 
constraints. 
The problem (\ref{Bex314}) is described using the polynomial class as follows. 

\begin{verbatim}
function [objPoly,ineqPolySys,lbd,ubd] = example1;
%%%%%%%%%%%%%%
% example1.m  
%%%%%%%%%%%%%%
%
% The SparsePOP format data for the example1: 
%
% minimize -2*x1 +3*x2 -2*x3
% subject to 
%       x1^2 + 3*x2^2 -2*x2*x3 +3*x3^2 -17*x1 +8*x2 -14*x3 >= -19, 
%       x1 + 2*x2 + x3 <= 5, 
%       5*x2 + 2*x3 = 7, 
%       0 <= x1 <= 2, 0 <= x2 <= 1. 
% 
% To solve the problem by sparsePOP.m: 
% >> param.relaxOrder = 3;
% >> sparsePOP('example1',param);
% 
% This problem is also described in terms of the GAMS scalar format in the 
% file example1.gms.  See Section 3 of the manual.
%
%'example1'
% objPoly 
% -2*x1 +3*x2 -2*x3
        objPoly.typeCone = 1;
        objPoly.dimVar   = 3;    
        objPoly.degree   = 1;    
        objPoly.noTerms  = 3; 
        objPoly.supports = [1,0,0; 0,1,0; 0,0,1]; 
        objPoly.coef     = [-2; 3; -2]; 
% ineqPolySys
% 19 -17*x1 +8*x2 -14*x3 +6*x1^2 +3*x2^2 -2*x2*x3 +3*x3^2 >= 0,
        ineqPolySys{1}.typeCone = 1;
        ineqPolySys{1}.dimVar   = 3;    
        ineqPolySys{1}.degree   = 2;    
        ineqPolySys{1}.noTerms  = 8; 
        ineqPolySys{1}.supports = [0,0,0; 1,0,0; 0,1,0; 0,0,1; ...
                                   2,0,0; 0,2,0; 0,1,1; 0,0,2]; 
        ineqPolySys{1}.coef     = [19; -17; 8; -14; 6; 3; -2; 3]; 
%    
% 5 -x1 -2*x2 -x3  >= 0.   
        ineqPolySys{2}.typeCone = 1;
        ineqPolySys{2}.dimVar   = 3;    
        ineqPolySys{2}.degree   = 1;    
        ineqPolySys{2}.noTerms  = 4; 
        ineqPolySys{2}.supports = [0,0,0; 1,0,0; 0,1,0; 0,0,1]; 
        ineqPolySys{2}.coef     = [5; -1; -2; -1];
%    
% 7 -5*x2 -2*x3 = 0. 
        ineqPolySys{3}.typeCone = -1;
        ineqPolySys{3}.dimVar   = 3;
        ineqPolySys{3}.degree   = 1;
        ineqPolySys{3}.noTerms  = 3;
        ineqPolySys{3}.supports = [0,0,0; 0,1,0; 0,0,1];
        ineqPolySys{3}.coef     = [7; -5; -2];
% lower bounds for variables x1, x2 and x3. 
% 0 <= x1, 0 <= x2, -infinity < x3: 
        lbd = [0,0,-1.0e10];
% upper bounds for variables x1, x2 and x3
% x1 <= 2, x2 <= 1, x3 < infinity: 
        ubd = [2,1,1.0e10];
return
% end of example1.m
\end{verbatim}

% Skim 5/15/07 indicating
We note that -1.0e10 in lbd and 1.0e10 in ubd mean $-\infty$ and $\infty$, respectively, indicating $x_3$ 
can take any value in the above example.  
% Kojima 5/18/07 
The functions simplifyPolynomial.m, 
plusPolynomials.m, and multiplyPolynomials.m in the directory subPrograms/Mfiles/ are useful in 
describing a POP in terms of the SparsePOP format. See Rosenbrock.m, and also other 
.m files in the directory \mbox{example/POPformat/} for more general description of the 
SparsePOP format.  


