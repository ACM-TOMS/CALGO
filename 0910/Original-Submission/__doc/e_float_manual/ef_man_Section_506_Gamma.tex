\lstsetCPlusPlus
\begin{lstlisting}
e_float    gamma               (const e_float& x);
ef_complex gamma               (const ef_complex& z);
e_float    incomplete_gamma    (const e_float& a, const e_float& x);
e_float    gen_incomplete_gamma(const e_float& a, const e_float& x0,
                                                  const e_float& x1);
\end{lstlisting}

\vspace{6.0pt}

\noindent {\it Effects:} These functions compute the gamma functions
for their respective arguments {\courier a}, {\courier x}, {\courier x0},
{\courier x1}, {\courier z}, with
{\courier a}, {\courier x}, {\courier x0}, {\courier x1}~$\in\mathbb{R}$;
{\courier z}~$\in\mathbb{Z}$.

\vspace{6.0pt}

\noindent {\it Returns:} The {\courier gamma} function
returns~\cite{wolframfunctions:website}

\begin{equation}
\Gamma(x) \ = \ \int_{0}^{\infty} t^{a-1} e^{-t} dt\quad.
\end{equation}

\vspace{6.0pt}

\noindent {\it Returns:} The {\courier incomplete\underline\ gamma} function
returns~\cite{wolframfunctions:website}

\begin{equation}
\Gamma(a, x) \ = \ \int_{x}^{\infty} t^{a-1} e^{-t} dt\quad.
\end{equation}

\noindent {\it Returns:} The (generalized) {\courier gen\underline\ incomplete\underline\ gamma}
function returns~\cite{wolframfunctions:website}

\begin{equation}
\Gamma(a, x1, x2) \ = \ \int_{x1}^{x2} t^{a-1} e^{-t} dt\quad.
\end{equation}

\noindent {\it Remark:} The {\courier incomplete\underline\ gamma} function
and the {\courier gen\underline\ incomplete\underline\ gamma} function
are implemented as utility functions. They do not always maintain precision
over the entire range of their parameters.

\vspace{6.0pt}

\lstsetCPlusPlus
\begin{lstlisting}
e_float beta           (const e_float& a,    const e_float& b);
e_float beta           (const ef_complex& a, const ef_complex& b);
e_float incomplete_beta(const e_float& x,    const e_float& a,
                                             const e_float& b);
\end{lstlisting}

\vspace{6.0pt}

\noindent {\it Effects:} These functions compute the beta functions
for their respective arguments {\courier a}, {\courier b}, {\courier x},
with {\courier a}, {\courier b}, {\courier x}~$\in\mathbb{R}$; or
{\courier a}, {\courier b}, {\courier z}~$\in\mathbb{Z}$.

\vspace{6.0pt}

\noindent {\it Returns:} The {\courier beta} function
returns~\cite{wolframfunctions:website}

\begin{equation}
B(a,b) \ = \ \frac{\Gamma(a) \Gamma(b)}{\Gamma(a+b)}\quad.
\end{equation}

\vspace{6.0pt}

\noindent {\it Returns:} The {\courier incomplete\underline\ beta} function
returns~\cite{wolframfunctions:website}

\begin{equation}
B_{x}(a,b) \ = \ \int_{0}^{x} t^{a-1} (1-t)^{b-1} dt\quad.
\end{equation}

\noindent {\it Remark:} The {\courier incomplete\underline\ beta} function
is implemented as a utility function. It does not cover the entire range of
its parameters.

\vspace{6.0pt}

\lstsetCPlusPlus
\begin{lstlisting}
e_float factorial (const UINT32 n);
e_float factorial2(const  INT32 n);
\end{lstlisting}

\vspace{6.0pt}

\noindent {\it Effects:} These functions compute the factorial or double factorial
functions for their respective argument {\courier n},
with {\courier n}~$\in\mathbb{N}^{+}$.

\vspace{6.0pt}

\noindent {\it Returns:} The {\courier factorial} function
returns~\cite{wolframfunctions:website}

\begin{equation}
n! \ = \ \prod_{k=1}^{n} k\quad.
\end{equation}

\noindent {\it Returns:} The {\courier factorial2} function (double factorial)
returns~\cite{wiki:website}

\begin{equation}
n!! = \left\{ \begin{array}{lll}
 &\displaystyle 2^{n/2} (n/2)! &\mbox{ if $n$ is even,} \\ \\
 &\displaystyle \left({\genfrac{}{}{0pt}{0}{\dfrac{n}{2}}{\dfrac{n-1}{2}}}\right) 2^{(n-1)/2}
 \left[(n-1)/2\right]! &\mbox{ if $n$ is odd.}
       \end{array} \right.
\end{equation}

\lstsetCPlusPlus
\begin{lstlisting}
e_float binomial(const UINT32 n,   const UINT32 k);
e_float binomial(const UINT32 n,   const e_float& y);
e_float binomial(const e_float& x, const UINT32 k);
e_float binomial(const e_float& x, const e_float& y);
\end{lstlisting}

\vspace{6.0pt}

\noindent {\it Effects:} These functions compute the binomial functions
for their respective arguments {\courier n}, {\courier k}, {\courier x}, {\courier y},
with {\courier n}, {\courier k}~$\in\mathbb{N}^{+}$;
{\courier x}, {\courier y}~$\in\mathbb{R}$.

\vspace{6.0pt}

\noindent {\it Returns:} The {\courier binomial} function in its integer form
returns~\cite{wolframfunctions:website}

\begin{equation}
\binom{n}{k} \ = \ \frac{n!}{k! (n-k)!}\quad.
\end{equation}

\noindent {\it Returns:} The {\courier binomial} function in its non-integer form
returns~\cite{wolframfunctions:website}

\begin{equation}
\binom{x}{y} \ = \ \frac{\Gamma(x+1)}{\Gamma(y+1) \Gamma(x-y+1)}\quad.
\end{equation}

\lstsetCPlusPlus
\begin{lstlisting}
e_float    pochhammer(const e_float& x,    const UINT32 n);
ef_complex pochhammer(const ef_complex& z, const UINT32 n);
e_float    pochhammer(const e_float& x,    const e_float&    a);
ef_complex pochhammer(const ef_complex& z, const ef_complex& a);
\end{lstlisting}

\vspace{6.0pt}

\noindent {\it Effects:} These functions compute the Pochhammer functions
for their respective arguments.

\vspace{6.0pt}

\noindent {\it Returns:} The {\courier pochhammer} function in its integer form
returns~\cite{wolframmathworld:website}

\begin{equation}
x_{n} \ = \ x(x+1)\dots(x+n-1)\quad.
\end{equation}

\noindent {\it Returns:} The {\courier pochhammer} function in its non-integer form
returns~\cite{wolframmathworld:website}

\begin{equation}
x_{a} \ = \ \frac{\Gamma(x+a)}{\Gamma(x)}\quad.
\end{equation}

