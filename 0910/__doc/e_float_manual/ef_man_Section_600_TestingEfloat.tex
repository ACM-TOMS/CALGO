\subsection{The Test System}

Testing is essential for establishing the reliability of a system
with such high complexity and exactness.
Consequently, a great deal of effort has been invested in testing
and verification of \efloat.

Testing has been performed using several test methods.
During the development and verification stages of \efloat,
there have been countless spot--checks,
Wronskian analyses, comparisons among the different
\efloatclass\ implementations, as well as comparisons with
\mathematica\ and MPFR. At times, some algorithms have been verified
by temporarily deactivating them via source code modification,
subsequently forcing a different computation method
such as recursion to be used instead.
For example, asymptotic Bessel function calculations have been
verified in part by forcing stable recursion over many orders
to be used instead of asymptotic expansion and verifying that the
results computed from both methods agree to full precision.

A large--scale dedicated automatic test system has been developed
to provide for reproducible testing of \efloat\ in combination with
detailed performance and code coverage analyses.
The automatic test system consists of
three parts --- the automatic test case generator,
the test suite and the test execution system
(see Figure~\ref{figure:architecture}).

The test case generator creates test cases in the form of
source code which are subsequently manually added
to the test suite, and then compiled and executed by the
test execution system. Every test case is designed to test
one or several functions, algorithms or convergence zones.
In order to improve function and block coverage in testing,
the design of the test cases has been partially guided by
code coverage analyses using
Intel{\footnotesize {\textregistered}}'s ``codecov'' tool.
Each test case contains a short, automatically generated
C++ code sequence, usually involving a loop calculation,
which calculates test values and compares these directly with
control values, themselves pre--computed to $400$ digits with
\mathematica\ and directly written in the test case as strings.

The test suite contains about $250$ real--num\-ber\-ed test cases
and about $30$ com\-plex--num\-ber\-ed test cases. Each test case
computes $\sim$$1$--$100$ individual numerical values,
resulting in a total of $\sim$$10$,$000$ test results.
In addition to automatic verification, the test results
are written to log files upon test execution.
A test case passes its execution if there is full agreement
between each individual numerical result and its corresponding
control value, up to and including the very last digit of precision
given by the precision setting. If any single digit of any single
result does not agree with that of its control value, then the
whole test case fails.

All test cases in the entire test suite pass fully for each one
of the main three
\efloatclass\ classes \efxefloatclass, \gmpefloatclass\ and
\mpfrefloatclass\ at $30$, $50$, $100$, $200$ and $300$ digits
of precision, using each compiler system listed in
Table~\ref{table:compilers}.
The afore--mentioned test result is a very significant technical result.
All testing evidence indicates
that \efloat\ correctly calculates all of the function values for the
parameter ranges listed in Table~\ref{table:functions} to full precision
ranging from $30$ to $300$ digits using any one of the main three
\efloatclass\ classes \efxefloatclass, \gmpefloatclass\ or \mpfrefloatclass.

Certainly, a great deal of effort has gone into verification of \efloat.
Nonetheless, some function values and parameter ranges remain
poorly tested. The author was not able to find independent
control data --- or even found conflicting results --- for some
parameter regions. Several non--confirmed values which have been
computed with \efloat\ are shown below.

\vspace{5.0pt}

{\raggedright
$P_{\pi+{\mbox{\tiny{20,000}}}}^{\gamma+{\mbox{\tiny{10,000}}}}\left(\frac{2}{3}\right)\,\approx\,$
{\courierNine -5.9808839017059291250180839952610805896015696701604662259639540559236
\nopagebreak
\\ \ \ \ \ \ \ \ \ \ \ \ \ \ \ 
14936288067418735478313503042667}\courierNineMakeTenPowX{42814}
}

\vspace{5.0pt}

{\raggedright
$Q_{{\mbox{\tiny{2347}}}}^{{\mbox{\tiny{-2099}}}}\left(\gamma\right)\,\approx\,$
{\courierNine 5.13447421240962834386055286923384862133503967053295289159821523454450314
\nopagebreak
\\ \ \ \ \ \ \ \ \ \ \ \ \ 
9892985668770122300019211348}\courierNineMakeTenPowX{-6860}
}

\vspace{5.0pt}

{\raggedright
$J_{\pi+10^{8}}\left(G\cdot 10^{8}\right)\,\approx\,$
{\courierNine 0.0000710158786472151060060900254513645945033058828414017520190813
\nopagebreak
\\ \ \ \ \ \ \ \ \ \ \ \ \ \ \ \ \ \ \ 
5062824517495900855118876479980481012041}
}

\vspace{5.0pt}

{\raggedright
$K_{{\mbox{\tiny{690}}}}\left(\pi+310\right)\,\approx\,$
{\courierNine 2.473093134580761699282091225752444991289365302904873098822780430687
\nopagebreak
\\ \ \ \ \ \ \ \ \ \ \ \ \ \ \ \ \ \ 
162258158192556310616234787924259}\courierNineMakeTenPowX{128}
}

\vspace{5.0pt}

{\raggedright
$D_{\frac{1201}{12}}\left(40\right)\,\approx\,$
{\courierNine 1.524674031836673520032956906376264551740269827916850281507352611567663
\nopagebreak
\\ \ \ \ \ \ \ \ \ \ \ \ \ \ 
272230385469253930112335260440}\courierNineMakeTenPowX{-15}
}

\subsection{Extending the Test System}

The test system can be extended by adding a dedicated test case.

\begin{itemize}
\item Each individual test case must be implemented in its own individual
automatically generated source file.
\item The test case sources are generated with a standalone project
called ``Test\-Case\-Generator'' located in the directory
{\courier{\underline\ \underline\ Test\-Case\-Generator}}.
\item Test case generation is only supported for
Windows{\footnotesize {\textregistered}} platforms.
\item Design the parameters of the test case and add the relevant
source code the the test case generator project. Use the many examples
of this in the source code in order to add the new case.
Note that it takes a long time to generate all of the test case
sources. Therefore, a judicious use of commenting can be used to
comment--out the test cases which do not need to be generated.
\item When the desired new test case file has been generated, add it
to the \efloat\ project in the appropriate directory and add it to the
solution either in Developer Studio or the the relevant file list
for GNUmake.
\item The new test case can now be compiled and the call of the
test case function can be added to the list of test cases, for example
in {\courier{test\underline\ real.cpp}} or {\courier{test\underline\ imag.cpp}}.
\end{itemize}
