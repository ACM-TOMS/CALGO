\lstsetCPlusPlus
\begin{lstlisting}
e_float hyperg_0f0    (const e_float& x);
e_float hyperg_0f1    (const e_float& b, const e_float& x);
e_float hyperg_0f1_reg(const e_float& a, const e_float& x);
e_float hyperg_1f0    (const e_float& a, const e_float& x);
e_float hyperg_1f1    (const e_float& a, const e_float& b, const e_float& x);
e_float hyperg_1f1_reg(const e_float& a, const e_float& b, const e_float& x);
e_float hyperg_2f0    (const e_float& a, const e_float& b, const e_float& x);
e_float hyperg_2f1    (const e_float& a, const e_float& b, const e_float& c,
                       const e_float& x);
e_float hyperg_2f1_reg(const e_float& a, const e_float& b, const e_float& c,
                       const e_float& x);
e_float hyperg_pfq    (const std::deque<e_float>& a, const 
                             std::deque<e_float>& b, const e_float& x);
\end{lstlisting}

\vspace{6.0pt}

\noindent {\it Effects:} These functions compute the hypergeometric series
for their respective real valued arguments.

\vspace{6.0pt}

\noindent {\it Remark:} There is support for hyper\-geometric functions,
but these are primarily implemented as utilities for other calculations
and not intended to cover the complete parameter ranges. These series
have limited convergence, generally only for arguments less than unity
in magnitude. Some of the series do not check for negative integer
numerator parameters. Others do.

\vspace{6.0pt}

\noindent {\it Remark:} The suffix `{\courier \underline\ reg}' denotes a
`regularized' hyper\-geometric function. For example the function
{\courier hyperg\underline\ 2f1\underline\ reg} is the regularized
hyper\-geometric function, defined in \cite{wolfram:textbook} and
\cite{wolframfunctions:website} by

\begin{equation}
_{2}\widetilde{F}_{1}(a, b; c; x) = {_{2}F_{1}}(a, b; c; x)/\Gamma(c).
\end{equation}

\lstsetCPlusPlus
\begin{lstlisting}
e_float conf_hyperg(const e_float& a, const e_float& c,
                    const e_float& x);
e_float      hyperg(const e_float& a, const e_float& b, const e_float& c,
                    const e_float& x);
\end{lstlisting}

\vspace{6.0pt}

\noindent {\it Remark:} The function name of the hyper\-geometric series
{\courier conf\underline\ hyperg} is synonymous with the function name of
the hyper\-geometric series {\courier hyperg\underline\ 1f1}.

\vspace{6.0pt}

\noindent {\it Remark:} The function name of the hyper\-geometric series
{\courier hyperg} is synonymous with the function name of the hyper\-geometric
series {\courier hyperg\underline\ 2f1}.


