\documentstyle[12pt]{article}
\topmargin       0in
\textheight      8in 
\oddsidemargin   0in
\evensidemargin  0in
\textwidth       6.5in
\pagestyle{empty}
\newfont{\msbm}{msbm10 scaled\magstephalf}
\newcommand{\Z}{\mbox{\msbm Z}}
\newcommand{\dfrac}{\displaystyle\frac}
\newcommand{\ul}{\underline}
\newenvironment{romlist}{\begin{list}{(\roman{romnum})}
    {\usecounter{romnum}\setlength{\topsep}{1pt}
    \setlength{\itemsep}{1pt}}
    \rm}{\end{list}}

\newcommand{\zettleq}{\raise.08in\hbox{.} 
\!\!\! =\!\!\!\raise-.03in\hbox{.} }

\begin{document}
\baselineskip=24pt
\begin{center}
SLEIGN2
\end{center}
\vspace*{.2in}
\noindent
{\bf Commentary on the individual examples in xamples.f}

These examples have been chosen to illustrate the capabilities and
limitations of the program SLEIGN2.  Many of
the examples have been chosen from special cases of the well known and
well studied ``special functions" of mathematical analysis.  All
possible cases of end-point classifications are represented; general separated
regular and singular boundary conditions may be applied to these examples
as appropriate. Also coupled periodic-type boundary conditions can be 
used in the regular case.

For some of these examples it is possible to give explicit information
on the spectrum of associated boundary value problems; this can take the
form of providing explicit formulas for eigenvalues against which the
program calculated results can be compared.

In all cases of limit-circle end-points boundary condition functions $u$
and $v$ have been entered as part of the example data.  In the case of
limit-circle non-oscillatory end-points we use the convention that the
boundary condition function $u$ determines the principal or Friedrichs
boundary condition.

On selecting a numbered example the the differential
equation is displayed in FORTRAN, and details of the end-point
classification given.  If information on the form of the boundary condition
functions $u$ and $v$ is required then the user should scroll through
the xamples.f file to the appropriate  numbered part of the $u, \ v$
section.  

Some regular and weakly regular problems can be more successfully run using the 
limit-circle non-oscillatory (LCNO) algorithm; details are given below for some of
the examples.  

It should be noted that for the limit-circle oscillatory problems it is
sometimes difficult to compute numerically more than a few of the
eigenvalues.  

One example is included for which the program fails; see Laguerre
\#22.  This problem has a discrete spectrum and for one particular
boundary condition the eigenvalues are known explicitly.  In this case
numerical values to confirm the details of the spectrum can be obtained
by use of the Liouville transformation; this leads to
the Laguerre/Liouville
example \#23 for which the program is successful.

The Liouville transformation has also been applied to the Jacobi
equation, \#16 to yield the Jacobi/Liouville example \#24.

The Liouville transformation is sometimes useful to get a
Sturm-Liouville differential equation into a form more suitable for numerical
computation.

{\bf Parameters.}  Many of the examples involve the choice of one or more
parameters; the range of these parameters is given when the numbered
differential equation is displayed.  If a choice of parameter is made
outside of the stated range the program may abort.

{\bf Remarks on the individual examples.}
\begin{enumerate}
\item%1
{\bf Classical Legendre equation}
\begin{itemize}
\item[(i)]
The Legendre polynomials are obtained by taking the principal
(Friedrichs) boundary condition at both end-points $\pm 1:$ enter $A1=1,
\; A2=0, \; B1 = 1 , B2 = 0 $ i.e. take the boundary condition function
$u$ at $ \pm 1$; eigenvalues:  $\lambda_n
 = (n+1/2 ) (n+ 1/2) \ ; n = 0,1,2, \cdots  $; eigenfunctions:
Legendre polynomials $P_n(x)$.
\item[(ii)]
Enter $A1=0, \; A2 = 1, \; B1 = 0, \; B2=1, $ i.e. use the b.c. function 
$v$ at $\pm 1$;
eigenvalues:  $\mu_n ; n=0,1,2,\cdots $ no explicit formula is available;
eigenfunctions: are logarithmically unbounded at $\pm 1$.  
Observe that $\mu_n < \lambda_n  < \mu_{n+1} \ ; n=0,1,2 \cdots $.  \cite{T}:
Chapter IV.
\end{itemize}
\item %2
{\bf Bessel equation}\\
This is the Liouville form of the classical Bessel equation.
\begin{itemize}
\item[(1)]
Problems on $(0,1] $ with $y(1) = 0$
\begin{itemize}
\item[(i)]
$0 \leq \nu < 1 , \; \nu \neq \dfrac{1}{2} $; the Friedrichs case: $A1=1$,
$A2 =0$ yields the classical Fourier-Bessel series; $ \lambda_n = j_{\nu,n}^2 
$ where $\{ j_{\nu,n}, n : n=0,1,2,...\}$ are the zeros (positive) of the
Bessel function $J_{\nu}(x)$.
\item[(ii)]
$\nu  \geq 1$; limit-point at $0$ so unique boundary value problem with $
\lambda_n = j_{\nu,n}^2 $ as before
\item[(iii)]
these are similar to (i) and (ii) when $\nu < 0$
\end{itemize}
\item[(2)]
Problems on $[1, \infty );$ continuous spectrum on $[0, \infty )$
\begin{itemize}
\item[(i)]
for Dirichlet and Neumann b.c. at $1$ there are no eigenvalues.
\item[(ii)]
for $A1 = A2 = 1$ at $1$ there is one isolated eigenvalue.
\end{itemize}
\item[(3)]
Problems on $(0, \infty);$ continuous spectrum on $[0, \infty)$
\begin{itemize}
\item[(i)]
no eigenvalues if $\nu \geq 1$
\item[(ii)]
for $0 \leq \nu < 1 $ the Friedrichs case is given by $A1=1$, $A2=0$;
there are no eigenvalues
\item[(iii)]
if $A1 = 5, \; A2=1$ there is one isolated eigenvalue near -46.64.\\
\hfill \cite{T}: Chapter IV and  \cite{W}: Chapter XVIII.
\end{itemize}
\end{itemize}
\item %3
{\bf The Halvorsen equation}

This equation is $R$ at $0$ and LCNO at $+ \infty$ so that the spectrum
is discrete and bounded below for all b.c..  However, this example
illustrates that even a regular end-point can cause difficulties for computation.
The program fails on $R$ at $0$; is successful for $WR$ at $0$; is
successful for LCNO at $0$ (use $u(x) = x, \ v(x) = 1$ at $0$; the given
$u$ and $v$ at $\infty $ and employ the double limit-circle entry at $0$
and $\infty $ with $c=1$ - see "Help" h4 for more details).

At $0$, with $u(x) = x$ and $v(x) =1$, the principal entry is $A1 = 1,
\ A2 = 0$; at $\infty $ with $u(x) = 1, \ v(x) = x$ the principal entry
is also $A1=1, \; A2=0$ but note the interchange of $u$ and $v$.
\item %4
{\bf The Boyd equation}

This equation arises in a model studying eddies in the atmosphere;
see \cite{B}.  There is no explicit formula for the eigenvalues of any
particular boundary condition; eigenfunctions can be given in terms of
Whittaker functions; see \cite{BEZ}; example 3.
\item%5
{\bf The regularized Boyd equation}

This is a WR form of equation 4; the singularity at  zero has been
regularized using quasi-derivatives.  There is a close relationship
between the examples 4 and 5; in particular they have the same eigenvalues 
- see \cite{AEZ}.  For a general discussion of
regularization using non-principal solutions see \cite{NZ}.  For numerical
results see \cite{BEZ}; example 3.
\item %6
{\bf The Sears-Titchmarsh equation}

This differential equations has one LP and one LCO end-point.  For
details of boundary value problems on $[1, \infty )$ see \cite{BEZ}: example
4.  The equation was studied originally in \cite{T}; Chapter IV; but see
\cite{ST}.

For problems on $[1, \infty)$, with separated boundary conditions, the
spectrum is simple and discrete but unbounded above and below.

Numerical results are given in \cite{BEZ}: example 4.
\item %7
{\bf The BEZ equation}

This equation is similar to equation 6.  On the interval $(0, 1]$ there
is a singularity at $0$ in LCO; the equation is R at 1.

For numerical results see \cite{BEZ}: example 5.
\item %8
{\bf The LaPlace tidal wave equation}

This equation is a particular case of the move general equation with
this name; for details and references see \cite{H}

There are no representations for solutions in terms of the well-known
special functions.  Thus to determine boundary conditions at the LCNO
end-point $0$ use has to be made of maximal domain functions; see the
$u, \ v$ section for this equation.  Numerical results are given in
\cite{BEZ}: example 8.
\item %9
{\bf The Latzko equation}

This differential equation has a long and celebrated history; see \cite{F}:
pages 43 to 45. There is a LCNO singularity at 1 which requires the
use of maximal domain functions; see the $u, \ v$ section.  The
end-point $0$ is WR due to the fact that $w(0) = 0$.

This example is similar in some respects to the Legendre equation of
example 1.

For numerical results see \cite{BEZ}: example 7.
\item%10
{\bf A weakly regular equation}

This is a devised example to illustrate the computational difficulties
of weakly regular problems.

The differential equation gives $p(0) = 0$
and $w(0) = \infty $ but nevertheless $0$ is a regular end-point in the
Lebesgue integral sense, but has to be classified as weakly regular in
the computational sense.

The Liouville normal form of this equation is the Fourier equation; see
example 21.

There are explicit solutions of this equation when $\lambda = 0$ given by
$$ cos (2x^{1/2} \surd \lambda ) \  \ ; sin (2x^{1/2} \surd \lambda ) /
\surd \lambda . $$

If $0$ is treated as a LCNO end-point then $u, \ v$ boundary condition
functions are
$$ u(x) = 2x^{1/2} \ \ ; v(x) = 1 . $$
{\bf WR at 0} The regular condition $D \; y(0) = 0 $ is equivalent to the 
singular condition $[y,u](0) = 0$ {\bf LCNO}
\\
Similarly the regular condition $N \; (py')(0) = 0$ is equivalent to the 
singular condition $ [y, \ v] (0) = 0$.

The following indicated boundary value problems have the given explicit
formulae for the eigenvalues:
$$ y(0) = 0 {\mbox{ or }} [y, \ u] (0) = 0 {\mbox{ and }} y(1) = 0 \; \;
\lambda_n = ((n+1) \pi)^2 /4 \ (n=0, 1,... ) $$
$$ (py)'(0) = 0 {\mbox{ or }} [y,v](0) = 0 {\mbox{ and }} (py')(1) = 0
\; \; \lambda_n  = ((n+ \dfrac{1}{2} ) \pi ) ^2 /4 \ (n=0,1, ...). $$
\item %11
{\bf The Plum equation}

Plum \cite{P} computed the first seven periodic eigenvalues using a numerical
homotopy method together with interval arithmetic and obtained rigorous
bounds for these seven computed eigenvalues.
\item %12
{\bf The Mathieu equation}

The classical Mathieu equation has a celebrated history and
voluminous literature. There are no eigenvalues for this problem on $(- \infty , +
\infty )$. There may be one negative eigenvalue of the problem on $[0, \infty)$
depending on the boundary condition at the end-point $0$.  The continuous
(essential) spectrum is the same for the whole line or half-line
problems and consists of an infinite number of disjoint closed
intervals. The endpoints of these - and thus the spectrum of the
problem - can be characterized in terms of periodic and semi-periodic
eigenvalues of S-L problems on the compact interval $[0, 2 \pi]$.  These
can be computed with SLEIGN2. These remarks also apply to the general S-L 
equation with periodic coefficients of the same period; the 
so-called Hill's equation.

Of special interest is the starting point of the continuous spectrum -
this is also the oscillation number of the equation.  It can be computed
with SLEIGN2. For the Mathieu equation ($p=1, \ q=sin(x), w=1$) on both the 
whole line and the half line it is appr. -0.378.
\item %13
{\bf The hydrogen atom equation}

This is the classical one-dimensional equation for quantum modelling of
the hydrogen atom; see \cite{T}: Chapter IV.

For the parameter $L=0$ the equation is LCNO at end-point $0$; the boundary
condition $A1=1, \; A2=0$ gives the Friedrichs extension which leads to
the classical eigenvalues and eigenfunctions.

For $L>0$ the end-point zero is LP.

For all these boundary value problems the continuous spectrum is 
$[0, \infty )$, and there are a countably infinite number 
of negative eigenvalues with a single cluster point at 0. The eigenvalues 
for the classical i.e. Friedrichs case are given by 
$$
\lambda_n = - \dfrac{1}{4(L+n-1)^2} \, , \ n=0,1,2, \ldots 
$$

\item %14
{\bf The Marletta equation}

This equation is R at 0 and LP at $\infty $; there is a continuous
spectrum on $[0, \infty )$; there is an isolated negative eigenvalue for
some boundary condition at 0. Both codes SLEIGN2 and SLEDGE report a second 
eigenvalue near 0, this may be due to the fact that there is a solution which is 
 NOT in
$L^2 (0, \infty)$ but is "nearly" in this space, thus deceiving these codes; 
details are in the Marletta certification 
report on SLEIGN (not SLEIGN2) \cite{M}.
\item %15
{ \bf The harmonic oscillator equation}

This is another classic equation. It is also the
Liouville normal form of the differential equation for the Hermite
orthogonal polynomials. On the whole real line the boundary value
problem requires no boundary conditions at the end-points of $\pm \infty
$. Thus there is a unique self-adjoint extension with discrete spectrum given by : 
$$ \{ \lambda_n = 2n+1 ; \; n=0,
1,2,...\}. $$
For a classical treatment see \cite{T}; Chapter IV, section 2.
\item %16
{\bf The Jacobi equation}

To obtain the classical orthogonal polynomials use the Friedrichs extension. This 
is determined by the boundary conditions as follows: \\
Endpoint +1 :  $$ -1 < \alpha < 0,  \ -1 < \beta, \ WR,  \ (py')(1) =0 $$  
$$ 0 \le \alpha < 1 : LCNO, 
\quad \ [y,u](1) = 0 = [y,v](1) =0. $$
$$ 1 \le \alpha : \ LP $$
Endpoint -1 :  $$ -1 < \beta  < 0,  \ -1 < \alpha \ WR,  \ (py')(-1) =0 $$  
$$ 0 \le \beta  < 1 : LCNO, 
\quad \ [y,u](-1) = 0 = [y,v](-1) =0. $$
$$ 1 \le \beta  : \ LP $$
For the classical orthogonal polynomials the eigenvalues are given by : 
$$
\lambda_n = n(n + \alpha + \beta + 1), \ n = 0,1,2, \ldots 
$$

\item % 17
{\bf The rotation Morse oscillator equation}

This classical problem has continuous spectrum $[0, \inf ]$ and 26 negative 
eigenvalues.

\item %18
{\bf The Dunsch equation}

Discussed in chapter VIII, pp. 1510-1520 of \cite{DS}. For $ 0 < \alpha < 1/2 $ , 
and $0 < \beta < 1/2$ we choose : \\
$$
at -1 : \ u_-(x) = (1 + x)^{\alpha}, \quad v_-(x) = (1+x)^{-\alpha}
$$
$$
at +1 : \ u_+(x) = (1 + x)^{\beta }, \quad v_+(x) = (1+x)^{-\beta}
$$
Note that these $u$ and $v$ are not solutions but maximal domain functions. D and S
state on p.1519 that the boundary value problem determined by  
$$
[y,u_-](-1) = 0 = [y,u_+](1)
$$
has eigenvalues given by :
$$
\lambda_n = (n + \alpha + \beta + 1)(n + \alpha + \beta), n = 0, 1, 2, \ldots 
$$

\item %19
{\bf The Donsch equation}

This is a modification of problem 18 which illustrates an LCNO/LCO mix. Replace 
$\alpha$ in 18 by $ i \gamma$. This changes the singularity at -1 from LCNO to 
LCO. Take $\gamma > 0$ and $ 0 \le \beta < 1/2 $. 
$$
At +1 : u(x) = (1-x)^{\beta}, \quad v(x) = (1-x)^{-\beta} 
$$
$$
At -1 : u(x) = cos(log(1+x)), \quad v(x) = sin(log(1+x)) 
$$
Again these $u$ and $v$ are not solutions but maximal domain functions.

\item %20.

{ \bf The Krall equation}
This example should be seen as a special case of the Bessel equation 2
above. Solutions can be obtained in terms of the modified Bessel
functions.

To help with the computations for this example the spectrum is
translated by a term $+1$; this simple devise is used for convenience. 

For problems with separated conditions at end-points $0$ and $\infty $
there is a continuous spectrum on $[1, \infty )$ with a discrete (and
simple) spectrum on $(- \infty , 1)$.  This discrete spectrum has
cluster points at $- \infty $ and $1$.

With the $u, \; v$ boundary condition function as given, in particular
$u(x) = x^{1/2} \; cos (k \ log(x))$, the problem with boundary
condition $[y, u](0) = 0$ has eigenvalues given explicitly by:
\begin{itemize}
\item[(i)]
suppose $\Gamma (1+i) = \alpha + i \beta $ and $ \mu > 0$ satisfies $
tan(log (\dfrac{1}{2} \mu )) = - \alpha / \beta $
\item[(ii)]
$\theta = im (log ( \Gamma (1+i))) $
\item[(iii)]
$log (\dfrac{1}{2} \mu ) = \dfrac{\pi}{2} + \theta + s \pi \; \; s = 0,
\pm 1, \pm 2, ... $
\item[(iv)]
$- \mu_s^2 = - (2exp (\theta + \dfrac{1}{2} \pi ))^2 \; exp (2s \pi ) \;
s=0, \pm 1 , \pm 2 ... $
\end{itemize}
then the eigenvalues are $ \lambda_n = - \mu_{-(n+1)}^2 + 1 \ (n\in
\Z )$.

SLEIGN2 can compute only six of these eigenvalues on our Sun workstation, 
$\lambda_{-3}$ to
$\lambda_2 $; other eigenvalues are, numerically, too close to 1 or too close to 
$-\infty $.  We have $$ \lambda_2 \; \zettleq \; 0.999997  \quad 
\lambda_{-3} \; \zettleq \; -14,519,130. $$  See \cite{K}.

\item % 21

{\bf The Fourier equation}

This is a simple constant coefficient equation whose eigenvalues, for any 
self-adjoint boundary condition, can be characterized in terms of a transcendental 
equation involving only trigonometric functions.

\item % 22 

{\bf The Laguerre equation}

This is the classical form of the differential equation which for
parameter $ \alpha > 1 $ produces the Laguerre polynomials as
eigenfunctions; for the appropriate boundary condition at 0, when
required, the eigenvalues are then (remarkably!) independent of $ \alpha $ and given
by $ \lambda_n = n \; (n=0, 1,2,...)$; see \cite{AS}; Chapter 22, section
22.6.

SLEIGN2 fails to compute eigenvalues with this differential equation on $(0, \infty
)$ on our Sun workstation; this appears to be due to numerical problems resulting 
from the exponentially
small coefficients; however, see example 23 below.
\item %23
{\bf The Laguerre/Liouville equation}

This is the Liouville normal form of the Laguerre equation. It seems to be in a
form more suitable for eigenvalue computations in contrast to
the previous example.

The Laguerre polynomials are produced when $ \alpha > -1$. 
For $\alpha \geq 1$ the LP condition holds at 0.  For
$0 \leq \alpha < 1$ the appropriate boundary condition is $[y,u] (0) =
0; $ for $ -1 < \alpha < 0 $ use $[y, \ v] (0) = 0$.  In all these cases
$ \lambda_n = n \; (n=0, 1, 2, ...)$.
\item %24
{\bf The Jacobi/Liouville equation}

This is the Liouville normal form of the Jacobi equation of example 16.
\item%25
{\bf The Meissner equation}

This equation arose in a model of a one dimensional crystal.  For
this constant coefficient equation with a weight function which has a
jump discontinuity the eigenvalues can be characterized as roots of a
transcendental equation involving only trig. and inverse trig. functions. 
There are infinitely many simple eigenvalues and infinitely many double ones 
for the periodic case; they are given by:

{\bf Periodic boundary conditions on $(-0.5, 0.5)$.} 

We have $\lambda_0
 = 0$ and 
$$
 \lambda_{4n+1} = ( 2m \pi + \alpha)^2 ; \ 
 \ \lambda_{4n+2} = ( 2(n+1) \pi - \alpha))^2 ; 
$$
$$
 \ \lambda_{4n+3} \ = \  
 \ \lambda_{4n+4} = ( 2(n+1) \pi ))^2 ; \ n=0,1,2, \ldots
$$
where
$
\alpha \, = \, cos^{-1}(-7/8)
$
{\bf Semi-Periodic eigenvalues}

With $ \beta = cos^{-1}( (1 + \sqrt(33))/16)$ and 
 $ \gamma  = cos^{-1}( (1 - \sqrt(33))/16)$  these are all simple and given by :
$$
\lambda_{4n} \, = \, ( 2n \pi \, + \, \beta)^2 ; \  
\lambda_{4n+1} \, = \, ( 2n \pi \, + \, \gamma )^2 ; \  
\lambda_{4n+2} \, = \, ( 2(n+1) \pi \, - \, \gamma )^2 ; \  
$$
$$
\lambda_{4n+3} \, = \, ( 2(n+1) \pi \, - \, \beta)^2 ; \  n =0, 1, 2, \ldots
$$
See \cite{E} and \cite{Hoc}.
\end{enumerate}


\begin{thebibliography}{999}

\bibitem{AS} M.Abramovitz and I.Stegun, {\em Handbook of Mathematical Functions 
with formulas and graphs and mathematical tables}, Dover Publications Inc., 
New York, 1965.

\bibitem{AEZ} F.V.Atkinson, W.N.Everitt and A.Zettl, {\em Regularization of 
Sturm-Liouville problem with an interior singularity using quasi-derivatives}
 Diff. and Int. Equations 1 (1988), 213-222.

\bibitem{BEZ} P.B.Bailey, W.N.Everitt and A.Zettl, {\em Computing eigenvalues 
of singular Sturm-Liouville problems}, Results in Mathematics, v.20(1991), 
391-423.

\bibitem{B} J.P.Boyd, {\em Sturm-Liouville eigenvalue problems with 
an interior pole}, J.Math. Physics, 22(1981), 1575-1590.

\bibitem{DS} N.Dunford and J.T.Schwartz, {\em Linear Operators, part II}, 
Interscience Publishers, New York, 1963.

\bibitem{E} M.S.P. Eastham, {\em The spectral theory of periodic differential 
equations}, Scottish Academic Press, Edinburgh and London, 1973.

\bibitem{EGZ} W.N.Everitt, J.Gunson and A.Zettl, {\em Some comments on  
Sturm-Liouville eigenvalue problems with interior singularities}, J.
Appl. Math. Phys. (ZAMP) 38 (1987), 813-838.

\bibitem{F} G.Fichera, {\em Numerical and quantitative analysis}, 
Pitman Press, London, 1978.

\bibitem{H} M.S.Homer, {\em Boundary value problems for the La Place tidal wave  
equation}, Proc. Roy. Soc. of London (A) 428 (1990), 157-180.

\bibitem{Hoc} H.Hochstadt, {\em A special Hill's equation with discontinuous 
coefficients}, Amer. Math. Monthly, 70(1963), 18-26.

\bibitem{K} A.M.Krall, {\em Boundary value problems for an eigenvalue problem 
with a singular potential}, J. Diff. Equations, 45 (1982), 128-138.

\bibitem{M} M.Marletta, {\em Numerical tests of the SLEIGN software for 
Sturm-Liouville problems}, ACM TOMS, v17(1991), 501-503.

\bibitem{NZ} H.-D.Niessen and A.Zettl, {\em Singular Sturm-Liouville problems;
the Friedrichs extension and comparison of eigenvalues}, Proc. London
Math. Soc. 64 (1992), 545-578.

\bibitem{P} M.Plum, {\em Eigenvalue inclusions for second-order ordinary 
differential operators by a numerical homotopy method}, ZAMP, v41(1990),
205-226.

\bibitem{ST} D.B.Sears and E.C.Titchmarsh, {\em Some eigenfunction formulae}, 
Quart. J. Math. Oxford (2) 1 (1950), 165-175.

\bibitem{T} E.C. Tichmarsh, {\em Eigenfunction expansions associated with second 
order differential equations}, v. I, Clarendon Press, Oxford; 1962.

\bibitem{W} G.N.Watson, {\em A treatise on the theory of Bessel functions},
Cambridge University Press, Cambridge, England, 1958.


\end{thebibliography}

\end{document}

