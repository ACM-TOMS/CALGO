The directory {\ttfamily examples/} includes five examples of utilizing the R\+I\+D\+C library, and one example, {\ttfamily examples/brusselator\+\_\+radau\+\_\+mkl} that implements a three stage, fifth-\/order Radau method to provide a basis of comparison with the R\+I\+D\+C integrators. Depending on the options selected in the {\ttfamily ./configure} step, some or all of these examples are built and run during during the {\ttfamily make check} process. Alternatively, a user can compile and run an example seperately after the {\ttfamily ./configure} step. For example, the subdirectory {\ttfamily examples/explicit/} contains the code to solve a system of O\+D\+E\+S using R\+I\+D\+C with an explicit Euler step function. To compile this specific example, move into the {\ttfamily examples/explicit} subdirectory and type {\ttfamily make explicit}. The executable {\ttfamily explicit} takes as input the order required and the number of time steps. For example {\ttfamily ./explicit.exe 4 100} solves the system of O\+D\+Es using fourth order R\+I\+D\+C with 100 time steps. A shell script {\ttfamily run.\+sh} is provided to run the R\+I\+D\+C integrator with different numbers of time steps for a convergence study. A simple matlab or octave script {\ttfamily convergence.\+m} is included in that subdirectory to test the order of convergence. {\ttfamily octave convergence.\+m} gives the slope and intercept for the linear fit of log of the error versus log of the time step. In this example we obtain a slope of -\/4.\+0630 indicating the we indeed have an order 4 method. 