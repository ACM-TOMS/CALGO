\documentclass[a4paper,12pt]{article}
\usepackage{amsfonts,amssymb}
\usepackage{graphicx}

\evensidemargin 0cm \oddsidemargin 0cm \topmargin -1cm
\textheight 23 true cm
\textwidth 16 true cm

%opening

\title{Documentation and implementation}
\author{}

\begin{document}



\maketitle


The main files of the package are listed below
\begin{itemize}
\item {\tt z=evalArgy(x,y,p)}: {\tt x}, {\tt y} are scalars (or row vectors of
the same
length) with the coordinates of a point (or a set of points) where we
will evaluate the basis functions; {\tt p} is a $2\times 3$ matrix
that determines the triangle. Its $j$th column are the coordinates
of the $j$th vertex. The only output argument ({\tt z}) is  a
$21\times k$ matrix $k$ being the length of ${\tt x}$ or ${\tt y}$,
where the $i$th row is  the values of the $i$th element of the basis
at the points given by {\tt x}, {\tt y}.


\item {\tt [dx,dy]=evalGradArgy(x,y,p)} returns in {\tt dx},{\tt dy}
the values of the  $x-$ and $y-$ derivatives of the  elements of the
local basis of the Argyris element at $({\tt x},{\tt y})$. All the
input and output arguments follow the convention above.



\item {\tt [dxx,dxy,dyy]=evalHessArgy(x,y,p)} evaluates
the second  derivatives of the elements of the Argyris basis at
$({\tt x},{\tt y})$.
\end{itemize}

We have some auxiliary routines used in the computations:
\begin{itemize}

\item {\tt [C,B,b,Th]=changeOfBasis(p)} computes $C$, $B$, ${\bf b}$ and
$\Theta$, the matrices (and vector) involved in the changes of
coordinates between the reference triangle and the triangle given by
{\tt p}.

\item {\tt [zx,zy]=khat2k(x,y,B,b)} returns in ({\tt zx},{\tt zy}) the image
of points by the linear map $(x,y)^\top \mapsto B (x,y)^\top + b$.
In our computations, $B$ and ${\bf b}$ are the coefficients of the
transformation mapping $\widehat K$ to $K$. {\tt
[zx,zy]=k2khat(x,y,B,b)} computes images with the inverse mapping.
As before, ${\tt x}$ and ${\tt y}$ can be row vectors.
\end{itemize}

There is a data file
\begin{itemize}
\item {\tt coefRef.dat}: in this file it is saved a $21\times21$
matrix. The $i$th row contains the coefficients in the monomial
basis of the $i$th element of the Argyris basis in the reference
triangle.
\end{itemize}

This matrix is used by the main routines {\tt evalArgy,
evalGradArgy, evalHessArgy}. These functions proceed to check at the
beginning of the code the existence of the variable {\tt coefRef}.
In case that it has not been declared, the file {\tt coefRef.dat} is
loaded and stored in a global variable of the same name. Then {\tt
coefRef} is now defined and becomes accessible for the all the
routines from now on. Therefore, the file {\tt coefRef.dat} is loaded
only once per session.

Finally, we have the routine
\begin{itemize}
\item {\tt coefRef=reference:}  returns the  array {\tt coefRef} with
the coefficients of the Argyris basis on the reference triangle.
This function can be used also to compute the Argyris basis on a
user-chosen triangle. It requires the symbolic toolbox.
\end{itemize}
In practise, there is no need to execute this file since the matrix
{\tt coefRef} is available as a data file. We have included it here
for the sake of completeness.

 In the remainder of the section we give some
details about the implementation of this package, starting with the
function {\tt changeOfBasis}. The computation of $B$, ${\bf b}$ and
$\Theta$ is straightforward and is made in an internal function
called ${\tt afftrans}$. The bulk of the code  is devoted to the
computation of $C$. We distinguish three parts. The first part
defines some geometric quantities
\begin{verbatim}
      v=[p(:,2)- p(:,1), p(:,3)-p(:,1), p(:,3)-p(:,2)];
      [B,b,Th]=afftrans(p);
      sides=diag([norm(v(:,1)),norm(v(:,2)),norm(v(:,3))]);
      aux=sides^(-2)*[0  1; -1   0; -1/sqrt(2) -1/sqrt(2)]*B';  % see (3)
      R=[0 -1; 1 0];
\end{verbatim}
Next we construct the matrix $D$:
\begin{verbatim}
      f=dot(aux',v);
      g=dot(aux',R*v);
      D=blkdiag(eye(3),B',B',B',Th,Th,Th,[diag(g) diag(f)]);
\end{verbatim}
Note the use of the Matlab command {\tt dot} to compute $f_\alpha$
and $g_\alpha$ by doing the dot product between the columns of
appropriate matrices. The command {\tt blkdiag}
is finally employed to assembly the matrix $D$.

The construction of the  matrix $E$ is done similarly
\begin{verbatim}
      E=zeros(24,21);
      E(1:21,:)=blkdiag(eye(18),sides);
      E(22:24,1:3)= 15/8*[-1  1 0;  -1  0 1; 0 -1 1];
      E(22:24,4:9)=-7/16*[v(:,1)' v(:,1)'  0  0;
                          v(:,2)' 0  0     v(:,2)';...
                          0  0    v(:,3)'  v(:,3)'];
      w=[v(1,:).^2; 2*v(1,:).*v(2,:); v(2,:).^2]';
      E(22:24,10:18)=1/32*[-w(1,:)    w(1,:)    0  0  0;.
                           -w(2,:)   0  0  0     w(2,:);...
                           0  0  0   -w(3,:)     w(3,:)];
\end{verbatim}
The program finishes by  computing $C$
\begin{verbatim}
      C=D*E;
\end{verbatim}

Function {\tt evalArgy} performs the evaluation of the Argyris
basis in the triangle specified by {\tt p}. This is done in the
following lines
\begin{verbatim}
      [C,B,b]=changeOfBasis(p);
      [x,y]=k2khat(x,y,B,b);
      z=monomials(x,y);
      z=C'*coefRef*z;
\end{verbatim}
The point $(x,y)$ is mapped first into the reference triangle and
next the elements of the  monomial basis are evaluated at this point
(note that the $i$th row of {\tt coefRef*z} corresponds to $\widehat
N_i(x,y)$). The change of basis, and therefore the evaluation of the
local basis $N_i(x,y)$ in the user-specified triangle, is carried
out by the left multiplication by {\tt C'}.

This code can be vectorized just by allowing both {\tt x,y} to be
row vectors of the same length. If  $k$ is the length of {\tt x},
{\tt y}, {\tt z}, becomes a $21 \times k$ matrix in all the
occurrences.

To compute the first derivatives we use the chain rule (recall that
the gradient is seen columnwise)
\[
(\nabla N_j({\bf x}))^\top= \sum_{i=1}^n c_{ij} (\nabla
\widehat{N}_i)^\top \circ F^{-1}({\bf x}) B^{-1}
\]
The following lines, which belong to the function {\tt
evalGradArgy}, evaluate the gradient
\begin{verbatim}
      [C,B,b]=changeOfBasis(p);
      [x,y]=k2khat(x,y,B,b);
      k=length(x);
      mx=derx(x,y);
      my=dery(x,y);
      grads=zeros(21,2*k);
      grads(:)=[mx(:) my(:)]/B;
      grads=C'*coefRef*grads;
      dx=grads(:,1:k);
      dy=grads(:,k+1:2*k);
\end{verbatim}
Functions {\tt derx}, {\tt dery} return a column vector with the
derivatives of the monomial basis evaluated at $({\tt x},{\tt y})$.
The columnwise access to the elements of a matrix in Matlab is used
here to set {\tt grads} in such a way that after running  the six
first lines, {\tt grads} has in the first  $k$ columns
$\partial_x(m_i\circ F^{-1})(x_j,y_j)$ (here $m_{i}$ denotes the
$i$th element of the basis of monomials and $F$ the affine mapping
from $\widehat{K}$ onto $K$) whereas $\partial_y(m_i\circ
F^{-1})(x_j,y_j)$ are stored in the last $k$ columns.  Finally,
left multiplication by {\tt coefRef} and {\tt C'} makes the change of
basis.


The evaluation of the second derivatives, which is done in {\tt
evalHessArgy}, is implemented in the same manner:
\begin{verbatim}
      [C,B,b,Th]=changeOfBasis(p);
      [x,y]=k2khat(x,y,B,b);
      k=length(x);
      mxx=derxx(x,y);
      mxy=derxy(x,y);
      myy=deryy(x,y);
      hessian=zeros(21,3*k);
      hessian(:)=[mxx(:) mxy(:) myy(:)]/Th';
      hessian=C'*coefRef*hessian;
      dxx=hessian(:,1:k);
      dxy=hessian(:,k+1:2*k);
      dyy=hessian(:,2*k+1:3*k);
\end{verbatim}




\end{document}
