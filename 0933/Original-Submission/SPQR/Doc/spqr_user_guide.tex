\documentclass[12pt]{article}
\usepackage{hyperref}

\topmargin -0.5in
\textheight 9.0in
\oddsidemargin 0pt
\evensidemargin 0pt
\textwidth 6.5in

%-------------------------------------------------------------------------------
% get epsf.tex file, for encapsulated postscript files:
\input epsf
%-------------------------------------------------------------------------------
% macro for Postscript figures the easy way
% usage:  \postscript{file.ps}{scale}
% where scale is 1.0 for 100%, 0.5 for 50% reduction, etc.
%
\newcommand{\postscript}[2]
{\setlength{\epsfxsize}{#2\hsize}
\centerline{\epsfbox{#1}}}
%-------------------------------------------------------------------------------

\title{User's Guide for SuiteSparseQR, a multifrontal multithreaded sparse
QR factorization package}
\author{Timothy A. Davis\thanks{
email: DrTimothyAldenDavis@gmail.com.
http://www.suitesparse.com.
Portions of this work were supported by the National
Science Foundation, under grants 0203270, 0620286, and 0619080.}}

\date{May 28, 2012}

%-------------------------------------------------------------------------------
\begin{document}
%-------------------------------------------------------------------------------
\maketitle

\begin{abstract}

SuiteSparseQR is an implementation of the multifrontal sparse QR factorization
method.  Parallelism is exploited both in the BLAS and across different frontal
matrices using Intel's Threading Building Blocks, a shared-memory programming
model for modern multicore architectures.  It can obtain a substantial fraction
of the theoretical peak performance of a multicore computer.  The package is
written in C++ with user interfaces for MATLAB, C, and C++.  Both real and
complex sparse matrices are supported.

\end{abstract}

\maketitle

%-------------------------------------------------------------------------------
\section{Introduction}
\label{intro}
%-------------------------------------------------------------------------------

The algorithms used in SuiteSparseQR are discussed in a companion paper,
\cite{Davis08a}, and an overview of how to use the software is given in
\cite{Davis08b}.  This document gives detailed information on the installation
and use of SuiteSparseQR.

%-------------------------------------------------------------------------------
\section{Using SuiteSparseQR in MATLAB}
%-------------------------------------------------------------------------------

The simplest way to use SuiteSparseQR is via MATLAB.  Its syntax includes every
feature of the MATLAB \verb'qr' in version R2009a and earlier
\cite{GilbertMolerSchreiber}, plus additional features not available in MATLAB.
It is also a replacement for \verb'x=A\b' for least-squares problems and
underdetermined systems.  In addition to substantial gains in performance (10x
to 100x is not uncommon, up to 10,000x has been observed), SuiteSparseQR adds
new capabilities that are not present in MATLAB.  For example, it provides an
efficient method for finding the minimum 2-norm solution to an underdetermined
system.

%-------------------------------------------------------------------------------
\subsection{Installing SuiteSparseQR for use in MATLAB}
%-------------------------------------------------------------------------------

All packages in SuiteSparse, including SuiteSparseQR and the codes it relies on
(AMD, COLAMD, CHOLMOD, METIS, CCAMD, and CCOLAMD) are compiled with a single
command typed into the MATLAB Command Window.  SuiteSparseQR uses the LAPACK
and BLAS libraries provided with MATLAB; you do not need to do anything to use
these.  Below are step-by-step instructions for compiling all of SuiteSparse
(including SuiteSparseQR), and optional instructions on using METIS and/or
Intel's Threading Building Blocks (TBB).

%-------------------------------------------------------------------------------
\subsubsection{Required instructions for Windows}
%-------------------------------------------------------------------------------

For Windows, you cannot use the \verb'lcc' compiler that ships with MATLAB; it
is not a C++ compiler.  To compile SuiteSparseQR, you must obtain a C++
compiler; Microsoft Visual Studio C++ Express Edition will work fine.  Install
this compiler from \newline
\htmladdnormallink{http://www.microsoft.com/express/vc/}{http://www.microsoft.com/express/vc/}
and then type \verb'mex -setup' in the MATLAB Command Window.

%-------------------------------------------------------------------------------
\subsubsection{Optional instructions on using METIS for any operating system}
%-------------------------------------------------------------------------------

To use METIS \cite{KarypisKumar98e}, obtain a copy of the metis-4.0.1
distribution from \newline
\htmladdnormallink{http://glaros.dtc.umn.edu/gkhome/views/metis}{http://glaros.dtc.umn.edu/gkhome/views/metis}.
Do not get the \verb'metis-4.0.zip' file since it does not contain source code;
you must download the \verb'metis-4.0.tar.gz' file instead.  Uncompress this
file and place the \verb'metis-4.0' directory in the SuiteSparse directory.
You must not have a nested \verb'SuiteSparse/metis-4.0/metis-4.0' directory;
this can occur on Windows when you extract a compressed archive.  If you have
this directory, move the lowest \verb'metis-4.0' directory up one level so that
(for example) the file \newline \verb'SuiteSparse/metis-4.0/Lib/metis.h'
exists.

METIS tends to give orderings that are good for the parallelism exploited by
TBB, so its use with TBB is recommended.  Note however that METIS is not
bug-free; it can occasionally cause segmentation faults, particularly if used
when finding basic solutions to underdetermined systems with many more columns
than rows (SuiteSparseQR does not use METIS, by default, for those systems).
This (rare) faulty behavior has been confirmed with valid inputs to the METIS
test programs themselves; it is not a bug in the SuiteSparse interface to
METIS.  Use METIS at your own risk.

%-------------------------------------------------------------------------------
\subsubsection{Optional instructions for using TBB on Linux/Unix/Mac}
%-------------------------------------------------------------------------------

If you are using a Debian-based Linux system (such as Ubuntu), you're in luck!
You can install TBB via the Synaptic Package Manager.  Just search for TBB,
select it, and click Apply.  This will place the right files in \verb'/usr/lib'
and it will create the \verb'/usr/include/tbb' directory.  It's by far the
simplest way to install TBB.  If you do this, skip the rest of this section.

Alternatively, obtain a copy of TBB from
\htmladdnormallink{http://www.threadingbuildingblocks.org}{http://www.threadingbuildingblocks.org}
as a \verb'tbb*.tgz' file appropriate for your version of Linux/Unix.  If you
install via the \verb'tbb_*.tgz' file, make sure the \verb'libtbb*.so*' and
\verb'libtbbmalloc*.so*' files are placed in the \verb'/usr/lib' directory.
Make sure the \verb'tbb' directory containing all of the include files is
placed in \verb'/usr/include' (for example, the
\verb'/usr/include/tbb/task_scheduler_init.h' file must exist).

If you do not have permission to install TBB properly into the \verb'/usr/lib'
and \verb'/usr/include' directories, you can place it in your own directory and
modify the \verb'tbb_path' variable in \verb'spqr_make.m' to specify
the location of your own copy of TBB (refer to that file for instructions).
You must also set your \verb'LD_LIBRARY_PATH' environment variable to include
the directory containing the \verb'libtbb*so' files.  You need to first
determine which subdirectory of the TBB distribution contains the libraries
appropriate for your system; in the examples below this is just called
\verb'/path'.  It must be an absolute path, starting with the \verb'/'
character.  There must be no spaces in the path name.

For Linux use this command at the system command line before starting MATLAB,
where \verb'/path' should be replaced with the actual full path of the TBB
\verb'lib' directory:

\begin{verbatim}
    setenv LD_LIBRARY_PATH /path:$LD_LIBRARY_PATH
\end{verbatim}

If you use the \verb'csh' shell, place the command in your \verb'~/.cshrc'
file so you don't have to type it each time you start MATLAB.

For the Mac, edit the \verb'/etc/profile' file and add this line to the end of
the file:

\begin{verbatim}
    export DYLD_LIBRARY_PATH=/path:$DYLD_LIBRARY_PATH
\end{verbatim}

For example, using the 64-bit Linux version placed in my home directory, this
command would be placed in my \verb'~/.cshrc' file, prior to starting MATLAB.

{\scriptsize
\begin{verbatim}
    setenv LD_LIBRARY_PATH /home/davis/tbb21_009oss/em64t/cc4.1.0_libc2.4_kernel2.6.16.21/lib:$LD_LIBRARY_PATH
\end{verbatim}
}

Then, before starting MATLAB, make sure this variable is set by typing this
command at the system command line (you only have to do this for this session;
whenever you start a new command shell this will be done automatically): 

\begin{verbatim}
    source ~/.cshrc
\end{verbatim}

For this example, my \verb'spqr_make.m' file would contain this line:

\begin{verbatim}
    tbb_path = '/home/davis/tbb21_009oss' ;
\end{verbatim}

%-------------------------------------------------------------------------------
\subsubsection{Optional instructions for using TBB on Windows}
%-------------------------------------------------------------------------------

Obtain a copy of TBB from
\htmladdnormallink{http://www.threadingbuildingblocks.org}{http://www.threadingbuildingblocks.org}
; for TBB Version 2.1 this file is \verb'tbb21_009oss_win.zip'.

Create a folder and place the \verb'tbb21_009oss_win.zip' file there, and
extract the file.  In the example below, I extracted to
\verb'C:\TBB\tbb21_009oss' (if you do the same then you do not have to edit
\verb'spqr_make.m').  There should be no spaces in the path (for example,
placing TBB under the \verb'Program Files' directory will not work).

In Windows XP, right-click \verb'My Computer' and select \verb'Properties'.
Click the \verb'Advanced' tab.  Click \verb'Environment Variables'.  Under
\verb'System' variables, edit the \verb'Path' to append the name of the folder
containing the TBB \verb'bin' folder appropriate for your system, preceded by a
semicolon.  For example, in a 32-bit Windows system if TBB is installed in
\verb'C:\TBB' you would append the string
\verb';C:\TBB\tbb21_009oss\ia32\vc9\bin\' to the end of your system \verb'Path'
variable.

For Vista, the instructions are the same, except that you choose
\verb'Computer' instead of \verb'My Computer', and you click the
\verb'Advanced System Settings' tab instead of the \verb'Advanced' tab.

Next, edit the \verb'SPQR\MATLAB\spqr_make.m' file.  Change the \verb'tbb_path'
variable to point to your copy of TBB.  For example, if you installed TBB into
\verb'C:\TBB' your \verb'spqr_make.m' file would contain this line:

\begin{verbatim}
    tbb_path = 'C:\TBB\tbb21_009oss' ;
\end{verbatim}

That line is already in \verb'spqr_make.m', so if you choose to install TBB
in that location, you do not have to edit the file.

%-------------------------------------------------------------------------------
\subsubsection{Now you're ready to compile (on any operating system)}
%-------------------------------------------------------------------------------

Type these commands in the MATLAB window:
\begin{verbatim}
    cd SuiteSparse
    SuiteSparse_install
\end{verbatim}
You will be asked if you want to run some demos.  I recommend that you do this
to ensure your functions have been installed correctly.  Next type the command
\begin{verbatim}
    pathtool
\end{verbatim}
and examine your MATLAB path.  The various SuiteSparse directories have been
placed in your path.  Click ``save'' to save this path for future MATLAB
sessions.  If this fails, you do not have permission to modify the
\verb'pathdef.m' file (it is shared by all users).  An alternative is to type
the command:
\begin{verbatim}
    path
\end{verbatim}
and cut-and-paste the paths displayed there into your own \verb'startup.m'
file, prepending the command \verb'addpath' to each line.  For example, if I
installed SuiteSparse into my home directory (\verb'/home/davis') then my
\verb'startup.m' file should look like this:

{\footnotesize
\begin{verbatim}
    addpath  /home/davis/SuiteSparse/SPQR/MATLAB
    addpath  /home/davis/SuiteSparse/RBio
    addpath  /home/davis/SuiteSparse/MATLAB_Tools/spok
    addpath  /home/davis/SuiteSparse/MATLAB_Tools/waitmex
    addpath  /home/davis/SuiteSparse/MATLAB_Tools/shellgui
    addpath  /home/davis/SuiteSparse/MATLAB_Tools/GEE
    addpath  /home/davis/SuiteSparse/LINFACTOR
    addpath  /home/davis/SuiteSparse/MESHND
    addpath  /home/davis/SuiteSparse/UFcollection
    addpath  /home/davis/SuiteSparse/SSMULT
    addpath  /home/davis/SuiteSparse/KLU/MATLAB
    addpath  /home/davis/SuiteSparse/BTF/MATLAB
    addpath  /home/davis/SuiteSparse/LDL/MATLAB
    addpath  /home/davis/SuiteSparse/CXSparse/MATLAB/UFget
    addpath  /home/davis/SuiteSparse/CXSparse/MATLAB/Demo
    addpath  /home/davis/SuiteSparse/CXSparse/MATLAB/CSparse
    addpath  /home/davis/SuiteSparse/CAMD/MATLAB
    addpath  /home/davis/SuiteSparse/CCOLAMD/MATLAB
    addpath  /home/davis/SuiteSparse/COLAMD/MATLAB
    addpath  /home/davis/SuiteSparse/AMD/MATLAB
    addpath  /home/davis/SuiteSparse/CHOLMOD/MATLAB
    addpath  /home/davis/SuiteSparse/UMFPACK/MATLAB
    addpath  /home/davis/SuiteSparse
    addpath  /home/davis/SuiteSparse/MATLAB_Tools
\end{verbatim}
}

On a Windows system, I might see paths like this instead in my \verb'startup.m'
file:

{\footnotesize
\begin{verbatim}
    addpath C:\Documents and Settings\davis\My Documents\SuiteSparse\SPQR\MATLAB
    ...
\end{verbatim}
}

Your \verb'startup.m' file should appear in the directory in which MATLAB
starts.  Failing that, every time you start MATLAB, find your \verb'startup.m'
file and run it.  For more help, type \verb'doc startup' in MATLAB.

The \verb'SuiteSparse_install' script works on any version of MATLAB
(Linux/Unix, Mac, or Windows) if you have a C++ compiler.  The install script
will detect if you have placed the METIS directory in the right place, and will
compile it for use with SuiteSparseQR if it finds it there.  Otherwise METIS
will be skipped (the install script will tell you if it finds METIS or not).

%-------------------------------------------------------------------------------
\subsubsection{Optional instructions for using TBB on any system}
%-------------------------------------------------------------------------------

If you have followed the steps above (as I recommend that you do), you have
just compiled SuiteSparseQR, but it will not yet be using TBB.  To use TBB with
SuiteSparseQR, first install TBB as described above.  Once TBB is installed,
type the following commands in the MATLAB Command Window (assuming you have
METIS):

\begin{verbatim}
    cd SuiteSparse/SPQR/MATLAB
    spqr_make metis tbb
\end{verbatim}

Or if you do not have METIS, do this instead:

\begin{verbatim}
    cd SuiteSparse/SPQR/MATLAB
    spqr_make nometis tbb
\end{verbatim}

For more options, type \verb'help spqr_make'.

%-------------------------------------------------------------------------------
\subsection{Functions provided to the MATLAB user}
%-------------------------------------------------------------------------------

Three primary functions are available:

\begin{enumerate}

\item \verb'spqr', a replacement for the MATLAB \verb'qr'

\item \verb'spqr_solve', a replacement for \verb'x=A\b' when \verb'A' is sparse
and rectangular.  It works for the square case, too, but \verb'x=A\b' will be
faster (using LU or Cholesky factorization). \verb'spqr_solve' is a good method
for ill-conditioned or rank-deficient square matrices, however.

\item \verb'spqr_qmult', which multiplies \verb'Q' (stored in Householder
vector form) times a matrix \verb'x'.

\end{enumerate}

Their syntax is described below in the table below.
The permutation \verb'P' is chosen to reduce fill-in and to return \verb'R' in
upper trapezoidal form if \verb'A' is estimated to have less than full rank.
The \verb'opts' parameter provides non-default options (refer to the next
section).  The output \verb'Q' can be optionally returned in Householder form,
which is far sparser than returning \verb'Q' as a sparse matrix.

\vspace{0.1in}

{\footnotesize
\begin{tabular}{|ll|}
\hline
\verb'R = spqr (A)'             & Q-less QR factorization \\
\verb'R = spqr (A,0)'           & economy variant (\verb'size(R,1) = min(m,n)') \\
\verb'R = spqr (A,opts)'        & as above, with non-default options \\
\hline
\verb'[Q,R] = spqr (A)'         & \verb'A=Q*R' factorization \\
\verb'[Q,R] = spqr (A,0)'       & economy variant (\verb'size(Q,2) = size(R,1) = min(m,n)') \\
\verb'[Q,R] = spqr (A,opts)'    & \verb'A=Q*R', with non-default options \\
\hline
\verb'[Q,R,P] = spqr (A)'       & \verb'A*P=Q*R' where P reduces fill-in \\
\verb'[Q,R,P] = spqr (A,0)'     & economy variant (\verb'size(Q,2) = size(R,1) = min(m,n)') \\
\verb'[Q,R,P] = spqr (A,opts)'  & as above, with non-default options \\
\hline
\verb'[C,R] = spqr (A,B)'       & as \verb'R=spqr(A)', also returns \verb"C=Q'*B" \\
\verb'[C,R] = spqr (A,B,0)'     & economy variant (\verb'size(C,1) = size(R,1) = min(m,n)') \\
\verb'[C,R] = spqr (A,B,opts)'  & as above, with non-default options \\
\hline
\verb'[C,R,P] = spqr (A,B)'     & as \verb'R=spqr(A*P)', also returns \verb"C=Q'*B" \\
\verb'[C,R,P] = spqr (A,B,0)'   & economy variant (\verb'size(C,1) = size(R,1) = min(m,n)') \\
\verb'[C,R,P] = spqr (A,B,opts)'& as above, with non-default options \\
\hline
\verb'x = spqr_solve (A,B)'     & \verb'x=A\B' \\
\verb'[x,info] = spqr_solve (A,B,opts)' & as above, with statistics and non-default parameters \\
\hline
\verb'Y = spqr_qmult (Q,X,k)'   & computes \verb"Q'*X", \verb"Q*X", \verb"X*Q'", or \verb"X*Q" 
(selected with \verb'k') \\
\hline
\end{tabular}
}

%-------------------------------------------------------------------------------
\subsection{The {\tt opts} parameter}
%-------------------------------------------------------------------------------

The \verb'opts' struct provides control over non-default parameters for
SuiteSparseQR.  Entries not present in \verb'opts' are set to their defaults.

\begin{itemize}

    \item \verb'opts.tol':   columns that have 2-norm \verb'<= opts.tol' are
    treated as zero. The default is \verb"20*(m+n)*eps*sqrt(max(diag(A'*A)))"
    where \verb'[m n]=size(A)'.

    \item \verb'opts.econ':  number of rows of \verb'R' and columns of \verb'Q'
    to return.  The default is \verb'm'.  Using \verb'n' gives the standard
    economy form (as in the MATLAB \verb'qr(A,0)').  A value less than the
    estimated rank \verb'r' is set to \verb'r', so \verb'opts.econ=0' gives
    the ``rank-sized'' factorization, where \verb'size(R,1)==nnz(diag(R))==r'.

    \item \verb'opts.ordering': a string describing which column ordering
    method to use.  Let \newline 
    \verb'[m2 n2]=size(S)' where \verb'S' is obtained by
    removing singletons from \verb'A'.  The singleton permutation places
    \verb'A*P' in the form \verb'[A11 A12 ; 0 S]' where \verb'A11' is upper
    triangular with diagonal entries all greater than \verb'opts.tol'.

    The default is to use COLAMD if \verb'm2<=2*n2'; otherwise try AMD.  Let
    \verb'f' be the flops for \verb"chol((S*P)'*(S*P))" with the ordering
    \verb'P' found by AMD.  Then if \verb'f/nnz(R) >= 500' and
    \verb'nnz(R)/nnz(S) >= 5' then try METIS, and take the best ordering found
    (AMD or METIS); otherwise use AMD without trying METIS.  If METIS is not
    installed then the default ordering is to use COLAMD if \verb'm2<=2*n2' and
    to use AMD otherwise.

    The available orderings are:

       {\tt 'default'}: the default ordering.

       {\tt 'amd'}: use \verb"amd(S'*S)".

       {\tt 'colamd'}: use \verb"colamd(S)".

       {\tt 'metis'}: use \verb"metis(S'*S)", only if METIS is
       installed.

       {\tt 'best'}: try all three (AMD, COLAMD, METIS) and take the
       best.

       {\tt 'bestamd'}: try AMD and COLAMD and take the best.

       {\tt 'fixed'}: use \verb"P=I"; this is the only option if
       \verb"P" is not present in the output.

       {\tt 'natural'}: singleton removal only.

    \item \verb'opts.Q': a string describing how \verb'Q' is to be returned.
    The default is \verb"'discard'" if \verb'Q' is not present in the output,
    or \verb"'matrix'" otherwise.  If \verb'Q' is present and \verb'opts.Q' is
    \verb"'discard'", then \verb"Q=[]" is returned (thus \verb"R=spqr(A*P)" is
    \verb"[Q,R,P]=spqr(A)" where \verb"spqr" finds \verb"P" but \verb"Q" is
    discarded instead). The usage \verb"opts.Q='matrix'" returns \verb'Q' as a
    sparse matrix where \verb'A=Q*R' or \verb"A*P=Q*R".  Using
    \verb"opts.Q='Householder'" returns \verb'Q' as a struct containing the
    Householder reflections applied to \verb'A' to obtain \verb'R', resulting
    in a far sparser \verb'Q' than the \verb"'matrix'" option.  

    \item \verb'opts.permutation': a string describing how \verb'P' is to be
    returned.  The default is \verb"'matrix'", so that \verb"A*P=Q*R".  Using
    \verb"'vector'" gives \verb"A(:,P)=Q*R" instead.

    \item \verb'opts.spumoni': an integer \verb'k' that
        acts just like \verb"spparms('spumoni',k)".

    \item \verb'opts.min2norm': used by \verb'spqr_solve'; you can use
    \verb"'basic'" (the default), or \verb"'min2norm'".  Determines the kind of
    solution that \verb'spqr_solve' computes for underdetermined systems.  Has
    no effect for least-squares problems; ignored by \verb'spqr' itself.

    \item \verb'opts.grain', \verb'opts.small', \verb'opts.nthreads':
    multitasking control (if compiled with TBB).  Let \verb'f' be the total
    flop count.  The analysis phase tries to ensure that all parallel tasks
    have at least \verb'max(total_flops/opts.grain,opts.small)' flops.  No TBB
    parallelism is exploited if \verb'opts.grain <= 1'.  The parameter
    \verb'opts.nthreads' gives the number of threads to use for TBB (which is
    different than the number of threads used by the BLAS).  Setting
    \verb'opts.nthreads <= 0' means to let TBB determine the number of threads
    (normally equal to the number of cores); otherwise, exactly
    \verb'opts.nthreads' threads are used.  The defaults are
    \verb'opts.grain=1', \verb'opts.small=1e6', and \verb'opts.nthreads=0',
    respectively.  That is, TBB is disabled by default since it conflicts with
    BLAS multithreading.  If you enable TBB, be sure to disable BLAS
    multithreading with the MATLAB command \verb'maxNumCompThreads(1)', or
    choose \newline \verb'opts.nthreads=k' and \verb'maxNumCompThreads(b)' so
    that the product \verb'k*b' is equal to the number of cores.  Note that
    these recommendations may change for future versions of TBB.  A good value
    of \verb'opts.grain' is twice that of \verb'opts.nthreads'.  If TBB
    parallelism is enabled, the METIS ordering normally gives the best speedup
    for large problems.

\end{itemize}

%-------------------------------------------------------------------------------
\subsection{Examples on how to use the MATLAB interface}
%-------------------------------------------------------------------------------

To solve a least-squares problem, or to find the basic solution to an
underdetermined system, just use \verb'x = spqr_solve(A,b)' in place of
\verb'x=A\b'.  To compute the QR factorization, use \verb'[Q,R]=spqr(A)'
instead of \verb'[Q,R]=qr(A)'.  Better results can be obtained by discarding
\verb'Q' with the usage \verb'R=spqr(A)' (in place of \verb'R=qr(A)'), or by
requesting \verb'Q' in Householder form with \verb'[Q,R]=spqr(A,opts)' where
\verb"opts.Q='Householder'".  The latter option is not available in MATLAB.  To
use a fill-reducing ordering, simply use any of the syntaxes above with
\verb'P' as an output parameter.

The least-squares solution of an overdetermined system \verb'A*x=b' with
\verb'm>n' (where \verb'A' has rank \verb'n') can be found in one of at least
seven ways (in increasing order of efficiency, in time and memory):

{\footnotesize
\begin{tabular}{|l|l|}
    \hline
    \verb"x = pinv(full(A)) * b ;"
        & impossible for large \verb'A' \\
    \hline
    \verb"[Q,R] = spqr (A) ;"
        & high fill-in in \verb'R', \\
    \verb"x = R\(Q'*b) ;"
        & \verb'Q' costly in matrix form \\
    \hline
    \verb"[Q,R,P] = spqr (A) ;"
        & low fill-in in \verb'R', \\
    \verb"x = P*(R\(Q'*b)) ;"
        & \verb'Q' costly in matrix form \\
    \hline
    \verb"[Q,R,P] = spqr (A,struct('Q','Householder')) ;"
        & low fill-in in \verb'R', \\
    \verb"x = P*(R\spqr_qmult (Q,b,0)) ;"
        & \verb'Q' in efficient Householder form \\
    \hline
    \verb"[c,R,P] = spqr (A,b) ;"
        & \verb'Q' not kept, \\
    \verb"x = P*(R\c) ;"
        & \verb'P' a permutation matrix \\
    \hline
    \verb"[c,R,p] = spqr (A,b,0) ;"
        & \verb'Q' not kept, \\
    \verb"y = (R\c) ; x(p) = y"
        & \verb'p' a permutation vector \\
    \hline
    \verb"x = spqr_solve (A,b) ;"
        & less memory and better handling \\
        & of rank-deficient matrices \\
    \hline
\end{tabular}
}

\vspace{0.1in}

The minimum-norm solution of an underdetermined system \verb'A*x=b' with
\verb'm<n' can be found in one of five ways (in increasing order of
efficiency, in time and memory):

\vspace{0.1in}

{\footnotesize
\begin{tabular}{|l|l|}
    \hline
    \verb"x = pinv(full(A)) * b ;"
        & impossible for large \verb'A' \\
    \hline
    \verb"[Q,R] = spqr (A') ;"
        & high fill-in in \verb'R', \\
    \verb"x = Q*(R'\b) ;"
        & \verb'Q' costly in matrix form \\
    \hline
    \verb"[Q,R,P] = spqr (A') ;"
        & low fill-in in \verb'R', \\
    \verb"x = Q*(R'\(P'*b)) ;"
        & \verb'Q' costly in matrix form \\
    \hline
    \verb"[Q,R,P] = spqr (A',struct('Q','Householder')) ;"
        & low fill-in in \verb'R', \\
    \verb"x = spqr_qmult (Q,R'\(P'*b),1) ;"
        & \verb'Q' in efficient Householder form \\
    \hline
    \verb"opts.solution = 'min2norm' ;"
        & as 4th option above, but faster, \\
    \verb"x = spqr_solve (A,b,opts) ;"
        & less memory, and better handling \\
        & of rank-deficient matrices \\
    \hline
\end{tabular}
}

Note that \verb'spqr_solve' uses a fill-reducing ordering, by default.
It can be disabled or modified using a non-default \verb'opts' parameter
(\verb'opts.ordering', specifically).

%-------------------------------------------------------------------------------
\section{Using SuiteSparseQR in C and C++}
%-------------------------------------------------------------------------------

SuiteSparseQR relies on CHOLMOD for its basic sparse matrix data structure, a
compressed sparse column format.  CHOLMOD provides interfaces to the AMD,
COLAMD, and METIS ordering methods, supernodal symbolic Cholesky factorization
(namely, \verb'symbfact' in MATLAB), functions for converting between different
data structures, and for basic operations such as transpose, matrix multiply,
reading a matrix from a file, writing a matrix to a file, and many other
functions.

%-------------------------------------------------------------------------------
\subsection{Installing the C/C++ library on Linux/Unix}
%-------------------------------------------------------------------------------

Before you compile the SuiteSparseQR library and demo programs, you may wish to
edit the \verb'SuiteSparse/SuiteSparse_config/SuiteSparse_config.mk' configuration file.  The
defaults should be fine on most Linux/Unix systems.  Next, type \verb'make' at
the Linux/Unix command line, in either the \verb'SuiteSparse' directory (which
compiles all of SuiteSparse) or in the \verb'SuiteSparse/SPQR' directory (which
just compiles SuiteSparseQR and the libraries it requires).  SuiteSparseQR will
be compiled, and a set of simple demos will be run (including the one in the
next section).

The configuration file defines where the LAPACK and BLAS libraries are to be
found.  Selecting the right BLAS is critical.  There is no standard naming
scheme for the name and location of these libraries.  The defaults in the
\verb'SuiteSparse_config.mk' file use \verb'-llapack' and \verb'-lblas'; the latter may
link against the standard Fortran reference BLAS, which will not provide
optimal performance.  For best results, you should use the Goto BLAS
\cite{GotoVanDeGeijn08}, or high-performance vendor-supplied BLAS such as the
Intel MKL, AMD ACML, or the Sun Performance Library.  Selection of LAPACK and
the BLAS is done with the \verb'LAPACK=' and \verb'BLAS=' lines in the
\verb'SuiteSparse_config.mk' file.

Four compile-time options can be used to modify how SuiteSparseQR is compiled.
Select these via the \verb'SPQR_CONFIG=' line in the \verb'SuiteSparse_config.mk' file.

\begin{itemize}

    \item \verb'-DNPARTITION': do not compile with METIS, CAMD, or CCOLAMD.
    These packages are included by default.

    \item \verb'-DNEXPERT': do not compile with the ``expert'' routines in
    \verb'SuiteSparseQR_expert.cpp'.  The expert routines are included by
    default.

    \item \verb'-DHAVE_TBB': enable the Intel Threading Building Blocks, TBB.
    The use of TBB is disabled by default.  It is disabled because not all
    installations have TBB available.  The use of TBB is recommended, however,
    if you have a multicore computer.

\end{itemize}

To fully test 100\% of the lines of SuiteSparseQR, go to the \verb'Tcov'
directory and type \verb'make'.  This will work for Linux only.

%-------------------------------------------------------------------------------
\subsection{C/C++ Example}
%-------------------------------------------------------------------------------

The C++ interface is written using templates for handling both real and complex
matrices.  The simplest function computes the MATLAB equivalent of
\verb'x=A\b' and is almost as simple:
{\footnotesize
\begin{verbatim}
    #include "SuiteSparseQR.hpp"
    X = SuiteSparseQR <double> (A, B, cc) ;
\end{verbatim}
}
The C version of this function is almost identical:
{\footnotesize
\begin{verbatim}
    #include "SuiteSparseQR_C.h"
    X = SuiteSparseQR_C_backslash_default (A, B, cc) ;
\end{verbatim}
}

Below is a simple C++ program that illustrates the use of SuiteSparseQR.  The
program reads in a least-squares problem from \verb'stdin' in MatrixMarket
format \cite{BoisvertPozoRemingtonBarrettDongarra97}, solves it, and prints the
norm of the residual and the estimated rank of \verb'A'.  The comments reflect
the MATLAB equivalent statements.  The C version of this program is identical
except for the \verb'#include' statement and call to SuiteSparseQR which are
replaced with the C version of the statement above, and C-style comments.

{\footnotesize
\begin{verbatim}
    #include "SuiteSparseQR.hpp"
    int main (int argc, char **argv)
    {
        cholmod_common Common, *cc ;
        cholmod_sparse *A ;
        cholmod_dense *X, *B, *Residual ;
        double rnorm, one [2] = {1,0}, minusone [2] = {-1,0} ;
        int mtype ;

        // start CHOLMOD
        cc = &Common ;
        cholmod_l_start (cc) ;

        // load A
        A = (cholmod_sparse *) cholmod_l_read_matrix (stdin, 1, &mtype, cc) ;

        // B = ones (size (A,1),1)
        B = cholmod_l_ones (A->nrow, 1, A->xtype, cc) ;

        // X = A\B
        X = SuiteSparseQR <double> (A, B, cc) ;

        // rnorm = norm (B-A*X)
        Residual = cholmod_l_copy_dense (B, cc) ;
        cholmod_l_sdmult (A, 0, minusone, one, X, Residual, cc) ;
        rnorm = cholmod_l_norm_dense (Residual, 2, cc) ;
        printf ("2-norm of residual: %8.1e\n", rnorm) ;
        printf ("rank %ld\n", cc->SPQR_istat [4]) ;

        // free everything and finish CHOLMOD
        cholmod_l_free_dense (&Residual, cc) ;
        cholmod_l_free_sparse (&A, cc) ;
        cholmod_l_free_dense (&X, cc) ;
        cholmod_l_free_dense (&B, cc) ;
        cholmod_l_finish (cc) ;
        return (0) ;
    }
\end{verbatim}
}

%-------------------------------------------------------------------------------
\subsection{C++ Syntax}
%-------------------------------------------------------------------------------

All features available to the MATLAB user are also available to both the C and
C++ interfaces using a syntax that is not much more complicated than the MATLAB
syntax.  Additional features not available via the MATLAB interface include the
ability to compute the symbolic and numeric factorizations separately (for
multiple matrices with the same nonzero pattern but different numerical
values).  The following is a list of user-callable C++ functions and what they
can do:

\begin{enumerate}

    \item \verb'SuiteSparseQR': an overloaded function that provides functions
    equivalent to \verb'spqr' and \verb'spqr_solve' in the SuiteSparseQR MATLAB
    interface.

    \item \verb'SuiteSparseQR_factorize': performs both the symbolic and
    numeric factorizations and returns a QR factorization object such that
    \verb'A*P=Q*R'.  It always exploits singletons.

    \item \verb'SuiteSparseQR_symbolic': performs the symbolic factorization
    and returns a QR factorization object to be passed to
    \verb'SuiteSparseQR_numeric'.  It does not exploit singletons.

    \item \verb'SuiteSparseQR_numeric': performs the numeric factorization on a
    QR factorization object, either one constructed by
    \verb'SuiteSparseQR_symbolic', or reusing one from a prior call to
    \verb'SuiteSparseQR_numeric' for a matrix \verb'A' with the same pattern as
    the first one, but with different numerical values.

    \item \verb'SuiteSparseQR_solve': solves a linear system using the object
    returned by \newline \verb'SuiteSparseQR_factorize' or
    \verb'SuiteSparseQR_numeric', namely \verb"x=R\b", \newline \verb"x=P*R\b",
    \verb"x=R'\b", or \verb"x=R'\(P'*b)".

    \item \verb'SuiteSparseQR_qmult': provides the same function as
    \verb'spqr_qmult' in the MATLAB interface, computing \verb"Q*x",
    \verb"Q'*x", \verb"x*Q", or \verb"x*Q'".  It uses either a QR factorization
    in MATLAB-style sparse matrix format, or the QR factorization object
    returned by \newline \verb'SuiteSparseQR_factorize' or
    \verb'SuiteSparseQR_numeric'.

    \item \verb'SuiteSparseQR_min2norm': finds the minimum 2-norm solution to
    an underdetermined linear system.

    \item \verb'SuiteSparseQR_free': frees the QR factorization object.

\end{enumerate}

%-------------------------------------------------------------------------------
\subsection{Details of the C/C++ Syntax}
%-------------------------------------------------------------------------------

For further details of how to use the C/C++ syntax, please refer to the
definitions and descriptions in the following files:

\begin{enumerate}
\item \verb'SuiteSparse/SPQR/Include/SuiteSparseQR.hpp' describes each
C++ function.  Both \verb'double' and \verb'std::complex<double>' matrices
are supported.

\item \verb'SuiteSparse/SPQR/Include/SuiteSparseQR_definitions.h' describes
definitions \newline common to both C and C++ functions.  For example, each of
the ordering methods is given a \verb'#define''d name.  The default is
\verb'ordering = SPQR_ORDERING_DEFAULT', and the default tolerance is given by
\verb'tol = SPQR_DEFAULT_TOL'.

\item \verb'SuiteSparse/SPQR/Include/SuiteSparseQR_C.h' describes
the C-callable functions.

\end{enumerate}

Most of the packages in SuiteSparse come in multiple versions with different
sized integers.  The first is the plain C/C++ \verb'int'.  The second the
\verb'SuiteSparse_long' integer, defined in the
\verb'SuiteSparse/SuiteSparse_config/SuiteSparse_config.h'
file.  This integer is \verb'long' except on a Windows-64 platform for which it
is the  \verb'__int64' type.  The intent of \verb'SuiteSparse_long'
is that it should be
32-bits on a 32-bit platform, and 64-bits on a 64-bit platform.  

By contrast, SuiteSparseQR only provides a \verb'SuiteSparse_long' version.
Most users (except Windows-64) can simply use \verb'long' as the basic integer
type passed to and returned from SuiteSparseQR.

The C/C++ options corresponding to the MATLAB \verb'opts' parameters and the
contents of the optional \verb'info' output of \verb'spqr_solve' are described
below.  Let \verb'cc' be the CHOLMOD \verb'Common' object, containing parameter
settings and statistics.  All are of type \verb'double', except for
\verb'SPQR_istat' which is \verb'SuiteSparse_long',
\verb'cc->memory_usage' which is
\verb'size_t', and \verb'cc->SPQR_nthreads' which is \verb'int'.  Parameters
include:

\vspace{0.1in}
{\footnotesize
\begin{tabular}{|ll|}
\hline
\verb'cc->SPQR_grain' & the same as \verb'opts.grain' in the MATLAB interface \\
\verb'cc->SPQR_small' & the same as \verb'opts.small' in the MATLAB interface \\
\verb'cc->SPQR_nthreads'
    & the same as \verb'opts.nthreads' in the MATLAB interface \\
\hline
\end{tabular}
}
\vspace{0.1in}

Other parameters, such as \verb'opts.ordering' and \verb'opts.tol',
are input parameters to the various C/C++ functions.  Others such as
\verb"opts.solution='min2norm'" are separate functions in the C/C++
interface.  Refer to the files listed above for details.
Output statistics include:

\vspace{0.1in}
{\footnotesize
\begin{tabular}{|ll|}
\hline
\verb'cc->SPQR_xstat [0]' & an upper bound on the flop count \\
\verb'cc->SPQR_xstat [1]' & the tolerance used (\verb'opts.tol') \\
\hline
\verb'cc->SPQR_istat [0]' & upper bound on \verb'nnz(R)' \\
\verb'cc->SPQR_istat [1]' & upper bound on \verb'nnz(H)' \\
\verb'cc->SPQR_istat [2]' & number of frontal matrices \\
\verb'cc->SPQR_istat [3]' & number of TBB tasks \\
\verb'cc->SPQR_istat [4]' & estimate of the rank of \verb'A' \\
\verb'cc->SPQR_istat [5]' & number of column singletons \\
\verb'cc->SPQR_istat [6]' & number of row singletons \\
\verb'cc->SPQR_istat [7]' & ordering used \\
\hline
\verb'cc->memory_usage'   & memory used, in bytes \\
\hline
\end{tabular}
}
\vspace{0.1in}

The upper bound on the flop count is found in the analysis phase, which ignores
the numerical values of \verb'A' (the same analysis phase operates on both real
and complex matrices).  Thus, if you are factorizing a complex matrix, multiply
this statistic by 4.

%-------------------------------------------------------------------------------
\section{Requirements and Availability}
\label{summary}
%-------------------------------------------------------------------------------

SuiteSparseQR requires four prior Collected Algorithms of the ACM: CHOLMOD
\cite{ChenDavisHagerRajamanickam09,DavisHager09} (version 1.7 or later), AMD
\cite{AmestoyDavisDuff96,AmestoyDavisDuff03}, and COLAMD
\cite{DavisGilbertLarimoreNg00_algo,DavisGilbertLarimoreNg00} for its
ordering/analysis phase and for its basic sparse matrix data structure, and the
BLAS \cite{dddh:90} for dense matrix computations on its frontal matrices; also
required is LAPACK \cite{LAPACK} for its Householder reflections.  An efficient
implementation of the BLAS is strongly recommended, either vendor-provided
(such as the Intel MKL, the AMD ACML, or the Sun Performance Library) or other
high-performance BLAS such as those of \cite{GotoVanDeGeijn08}.

The use of Intel's Threading Building Blocks is optional \cite{Reinders07}, but
without it, only parallelism within the BLAS can be exploited (if available).
Suite\-SparseQR can optionally use METIS 4.0.1 \cite{KarypisKumar98e} and two
constrained minimum degree ordering algorithms, CCOLAMD and CAMD
\cite{ChenDavisHagerRajamanickam09}, for its fill-reducing ordering options.
SuiteSparseQR can be compiled without these ordering methods and without TBB.

In addition to appearing as Collected Algorithm 8xx of the ACM, SuiteSparseQR
is available at
\htmladdnormallink{http://www.suitesparse.com}{http://www.suitesparse.com}
and at MATLAB Central
in the user-contributed File Exchange (
\htmladdnormallink{http://www.mathworks.com/matlabcentral}{http://www.mathworks.com/matlabcentral}
).
SuiteSparseQR is licensed under the GNU GPL.  Commercial licenses are also
available; contact the author for details.

%-------------------------------------------------------------------------------
% References
%-------------------------------------------------------------------------------

\bibliographystyle{plain}
\bibliography{spqr_user_guide}
\end{document}

