\documentclass[acmtoms]{acmtrans2m}
\title{User Documentation}
\author{MASAO KODAMA}
\begin{document}
\maketitle
\section{The subprograms}
The subprograms belonging to the present algorithm are classified into 
the following Groups (A), (B), (C).\\
%
Group (A)
\begin{itemize}
\item SUBROUTINE bessel(znu, xx, zans, info)

zans: output, the complex value of $J_\nu(x)$.
\item SUBROUTINE neumann(znu, xx, zans, info)

zans: output, the complex value of $N_\nu(x)$.
\item SUBROUTINE hankel1(znu, xx, zans, info)

zans: output, the complex value of $H_\nu^{(1)}\!(x)$.
\item SUBROUTINE hankel2(znu, xx, zans, info)

zans: output, the complex value of $H_\nu^{(2)}\!(x)$.
\end{itemize}
The dummy arguments znu, xx, info have the following common roles.
\begin{itemize}
\item znu: input, complex order $\nu$.
\item xx: input, nonnegative argument $x$.
\item info: output, integer, information about the output zans.
\begin{itemize}
\item info=0:  normal output, $e_r\le5\!\times\! 10^3\epsilon_1$, where
$e_r$ is the relative error of zans.
\item info=5:  low accuracy, 
$5\!\times\! 10^3\epsilon_1 <e_r \le 2\!\times\! 10^7\epsilon_1$.
\item info=10: rough accuracy, $2\!\times\! 10^7\epsilon_1<e_r$.
\item info=20: (1) An overflow occurred. The value of zans
is the maximum available real number. (2) The answer zans is 
indefinite theoretically. For example, $J_{i}(0)$ is indefinite.
\item info=30:  out of range, $x<0$ for example.
\end{itemize}
\end{itemize}
%
Group (B)
\begin{itemize}
\item 
bes\_series: this calculates $J_\nu(x)$ with the series expansion in Section 
2.2.1.
\item 
neu\_series: this calculates $N_\nu(x)$ with the method stated in Section 
2.2.2.
\item 
bes\_han\_dby: this calculates $J_\nu(x)$, $H_\nu^{(1)}\!(x)$, 
$H_\nu^{(2)}\!(x)$ with Debye's expansions in Section 2.2.3.
\item 
bes\_olver: this calculates $J_\nu(x)$ with Olver's expansion stated in 
Section 2.2.4.
\item 
han2\_olver: this calculates $H_\nu^{(2)}\!(x)$ with Olver's expansion in 
Section 2.2.4.
\item 
bes\_recur: this calculates $J_\nu(x)$ with the recurrence method in Section 
2.2.5.
\item 
han2\_temme: this calculates $H_\nu^{(2)}\!(x)$ with Temme's algorithm
in Section 2.2.6.
\end{itemize}
Group (C)
\begin{itemize}
\item 
num\_region: this determines the region number $n$ that is the suffix $n$ of 
$R_n$.
\item 
neu\_srs\_init: this is invoked by neu\_series.
\item 
def\_bessel: this is invoked by neu\_srs\_init.
\item 
sumaabb: this is invoked by bes\_olver, han2\_olver.
\item 
fzeta: this is invoked by bes\_olver, han2\_olver.
\item aiz: 
this is invoked by bes\_olver, han2\_olver;
this subprogram calculates the Airy function Ai$(z)$ and its 
derivative Ai$'(z)$, and the subprogram is cited from Algorithm 819 \cite{gil}.
\item cdlgam: 
this is invoked by bes\_series, bes\_recur;
this subprogram calculates $\Gamma(\nu)$, $\log\Gamma(\nu)$
and is cited from Algorithm 421 \cite{kuki72b}. 
\item 
abs2: this calculates a rough absolute value of a complex number.
\end{itemize}

All the above subprograms are the module subprograms of MODULE mod\_bes. 
The four subroutines of Group (A) can be invoked by a user. 
If a subroutine of Group (A) is invoked, this 
subroutine invokes FUNCTION num\_region first, determines the region number 
$n$ and invokes one or two of the subroutines of Group (B) 
according to $n$.
In addition, subprograms of Group (C) help numerical calculation in 
subroutines of Group (B).
\bibliographystyle{acmtrans}
%\bibliography{manuscript_bes}
\begin{thebibliography}{}

\bibitem[\protect\citeauthoryear{Gil, Segura, and Temme}{Gil
  et~al\mbox{.}}{2002}]{gil}
{\sc Gil, A.}, {\sc Segura, J.}, {\sc and} {\sc Temme, N.~M.} 2002.
\newblock Algorithm 819: \protect{AIZ}, \protect{BIZ}: Two \protect{Fortran} 77
  routines for the computation of complex \protect{Airy} functions.
\newblock {\em ACM Trans. Math. Soft.\/}~{\em 28,\/}~3 (Sept.), 325--336.

\bibitem[\protect\citeauthoryear{Kuki}{Kuki}{1972}]{kuki72b}
{\sc Kuki, H.} 1972.
\newblock Algorithm 421\quad \protect{Complex} gamma function with error
  control [\protect{S14}].
\newblock {\em Commun. ACM\/}~{\em 15,\/}~4 (Apr.), 271--272.

\end{thebibliography}
\end{document}
