\begin{table}[ht]\noindent
\begin{center}
\renewcommand{\arraystretch}{1.20}
\begin{tabular}{c|l|c|c}

Class Name
            & Internal Representation
						& Digits
						& Padding \\

\hline

\multirow{4}{*}{{\courier{efx::e{\ttfamily\underline\ }float}}}
            & data: {\courier{UINT32}} base--$10^{8}$ array
						& \multirow{4}{*}{$30$--$300$}
						& \multirow{4}{*}{$15$\%} \\

\ %space
            & Boolean negative sign
            & \ %space
						& \ \\

\ %space
            & {\courier{INT64}} base--10 exponent
            & \ %space
						& \ \\

\ %space
            & floating point class (finite, NaN, {\emph{etc}}.)
            & \ %space
						& \ \\

\hline

\multirow{2}{*}{{\courier{gmp::e{\ttfamily\underline\ }float}}}
            & data: GMP's type {\courier{::mpf{\ttfamily\underline\ }t}}
						& \multirow{2}{*}{$30$--$300$}
						& \multirow{2}{*}{$15$\%} \\

\ %space
            & floating point class (finite, NaN, {\emph{etc}}.)
            & \ %space
						& \ \\

\hline

{\courier{mpfr::e{\ttfamily\underline\ }float}}
            & data: MPFR's type {\courier{::mpfr{\ttfamily\underline\ }t}}
						& $30$--$300$
						& $15$\% \\

\hline

{\courier{f90::e{\ttfamily\underline\ }float}}
            & data: Fortran's type {\courier{REAL(KIND\ensuremath{=}16)}}
						& $\sim$$\,30$
						& $4$--$5$ digits \\

\hline

\end{tabular}
\vspace{2.0mm}
\caption{The MP classes in \efloat\ are shown. Details about the internal representation
as well as the digit range and the added internal extra digits ({\emph{i.e.}} the padding) are given.}
\label{table:mpclasses}
\end{center}
\end{table}

