\documentclass{article}
\begin{document}

\section*{DESCRIPTION OF SUBROUTINES}

\noindent In this section we describe easy-to-use subroutines
which can be called from the user's program. In the description of
formal parameters we introduce a type of the argument denoted by
two letters. The first letter is either \texttt{I} for integer
arguments or \texttt{R} for double precision real arguments. The
second letter specifies whether the argument must have a value
defined on the entry to the subroutine (\texttt{I}), whether it is a
value which will be returned (\texttt{O}), or both (\texttt{U}), or
whether it is an auxiliary value (\texttt{A}). Notice that the input
type arguments can be changed on the output under some
circumstances, especially if improper input values were given.
Besides the formal parameters, we use a \verb|COMMON /STAT/| block
containing statistical information. This block, used in each
subroutine, has the following form:

{\small

\begin{verbatim}
      COMMON /STAT/ NDECF,NRES,NRED,NREM,NADD,NIT,NFV,NFG,NFH
\end{verbatim}

}

Its elements have the following meanings:

\vspace{2mm}

{\small

\noindent\parbox{20mm}{Element}\parbox{10mm}{$\!$Type}\parbox[t]{91mm}
  {Significance}\par\noindent\rule[1mm]{121mm}{.4pt}
  \par
\noindent\parbox{20mm}{\texttt{NDECF}}\parbox{10mm}{\texttt{IO}}\parbox[t]{91mm}{
  Number of matrix decompositions.}
  \par\vspace{2mm}
\noindent\parbox{20mm}{\texttt{NRES}}\parbox{10mm}{\texttt{IO}}\parbox[t]{91mm}{
  Number of restarts.}
  \par\vspace{2mm}
\noindent\parbox{20mm}{\texttt{NRED}}\parbox{10mm}{\texttt{IO}}\parbox[t]{91mm}{
  Number of reductions.}
  \par\vspace{2mm}
\noindent\parbox{20mm}{\texttt{NREM}}\parbox{10mm}{\texttt{IO}}\parbox[t]{91mm}{
  Number of constraint deletions during the QP solutions.}
  \par\vspace{2mm}
\noindent\parbox{20mm}{\texttt{NADD}}\parbox{10mm}{\texttt{IO}}\parbox[t]{91mm}{
  Number of constraint additions during the QP solutions.}
  \par\vspace{2mm}
\noindent\parbox{20mm}{\texttt{NIT}}\parbox{10mm}{\texttt{IO}}\parbox[t]{91mm}{
  Number of iterations.}
  \par\vspace{2mm}
\noindent\parbox{20mm}{\texttt{NFV}}\parbox{10mm}{\texttt{IO}}\parbox[t]{91mm}{
  Number of function evaluations.}
  \par\vspace{2mm}
\noindent\parbox{20mm}{\texttt{NFG}}\parbox{10mm}{\texttt{IO}}\parbox[t]{91mm}{
  Number of gradient evaluations.}
  \par\vspace{2mm}
\noindent\parbox{20mm}{\texttt{NFH}}\parbox{10mm}{\texttt{IO}}\parbox[t]{91mm}{
  Number of Hessian evaluations.}

\vspace{2mm}

}

\noindent Easy-to-use subroutines are called by the following statements:

{\small

\begin{verbatim}
      CALL PMINU(NF,NA,X,AF,IA,RA,IPAR,RPAR,F,GMAX,IEXT,ITERM)
      CALL PMINS(NF,NA,NB,X,IX,XL,XU,AF,IA,RA,IPAR,RPAR,F,GMAX,IEXT,ITERM)
      CALL PMINL(NF,NA,NB,NC,X,IX,XL,XU,CF,IC,CL,CU,CG,AF,IA,RA,IPAR,RPAR,
     & F,GMAX,IEXT,ITERM)
      CALL PBUNU(NF,NA,X,IA,RA,IPAR,RPAR,F,GMAX,ITERM)
      CALL PBUNS(NF,NA,NB,X,IX,XL,XU,IA,RA,IPAR,RPAR,F,GMAX,ITERM)
      CALL PBUNL(NF,NA,NB,NC,X,IX,XL,XU,CF,IC,CL,CU,CG,IA,RA,IPAR,RPAR,
     & F,GMAX,ITERM)
      CALL PNEWU(NF,NA,X,IA,RA,IPAR,RPAR,F,GMAX,IHES,ITERM)
      CALL PNEWS(NF,NA,NB,X,IX,XL,XU,IA,RA,IPAR,RPAR,F,GMAX,IHES,ITERM)
      CALL PNEWL(NF,NA,NB,NC,X,IX,XL,XU,CF,IC,CL,CU,CG,IA,RA,IPAR,RPAR,
     & F,GMAX,IHES,ITERM)
      CALL PVARU(NF,NA,X,RA,IPAR,RPAR,F,GMAX,ITERM)
      CALL PVARS(NF,NA,NB,X,IX,XL,XU,RA,IPAR,RPAR,F,GMAX,ITERM)
      CALL PVARL(NF,NA,NB,NC,X,IX,XL,XU,CF,IC,CL,CU,CG,IA,RA,IPAR,RPAR,
     & F,GMAX,ITERM)
\end{verbatim}

}

Their arguments have the following meanings:

%\newpage

\vspace{2mm}

{\small

\noindent\parbox{20mm}{Argument}\parbox{10mm}{$\!$Type}\parbox[t]{91mm}
  {Significance}\par\noindent\rule[1mm]{121mm}{.4pt}
  \par
\noindent\parbox{20mm}{\texttt{NF}}\parbox{10mm}{\texttt{II}}\parbox[t]{91mm}{
  Number of variables of the objective function.}
  \par\vspace{2mm}
\noindent\parbox{20mm}{\texttt{NA}}\parbox{10mm}{\texttt{II}}\parbox[t]{91mm}{
  Number of functions in the minimax
  criterion for subroutines {\tt PMINU}, {\tt PMINS}, {\tt PMINL} or the maximum bundle
  dimension for the other subroutines (choice $\texttt{NA}=0$ causes that the
  default value $\texttt{NA}=$\texttt{NF}+3 will be taken in a later case)}
  \par\vspace{2mm}
\noindent\parbox{20mm}{\texttt{NB}}\parbox{10mm}{\texttt{II}}\parbox[t]{91mm}{
  Specification whether the simple bounds are suppressed
  ($\texttt{NB}=0$) or accepted ($\texttt{NB}>0$).}
  \par\vspace{2mm}
\noindent\parbox{20mm}{\texttt{NC}}\parbox{10mm}{\texttt{II}}\parbox[t]{91mm}{
  Number of linear constraints; if $\texttt{NC}=0$ the linear constraints
  are suppressed.}
  \par\vspace{2mm}
\noindent\parbox{20mm}{\texttt{X(NF)}}\parbox{10mm}{\texttt{RU}}\parbox[t]{91mm}{
  On input, vector with the initial estimate to the solution.
  On output, the approximation to the minimum.}
  \par\vspace{2mm}
\noindent\parbox{20mm}{\texttt{IX(NF)}}\parbox{10mm}{\texttt{II}}\parbox[t]{91mm}{
  Vector containing the simple bound types (significant only if
  $\texttt{NB}>0$):}
  \par\vspace{1mm}
\noindent\parbox{30mm}{$\;$}\parbox{20mm}{$\texttt{IX(I)}=0$:}\parbox[t]{71mm}{
  the variable X(I) is unbounded,}
  \par
\noindent\parbox{30mm}{$\;$}\parbox{20mm}{$\texttt{IX(I)}=1$:}\parbox[t]{71mm}{
  the lower bound $\texttt{X(I)}\ge\texttt{XL(I)}$,}
  \par
\noindent\parbox{30mm}{$\;$}\parbox{20mm}{$\texttt{IX(I)}=2$:}\parbox[t]{71mm}{
  the upper bound $\texttt{X(I)}\le\texttt{XU(I)}$,}
  \par
\noindent\parbox{30mm}{$\;$}\parbox{20mm}{$\texttt{IX(I)}=3$:}\parbox[t]{71mm}{
  the two-side bound $\texttt{XL(I)}\le\texttt{X(I)}\le\texttt{XU(I)}$,}
  \par
\noindent\parbox{30mm}{$\;$}\parbox{20mm}{$\texttt{IX(I)}=5$:}\parbox[t]{71mm}{
  the variable \texttt{X(I)} is fixed (given by its initial estimate).}
  \par\vspace{2mm}
\noindent\parbox{20mm}{\texttt{XL(NF)}}\parbox{10mm}{\texttt{RI}}\parbox[t]{91mm}{
  Vector with lower bounds for variables
  (significant only if $\texttt{NB}>0$).}
  \par\vspace{2mm}
\noindent\parbox{20mm}{\texttt{XU(NF)}}\parbox{10mm}{\texttt{RI}}\parbox[t]{91mm}{
  Vector with upper bounds for variables
  (significant only if $\texttt{NB}>0$).}
  \par\vspace{2mm}
\noindent\parbox{20mm}{\texttt{CF(NC)}}\parbox{10mm}{\texttt{RA}}\parbox[t]{91mm}{
  Vector which contains values of constraint functions
  (significant only if $\texttt{NC}>0$).}
  \par\vspace{2mm}
\noindent\parbox{20mm}{\texttt{IC(NC)}}\parbox{10mm}{\texttt{II}}\parbox[t]{91mm}{
  INTEGER vector which contains constraint types (significant only if
  $\texttt{NC}>0$):}
  \par\vspace{1mm}
\noindent\parbox{30mm}{$\;$}\parbox{20mm}{$\texttt{IC(K)}=0$:}\parbox[t]{71mm}{
  the constraint \texttt{CF(K)} is not used,}
  \par
\noindent\parbox{30mm}{$\;$}\parbox{20mm}{$\texttt{IC(K)}=1$:}\parbox[t]{71mm}{
  the lower constraint $\texttt{CF(K)}\ge\texttt{CL(K)}$,}
  \par
\noindent\parbox{30mm}{$\;$}\parbox{20mm}{$\texttt{IC(K)}=2$:}\parbox[t]{71mm}{
  the upper constraint $\texttt{CF(K)}\le\texttt{CU(K)}$,}
  \par
\noindent\parbox{30mm}{$\;$}\parbox{20mm}{$\texttt{IC(K)}=3$:}\parbox[t]{71mm}{
  the two-side constraint $\texttt{CL(K)}\le\texttt{CF(K)}\le\texttt{CU(K)}$,}
  \par
\noindent\parbox{30mm}{$\;$}\parbox{20mm}{$\texttt{IC(K)}=5$:}\parbox[t]{71mm}{
  the equality constraint $\texttt{CF(K)}=\texttt{CL(K)}$.}
  \par\vspace{2mm}
\noindent\parbox{20mm}{\texttt{CL(NC)}}\parbox{10mm}{\texttt{RI}}\parbox[t]{91mm}{
  Vector with lower bounds for constraint functions
  (significant only if $\texttt{NC}>0$).}
  \par\vspace{2mm}
\noindent\parbox{20mm}{\texttt{CU(NC)}}\parbox{10mm}{\texttt{RI}}\parbox[t]{91mm}{
  Vector with upper bounds for constraint functions
  (significant only if $\texttt{NC}>0$).}
  \par\vspace{2mm}
\noindent\parbox{20mm}{\texttt{CG(NF*NC)}}\parbox{10mm}{\texttt{RI}}\parbox[t]{91mm}{
  Matrix whose columns are normals of the linear constraints
  (significant only if $\texttt{NC}>0$).}
  \par\vspace{2mm}
\noindent\parbox{20mm}{\texttt{AF(NA)}}\parbox{10mm}{\texttt{RO}}\parbox[t]{91mm}{
  Vector which contains the values of functions in the minimax
  criterion.}
  \par\vspace{2mm}
\noindent\parbox{20mm}{\texttt{IA(NIA)}}\parbox{10mm}{\texttt{IA}}\parbox[t]{91mm}{
  Working array of the dimension \texttt{NIA}, where at least
  \texttt{NIA}=\texttt{NF}+\texttt{NA}+1.}
  \par\vspace{2mm}
\noindent\parbox{20mm}{\texttt{RA(NRA)}}\parbox{10mm}{\texttt{RA}}\parbox[t]{91mm}{
  Working array of the dimension \texttt{NRA}. The minimum
  values of \texttt{NRA} required in individual subroutines can be found in
  text files {\tt PMIN.TXT}, {\tt PBUN.TXT}, {\tt PNEW.TXT}, {\tt PVAR.TXT}.}
  \par\vspace{2mm}
\noindent\parbox{20mm}{\texttt{IPAR(7)}}\parbox{10mm}{\texttt{IA}}\parbox[t]{91mm}{
  Integer parameters (see Table 5.1).}
  \par\vspace{2mm}
\noindent\parbox{20mm}{\texttt{RPAR(7)}}\parbox{10mm}{\texttt{RA}}\parbox[t]{91mm}{
  Real parameters (see Table 5.1).}
  \par\vspace{2mm}
\noindent\parbox{20mm}{\texttt{F}}\parbox{10mm}{\texttt{RO}}\parbox[t]{91mm}{
  Value of the objective function at the solution \texttt{X}.}
  \par\vspace{2mm}
\noindent\parbox{20mm}{\texttt{GMAX}}\parbox{10mm}{\texttt{RO}}\parbox[t]{91mm}{
  value indicating the termination ($\|g^k\|_{\infty}$ in {\tt PMIN}
  or $w^k$ in {\tt PBUN}, {\tt PNEW} and {\tt PVAR}).}
  \par\vspace{2mm}
\noindent\parbox{20mm}{\texttt{IEXT}}\parbox{10mm}{\texttt{II}}\parbox[t]{91mm}{
  Variable that specifies the minimax criterion:}
  \par\vspace{1mm}
\noindent\parbox{30mm}{$\;$}\parbox{20mm}{$\texttt{IEXT}<0$:}\parbox[t]{71mm}{
  maximum of positive values,}
  \par
\noindent\parbox{30mm}{$\;$}\parbox{20mm}{$\texttt{IEXT}=0$:}\parbox[t]{71mm}{
  maximum of absolute values,}
  \par
\noindent\parbox{30mm}{$\;$}\parbox{20mm}{$\texttt{IEXT}>0$:}\parbox[t]{71mm}{
  maximum of negative values,}
  \par\vspace{2mm}
\noindent\parbox{20mm}{\texttt{IHES}}\parbox{10mm}{\texttt{II}}\parbox[t]{91mm}{
  Variable that specifies a way for computing second derivatives:}
  \par\vspace{1mm}
\noindent\parbox{30mm}{$\;$}\parbox{20mm}{$\texttt{IHES}=0$:}\parbox[t]{71mm}{
  numerical computation,}
  \par
\noindent\parbox{30mm}{$\;$}\parbox{20mm}{$\texttt{IHES}=1$:}\parbox[t]{71mm}{
  analytical computation by the user supplied subroutine HES.}
  \par\vspace{2mm}
\noindent\parbox{20mm}{\texttt{ITERM}}\parbox{10mm}{\texttt{IO}}\parbox[t]{91mm}{
  Variable that indicates the cause of termination:}
  \par\vspace{1mm}
\noindent\parbox{30mm}{$\;$}\parbox{20mm}{$\texttt{ITERM}=1$:}\parbox[t]{71mm}{
  if $|x - x_{old}|$ was less than or equal to \texttt{TOLX} in \texttt{MTESX}
  subsequent iterations,}
  \par
\noindent\parbox{30mm}{$\;$}\parbox{20mm}{$\texttt{ITERM}=2$:}\parbox[t]{71mm}{
  if $|F - F_{old}|$ was less than or equal to \texttt{TOLF} in \texttt{MTESF}
  subsequent iterations,}
  \par
\noindent\parbox{30mm}{$\;$}\parbox{20mm}{$\texttt{ITERM}=3$:}\parbox[t]{71mm}{
  if \texttt{F} is less than or equal to \texttt{TOLB},}
  \par
\noindent\parbox{30mm}{$\;$}\parbox{20mm}{$\texttt{ITERM}=4$:}\parbox[t]{71mm}{
  if \texttt{GMAX} is less than or equal to \texttt{TOLG},}
  \par
\noindent\parbox{30mm}{$\;$}\parbox{20mm}{$\texttt{ITERM}=11$:}\parbox[t]{71mm}{
  if \texttt{NFV} exceeded \texttt{MFV},}
  \par
\noindent\parbox{30mm}{$\;$}\parbox{20mm}{$\texttt{ITERM}=12$:}\parbox[t]{71mm}{
  if \texttt{NIT} exceeded \texttt{MIT},}
  \par
\noindent\parbox{30mm}{$\;$}\parbox{20mm}{$\texttt{ITERM}<0$:}\parbox[t]{71mm}{
  if the method failed ({$\texttt{ITERM}=-6$} if the required precision was not achieved,
  {$\texttt{ITERM}=-10$} if two consecutive restarts were required, {$\texttt{ITERM}=-12$}
  if the quadratic programming subroutine failed).}

}

The integer and real parameters are listed in the following table:

\vspace{2mm}

\begin{center}
\begin{tabular}{c|llll} \hline
Parameter & {\tt PMIN} & {\tt PBUN} & {\tt PNEW} & {\tt PVAR} \\ \hline
{\tt IPAR(1)} &   {\tt MET} &   {\tt MOT} &   {\tt MOS} &   {\tt MEX} \\
{\tt IPAR(2)} &   {\tt MEC} &   {\tt MES} &   {\tt MES} &   {\tt MOS} \\
{\tt IPAR(3)} &   {\tt MER} & {\tt MTESX} & {\tt MTESX} & {\tt MTESX} \\
{\tt IPAR(4)} &   {\tt MES} & {\tt MTESF} & {\tt MTESF} & {\tt MTESF} \\
{\tt IPAR(5)} &   {\tt MIT} &   {\tt MIT} &   {\tt MIT} &   {\tt MIT} \\
{\tt IPAR(6)} &   {\tt MFV} &   {\tt MFV} &   {\tt MFV} &   {\tt MFV} \\
{\tt IPAR(7)} & {\tt IPRNT} & {\tt IPRNT} & {\tt IPRNT} & {\tt IPRNT} \\ \hline
{\tt RPAR(1)} &  {\tt TOLX} &  {\tt TOLX} &  {\tt TOLX} &  {\tt TOLX} \\
{\tt RPAR(2)} &  {\tt TOLF} &  {\tt TOLF} &  {\tt TOLF} &  {\tt TOLF} \\
{\tt RPAR(3)} &  {\tt TOLB} &  {\tt TOLB} &  {\tt TOLB} &  {\tt TOLB} \\
{\tt RPAR(4)} &  {\tt TOLG} &  {\tt TOLG} &  {\tt TOLG} &  {\tt TOLG} \\
{\tt RPAR(5)} &  {\tt TOLD} &  {\tt TOLD} &  {\tt TOLD} &   {\tt ETA} \\
{\tt RPAR(6)} &  {\tt TOLS} &  {\tt TOLS} &  {\tt TOLS} &   {\tt EPS} \\
{\tt RPAR(7)} &  {\tt XMAX} &  {\tt TOLP }&  {\tt TOLP} &  {\tt XMAX} \\
{\tt RPAR(8)} &          -  &   {\tt ETA} &   {\tt ETA} &          -  \\
{\tt RPAR(9)} &          -  &  {\tt XMAX} &  {\tt XMAX} &          -  \\ \hline
\end{tabular}

\vspace{4mm}

Table 5.2 - Integer and real parameters
\end{center}

\vspace{2mm}

Integer and real parameters have the following meanings:

\vspace{2mm}

{\small

\noindent\parbox{20mm}{Argument}\parbox{10mm}{$\!$Type}\parbox[t]{91mm}
  {Significance}\par\noindent\rule[1mm]{121mm}{.4pt}
  \par
\noindent\parbox{20mm}{\texttt{MET}}\parbox{10mm}{\texttt{II}}\parbox[t]{91mm}{
  Variable that specifies self-scaling for variable metric updates:}
  \par\vspace{1mm}
\noindent\parbox{30mm}{$\;$}\parbox{20mm}{$\texttt{MET}=1$:}\parbox[t]{71mm}{
  self-scaling is suppressed,}
  \par
\noindent\parbox{30mm}{$\;$}\parbox{20mm}{$\texttt{MET}=2$:}\parbox[t]{71mm}{
  self-scaling is used only in the first iteration (initial self-scaling),}
  \par
\noindent\parbox{30mm}{$\;$}\parbox{20mm}{$\texttt{MET}=3$:}\parbox[t]{71mm}{
  self-scaling is controlled by a special procedure.}
  \par\vspace{1mm}
\noindent\parbox{30mm}{$\;$}\parbox[t]{91mm}{The choice $\texttt{MET}=0$
  causes that the default value $\texttt{MET}=3$ will be taken.}
  \par\vspace{2mm}
\noindent\parbox{20mm}{\texttt{MEC}}\parbox{10mm}{\texttt{II}}\parbox[t]{91mm}{
  Variable that specifies correction of variable metric updates if negative
  curvature occurs:}
  \par\vspace{1mm}
\noindent\parbox{30mm}{$\;$}\parbox{20mm}{$\texttt{MEC}=1$:}\parbox[t]{71mm}{
  correction is suppressed,}
  \par
\noindent\parbox{30mm}{$\;$}\parbox{20mm}{$\texttt{MEC}=2$:}\parbox[t]{71mm}{
  Powell's correction is used.}
  \par\vspace{1mm}
\noindent\parbox{30mm}{$\;$}\parbox[t]{91mm}{The choice $\texttt{MEC}=0$
  causes that the default value $\texttt{MEC}=1$ will be taken.}
  \par\vspace{2mm}
\noindent\parbox{20mm}{\texttt{MER}}\parbox{10mm}{\texttt{II}}\parbox[t]{91mm}{
  Variable that specifies restart after unsuccessgful variable metric updates:}
  \par\vspace{1mm}
\noindent\parbox{30mm}{$\;$}\parbox{20mm}{$\texttt{MER}=0$:}\parbox[t]{71mm}{
  restart is suppressed,}
  \par
\noindent\parbox{30mm}{$\;$}\parbox{20mm}{$\texttt{MER}=1$:}\parbox[t]{71mm}{
  variable metric method is restarted by using the unit matrix.}
  \par\vspace{1mm}
\noindent\parbox{20mm}{\texttt{MEX}}\parbox{10mm}{\texttt{II}}\parbox[t]{91mm}{
  Variable that specifies version of nonsmooth variable metric method:}
  \par\vspace{1mm}
\noindent\parbox{30mm}{$\;$}\parbox{20mm}{$\texttt{MEX}=0$:}\parbox[t]{71mm}{
  convex version is used,}
  \par
\noindent\parbox{30mm}{$\;$}\parbox{20mm}{$\texttt{MEX}=1$:}\parbox[t]{71mm}{
  nonconvex version is used.}
  \par\vspace{1mm}
\noindent\parbox{20mm}{\texttt{MOT}}\parbox{10mm}{\texttt{II}}\parbox[t]{91mm}{
  Variable that specifies the weight updating method:}
  \par\vspace{1mm}
\noindent\parbox{30mm}{$\;$}\parbox{20mm}{$\texttt{MOT}=1$:}\parbox[t]{71mm}{
  quadratic interpolation,}
  \par
\noindent\parbox{30mm}{$\;$}\parbox{20mm}{$\texttt{MOT}=2$:}\parbox[t]{71mm}{
  local minimization,}
  \par
\noindent\parbox{30mm}{$\;$}\parbox{20mm}{$\texttt{MOT}=3$:}\parbox[t]{71mm}{
  quasi-Newton condition.}
  \par\vspace{1mm}
\noindent\parbox{30mm}{$\;$}\parbox[t]{91mm}{The choice $\texttt{MOT}=0$
  causes that the default value $\texttt{MOT}=1$ will be taken.}
  \par\vspace{2mm}
\noindent\parbox{20mm}{\texttt{MOS}}\parbox{10mm}{\texttt{II}}\parbox[t]{91mm}{
  Distance measure exponent $\omega$ (either $1$ or $2$). The choice
  $\texttt{MOS}=0$ causes that the default value $\texttt{MOS}=1$ will be taken.}
  \par\vspace{2mm}
\noindent\parbox{20mm}{\texttt{MES}}\parbox{10mm}{\texttt{II}}\parbox[t]{91mm}{
  Variable that specifies the interpolation method selection in a
  line search (until a sufficient function decrease is reached; then only
  bisection will be used):}
  \par\vspace{1mm}
\noindent\parbox{30mm}{$\;$}\parbox{20mm}{$\texttt{MES}=1$:}\parbox[t]{71mm}{
  bisection,}
  \par
\noindent\parbox{30mm}{$\;$}\parbox{20mm}{$\texttt{MES}=2$:}\parbox[t]{71mm}{
  two-point quadratic interpolation.}
  \par\vspace{1mm}
\noindent\parbox{30mm}{$\;$}\parbox[t]{91mm}{The choice $\texttt{MES}=0$
  causes that the default value $\texttt{MES}=2$ will be taken.}
  \par\vspace{2mm}
\noindent\parbox{20mm}{\texttt{MTESX}}\parbox{10mm}{\texttt{II}}\parbox[t]{91mm}{
  Maximum number of iterations with changes
  of the coordinate vector \texttt{X} smaller than \texttt{TOLX}; the choice
  $\texttt{MTESX}=0$ causes that the default value $20$ will be taken.}
  \par\vspace{2mm}
\noindent\parbox{20mm}{\texttt{MTESF}}\parbox{10mm}{\texttt{II}}\parbox[t]{91mm}{
  Maximum number of iterations with changes
  of function values smaller than TOLF; the choice $\texttt{MTESF}=0$ causes that
  the default value $2$ will be taken.}
  \par\vspace{2mm}
\noindent\parbox{20mm}{\texttt{MIT}}\parbox{10mm}{\texttt{II}}\parbox[t]{91mm}{
  Maximum number of iterations; the choice
  $\texttt{MIT}=0$ causes that the default value $200$ will be taken.}
  \par\vspace{2mm}
\noindent\parbox{20mm}{\texttt{MFV}}\parbox{10mm}{\texttt{II}}\parbox[t]{91mm}{
  Maximum number of function evaluations;
  the choice $\texttt{MFV}=0$ causes that the default value $500$ will be taken.}
  \par\vspace{2mm}
\noindent\parbox{20mm}{\texttt{IPRNT}}\parbox{10mm}{\texttt{II}}\parbox[t]{91mm}{
  Print specification:}  \par\vspace{1mm}
\noindent\parbox{30mm}{$\;$}\parbox{20mm}{$\texttt{IPRNT}= 0$:}\parbox[t]{71mm}{
  print is suppressed,}
  \par
\noindent\parbox{30mm}{$\;$}\parbox{20mm}{$\texttt{IPRNT}= 1$:}\parbox[t]{71mm}{
  basic print of final results,}
  \par
\noindent\parbox{30mm}{$\;$}\parbox{20mm}{$\texttt{IPRNT}=-1$:}\parbox[t]{71mm}{
  extended print of final results,}
  \par
\noindent\parbox{30mm}{$\;$}\parbox{20mm}{$\texttt{IPRNT}= 2$:}\parbox[t]{71mm}{
  basic print of intermediate and final results,}
  \par
\noindent\parbox{30mm}{$\;$}\parbox{20mm}{$\texttt{IPRNT}=-2$:}\parbox[t]{71mm}{
  extended print of intermediate and final results,}
  \par
\noindent\parbox{20mm}{\texttt{TOLX}}\parbox{10mm}{\texttt{RI}}\parbox[t]{91mm}{
  Tolerance for the change of the coordinate vector \texttt{X};
  the choice $\texttt{TOLX}=0$ causes that the default value $10^{-16}$ will be
  taken.}
  \par\vspace{2mm}
\noindent\parbox{20mm}{\texttt{TOLF}}\parbox{10mm}{\texttt{RI}}\parbox[t]{91mm}{
  Tolerance for the change of the function value; the choice
  $\texttt{TOLF}=0$ causes that the default value $10^{-8}$ will be taken.}
  \par\vspace{2mm}
\noindent\parbox{20mm}{\texttt{TOLB}}\parbox{10mm}{\texttt{RI}}\parbox[t]{91mm}{
  Minimum acceptable function value; the choice $\texttt{TOLB}=
  0$ causes that the default value $-10^{60}$ will be taken.}
  \par\vspace{2mm}
\noindent\parbox{20mm}{\texttt{TOLG}}\parbox{10mm}{\texttt{RI}}\parbox[t]{91mm}{
  Tolerance for the gradient of the Lagrangian function; the
  choice $\texttt{TOLG}=0$ causes that the default value $10^{-6}$ will be taken.}
  \par\vspace{2mm}
\noindent\parbox{20mm}{\texttt{TOLD}}\parbox{10mm}{\texttt{RI}}\parbox[t]{91mm}{
  Restart tolerance $\underline{\varepsilon}$; the choice $\texttt{TOLD}=0$
  causes that the default value $10^{-4}$ will be taken.}
  \par\vspace{2mm}
\noindent\parbox{20mm}{\texttt{TOLS}}\parbox{10mm}{\texttt{RI}}\parbox[t]{91mm}{
  Line search tolerance $m_L$; the choice $\texttt{TOLS}=0$
  causes that the default value $10^{-2}$ will be taken.}
  \par\vspace{2mm}
\noindent\parbox{20mm}{\texttt{TOLP}}\parbox{10mm}{\texttt{RI}}\parbox[t]{91mm}{
  Line search tolerance $m_R$; the choice $\texttt{TOLP}=0$
  causes that the default value $0.5$ will be taken.}
  \par\vspace{2mm}
\noindent\parbox{20mm}{\texttt{ETA}}\parbox{10mm}{\texttt{RI}}\parbox[t]{91mm}{
  Distance measure parameter $\gamma$.}
  \par\vspace{2mm}
\noindent\parbox{20mm}{\texttt{EPS}}\parbox{10mm}{\texttt{RI}}\parbox[t]{91mm}{
  Parameter for constraint deletion; the choice $\texttt{EPS}=0$ causes
  that the default value $0.5$ will be taken.}
  \par\vspace{2mm}
\noindent\parbox{20mm}{\texttt{XMAX}}\parbox{10mm}{\texttt{RI}}\parbox[t]{91mm}{
  Maximum stepsize; the choice $\texttt{XMAX}=0$ causes that
  the default value $10^3$ will be taken.}

}

\vspace{2mm}

The choice of parameters {\texttt{ETA}} and {\texttt{XMAX}} is rather
delicate. It can considerably influence the efficiency of the
method. Therefore, these parameters should be tuned carefully.
Briefly, the parameter {\texttt{ETA}} should be smaller (e.g.
$10^{-12}-10^{-6}$) for convex problems and larger (e.g.
$10^{-4}-10^2$) for nonconvex problems. The parameter
{\texttt{XMAX}} reduces the stepsize so that it plays an important
role in the neighborhood of the kink. The other parameters are not
so important, but small {\texttt{MTESX}} or {\texttt{MTESF}} can lead
to premature termination of the iterative process and an
unsuitable value of {\texttt{EPS}} can unfavourably influence
constraint handling in the linearly constrained case.

Subroutines {\tt PMINU}, {\tt PMINS}, {\tt PMINL} require the user supplied
subroutines {\tt FUN} and {\tt DER} which define the values and the gradients
of the functions in the minimax criterion and have the form

\begin{verbatim}
      SUBROUTINE FUN(NF,KA,X,FA)
      SUBROUTINE DER(NF,KA,X,GA)
\end{verbatim}

\noindent Subroutines {\tt PBUNU}, {\tt PBUNS}, {\tt PBUNL}, {\tt PNEWU},
{\tt PNEWS}, {\tt PNEWL}, {\tt PVARU}, {\tt PVARS}, {\tt PVARL} require
the user supplied subroutine {\tt FUNDER} which defines the objective
function and its subgradient and has the form

\begin{verbatim}
      SUBROUTINE FUNDER(NF,X,F,G)
\end{verbatim}

\noindent Subroutines {\tt PNEWU}, {\tt PNEWS}, {\tt PNEWL} require the
additional user supplied subroutine {\tt HES} which defines the matrix of the
second-order information (usually the Hessian matrix) and has the form

\begin{verbatim}
      SUBROUTINE HES(NF,X,H)
\end{verbatim}

\noindent If \texttt{IHES}=0, then the user supplied subroutine {\tt HES}
can be empty.

The arguments of user supplied subroutines have the following
meanings:

\vspace{2mm}

%\newpage

{\small

\noindent\parbox{20mm}{Argument}\parbox{10mm}{$\!$Type}\parbox[t]{91mm}
  {Significance}\par\noindent\rule[1mm]{121mm}{.4pt}
  \par
\noindent\parbox{20mm}{\texttt{NF}}\parbox{10mm}{\texttt{II}}\parbox[t]{91mm}{
  Number of variables of the
  objective function.}
  \par\vspace{2mm}
\noindent\parbox{20mm}{\texttt{KA}}\parbox{10mm}{\texttt{II}}\parbox[t]{91mm}{
  Index of a function in the minimax criterion.}
  \par\vspace{2mm}
\noindent\parbox{20mm}{\texttt{X(NF)}}\parbox{10mm}{\texttt{RI}}\parbox[t]{91mm}{
  An estimate to the solution.}
  \par\vspace{2mm}
\noindent\parbox{20mm}{\texttt{FA}}\parbox{10mm}{\texttt{RO}}\parbox[t]{91mm}{
  Value of a function with the index \texttt{KA} at point \texttt{X}.}
  \par\vspace{2mm}
\noindent\parbox{20mm}{\texttt{GA(NF)}}\parbox{10mm}{\texttt{RO}}\parbox[t]{91mm}{
  Gradient of a function with the index \texttt{KA} at point \texttt{X}.}
  \par\vspace{2mm}
\noindent\parbox{20mm}{\texttt{F}}\parbox{10mm}{\texttt{RO}}\parbox[t]{91mm}{
  Value of the objective function at point \texttt{X}.}
  \par\vspace{2mm}
\noindent\parbox{20mm}{\texttt{G(NF)}}\parbox{10mm}{\texttt{RO}}\parbox[t]{91mm}{
  Subgradient of the objective function at point \texttt{X}.}
  \par\vspace{2mm}
\noindent\parbox{20mm}{\texttt{H(NH)}}\parbox{10mm}{\texttt{RO}}\parbox[t]{91mm}{
  Matrix of the second-order information at point \texttt{X}
  (NH is equal to NF*(NF+1)/2).}

\vspace{5mm}

}

\vspace{5mm}

\end{document}
