\PassOptionsToPackage{unicode=true}{hyperref} % options for packages loaded elsewhere
\PassOptionsToPackage{hyphens}{url}
%
\documentclass[]{article}
\usepackage{lmodern}
\usepackage{amssymb,amsmath}
\usepackage{ifxetex,ifluatex}
\usepackage{fixltx2e} % provides \textsubscript
\ifnum 0\ifxetex 1\fi\ifluatex 1\fi=0 % if pdftex
  \usepackage[T1]{fontenc}
  \usepackage[utf8]{inputenc}
  \usepackage{textcomp} % provides euro and other symbols
\else % if luatex or xelatex
  \usepackage{unicode-math}
  \defaultfontfeatures{Ligatures=TeX,Scale=MatchLowercase}
\fi
% use upquote if available, for straight quotes in verbatim environments
\IfFileExists{upquote.sty}{\usepackage{upquote}}{}
% use microtype if available
\IfFileExists{microtype.sty}{%
\usepackage[]{microtype}
\UseMicrotypeSet[protrusion]{basicmath} % disable protrusion for tt fonts
}{}
\IfFileExists{parskip.sty}{%
\usepackage{parskip}
}{% else
\setlength{\parindent}{0pt}
\setlength{\parskip}{6pt plus 2pt minus 1pt}
}
\usepackage{hyperref}
\hypersetup{
            pdftitle={MinDistance},
            pdfauthor={Basic installation and user guide},
            pdfborder={0 0 0},
            breaklinks=true}
\urlstyle{same}  % don't use monospace font for urls
\usepackage[margin=1.5cm]{geometry}
\usepackage{color}
\usepackage{fancyvrb}
\newcommand{\VerbBar}{|}
\newcommand{\VERB}{\Verb[commandchars=\\\{\}]}
\DefineVerbatimEnvironment{Highlighting}{Verbatim}{commandchars=\\\{\}}
% Add ',fontsize=\small' for more characters per line
\newenvironment{Shaded}{}{}
\newcommand{\AlertTok}[1]{\textcolor[rgb]{1.00,0.00,0.00}{\textbf{#1}}}
\newcommand{\AnnotationTok}[1]{\textcolor[rgb]{0.38,0.63,0.69}{\textbf{\textit{#1}}}}
\newcommand{\AttributeTok}[1]{\textcolor[rgb]{0.49,0.56,0.16}{#1}}
\newcommand{\BaseNTok}[1]{\textcolor[rgb]{0.25,0.63,0.44}{#1}}
\newcommand{\BuiltInTok}[1]{#1}
\newcommand{\CharTok}[1]{\textcolor[rgb]{0.25,0.44,0.63}{#1}}
\newcommand{\CommentTok}[1]{\textcolor[rgb]{0.38,0.63,0.69}{\textit{#1}}}
\newcommand{\CommentVarTok}[1]{\textcolor[rgb]{0.38,0.63,0.69}{\textbf{\textit{#1}}}}
\newcommand{\ConstantTok}[1]{\textcolor[rgb]{0.53,0.00,0.00}{#1}}
\newcommand{\ControlFlowTok}[1]{\textcolor[rgb]{0.00,0.44,0.13}{\textbf{#1}}}
\newcommand{\DataTypeTok}[1]{\textcolor[rgb]{0.56,0.13,0.00}{#1}}
\newcommand{\DecValTok}[1]{\textcolor[rgb]{0.25,0.63,0.44}{#1}}
\newcommand{\DocumentationTok}[1]{\textcolor[rgb]{0.73,0.13,0.13}{\textit{#1}}}
\newcommand{\ErrorTok}[1]{\textcolor[rgb]{1.00,0.00,0.00}{\textbf{#1}}}
\newcommand{\ExtensionTok}[1]{#1}
\newcommand{\FloatTok}[1]{\textcolor[rgb]{0.25,0.63,0.44}{#1}}
\newcommand{\FunctionTok}[1]{\textcolor[rgb]{0.02,0.16,0.49}{#1}}
\newcommand{\ImportTok}[1]{#1}
\newcommand{\InformationTok}[1]{\textcolor[rgb]{0.38,0.63,0.69}{\textbf{\textit{#1}}}}
\newcommand{\KeywordTok}[1]{\textcolor[rgb]{0.00,0.44,0.13}{\textbf{#1}}}
\newcommand{\NormalTok}[1]{#1}
\newcommand{\OperatorTok}[1]{\textcolor[rgb]{0.40,0.40,0.40}{#1}}
\newcommand{\OtherTok}[1]{\textcolor[rgb]{0.00,0.44,0.13}{#1}}
\newcommand{\PreprocessorTok}[1]{\textcolor[rgb]{0.74,0.48,0.00}{#1}}
\newcommand{\RegionMarkerTok}[1]{#1}
\newcommand{\SpecialCharTok}[1]{\textcolor[rgb]{0.25,0.44,0.63}{#1}}
\newcommand{\SpecialStringTok}[1]{\textcolor[rgb]{0.73,0.40,0.53}{#1}}
\newcommand{\StringTok}[1]{\textcolor[rgb]{0.25,0.44,0.63}{#1}}
\newcommand{\VariableTok}[1]{\textcolor[rgb]{0.10,0.09,0.49}{#1}}
\newcommand{\VerbatimStringTok}[1]{\textcolor[rgb]{0.25,0.44,0.63}{#1}}
\newcommand{\WarningTok}[1]{\textcolor[rgb]{0.38,0.63,0.69}{\textbf{\textit{#1}}}}
\setlength{\emergencystretch}{3em}  % prevent overfull lines
\providecommand{\tightlist}{%
  \setlength{\itemsep}{0pt}\setlength{\parskip}{0pt}}
\setcounter{secnumdepth}{0}
% Redefines (sub)paragraphs to behave more like sections
\ifx\paragraph\undefined\else
\let\oldparagraph\paragraph
\renewcommand{\paragraph}[1]{\oldparagraph{#1}\mbox{}}
\fi
\ifx\subparagraph\undefined\else
\let\oldsubparagraph\subparagraph
\renewcommand{\subparagraph}[1]{\oldsubparagraph{#1}\mbox{}}
\fi

% set default figure placement to htbp
\makeatletter
\def\fps@figure{htbp}
\makeatother


\title{MinDistance}
\author{Basic installation and user guide}
\date{May 14, 2018}

\begin{document}
\maketitle

MinDistance is a software package with several fast scalar, vector and
parallel implementations for computing the minimum distance of a random
linear code.

This guide complements the submitted paper \emph{Fast Algorithms for the
Computation of the Minimum Distance of a Random Linear Code} and
includes basic setup and execution steps for the package.

The provided software package includes all the algorithms described in
the paper (described in Section 3 of the manuscript) as well as
vectorized and parallelized versions of the implementations (described
in Section 4 of the manuscript).

\hypertarget{basic-installation-steps}{%
\section{Basic installation steps}\label{basic-installation-steps}}

\hypertarget{configuration-and-testing}{%
\subsection{Configuration and testing}\label{configuration-and-testing}}

The configuration of MinDistance comprises only two steps:

\begin{Shaded}
\begin{Highlighting}[]
\CommentTok{# Step 1. Configuration}
\ExtensionTok{./configure}

\CommentTok{# Step 2. Compilation}
\FunctionTok{make}
\end{Highlighting}
\end{Shaded}

\hypertarget{vectorization-support}{%
\subsubsection{Vectorization support}\label{vectorization-support}}

MinDistance is ready to exploit vector units present in many modern
processors. \textbf{By default, vectorization is disabled}.

To activate this functionality, use the
\texttt{-\/-enable-vectorization=yes} option in the configuration step:

\begin{Shaded}
\begin{Highlighting}[]
\ExtensionTok{./configure}\NormalTok{ --enable-vectorization=yes}
\end{Highlighting}
\end{Shaded}

The configuration script will automatically detect the vector
capabilities of the CPU and use the most advanced (SSE, AVX2 or AVX512)
to improve performance.

Alternatively, the user can force a specific vectorization level by
using the \texttt{sse}, \texttt{avx2} or \texttt{avx512} arguments for
the \texttt{-\/-enable-vectorization} option. Note that the selected
extension must be provided by the CPU and the compiler in use.

The vectorized codes heavily rely on the architectural support for the
POPCNT instruction. It is expected no performance gain (or even
performance decrease) in the case of processors with vector support but
without POPCNT support (e.g.~old Intel Core 2 Duo processors). On modern
processors, forcing SSE mode could produce higher performances than more
modern modes (AVX2 and AVX512) for matrices where
128\textless{}=n\textless{}256. If you are going to evaluate matrices
within those dimensions, you might achieve higher performances by
forcing sse mode (\texttt{-\/-enable-vectorization=sse}) instead of
using auto mode (\texttt{-\/-enable-vectorization=yes}).

\hypertarget{modifying-cflags}{%
\subsubsection{\texorpdfstring{Modifying
\texttt{CFLAGS}}{Modifying CFLAGS}}\label{modifying-cflags}}

In order to pass a specific flag to the compiler (e.g.~optimization
flags), set the value of the \texttt{CFLAGS} environment variable prior
to the configuration step, for example:

\begin{Shaded}
\begin{Highlighting}[]
\CommentTok{# Sets -O3}
\VariableTok{CFLAGS=}\NormalTok{-O3 }\ExtensionTok{./configure} 
\end{Highlighting}
\end{Shaded}

\hypertarget{testing-correctness}{%
\subsubsection{Testing correctness}\label{testing-correctness}}

After compilation, it is highly recommended to run a simple sanity check
to validate the installation. A script with name
\texttt{check\_smallset\_a.sh} is provided in the \texttt{test} folder
to validate the correctness of the solutions using small matrices.

The expected output of a correct execution is:

\begin{Shaded}
\begin{Highlighting}[]
\FunctionTok{sh}\NormalTok{ check_smallset_a.sh }
 
\ExtensionTok{Testing}\NormalTok{ file:  ../matrices/mat_smallset_a/atest001_4x4x0.in     Ok}
\ExtensionTok{Testing}\NormalTok{ file:  ../matrices/mat_smallset_a/atest002_5x3x5.in     Ok}
\ExtensionTok{Testing}\NormalTok{ file:  ../matrices/mat_smallset_a/atest003_7x3x4.in     Ok}
\ExtensionTok{Testing}\NormalTok{ file:  ../matrices/mat_smallset_a/atest004_15x5x7.in     Ok}
\ExtensionTok{Testing}\NormalTok{ file:  ../matrices/mat_smallset_a/atest005_9x4x2.in     Ok}
\ExtensionTok{Testing}\NormalTok{ file:  ../matrices/mat_smallset_a/atest006_10x4x3.in     Ok}
\ExtensionTok{Testing}\NormalTok{ file:  ../matrices/mat_smallset_a/atest007_11x4x3.in     Ok}
\ExtensionTok{Testing}\NormalTok{ file:  ../matrices/mat_smallset_a/atest008_12x4x2.in     Ok}
\ExtensionTok{Testing}\NormalTok{ file:  ../matrices/mat_smallset_a/atest009_14x7x2.in     Ok}
\ExtensionTok{Testing}\NormalTok{ file:  ../matrices/mat_smallset_a/atest010_15x7x3.in     Ok}
\ExtensionTok{Testing}\NormalTok{ file:  ../matrices/mat_smallset_a/atest011_16x7x4.in     Ok}
\ExtensionTok{Testing}\NormalTok{ file:  ../matrices/mat_smallset_a/atest012_17x7x3.in     Ok}
\ExtensionTok{Testing}\NormalTok{ file:  ../matrices/mat_smallset_a/atest013_18x7x5.in     Ok}
\ExtensionTok{Testing}\NormalTok{ file:  ../matrices/mat_smallset_a/atest014_40x20x6.in     Ok}
\ExtensionTok{Testing}\NormalTok{ file:  ../matrices/mat_smallset_a/atest015_100x70x7.in     Ok}
\ExtensionTok{Testing}\NormalTok{ file:  ../matrices/mat_smallset_a/atest016_94x70x5.in     Ok}
\ExtensionTok{Testing}\NormalTok{ file:  ../matrices/mat_smallset_a/atest017_90x63x6.in     Ok}
\ExtensionTok{Testing}\NormalTok{ file:  ../matrices/mat_smallset_a/atest018_84x47x8.in     Ok}
\ExtensionTok{Testing}\NormalTok{ file:  ../matrices/mat_smallset_a/atest019_73x47x7.in     Ok}
\ExtensionTok{Testing}\NormalTok{ file:  ../matrices/mat_smallset_a/atest020_90x50x9.in     Ok}
\ExtensionTok{Testing}\NormalTok{ file:  ../matrices/mat_smallset_a/atest021_50x20x8.in     Ok}
\ExtensionTok{Testing}\NormalTok{ file:  ../matrices/mat_smallset_a/atest022_30x10x7.in     Ok}
\ExtensionTok{Testing}\NormalTok{ file:  ../matrices/mat_smallset_a/atest023_23x10x6.in     Ok}
\ExtensionTok{Testing}\NormalTok{ file:  ../matrices/mat_smallset_a/atest024_100x15x33.in     Ok}
\ExtensionTok{Testing}\NormalTok{ file:  ../matrices/mat_smallset_a/atest025_277x15x110.in     Ok}
\ExtensionTok{Testing}\NormalTok{ file:  ../matrices/mat_smallset_a/atest026_100x20x28.in     Ok}
\ExtensionTok{Testing}\NormalTok{ file:  ../matrices/mat_smallset_a/atest027_115x20x33.in     Ok}
\ExtensionTok{Testing}\NormalTok{ file:  ../matrices/mat_smallset_a/atest028_154x20x50.in     Ok}
\ExtensionTok{Testing}\NormalTok{ file:  ../matrices/mat_smallset_a/atest029_153x20x50.in     Ok}
\ExtensionTok{Testing}\NormalTok{ file:  ../matrices/mat_smallset_a/atest030_600x20x244.in     Ok}
\ExtensionTok{Testing}\NormalTok{ file:  ../matrices/mat_smallset_a/atest031_20x10x1.in     Ok}
\end{Highlighting}
\end{Shaded}

\hypertarget{usage}{%
\subsection{Usage}\label{usage}}

The name of the generated executable is \texttt{test\_distance}. It can
be invoked as:

\begin{Shaded}
\begin{Highlighting}[]
\ExtensionTok{./src/test_distance} \OperatorTok{<}\NormalTok{matrix_name}\OperatorTok{>}
\end{Highlighting}
\end{Shaded}

where \texttt{\textless{}matrix\_name\textgreater{}} is the name of the
generator to use.

\hypertarget{execution-configuration}{%
\subsection{Execution configuration}\label{execution-configuration}}

The configuration file with name \texttt{config.in} must reside in the
current directory. Otherwise, the application will abort with an error
message. This text file contains the parameters of the execution to
launch.

A sample \texttt{config.in} is provided in the package. The following is
a sample of the contents of the file:

\begin{Shaded}
\begin{Highlighting}[]
\ExtensionTok{4}\NormalTok{ 10         Algorithm (1:bas}\KeywordTok{;} \ExtensionTok{2}\NormalTok{:opt}\KeywordTok{;} \ExtensionTok{3}\NormalTok{:sta}\KeywordTok{;} \ExtensionTok{4}\NormalTok{:sav}\KeywordTok{;} \ExtensionTok{5}\NormalTok{:sav+unr}\KeywordTok{;} \ExtensionTok{10}\NormalTok{:gray)}
 \ExtensionTok{5}\NormalTok{ 4         Number of saved generators stored in RAM (only for sav algs.)}
\ExtensionTok{1}\NormalTok{ 2 4 16 28  Number of cores used in the execution of the program}
 \ExtensionTok{10}\NormalTok{          Number of permutations to perform in the diagonalization (}\OperatorTok{>}\NormalTok{=1)}\ExtensionTok{.}
\ExtensionTok{0}\NormalTok{            Print matrices (0=no}\KeywordTok{;}\ExtensionTok{1}\NormalTok{=yes)}
\end{Highlighting}
\end{Shaded}

For each row, only the first integer number is considered; the rest of
the line is just ignored.

The following are some remarks about each line in the configuration
file:

\begin{enumerate}
\def\labelenumi{\arabic{enumi}.}
\item
  Algorithm selection:

  \begin{itemize}
  \tightlist
  \item
    \textbf{bas (1)}: Basic algorithm, described in Subsection 3.2 of
    the manuscript.
  \item
    \textbf{opt (2)}: Optimized algorithm, described in Subsection 3.3
    of the manuscript.
  \item
    \textbf{sta (3)}: Stack-based algorithm, described in Subsection 3.4
    of the manuscript.
  \item
    \textbf{sav (4)}: Algorithm with saved additions, described in
    Subsection 3.5 of the manuscript.
  \item
    \textbf{sav+unr (5)}: Algorithm with saved additions and unrollings,
    described in Subsection 3.6 of the manuscript.
  \item
    \textbf{gray (10)}: Brute-force algorithm with Gray code based
    enumeration, described in Subsection 3.1 of the manuscript.
  \end{itemize}
\item
  Number of generators that are saved to RAM (only for altogithms 4 and
  5).
\item
  In parallel executions, number of threads to deploy and use.
\item
  Number of permutations to perform in the diagonalization stage (must
  be a positive integer).
\item
  Produce verbose output (print the input and output matrices).
\end{enumerate}

\hypertarget{generators-provided-for-testing-purposes}{%
\subsection{Generators provided for testing
purposes}\label{generators-provided-for-testing-purposes}}

\begin{itemize}
\item
  \emph{mat\_smallset\_a}: A set of small code generators that can be
  employed for checking the installation. This set includes several
  generator dimensions and lengths. Depending on the architecture and
  the number of cores, the processing of all these generators usually
  takes a few minutes on the best algorithms.
\item
  \emph{mat\_smallset\_b}: Same as before, but the generators are a bit
  larger, and hence the processing can take longer.
\item
  \emph{mat\_perf}: A set of large generators. The processing of one
  generator can take from several minutes to several weeks.
\end{itemize}

\hypertarget{input-generator-file-format}{%
\subsection{Input generator file
format}\label{input-generator-file-format}}

Generators are provided as input data and follow a strict data format.
Users must be sure they correctly prepare and format the input matrices
following one of the supported formats, namely:

\begin{itemize}
\tightlist
\item
  Format 1:
\end{itemize}

\begin{verbatim}
 3 7   matrix dimensions
  1 0 0 1 0 1 1
  0 1 0 1 1 0 1
  0 0 1 0 1 1 1
\end{verbatim}

\begin{itemize}
\tightlist
\item
  Format 2:
\end{itemize}

\begin{verbatim}
 3 7   matrix dimensions
 [ [ 1, 0, 0, 1, 0, 1, 1 ],
   [ 0, 1, 0, 1, 1, 0, 1 ],
   [ 0, 0, 1, 0, 1, 1, 1 ] ]
\end{verbatim}

The generators included in the package usually follow a common name
scheme to easily identify them:

\begin{verbatim}
CODENAME_NxKxDIST.in
\end{verbatim}

When DIST (distance) is unknown, a \texttt{\_} character is used.

\hypertarget{sample-execution-and-output}{%
\subsection{Sample execution and
output}\label{sample-execution-and-output}}

The output of a sample execution with the configuration file described
above would include the following information:

\hypertarget{execution-configuration-information}{%
\subsubsection{Execution configuration
information}\label{execution-configuration-information}}

Includes vectorization type, matrix information and configuration
details extracted from \texttt{config.in}. A sample is shown below:

\begin{Shaded}
\begin{Highlighting}[]
\ExtensionTok{Vectorization}\NormalTok{ scheme: AVX2}
\ExtensionTok{Matrix}\NormalTok{ dimensions:  47 x 84 }
\ExtensionTok{Chars}\NormalTok{ in file after reading matrix:}
\ExtensionTok{Trailing}\NormalTok{ char: }\StringTok{']'} 
\ExtensionTok{Trailing}\NormalTok{ char: }\StringTok{'}
\StringTok{'} 
\ExtensionTok{Trailing}\NormalTok{ char: }\StringTok{'}
\StringTok{'} 
\ExtensionTok{End}\NormalTok{ of chars in file after reading matrix.}
\ExtensionTok{Read}\NormalTok{ input matrix. Elapsed time (s.)}\BuiltInTok{:}\NormalTok{ 0.000280}

\ExtensionTok\NormalTok{ Number of saved generators:      5}
\ExtensionTok\NormalTok{ Number of permutations:          10}
\ExtensionTok{%}\NormalTok{ Print matrices:                  0}
\end{Highlighting}
\end{Shaded}

\hypertarget{gamma-matrices-creation-information}{%
\subsubsection{Gamma matrices creation
information}\label{gamma-matrices-creation-information}}

Includes, if configured, information about the creation of data
structures for saving combinations. A sample is shown below:

\begin{Shaded}
\begin{Highlighting}[]
\ExtensionTok{Creating}\NormalTok{ vector of Gamma matrices.}
\ExtensionTok{End}\NormalTok{ of creating vector of Gamma matrices.}
\ExtensionTok{Creating}\NormalTok{ data structures for saving combinations.}
\ExtensionTok{Generators}\NormalTok{:  1  Combinations:         47  Required accum.mem.(MB)}\BuiltInTok{:}\NormalTok{ 0.0}
\ExtensionTok{Generators}\NormalTok{:  2  Combinations:       1081  Required accum.mem.(MB)}\BuiltInTok{:}\NormalTok{ 0.0}
\ExtensionTok{Generators}\NormalTok{:  3  Combinations:      16215  Required accum.mem.(MB)}\BuiltInTok{:}\NormalTok{ 0.3}
\ExtensionTok{Generators}\NormalTok{:  4  Combinations:     178365  Required accum.mem.(MB)}\BuiltInTok{:}\NormalTok{ 3.0}
\ExtensionTok{Generators}\NormalTok{:  5  Combinations:    1533939  Required accum.mem.(MB)}\BuiltInTok{:}\NormalTok{ 26.4}
\ExtensionTok{End}\NormalTok{ of creating data structures for saving combinations.}
\end{Highlighting}
\end{Shaded}

In this example, a total memory of about 26 MB is employed for saving
generators. The larger generators are processed, the larger the total
accumulated memory will be. The user must check that this required space
is not too large. The number of saved generators is defined in the
\texttt{config.in} file.

\hypertarget{permutations-and-diagonalization}{%
\subsubsection{Permutations and
diagonalization}\label{permutations-and-diagonalization}}

Includes information about the diagonalization stages. A sample is shown
below:

\begin{Shaded}
\begin{Highlighting}[]
\ExtensionTok{Creating}\NormalTok{ vector of Gamma matrices.}
\ExtensionTok{End}\NormalTok{ of creating vector of Gamma matrices.}

\ExtensionTok{Diagonalizing}\NormalTok{ permuted matrix 0}
\ExtensionTok{Matrix}\NormalTok{ diagonalized.}
\ExtensionTok{Rank}\NormalTok{ vector: 7 7 3 }

\ExtensionTok{Diagonalizing}\NormalTok{ permuted matrix 1}
\ExtensionTok{Matrix}\NormalTok{ diagonalized.}
\ExtensionTok{Rank}\NormalTok{ vector: 7 7 3 }

\ExtensionTok{Diagonalizing}\NormalTok{ permuted matrix 2}
\ExtensionTok{Matrix}\NormalTok{ diagonalized.}
\ExtensionTok{Rank}\NormalTok{ vector: 7 7 3 }

\ExtensionTok{Diagonalizing}\NormalTok{ permuted matrix 3}
\ExtensionTok{Matrix}\NormalTok{ diagonalized.}
\ExtensionTok{Rank}\NormalTok{ vector: 7 7 3 }

\ExtensionTok{Diagonalizing}\NormalTok{ permuted matrix 4}
\ExtensionTok{Matrix}\NormalTok{ diagonalized.}
\ExtensionTok{Rank}\NormalTok{ vector: 7 7 3 }

\ExtensionTok{Diagonalizing}\NormalTok{ permuted matrix 5}
\ExtensionTok{Matrix}\NormalTok{ diagonalized.}
\ExtensionTok{Rank}\NormalTok{ vector: 7 7 3 }

\ExtensionTok{Diagonalizing}\NormalTok{ permuted matrix 6}
\ExtensionTok{Matrix}\NormalTok{ diagonalized.}
\ExtensionTok{Rank}\NormalTok{ vector: 7 7 3 }

\ExtensionTok{Diagonalizing}\NormalTok{ permuted matrix 7}
\ExtensionTok{Matrix}\NormalTok{ diagonalized.}
\ExtensionTok{Rank}\NormalTok{ vector: 7 7 3 }

\ExtensionTok{Diagonalizing}\NormalTok{ permuted matrix 8}
\ExtensionTok{Matrix}\NormalTok{ diagonalized.}
\ExtensionTok{Rank}\NormalTok{ vector: 7 6 4 }

\ExtensionTok{Diagonalizing}\NormalTok{ permuted matrix 9}
\ExtensionTok{Matrix}\NormalTok{ diagonalized.}
\ExtensionTok{Rank}\NormalTok{ vector: 7 6 3 }

\ExtensionTok{Best}\NormalTok{ rank vector: 7 7 3 }

\ExtensionTok{Erasing}\NormalTok{ vector of Gamma matrices.}
\ExtensionTok{Finished}\NormalTok{ matrix diagonalizations. Elapsed time (s.)}\BuiltInTok{:}\NormalTok{ 0.00}
\end{Highlighting}
\end{Shaded}

As can be see, the application tests several permutations in order to
keep the one with the largest ranks in the Gamma matrices.

\hypertarget{detailed-distance-calculation-process}{%
\subsubsection{Detailed distance calculation
process}\label{detailed-distance-calculation-process}}

Includes information about the iterative process to find the minimum
distance. A sample is shown below:

\begin{Shaded}
\begin{Highlighting}[]
\ExtensionTok{MEMORY_ALIGNMENT}\NormalTok{:                   16}
\ExtensionTok{Dimensions}\NormalTok{ of input matrix:         47 x 84 }
\ExtensionTok{Dimensions}\NormalTok{ of compacted Gamma mats: 94 x 3 }
\ExtensionTok{Distance}\NormalTok{ Loop. lowerDist: 1  upperDist: 84 }
\ExtensionTok{kInput}\NormalTok{:            47}
\ExtensionTok{numGammaMatrices}\NormalTok{:  2}
  \ExtensionTok{Generators}\NormalTok{: 1}
  \ExtensionTok{numGMatProcessed}\NormalTok{: 1}
  \ExtensionTok{numGMatContrib}\NormalTok{:   2}
    \ExtensionTok{Gamma}\NormalTok{ Matrix: 1 of 2 }
    \ExtensionTok{generate_with_saved_alg}
    \ExtensionTok{process_prefix_with_saved_alg}
    \ExtensionTok{fill_structures_for_saved_data_for_1_s}
    \ExtensionTok{generate_with_saved_recursive}\NormalTok{ with cores: 1  0 1}
    \ExtensionTok{generate_with_saved_1_s}
    \ExtensionTok{End}\NormalTok{ Gamma. lower:  2  upper: 14        Elapsed time (s.)}\BuiltInTok{:}\NormalTok{ 0.02}

    \ExtensionTok{Gamma}\NormalTok{ Matrix: 2 of 2 }
    \ExtensionTok{generate_with_saved_alg}
    \ExtensionTok{process_prefix_with_saved_alg}
    \ExtensionTok{fill_structures_for_saved_data_for_1_s}
    \ExtensionTok{generate_with_saved_recursive}\NormalTok{ with cores: 1  0 1}
    \ExtensionTok{generate_with_saved_1_s}
    \ExtensionTok{End}\NormalTok{ Gamma. lower:  2  upper: 13        Elapsed time (s.)}\BuiltInTok{:}\NormalTok{ 0.02}

  \ExtensionTok{End}\NormalTok{ Combin.  lower:  2  upper: 13        Elapsed time (s.)}\BuiltInTok{:}\NormalTok{ 0.02}

  \ExtensionTok{Generators}\NormalTok{: 2}
  \ExtensionTok{numGMatProcessed}\NormalTok{: 1}
  \ExtensionTok{numGMatContrib}\NormalTok{:   2}
    \ExtensionTok{Gamma}\NormalTok{ Matrix: 1 of 2 }
    \ExtensionTok{generate_with_saved_alg}
    \ExtensionTok{process_prefix_with_saved_alg}
    \ExtensionTok{fill_structures_for_saved_data_for_1_s}
    \ExtensionTok{generate_with_saved_recursive}\NormalTok{ with cores: 1  0 2}
    \ExtensionTok{generate_with_saved_1_s}
    \ExtensionTok{End}\NormalTok{ Gamma. lower:  3  upper: 11        Elapsed time (s.)}\BuiltInTok{:}\NormalTok{ 0.02}

    \ExtensionTok{Gamma}\NormalTok{ Matrix: 2 of 2 }
    \ExtensionTok{generate_with_saved_alg}
    \ExtensionTok{process_prefix_with_saved_alg}
    \ExtensionTok{fill_structures_for_saved_data_for_1_s}
    \ExtensionTok{generate_with_saved_recursive}\NormalTok{ with cores: 1  0 2}
    \ExtensionTok{generate_with_saved_1_s}
    \ExtensionTok{End}\NormalTok{ Gamma. lower:  3  upper: 11        Elapsed time (s.)}\BuiltInTok{:}\NormalTok{ 0.02}

  \ExtensionTok{End}\NormalTok{ Combin.  lower:  3  upper: 11        Elapsed time (s.)}\BuiltInTok{:}\NormalTok{ 0.02}

  \ExtensionTok{Generators}\NormalTok{: 3}
  \ExtensionTok{numGMatProcessed}\NormalTok{: 1}
  \ExtensionTok{numGMatContrib}\NormalTok{:   2}
    \ExtensionTok{Gamma}\NormalTok{ Matrix: 1 of 2 }
    \ExtensionTok{generate_with_saved_alg}
    \ExtensionTok{process_prefix_with_saved_alg}
    \ExtensionTok{fill_structures_for_saved_data_for_1_s}
    \ExtensionTok{generate_with_saved_recursive}\NormalTok{ with cores: 1  0 3}
    \ExtensionTok{generate_with_saved_1_s}
    \ExtensionTok{End}\NormalTok{ Gamma. lower:  4  upper: 10        Elapsed time (s.)}\BuiltInTok{:}\NormalTok{ 0.02}

    \ExtensionTok{Gamma}\NormalTok{ Matrix: 2 of 2 }
    \ExtensionTok{generate_with_saved_alg}
    \ExtensionTok{process_prefix_with_saved_alg}
    \ExtensionTok{fill_structures_for_saved_data_for_1_s}
    \ExtensionTok{generate_with_saved_recursive}\NormalTok{ with cores: 1  0 3}
    \ExtensionTok{generate_with_saved_1_s}
    \ExtensionTok{End}\NormalTok{ Gamma. lower:  4  upper: 10        Elapsed time (s.)}\BuiltInTok{:}\NormalTok{ 0.02}

  \ExtensionTok{End}\NormalTok{ Combin.  lower:  4  upper: 10        Elapsed time (s.)}\BuiltInTok{:}\NormalTok{ 0.02}

  \ExtensionTok{Generators}\NormalTok{: 4}
  \ExtensionTok{numGMatProcessed}\NormalTok{: 1}
  \ExtensionTok{numGMatContrib}\NormalTok{:   1}
    \ExtensionTok{Gamma}\NormalTok{ Matrix: 1 of 2 }
    \ExtensionTok{generate_with_saved_alg}
    \ExtensionTok{process_prefix_with_saved_alg}
    \ExtensionTok{fill_structures_for_saved_data_for_1_s}
    \ExtensionTok{generate_with_saved_recursive}\NormalTok{ with cores: 1  0 4}
    \ExtensionTok{generate_with_saved_1_s}
    \ExtensionTok{End}\NormalTok{ Gamma. lower:  5  upper: 10        Elapsed time (s.)}\BuiltInTok{:}\NormalTok{ 0.02}

  \ExtensionTok{End}\NormalTok{ Combin.  lower:  5  upper: 10        Elapsed time (s.)}\BuiltInTok{:}\NormalTok{ 0.02}

  \ExtensionTok{Generators}\NormalTok{: 5}
  \ExtensionTok{numGMatProcessed}\NormalTok{: 1}
  \ExtensionTok{numGMatContrib}\NormalTok{:   1}
    \ExtensionTok{Gamma}\NormalTok{ Matrix: 1 of 2 }
    \ExtensionTok{generate_with_saved_alg}
    \ExtensionTok{process_prefix_with_saved_alg}
    \ExtensionTok{fill_structures_for_saved_data_for_1_s}
    \ExtensionTok{generate_with_saved_recursive}\NormalTok{ with cores: 1  0 5}
    \ExtensionTok{generate_with_saved_1_s}
    \ExtensionTok{End}\NormalTok{ Gamma. lower:  6  upper:  8        Elapsed time (s.)}\BuiltInTok{:}\NormalTok{ 0.04}

  \ExtensionTok{End}\NormalTok{ Combin.  lower:  6  upper:  8        Elapsed time (s.)}\BuiltInTok{:}\NormalTok{ 0.04}

  \ExtensionTok{Generators}\NormalTok{: 6}
  \ExtensionTok{numGMatProcessed}\NormalTok{: 1}
  \ExtensionTok{numGMatContrib}\NormalTok{:   1}
    \ExtensionTok{Gamma}\NormalTok{ Matrix: 1 of 2 }
    \ExtensionTok{generate_with_saved_alg}
    \ExtensionTok{process_prefix_with_saved_alg}
    \ExtensionTok{generate_with_saved_recursive}\NormalTok{ with cores: 1  0 6}
    \ExtensionTok{End}\NormalTok{ Gamma. lower:  7  upper:  8        Elapsed time (s.)}\BuiltInTok{:}\NormalTok{ 0.06}

  \ExtensionTok{End}\NormalTok{ Combin.  lower:  7  upper:  8        Elapsed time (s.)}\BuiltInTok{:}\NormalTok{ 0.06}

  \ExtensionTok{Generators}\NormalTok{: 7}
  \ExtensionTok{numGMatProcessed}\NormalTok{: 1}
  \ExtensionTok{numGMatContrib}\NormalTok{:   1}
    \ExtensionTok{Gamma}\NormalTok{ Matrix: 1 of 2 }
    \ExtensionTok{generate_with_saved_alg}
    \ExtensionTok{process_prefix_with_saved_alg}
    \ExtensionTok{generate_with_saved_recursive}\NormalTok{ with cores: 1  0 7}
    \ExtensionTok{End}\NormalTok{ Gamma. lower:  8  upper:  8        Elapsed time (s.)}\BuiltInTok{:}\NormalTok{ 0.22}

  \ExtensionTok{End}\NormalTok{ Combin.  lower:  8  upper:  8        Elapsed time (s.)}\BuiltInTok{:}\NormalTok{ 0.22}

\ExtensionTok{End}\NormalTok{ Distance.  lower:  8  upper:  8        Elapsed time (s.)}\BuiltInTok{:}\NormalTok{ 0.22}

\ExtensionTok{Erasing}\NormalTok{ vector of Gamma matrices.}
\ExtensionTok{Erasing}\NormalTok{ data structures for saving addition of combinations.}
\ExtensionTok{Computed}\NormalTok{ distance: 8     Elapsed time (s.)}\BuiltInTok{:}\NormalTok{ 0.23}



\ExtensionTok{Distance}\NormalTok{ of input matrix:  8 }
\end{Highlighting}
\end{Shaded}

As can be seen, the application first processes combinations of one row
(Generators: 1), then it processes combinations of two rows (Generators:
2), and so on. This process finishes when the lower bound is equal to or
larger than the upper bound.

\end{document}
